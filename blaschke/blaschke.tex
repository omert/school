\documentclass[11pt]{book} \usepackage{amssymb}
\usepackage{amsfonts} \usepackage{amsmath} \usepackage{bm}
\usepackage{latexsym} \usepackage{epsfig}
\usepackage{ccaption}
\usepackage{subfigure}

\renewcommand{\figurename}{Fig.}
\captiondelim{}

\newtheorem{theorem}{Theorem}[section]
\newtheorem{lemma}[theorem]{Lemma}
\newtheorem{proposition}[theorem]{Proposition}
\newtheorem{corollary}[theorem]{Corollary}
\newtheorem{fact}[theorem]{Fact}
\newtheorem{definition}[theorem]{Definition}
\newtheorem{remark}[theorem]{Remark}
\newtheorem{conjecture}[theorem]{Conjecture}
\newtheorem{claim}[theorem]{Claim}
\newtheorem{example}[theorem]{Example}
\newenvironment{proof}{\noindent \textbf{Proof:}}{$\Box$}

\newcommand{\myvec}[1]{\mathfrak{#1}}

\begin{document}
\title{Affine Differential Geometry}

 \author{Wilhelm Blaschke, Translated by Omer Tamuz}
\maketitle

\chapter{Local Theory of Planar Curves}
\section{Affine Mapping}
We would like to commence by recalling some well known facts of analytical 
geometry.

For our purposes, it will be useful to represent the points of a plane with
a parallel coordinate system. We choose two intersecting lines\footnote{We
refer to straight lines simply as ``lines'' throughout the text (translator).}
in the plane and mark on each some direction as positive. We call these lines
the $x_1$ and $x_2$ axes. Through a remaining point of the plane we pass
parallels to the axes, and assign the corresponding segments of the axes 
the numbers $x_1$ and $x_2$. We have thus defined a bijection between 
pairs of real number $x_1$ and $x_2$, to the points of the plane 
(Fig.~\ref{fig:axes}).

\begin{figure}[htp]
  \label{fig:axes}
  \centering
  \includegraphics{axes.pdf} 
  \caption{}
\end{figure}

We will thus deal with ``real'' points and lines, and in fact consider only
those points and lines that in the framework of projective geometry are 
considered ``proper'', as opposed to ``improper'' or infinitely remote elements.

In place of the pair $x_1$, $x_2$ we use the shorter vector notation and 
speak simply of the point $\myvec{x}$.

Let $\myvec{x}\to \myvec{x}^*$ be a mapping between the points of plane,
which is given by a system of linear relations between their coordinates:
\begin{equation}
  \label{eq:affine_trans}
  \begin{array}{rcl}
    x_1^* &=& c_{10}+c_{11}x_1+c_{12}x_2\\
    x_2^* &=& c_{20}+c_{21}x_1+c_{22}x_2.
  \end{array}
\end{equation}
This system of equations is solvable for $x$ when the determinant 
\begin{equation}
  d=c_{11}c_{22}-c_{21}c_{12}=
  \begin{vmatrix}
    c_{11} & c_{12}\\
    c_{21} & c_{22}
  \end{vmatrix}
\end{equation}
is different than zero ($\neq 0$). Then, to each $\myvec{x}^*$ corresponds 
one and only one $\myvec{x}$. Following Euler \cite{Euler:1749} and 
M\"obius \cite{Mobius:1827}, this kind of mapping
is called an ``affine mapping'', ``affine transformation'' or just ``affinity''.
 
The inverse transformation, which one can determine by 
solving~\eqref{eq:affine_trans}, has again the same form: 
\begin{equation}
  \label{eq:affine_trans_inv}
  \begin{array}{rcl}
    x_1 &=& c_{10}^*+c_{11}^*x_1^*+c_{12}^*x_2^*\\
    x_2 &=& c_{20}^*+c_{21}^*x_1^*+c_{22}^*x_2^*,
  \end{array}
\end{equation}
where
\begin{equation}
  \label{eq:inv_c}
  \begin{array}{lll}
    c_{10}^*=-{c_{10}c_{22}-c_{20}c_{12} \over d},& c_{11}^*=+{c_{22}\over d},&c_{12}^*=-{c_{12}\over d},\\
    c_{20}^*=-{c_{10}c_{21}-c_{20}c_{11} \over d},& c_{21}^*=+{c_{21}\over d},&c_{22}^*=-{c_{11}\over d}.
  \end{array}
\end{equation}
The determinant
\begin{equation}
  \label{eq:inv_det}
  d^*=c_{11}^*c_{22}^*-c_{12}^*c_{21}^*={1\over d}
\end{equation}
is different than zero and hence the inverse transformation is also affine.

The points $\myvec{x}$ of a line $\mathfrak{g}$ are mapped by an affine 
transformation to
the points $\myvec{x}^*$ of a line $\mathfrak{g}^*$. In fact, a line 
$\mathfrak{g}$ can be
represented by a linear equation ($g_1$, $g_2$ not both zero):
\begin{equation}
  \label{eq:line}
  g_0+g_1x_1+g_2x_2=0.
\end{equation}
The substitution of~\eqref{eq:inv_c} into this equation results, because of the
invertibilty of the transformation, in an equation which is not trivially
satisfied, and thus again the locus of $\myvec{x}^*$ is a line.

It further follows from~\eqref{eq:affine_trans} that affine transformations are
continuous mappings. That is, as $\myvec{x}$ tends to $\myvec{x}_0$ the 
corresponding $\myvec{x}^*$ tends to $\myvec{x}^*_0$.

Using M\"obius nets ~\cite{Mobius:1827} one can show that the last
two properties, whose dependence or independence remain to be decided, are
the definitive properties of affine transformations: {\em Affine mappings are
the only coordinate transformations (in the plane) which are without
exception bijective, continuous and map lines to lines.}

Affine transformations preserve parallelism, since parallel lines on the plane
are characterized by not having common points. Furthermore, given two 
affinities $\myvec{x}\to\myvec{x}^*$ and $\myvec{x}^*\to\myvec{x}^{**}$ , their
``product'' $\myvec{x}\to\myvec{x}^{**}$ is again an affinity, since all the
characteristics of an affinity hold. Naturally, this can also be derived 
from~\eqref{eq:affine_trans}. Now a set of transformations that is closed under
the transformation product is called a group. {\em Hence the 
affinities~\eqref{eq:affine_trans} form a group.}

An affinity~\eqref{eq:affine_trans} can be decomposed into a ``displacement''
\begin{equation}
  \label{eq:decompose_displacement}
  \begin{array}{ll}
    x_1^*=c_{10}+x_1,& x_2^*=c_{20}+x_2
  \end{array}
\end{equation}
and a preceding ``homogeneous affinity''
\begin{equation}
  \label{eq:decompose_homogeneous}
  \begin{array}{ll}
    \begin{array}{l}
      x_1^*=c_{11}x_1+c_{12}x_2,\\
      x_2^*=c_{21}x_1+c_{22}x_2,
    \end{array}
    & c_{11}c_{22}-c_{12}c_{21}\neq 0.
  \end{array}
\end{equation}

Evidently the displacements or ``translations'' alone, and likewise the 
homogeneous affinities by themselves, constitute a group of transformations.
We thus have two ``subgroups'' of the general affine group considered above.

For our purposes an additional subgroup is particularly important: namely the
so called ``equiareal affinities''. Consider then the effect of the homogeneous
affinity~\eqref{eq:decompose_homogeneous} on the determinant
\begin{equation}
  \label{eq:xy_determinant}
  (\myvec{x},\myvec{y})=\begin{vmatrix}x_1&x_2\\y_1&y_2\end{vmatrix}.
\end{equation}

Given that
\begin{equation}
  \label{eq:xy_affinity}
  \begin{array}{rclrcl}
    x_1^* &=& c_{10}+c_{11}x_1+c_{12}x_2, &y_1^* &=& c_{10}+c_{11}y_1+c_{12}y_2\\
    x_2^* &=& c_{20}+c_{21}x_1+c_{22}x_2, &y_2^* &=& c_{20}+c_{21}y_1+c_{22}y_2,
  \end{array}  
\end{equation}
we can calculate the determinant $(\myvec{x}^*,\myvec{y}^*)$:
\begin{equation}
  \label{eq:xy_det_transformed}
  \begin{vmatrix}x_1^*&x_2^*\\y_1^*&y_2^*\end{vmatrix}=\begin{vmatrix}c_{11}&c_{12}\\c_{21}&c_{22}\end{vmatrix}\cdot\begin{vmatrix}x_1&x_2\\y_1&y_2\end{vmatrix}
\end{equation}
or in abbreviated notation
\begin{equation}
  \label{eq:xy_det_transformed_short}
  (\myvec{x}^*,\myvec{y}^*)=(\myvec{x},\myvec{y})\cdot d.
\end{equation}

The geometric meaning of $(\myvec{x},\myvec{y})$ is well known. Explicitly:
\begin{equation}
  \label{eq:triangle_area}
  {1\over 2}(\myvec{x},\myvec{y})=\mbox{area of the triangle $\myvec{o}$ $\myvec{x}$ $\myvec{y}$}
\end{equation}
(i.e., the area of the triangle described by the origin $(0,0)$ and the 
points $\myvec{x}$ and $\myvec{y}$, in that order). The triangle's area may
be either positive or negative, depending on whether its orientation 
along $\myvec{o}$ $\myvec{x}$ $\myvec{y}$ is
positive (Fig.~\ref{fig:positive_orientation}) or negative 
(Fig.~\ref{fig:negative_orientation}). These sign assignments depend on the
choice of axes. We will always think of the positive $x_2$ axis as being,
as in the figure, on the left of the positive $x_1$ axis, and so will consider 
as positive that orientation for which a circumscribed triangle remains on 
the left.

{\bf (OT - Grenzuebergang?)}

Since one can construct, through addition and translation, any arbitrary planar 
figure using triangles with a corner at $\myvec{o}$, then a homogeneous 
affinity with determinant $d$ multiplies areas by factor $d$.
The area delimited by a closed directed curve $\mathfrak{C}$ is what the path
integral
\begin{equation}
  \label{eq:curve_area}
  F={1\over 2}\oint_\mathfrak{C}x_1dx_2-x_2dx_1
  ={1\over 2}\oint_\mathfrak{C}(\myvec{x},d\myvec{x})
\end{equation}
\begin{figure}[htp]
  \begin{center}
    \subfigure[]{
      \label{fig:positive_orientation}
      \includegraphics{positive_orientation.pdf}
    }
    \subfigure[]{
      \label{fig:negative_orientation}
      \includegraphics{negative_orientation.pdf}
    }
  \end{center}
  \caption{}
\end{figure}
is interpreted to be. Thence follows
\begin{equation}
  \label{eq:area_transformation}
  {1\over 2}\oint_\mathfrak{C}x^*_1dx^*_2-x^*_2dx^*_1
  =d\cdot{1\over 2}\oint_\mathfrak{C}x_1dx_2-x_2dx_1
\end{equation}
and since displacements conserve area then one may conclude that:

{\em An arbitrary affinity with determinant $d$ increases areas by factor $d$};
i.e., if the mapping $\myvec{x}\to\myvec{x}^*$ transforms a figure of area $F$
to a figure of area $F^*$ then
\begin{equation}
  \label{eq:area_transform2}
  F^*=F\cdot d.
\end{equation}

When $d>0$ then orientation is preserved and when $d<0$ it is reversed.

When in particular $d=1$ then areas are unchanged, the affinity is equiareal
and it is easy to see that the equiareal affinities form a group --- another
subgroup of the general affine group.

{\em We will consider almost exclusively those geometrical attributes of
planar curves that are conserved by equiareal affinities, so by transformations
of the form:}
\begin{equation}
  \label{eq:equiareal_affinities}
  \begin{array}{ll}
    \begin{array}{l}
      x_1^*=c_{11}x_1+c_{12}x_2,\\
      x_2^*=c_{21}x_1+c_{22}x_2,
    \end{array}
    & c_{11}c_{22}-c_{12}c_{21}=1.
  \end{array}
\end{equation}

\section{Calculation Rules}
It has proven advantageous to introduce some abbreviations, to shorten
writing and calculations. Let $\myvec{x}$ and $\myvec{y}$ be two points with
coordinates $x_1,x_2$ and $y_1,y_2$. Then $\myvec{x}+\myvec{y}$ shall denote
the point with coordinates $x_1+y_1,x_2+y_2$. 

Geometrically one may find $\myvec{x}+\myvec{y}$ by joining the segments from
$\myvec{o}$ to $\myvec{x}$ and $\myvec{y}$, using the Parallelogram law. In
accordance with this notation, we would like to occasionally say the
``vector $\myvec{x}$'' rather than the ``point $\myvec{x}$'', and by that refer
to the directed segment from the origin $\myvec{o}$ to $\myvec{x}$.

The rigorous arithmetic definition of the term ``vector'' is the following:
{\em a geometric construct $\myvec{x}'$ with the ``components'' $x_1'$ and
$x_2'$ is a vector when an affinity~\eqref{eq:affine_trans} transforms the
components thus:}
\begin{equation}
  \label{eq:vector_def}
  \begin{array}{rcl}
    {x'_1}^* &=& c_{11}x'_1+c_{12}x'_2\\
    {x'_2}^* &=& c_{21}x'_1+c_{22}x'_2.
  \end{array}  
\end{equation}

We have already denoted by $(\myvec{x},\myvec{y})$ the determinant
\begin{equation}
  \label{eq:det_redefined}
  (\myvec{x},\myvec{y})=x_1y_2-x_2y_1.
\end{equation}
Then it holds that
\begin{equation}
  \label{eq:det_antisymmetric}
  (\myvec{x},\myvec{y})=-(\myvec{y},\myvec{x})
\end{equation}
and if we denote by $k\myvec{x}$ the vector with coordinates $kx_1,kx_2$ then
\begin{equation}
  \label{eq:det_bilinear}
  (k_1\myvec{x}_1+k_2\myvec{x}_2,\myvec{y})=k_1(\myvec{x}_1,\myvec{y})+k_2(\myvec{x}_2,\myvec{y}).
\end{equation}

We encounter here the same assignment as for the position of the axes, so that
\begin{equation}
  (\myvec{x},\myvec{y}) > 0
\end{equation}
means that the vector $\myvec{y}$ is on the left of $\myvec{x}$ 
(Fig.~\ref{fig:positive_orientation}).

\begin{equation}
  (\myvec{x},\myvec{y}) = 0
\end{equation}
is the condition for identical or directly opposed directions of the vectors.
In this case one may speak of the {\em linear dependence} of the vectors
$\myvec{x}$ and $\myvec{y}$, which may be expressed thus:
There exist two real numbers $a$ and $b$ such that 
$a\myvec{x}+b\myvec{y}=\myvec{o}$ and $a$ and $b$ are not both zero.

The following example shows how the notation above can be used to perform
calculations without explicit use of coordinates. Let $\myvec{x}$ and
$\myvec{y}$ be two points, with lines crossing them, parallel to $\myvec{x}'$
and $\myvec{y}'$. If $\myvec{z}$ is the point of intersection of these lines
then we would like to calculate the area $f$ of the triangle $\myvec{x}$
$\myvec{y}$ $\myvec{z}$ (Fig.~\ref{fig:triangle_area_example}).

\begin{figure}[htp]
  \centering
  \includegraphics{triangle_area.pdf}
  \caption{}
  \label{fig:triangle_area_example}
\end{figure}

Since the vector from $\myvec{x}$ to $\myvec{z}$ has direction $\myvec{x}'$
and the vector from $\myvec{y}$ to $\myvec{z}$ has direction $\myvec{y}'$ then
\begin{equation}
  \myvec{z}=\myvec{x}+\alpha \myvec{x}'=\myvec{y}+\beta \myvec{y}'
\end{equation}
or
\begin{equation}
  \label{eq:triangle_area_example1}
  \myvec{z}-\myvec{x}=\alpha \myvec{x}'=(\myvec{y}-\myvec{x})+\beta \myvec{y}'.
\end{equation}
we can therefore claim that for the area we are interested in, it holds that 
\begin{equation}
  \label{eq:triangle_area_example2}
  f={1\over 2}(\myvec{z}-\myvec{x},\myvec{y}-\myvec{x})
  ={\alpha \over 2}(\myvec{x}',\myvec{y}-\myvec{x}).
\end{equation}
It follows from \eqref{eq:triangle_area_example1} that
\begin{equation}
  \alpha(\myvec{x}',\myvec{y}')=(\myvec{y}-\myvec{x},\myvec{y}')
\end{equation}
or
\begin{equation}
  \alpha={(\myvec{y}-\myvec{x},\myvec{y}')\over (\myvec{x}',\myvec{y}')}
\end{equation}
and this substituted into \eqref{eq:triangle_area_example2} yields the desired
\begin{equation}
  f={1\over 2}{(\myvec{x}',\myvec{y}-\myvec{x})(\myvec{y}-\myvec{x},\myvec{y}')
    \over (\myvec{x}',\myvec{y}')}
\end{equation}

\section{Affine Distance}

We would like to return to the result of the exercise above, and understand it
a somewhat differently. We saw that given two points $\myvec{x}$, $\myvec{y}$
and two directions $\myvec{x}'$, $\myvec{y}'$, that is two ``line elements'', 
we can assign each such a construct a number 
$f(\myvec{x},\myvec{x}';\myvec{y},\myvec{y}')$ that is invariant under all
equiareal affinities. This number is namely the area of the triangle 
$\myvec{x}$ $\myvec{y}$ $\myvec{z}$ (Fig.~\ref{fig:triangle_area_example}).

We now claim that $f(\myvec{x},\myvec{x}';\myvec{y},\myvec{y}')$ is in some
sense the only operation that thus maps numbers to line elements. More 
precisely: If $\phi(\myvec{x},\myvec{x}';\myvec{y},\myvec{y}')$ is invariant
under an arbitrary equiareal affinity, where $\phi$ is a function that maps
pairs of line elements to numbers, then
$$ \phi(\myvec{x},\myvec{x}';\myvec{y},\myvec{y}') = \Phi\left[f(\myvec{x},\myvec{x}';\myvec{y},\myvec{y}')\right].$$

This follows from the fact that my means of the transformations 
\eqref{eq:equiareal_affinities} a triangle is mapped to an arbitrary triangle
of the same area. Hence $\phi$ cannot depend on the specific values
$\myvec{x},\myvec{x}';\myvec{y},\myvec{y}'$. By similar consideration one can
show in motion geometry that the motion invariants of two points can be 
expressed as functions of the distance between them.

{\bf (OT: Bewegungsgeomerie = motion geometry?)}

There is there a reason, however, to choose in particular the distance $r_{ik}$
between the points $\myvec{x}_i$ and $\myvec{x}_k$ rather than any other
invariant: namely the property of distance for points on a line, since
$$r_{12}+r_{23}=r_{13},$$
when $\myvec{x}_1$,  $\myvec{x}_2$ and $\myvec{x}_3$ lay on a line, in this 
order. This brings us to the following consideration.
 
The line elements $\myvec{x},\myvec{x}';\myvec{y},\myvec{y}'$ uniquely define
a parabola: one that passes through  $\myvec{x}$ and $\myvec{y}$ and is there
tangent to the directions $\myvec{x}'$ and $\myvec{y}'$ 
(Fig~\ref{fig:triangle_parabola}). This holds since four lines --- two of
which lie ``infinitely close'' --- determine a parabola. Let then
\begin{equation}
  \label{eq:parabola_def1}
  x_i=x_{0i}+\dot{x}_{0i}t+{1\over 2!}\ddot{x}_{0i}t^2, \quad(i=1,2)
\end{equation}
or in vector notation
\begin{equation}
  \label{eq:parabola_def2}
  \myvec{x}=\myvec{x}_0+\dot{\myvec{x}}_0t+{1\over 2!}\ddot{\myvec{x}}_0t^2
\end{equation}
be the parabola's equation in parameterized form, and
\begin{equation}
  \label{eq:parabola_initial_def}
  \begin{array}{lrcl}
    \myvec{x}_0=\myvec{x}(0)=\myvec{x},& \left({d\myvec{x}\over dt}\right)_{t=0}=\mbox{const}\cdot \myvec{x}',\\
    \myvec{x}_1=\myvec{x}(t_1)=\myvec{y},& \left({d\myvec{x}\over dt}\right)_{t=t_1}=\mbox{const}\cdot \myvec{y}'.\\
  \end{array}  
\end{equation}



\bibliographystyle{plain} \bibliography{blaschke}

\end{document}


 