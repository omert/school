\documentclass[11pt]{book} \usepackage{amssymb}
\usepackage{amsfonts} \usepackage{amsmath} \usepackage{bm}
\usepackage{latexsym} \usepackage{epsfig}
\usepackage{ccaption}
\usepackage{subfigure}

\renewcommand{\figurename}{Fig.}
\captiondelim{}

\newtheorem{theorem}{Theorem}[section]
\newtheorem{lemma}[theorem]{Lemma}
\newtheorem{proposition}[theorem]{Proposition}
\newtheorem{corollary}[theorem]{Corollary}
\newtheorem{fact}[theorem]{Fact}
\newtheorem{definition}[theorem]{Definition}
\newtheorem{remark}[theorem]{Remark}
\newtheorem{conjecture}[theorem]{Conjecture}
\newtheorem{claim}[theorem]{Claim}
\newtheorem{example}[theorem]{Example}
\newenvironment{proof}{\noindent \textbf{Proof:}}{$\Box$}

\newcommand{\myvec}[1]{\mathfrak{#1}}
\newcommand{\vecx}{\myvec{x}}
\newcommand{\vecy}{\myvec{y}}
\newcommand{\vecz}{\vecz}
\newcommand{\vecderiv}[1]{\dot{\myvec{#1}}}
\newcommand{\vecderivv}[1]{\ddot{\myvec{#1}}}

\begin{document}
\title{Affine Differential Geometry}

 \author{Wilhelm Blaschke, Translated by Omer Tamuz}
\maketitle

\chapter{Local theory of planar curves}
\section{Affine mapping}
We would like to commence by recalling some well known facts of analytical 
geometry.

For our purposes, it will be useful to represent the points of a plane with
a parallel coordinate system. We choose two intersecting lines\footnote{We
refer to straight lines simply as ``lines'' throughout the text (translator).}
in the plane and mark on each some direction as positive. We call these lines
the $x_1$ and $x_2$ axes. Through a remaining point of the plane we pass
parallels to the axes, and assign the corresponding segments of the axes 
the numbers $x_1$ and $x_2$. We have thus defined a bijection between 
pairs of real number $x_1$ and $x_2$, to the points of the plane 
(Fig.~\ref{fig:axes}).

\begin{figure}[htp]
  \label{fig:axes}
  \centering
  \includegraphics{axes.pdf} 
  \caption{}
\end{figure}

We will thus deal with ``real'' points and lines, and in fact consider only
those points and lines that in the framework of projective geometry are 
considered ``proper'', as opposed to ``improper'' or infinitely remote elements.

In place of the pair $x_1$, $x_2$ we use the shorter vector notation and 
speak simply of the point $\vecx$.

Let $\vecx\to \vecx^*$ be a mapping between the points of plane,
which is given by a system of linear relations between their coordinates:
\begin{equation}
  \label{eq:affine_trans}
  \begin{array}{rcl}
    x_1^* &=& c_{10}+c_{11}x_1+c_{12}x_2\\
    x_2^* &=& c_{20}+c_{21}x_1+c_{22}x_2.
  \end{array}
\end{equation}
This system of equations is solvable for $x$ when the determinant 
\begin{equation}
  d=c_{11}c_{22}-c_{21}c_{12}=
  \begin{vmatrix}
    c_{11} & c_{12}\\
    c_{21} & c_{22}
  \end{vmatrix}
\end{equation}
is different than zero ($\neq 0$). Then, to each $\vecx^*$ corresponds 
one and only one $\vecx$. Following Euler \cite{Euler:1749} and 
M\"obius \cite{Mobius:1827}, this kind of mapping
is called an ``affine mapping'', ``affine transformation'' or just ``affinity''.
 
The inverse transformation, which one can determine by 
solving~\eqref{eq:affine_trans}, has again the same form: 
\begin{equation}
  \label{eq:affine_trans_inv}
  \begin{array}{rcl}
    x_1 &=& c_{10}^*+c_{11}^*x_1^*+c_{12}^*x_2^*\\
    x_2 &=& c_{20}^*+c_{21}^*x_1^*+c_{22}^*x_2^*,
  \end{array}
\end{equation}
where
\begin{equation}
  \label{eq:inv_c}
  \begin{array}{lll}
    c_{10}^*=-{c_{10}c_{22}-c_{20}c_{12} \over d},& c_{11}^*=+{c_{22}\over d},&c_{12}^*=-{c_{12}\over d},\\
    c_{20}^*=-{c_{10}c_{21}-c_{20}c_{11} \over d},& c_{21}^*=+{c_{21}\over d},&c_{22}^*=-{c_{11}\over d}.
  \end{array}
\end{equation}
The determinant
\begin{equation}
  \label{eq:inv_det}
  d^*=c_{11}^*c_{22}^*-c_{12}^*c_{21}^*={1\over d}
\end{equation}
is different than zero and hence the inverse transformation is also affine.

The points $\vecx$ of a line $\mathfrak{g}$ are mapped by an affine 
transformation to
the points $\vecx^*$ of a line $\mathfrak{g}^*$. In fact, a line 
$\mathfrak{g}$ can be
represented by a linear equation ($g_1$, $g_2$ not both zero):
\begin{equation}
  \label{eq:line}
  g_0+g_1x_1+g_2x_2=0.
\end{equation}
The substitution of~\eqref{eq:inv_c} into this equation results, because of the
invertibilty of the transformation, in an equation which is not trivially
satisfied, and thus again the locus of $\vecx^*$ is a line.

It further follows from~\eqref{eq:affine_trans} that affine transformations are
continuous mappings. That is, as $\vecx$ tends to $\vecx_0$ the 
corresponding $\vecx^*$ tends to $\vecx^*_0$.

Using M\"obius nets ~\cite{Mobius:1827} one can show that the last
two properties, whose dependence or independence remain to be decided, are
the definitive properties of affine transformations: {\em Affine mappings are
the only coordinate transformations (in the plane) which are without
exception bijective, continuous and map lines to lines.}

Affine transformations preserve parallelism, since parallel lines on the plane
are characterized by not having common points. Furthermore, given two 
affinities $\vecx\to\vecx^*$ and $\vecx^*\to\vecx^{**}$ , their
``product'' $\vecx\to\vecx^{**}$ is again an affinity, since all the
characteristics of an affinity hold. Naturally, this can also be derived 
from~\eqref{eq:affine_trans}. Now a set of transformations that is closed under
the transformation product is called a group. {\em Hence the 
affinities~\eqref{eq:affine_trans} form a group.}

An affinity~\eqref{eq:affine_trans} can be decomposed into a ``displacement''
\begin{equation}
  \label{eq:decompose_displacement}
  \begin{array}{ll}
    x_1^*=c_{10}+x_1,& x_2^*=c_{20}+x_2
  \end{array}
\end{equation}
and a preceding ``homogeneous affinity''
\begin{equation}
  \label{eq:decompose_homogeneous}
  \begin{array}{ll}
    \begin{array}{l}
      x_1^*=c_{11}x_1+c_{12}x_2,\\
      x_2^*=c_{21}x_1+c_{22}x_2,
    \end{array}
    & c_{11}c_{22}-c_{12}c_{21}\neq 0.
  \end{array}
\end{equation}

Evidently the displacements or ``translations'' alone, and likewise the 
homogeneous affinities by themselves, constitute a group of transformations.
We thus have two ``subgroups'' of the general affine group considered above.

For our purposes an additional subgroup is particularly important: namely the
so called ``equiareal affinities''. Consider then the effect of the homogeneous
affinity~\eqref{eq:decompose_homogeneous} on the determinant
\begin{equation}
  \label{eq:xy_determinant}
  (\vecx,\vecy)=\begin{vmatrix}x_1&x_2\\y_1&y_2\end{vmatrix}.
\end{equation}

Given that
\begin{equation}
  \label{eq:xy_affinity}
  \begin{array}{rclrcl}
    x_1^* &=& c_{10}+c_{11}x_1+c_{12}x_2, &y_1^* &=& c_{10}+c_{11}y_1+c_{12}y_2\\
    x_2^* &=& c_{20}+c_{21}x_1+c_{22}x_2, &y_2^* &=& c_{20}+c_{21}y_1+c_{22}y_2,
  \end{array}  
\end{equation}
we can calculate the determinant $(\vecx^*,\vecy^*)$:
\begin{equation}
  \label{eq:xy_det_transformed}
  \begin{vmatrix}x_1^*&x_2^*\\y_1^*&y_2^*\end{vmatrix}=\begin{vmatrix}c_{11}&c_{12}\\c_{21}&c_{22}\end{vmatrix}\cdot\begin{vmatrix}x_1&x_2\\y_1&y_2\end{vmatrix}
\end{equation}
or in abbreviated notation
\begin{equation}
  \label{eq:xy_det_transformed_short}
  (\vecx^*,\vecy^*)=(\vecx,\vecy)\cdot d.
\end{equation}

The geometric meaning of $(\vecx,\vecy)$ is well known. Explicitly:
\begin{equation}
  \label{eq:triangle_area}
  {1\over 2}(\vecx,\vecy)=\mbox{area of the triangle $\myvec{o}$ $\vecx$ $\vecy$}
\end{equation}
(i.e., the area of the triangle described by the origin $(0,0)$ and the 
points $\vecx$ and $\vecy$, in that order). The triangle's area may
be either positive or negative, depending on whether its orientation 
along $\myvec{o}$ $\vecx$ $\vecy$ is
positive (Fig.~\ref{fig:positive_orientation}) or negative 
(Fig.~\ref{fig:negative_orientation}). These sign assignments depend on the
choice of axes. We will always think of the positive $x_2$ axis as being,
as in the figure, on the left of the positive $x_1$ axis, and so will consider 
as positive that orientation for which a circumscribed triangle remains on 
the left.

{\bf (OT - Grenzuebergang?)}

Since one can construct, through addition and translation, any arbitrary planar 
figure using triangles with a corner at $\myvec{o}$, then a homogeneous 
affinity with determinant $d$ multiplies areas by factor $d$.
The area delimited by a closed directed curve $\mathfrak{C}$ is what the path
integral
\begin{equation}
  \label{eq:curve_area}
  F={1\over 2}\oint_\mathfrak{C}x_1dx_2-x_2dx_1
  ={1\over 2}\oint_\mathfrak{C}(\vecx,d\vecx)
\end{equation}
\begin{figure}[htp]
  \begin{center}
    \subfigure[]{
      \label{fig:positive_orientation}
      \includegraphics{positive_orientation.pdf}
    }
    \subfigure[]{
      \label{fig:negative_orientation}
      \includegraphics{negative_orientation.pdf}
    }
  \end{center}
  \caption{}
\end{figure}
is interpreted to be. Thence follows
\begin{equation}
  \label{eq:area_transformation}
  {1\over 2}\oint_\mathfrak{C}x^*_1dx^*_2-x^*_2dx^*_1
  =d\cdot{1\over 2}\oint_\mathfrak{C}x_1dx_2-x_2dx_1
\end{equation}
and since displacements conserve area then one may conclude that:

{\em An arbitrary affinity with determinant $d$ increases areas by factor $d$};
i.e., if the mapping $\vecx\to\vecx^*$ transforms a figure of area $F$
to a figure of area $F^*$ then
\begin{equation}
  \label{eq:area_transform2}
  F^*=F\cdot d.
\end{equation}

When $d>0$ then orientation is preserved and when $d<0$ it is reversed.

When in particular $d=1$ then areas are unchanged, the affinity is equiareal
and it is easy to see that the equiareal affinities form a group --- another
subgroup of the general affine group.

{\em We will consider almost exclusively those geometrical attributes of
planar curves that are conserved by equiareal affinities, so by transformations
of the form:}
\begin{equation}
  \label{eq:equiareal_affinities}
  \begin{array}{ll}
    \begin{array}{l}
      x_1^*=c_{11}x_1+c_{12}x_2,\\
      x_2^*=c_{21}x_1+c_{22}x_2,
    \end{array}
    & c_{11}c_{22}-c_{12}c_{21}=1.
  \end{array}
\end{equation}

\section{Calculation rules}
\label{section:calculation_rules}
It has proven advantageous to introduce some abbreviations, to shorten
writing and calculations. Let $\vecx$ and $\vecy$ be two points with
coordinates $x_1,x_2$ and $y_1,y_2$. Then $\vecx+\vecy$ shall denote
the point with coordinates $x_1+y_1,x_2+y_2$. 

Geometrically one may find $\vecx+\vecy$ by joining the segments from
$\myvec{o}$ to $\vecx$ and $\vecy$, using the Parallelogram law. In
accordance with this notation, we would like to occasionally say the
``vector $\vecx$'' rather than the ``point $\vecx$'', and by that refer
to the directed segment from the origin $\myvec{o}$ to $\vecx$.

The rigorous arithmetic definition of the term ``vector'' is the following:
{\em a geometric construct $\vecx'$ with the ``components'' $x_1'$ and
$x_2'$ is a vector when an affinity~\eqref{eq:affine_trans} transforms the
components thus:}
\begin{equation}
  \label{eq:vector_def}
  \begin{array}{rcl}
    {x'_1}^* &=& c_{11}x'_1+c_{12}x'_2\\
    {x'_2}^* &=& c_{21}x'_1+c_{22}x'_2.
  \end{array}  
\end{equation}

We have already denoted by $(\vecx,\vecy)$ the determinant
\begin{equation}
  \label{eq:det_redefined}
  (\vecx,\vecy)=x_1y_2-x_2y_1.
\end{equation}
Then it holds that
\begin{equation}
  \label{eq:det_antisymmetric}
  (\vecx,\vecy)=-(\vecy,\vecx)
\end{equation}
and if we denote by $k\vecx$ the vector with coordinates $kx_1,kx_2$ then
\begin{equation}
  \label{eq:det_bilinear}
  (k_1\vecx_1+k_2\vecx_2,\vecy)=k_1(\vecx_1,\vecy)+k_2(\vecx_2,\vecy).
\end{equation}

We encounter here the same assignment as for the position of the axes, so that
\begin{equation}
  (\vecx,\vecy) > 0
\end{equation}
means that the vector $\vecy$ is on the left of $\vecx$ 
(Fig.~\ref{fig:positive_orientation}).

\begin{equation}
  (\vecx,\vecy) = 0
\end{equation}
is the condition for identical or directly opposed directions of the vectors.
In this case one may speak of the {\em linear dependence} of the vectors
$\vecx$ and $\vecy$, which may be expressed thus:
There exist two real numbers $a$ and $b$ such that 
$a\vecx+b\vecy=\myvec{o}$ and $a$ and $b$ are not both zero.

The following example shows how the notation above can be used to perform
calculations without explicit use of coordinates. Let $\vecx$ and
$\vecy$ be two points, with lines crossing them, parallel to $\vecx'$
and $\vecy'$. If $\vecz$ is the point of intersection of these lines
then we would like to calculate the area $f$ of the triangle $\vecx$
$\vecy$ $\vecz$ (Fig.~\ref{fig:triangle_area_example}).

\begin{figure}[htp]
  \centering
  \includegraphics{triangle_area.pdf}
  \caption{}
  \label{fig:triangle_area_example}
\end{figure}

Since the vector from $\vecx$ to $\vecz$ has direction $\vecx'$
and the vector from $\vecy$ to $\vecz$ has direction $\vecy'$ then
\begin{equation}
  \vecz=\vecx+\alpha \vecx'=\vecy+\beta \vecy'
\end{equation}
or
\begin{equation}
  \label{eq:triangle_area_example1}
  \vecz-\vecx=\alpha \vecx'=(\vecy-\vecx)+\beta \vecy'.
\end{equation}
we can therefore claim that for the area we are interested in, it holds that 
\begin{equation}
  \label{eq:triangle_area_example2}
  f={1\over 2}(\vecz-\vecx,\vecy-\vecx)
  ={\alpha \over 2}(\vecx',\vecy-\vecx).
\end{equation}
It follows from \eqref{eq:triangle_area_example1} that
\begin{equation}
  \alpha(\vecx',\vecy')=(\vecy-\vecx,\vecy')
\end{equation}
or
\begin{equation}
  \alpha={(\vecy-\vecx,\vecy')\over (\vecx',\vecy')}
\end{equation}
and this substituted into \eqref{eq:triangle_area_example2} yields the desired
\begin{equation}
  f={1\over 2}{(\vecx',\vecy-\vecx)(\vecy-\vecx,\vecy')
    \over (\vecx',\vecy')}
\end{equation}

\section{Affine distance}

We would like to return to the result of the exercise above, and understand it
a somewhat differently. We saw that given two points $\vecx$, $\vecy$
and two directions $\vecx'$, $\vecy'$, that is two ``line elements'', 
we can assign each such a construct a number 
$f(\vecx,\vecx';\vecy,\vecy')$ that is invariant under all
equiareal affinities. This number is namely the area of the triangle 
$\vecx$ $\vecy$ $\vecz$ (Fig.~\ref{fig:triangle_area_example}).

We now claim that $f(\vecx,\vecx';\vecy,\vecy')$ is in some
sense the only operation that thus maps numbers to line elements. More 
precisely: If $\phi(\vecx,\vecx';\vecy,\vecy')$ is invariant
under an arbitrary equiareal affinity, where $\phi$ is a function that maps
pairs of line elements to numbers, then
$$ \phi(\vecx,\vecx';\vecy,\vecy') = \Phi\left[f(\vecx,\vecx';\vecy,\vecy')\right].$$

This follows from the fact that my means of the transformations 
\eqref{eq:equiareal_affinities} a triangle is mapped to an arbitrary triangle
of the same area. Hence $\phi$ cannot depend on the specific values
$\vecx,\vecx';\vecy,\vecy'$. By similar consideration one can
show in motion geometry that the motion invariants of two points can be 
expressed as functions of the distance between them.

{\bf (OT: Bewegungsgeomerie = motion geometry?)}

There is there a reason, however, to choose in particular the distance $r_{ik}$
between the points $\vecx_i$ and $\vecx_k$ rather than any other
invariant: namely the property of distance for points on a line, since
$$r_{12}+r_{23}=r_{13},$$
when $\vecx_1$,  $\vecx_2$ and $\vecx_3$ lay on a line, in this 
order. This brings us to the following consideration.
 
The line elements $\vecx,\vecx';\vecy,\vecy'$ uniquely define
a parabola: one that passes through  $\vecx$ and $\vecy$ and is there
tangent to the directions $\vecx'$ and $\vecy'$ 
(Fig~\ref{fig:triangle_parabola}). This holds since four lines --- two of
which lie ``infinitely close'' --- determine a parabola. Let then
\begin{equation}
  \label{eq:parabola_def1}
  x_i=x_{0i}+\dot{x}_{0i}t+{1\over 2!}\ddot{x}_{0i}t^2, \quad(i=1,2)
\end{equation}
or in vector notation
\begin{equation}
  \label{eq:parabola_def2}
  \vecx=\vecx_0+\vecderiv{x}_0t+{1\over 2!}\vecderivv{x}_0t^2
\end{equation}
be the parabola's equation in parameterized form, and
\begin{equation}
  \label{eq:parabola_initial_def}
  \begin{array}{lrcl}
    \vecx_0=\vecx(0)=\vecx,& \left({d\vecx\over dt}\right)_{t=0}=\mbox{const}\cdot \vecx',\\
    \vecx_1=\vecx(t_1)=\vecy,& \left({d\vecx\over dt}\right)_{t=t_1}=\mbox{const}\cdot \vecy'.\\
  \end{array}  
\end{equation}

\begin{figure}[htp]
  \centering
  \includegraphics{triangle_parabola.pdf}
  \caption{}
  \label{fig:triangle_parabola}
\end{figure}

We calculate the area $f$ of the triangle that is defined by the line elements
$\vecx,\vecx';\vecy,\vecy'$. We denote, as in 
\eqref{eq:parabola_def2}, the derivative with respect to $t$ by dots,
and so
\begin{equation}
  \label{eq:triangle_area_vector}
  f={1\over 2}{(\vecderiv{x}_0,\vecx_1-\vecx_0)(\vecx_1-\vecx_0,\vecderiv{x}_1)
    \over (\vecderiv{x}_0,\vecderiv{x}_1)}.
\end{equation}
By substituting 
\begin{equation*} 
  \vecx_1-\vecx_0=\vecderiv{x}_0t_1+\vecderivv{x}_0{t_1^2\over 2!},\quad \vecderiv{x}_1=\vecderiv{x}_0+t_1\vecderivv{x}_0,
\end{equation*}
we find that
\begin{equation*}
  \begin{array}{rcl}
    (\vecderiv{x}_0,\vecx_1-\vecx_0)&=&{t_1^2\over 2}(\vecderiv{x}_0,\vecderivv{x}_0),\\
    (\vecx_1-\vecx_0,\vecderiv{x}_1)&=&{t_1^2\over 2}(\vecderiv{x}_0,\vecderivv{x}_0),\\
    (\vecderiv{x}_0,\vecderiv{x}_1)&=&t_1(\vecderiv{x}_0,\vecderivv{x}_0)
  \end{array}
\end{equation*}
and hence
\begin{equation}
  \label{eq:triangle_area_parabola}
  f={1\over 8}t_1^3(\vecderiv{x}_0,\vecderivv{x}_0).
\end{equation}
Accordingly, it is clear that through the points $t_1$ and $t_2$ the parabola
defines a triangle of area
\begin{equation}
  \label{eq:parabola_defines_area}
  f(t_1,t_2)={1\over 8}(t_2-t_1)^3(\vecderiv{x}_0,\vecderivv{x}_0).
\end{equation}
We have then (Fig.~\ref{fig:triangle_parabola})
\begin{equation}
  \label{eq:affine_distance_additive}
  f^{1/3}(t_1,t_2)+f^{1/3}(t_2,t_3)={1\over 2}(\vecderiv{x}_0,\vecderivv{x}_0)^{1/3}\left[t_2-t_1+t_3-t_2\right]=f^{1/3}(t_1,t_3),
\end{equation}
where the third root is taken to be real. $f^{1/3}$ possesses then an 
additivity theorem that is analogous to that of distance in motion geometry. we
state then:

{\em The affine distance $r$ of the line elements 
$\vecx,\vecx';\vecy,\vecy'$ is 
$$r=2\cdot f^{1/3},$$
where $f$ is the area of the triangle determined by 
$\vecx,\vecx';\vecy,\vecy'$, and the third root is taken to be
real.}

We would like to reformulate this newly found property of parabolas. Instead of
$t$ we introduce a new parameter $s$ by the condition
\begin{equation}
  \label{eq:natural_parameter}
  \left({d\vecx\over ds},{d^2\vecx\over ds^2}\right)=1.
\end{equation}
This condition is invariant under equiareal affinities, as the following 
observation shows.

Consider a curve in parameterized form:
\begin{equation}
  \label{eq:parameterized_curve}
  x_1=x_1(t),\quad x_2=x_2(t),
\end{equation}
and subject it to the transformation 
\begin{equation*}
  x_i^*=c_{i0}+c_{i1}x_1+c_{i2}x_2,\quad (i=1,2;\: c_{11}c_{22}-c_{12}c_{21}=1),
  \tag{\ref{eq:equiareal_affinities}}
\end{equation*}
so that the $k$-th derivatives of the coordinates experience the appropriate
homogeneous substitution
\begin{equation}
  \label{eq:deriv_transform}
  x_i^{(k)*}=c_{i1}x_1^{(k)}+c_{i2}x_2^{(k)}, \quad (i=1,2).
\end{equation}
By the definitions in Section~\ref{section:calculation_rules}, $x_1^{(k)}$ and
$x_2^{(k)}$ are the components of a vector $\vecx^{(k)}$, and by 
\eqref{eq:xy_det_transformed_short} the determinants 
$(\vecx^{(k)},\vecx^{(l)})$ are invariant under equiareal affinities.

The parameter $s$ for the parabola is therefore by \eqref{eq:natural_parameter}
affinely invariant, and hence fixed, up to the choice of the point $s=0$. It
is
\begin{equation*}
  s=(\vecderiv{x}_0,\vecderivv{x}_0)^{1/3}t+\mbox{const}
\end{equation*}
and therefore by \eqref{eq:parabola_defines_area}
\begin{equation}
  \label{eq:area_by_natural_param}
  f(s_1,s_2)={1\over 8}(s_2-s_1)^3.
\end{equation}
We have therefore proved the following theorem:

{\em Given a parabola parameterized by parameter $s$ such that
  \begin{equation}
    \left({d\vecx\over ds},{d^2\vecx\over ds^2}\right)=1.
    \tag{\ref{eq:natural_parameter}}
  \end{equation}
  then the affine distance between line elements $s_1$ and $s_2$ is $s_2-s_1$.}
Instead of this we would also like to say: {\em The parabola between the points
$s_1$ and $s_2$ has the affine length $s_2-s_1$.}

The addition theorem \eqref{eq:affine_distance_additive} for areas 
(Fig.~\ref{fig:triangle_parabola}) was already known to A.F. M\"obius.

\section{Affine length of curve arcs}

Following the reflections of the previous section, there emerge two options,
under appropriate differentiability conditions, for the definition of the 
affine length of a curve arc.

We can proceed in a similar fashion to the parabola and define a parameter
using affinely invariant normalization. Let $\vecx(t)$ be an arbitrary
curve vector, twice differentiable. Then by the considerations above it 
follows that
$$\left({d\vecx\over dt},{d^2\vecx\over dt^2}\right)
=(\vecderiv{x},\vecderivv{x})$$
is invariant under equiareal affinities. Let 
\begin{equation}
  \label{eq:no_inflection}
  (\vecderiv{x},\vecderivv{x})\neq 0
\end{equation}
for all points of our curve arc. What is the geometric meaning of this? Assume
$(\vecderiv{x},\vecderivv{x}) > 0$. Then the curve, and the vector
$\vecx(t+h)-\myvec(t)$, lie on the left of the (directed) tangent vector
$\vecderiv{x}(t)$, in a neighborhood of a point $t$, for small enough $|h|$.

To show this, we must (following Section~\ref{section:calculation_rules}) 
only demonstrate that
$$ D=(\vecderiv{x}, \vecx(t+h)-\vecx(t))$$
is positive for small enough $|h|$. Let
$$\Delta(t,\tau)=(\vecderiv{x}(t),\vecx(\tau)).$$
Then by \eqref{eq:det_bilinear}
$$D=\Delta(t,t+h)-\Delta(t,t),$$
and if we denote derivatives with respect to $\tau$ by circles, then by 
Taylor's Theorem
\begin{equation}
  \label{eq:distance_taylor}
  \begin{array}{c}    
    D=h\Delta^{\circ}(t,t)+{h^2\over 2}\Delta^{\circ\circ}(t,t+\vartheta h)
    = {h^2\over 2}\Delta^{\circ\circ}(t,t+\vartheta h),\\
    (0<\vartheta<1),
  \end{array}      
\end{equation}
since $\Delta^\circ(t,t)=(\vecderiv{x},\vecderiv{x})=0$. Because of the 
continuity of $\vecderiv{x}(t)$ then $\Delta^{\circ\circ}$ is also, for constant
$\tau$, a continuous function of $t$. Therefore $D$ is arbitrarily close to
\begin{equation}
  \label{eq:d_approx}
  {h^2\over 2}\Delta^{\circ\circ}(t+\vartheta h, t+ \vartheta h)>0
\end{equation}
and hence positive.

We find then that curve arcs with $(\vecderiv{x},\vecderivv{x})\neq 0$ are 
locally convex and have no inflection points. We will mostly limit ourselves
to consider such curve arcs. Furthermore we assume that the integral
\begin{equation}
  \label{eq:affine_arc_length}
  s_{12}=\int_{t_1}^{t_2}(\vecderiv{x},\vecderivv{x})^{1/3}dt
\end{equation}
exists. Then a parameter $s$ may be defined by the condition
\begin{equation}
  \label{eq:affine_natural_param}
  \left({d\vecx\over ds},{d^2\vecx\over ds^2}\right)=
  (\vecx',\vecx'')=1,
\end{equation}
and is unique up to the choice of the point $s=0$. Then:
\begin{equation}
  \label{eq:param_transform}
  \vecx'=\vecderiv{x}{dt\over ds},\quad 
  \vecx''=\vecderivv{x}\left({dt\over ds}\right)^2+\vecderiv{x}{d^2t\over ds^2}.
\end{equation}
From this follows
\begin{equation}
  \label{eq:affine_deriv_transform1}
  (\vecx',\vecx'')=(\vecderiv{x},\vecderivv{x})\left({}dt\over ds\right)^2
\end{equation}
and then by \eqref{eq:affine_natural_param}
\begin{equation}
  \label{eq:affine_arc_length2}
  s=\int(\vecderiv{x},\vecderivv{x})^{1/3}dt.
\end{equation}
By \eqref{eq:no_inflection} $s$ is a monotonous and invertible function of $t$.

We hence understand the integral \eqref{eq:affine_arc_length} as the 
{\em affine length} of an arc $\vecx(t)\:(t_1\leq t \leq t_2)$.

But it would be equally natural to define affine length by a limit process, 
as follows: Let (Fig.~\ref{fig:affine_length}) $\vecx_0\vecx^0$ be an
arc with a certain tangent at every point $\vecx$.  

\bibliographystyle{plain} \bibliography{blaschke}

\end{document}


  