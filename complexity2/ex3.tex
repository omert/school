\documentclass[11pt]{article} \usepackage{amssymb}
\usepackage{amsfonts} \usepackage{amsmath} \usepackage{bm}
\usepackage{latexsym} \usepackage{epsfig}

\setlength{\textwidth}{6.5 in} \setlength{\textheight}{8.25in}
\setlength{\oddsidemargin}{0in} \setlength{\topmargin}{0in}
\addtolength{\textheight}{.8in} \addtolength{\voffset}{-.5in}

\newtheorem{theorem}{Theorem}[section]
\newtheorem{lemma}[theorem]{Lemma}
\newtheorem{proposition}[theorem]{Proposition}
\newtheorem{corollary}[theorem]{Corollary}
\newtheorem{fact}[theorem]{Fact}
\newtheorem{definition}[theorem]{Definition}
\newtheorem{remark}[theorem]{Remark}
\newtheorem{conjecture}[theorem]{Conjecture}
\newtheorem{claim}[theorem]{Claim}
\newtheorem{example}[theorem]{Example}
\newenvironment{proof}{\noindent \textbf{Proof:}}{$\Box$}

\newcommand{\ignore}[1]{}

\newcommand{\enote}[1]{} \newcommand{\knote}[1]{}
\newcommand{\rnote}[1]{}



% \newcommand{\enote}[1]{{\bf [[Elchanan:} {\emph{#1}}{\bf ]]}}
% \newcommand{\knote}[1]{{\bf [[Krzysztof:} {\emph{#1}}{\bf ]]}}
% \newcommand{\rnote}[1]{{\bf [[Ryan:} {\emph{#1}}{\bf ]]}}



\DeclareMathOperator{\Support}{Supp} \DeclareMathOperator{\Opt}{Opt}
\DeclareMathOperator{\Ordo}{\mathcal{O}}
\newcommand{\MaxkCSP}{\textsc{Max $k$-CSP}}
\newcommand{\MaxkCSPq}{\textsc{Max $k$-CSP$_{q}$}}
\newcommand{\MaxCSP}[1]{\textsc{Max CSP}(#1)} \renewcommand{\Pr}{{\bf
    P}} \renewcommand{\P}{{\bf P}} \newcommand{\Px}{\mathop{\bf P\/}}
\newcommand{\E}{{\bf E}} \newcommand{\Cov}{{\bf Cov}}
\newcommand{\Var}{{\bf Var}} \newcommand{\Varx}{\mathop{\bf Var\/}}

\newcommand{\bits}{\{-1,1\}}

\newcommand{\nsmaja}{\textstyle{\frac{2}{\pi}} \arcsin \rho}

\newcommand{\Inf}{\mathrm{Inf}} \newcommand{\I}{\mathrm{I}}
\newcommand{\J}{\mathrm{J}}

\newcommand{\eps}{\epsilon} \newcommand{\lam}{\lambda}

% \newcommand{\trunc}{\ell_{2,[-1,1]}}
\newcommand{\trunc}{\zeta} \newcommand{\truncprod}{\chi}

\newcommand{\N}{\mathbb N} \newcommand{\R}{\mathbb R}
\newcommand{\Z}{\mathbb Z} \newcommand{\CalE}{{\mathcal{E}}}
\newcommand{\CalC}{{\mathcal{C}}} \newcommand{\CalM}{{\mathcal{M}}}
\newcommand{\CalR}{{\mathcal{R}}} \newcommand{\CalS}{{\mathcal{S}}}
\newcommand{\CalV}{{\mathcal{V}}}
\newcommand{\CalX}{{\boldsymbol{\mathcal{X}}}}
\newcommand{\CalG}{{\boldsymbol{\mathcal{G}}}}
\newcommand{\CalH}{{\boldsymbol{\mathcal{H}}}}
\newcommand{\CalY}{{\boldsymbol{\mathcal{Y}}}}
\newcommand{\CalZ}{{\boldsymbol{\mathcal{Z}}}}
\newcommand{\CalW}{{\boldsymbol{\mathcal{W}}}}
\newcommand{\CalF}{{\mathcal{Z}}}
% \newcommand{\boldG}{{\boldsymbol G}}
% \newcommand{\boldQ}{{\boldsymbol Q}}
% \newcommand{\boldP}{{\boldsymbol P}}
% \newcommand{\boldR}{{\boldsymbol R}}
% \newcommand{\boldS}{{\boldsymbol S}}
% \newcommand{\boldX}{{\boldsymbol X}}
% \newcommand{\boldB}{{\boldsymbol B}}
% \newcommand{\boldY}{{\boldsymbol Y}}
% \newcommand{\boldZ}{{\boldsymbol Z}}
% \newcommand{\boldV}{{\boldsymbol V}}
\newcommand{\boldi}{{\boldsymbol i}} \newcommand{\boldj}{{\boldsymbol
    j}} \newcommand{\boldk}{{\boldsymbol k}}
\newcommand{\boldr}{{\boldsymbol r}}
\newcommand{\boldsigma}{{\boldsymbol \sigma}}
\newcommand{\boldupsilon}{{\boldsymbol \upsilon}}
\newcommand{\hone}{{\boldsymbol{H1}}}
\newcommand{\htwo}{\boldsymbol{H2}}
\newcommand{\hthree}{\boldsymbol{H3}}
\newcommand{\hfour}{\boldsymbol{H4}}


\newcommand{\sgn}{\mathrm{sgn}} \newcommand{\Maj}{\mathrm{Maj}}
\newcommand{\Acyc}{\mathrm{Acyc}}
\newcommand{\UniqMax}{\mathrm{UniqMax}}
\newcommand{\Thr}{\mathrm{Thr}} \newcommand{\littlesum}{{\textstyle
    \sum}}

\newcommand{\half}{{\textstyle \frac12}}
\newcommand{\third}{{\textstyle \frac13}}
\newcommand{\fourth}{{\textstyle \frac14}}

\newcommand{\Stab}{\mathbb{S}}
\newcommand{\StabThr}[2]{\Gamma_{#1}(#2)}
\newcommand{\StabThrmin}[2]{{\underline{\Gamma}}_{#1}(#2)}
\newcommand{\StabThrmax}[2]{{\overline{\Gamma}}_{#1}(#2)}
\newcommand{\TestFcn}{\Psi}

\renewcommand{\phi}{\varphi}

\begin{document}
\title{Complexity - Exercise 3}

 \author{Omer Tamuz, 035696574}
\maketitle


\begin{enumerate}
\item
  {\bf Note}: I'm assuming here that $|Z_k|$ is a polynomial time calculable 
  function of $k$, and vice versa.
  \begin{itemize}
  \item 
    
    Assume $\{Z_k\}_{k\in \N}$ is predictable. Then there exists a 
    probabilistic polynomial time algorithm $A$ such
    that $\P_i[A(1_k, F_i(Z_k)) = B_{i+1} (Z_k )]-1/2$ is larger than 
    $1/p(|Z_k|)$ for
    some polynomial $p$ and infinitely many values of $k$.
    
    Let $M$ be the following algorithm: Given a string $X$ of length 
    $|Z_k|$, it runs $A$ on $1_k$ and $F_i(X)$ for $i$ picked uniformly in $[|Z_k|-1]$ 
    and accepts if and only if $A$ returned $B_{i+1}(X)$.

    Let $U_k$ be a string picked from the uniform distribution over 
    $\{0,1\}^{|Z_k|}$.
    Then $\P[M(U_k)=1]$ is clearly one half. On the other hand, 
    $\P[M(Z_k)=1]$ is equal to $\P_i[A(1_k,F_i(Z_k))]=B_{i+1}(Z_k)$.
    Hence $\P[M(Z_k)=1]-\P[M(U_k)=1]>1/p(|Z_k|)$ for infinitely many values of 
    $k$. Since $M$ is clearly polynomial time, this means that $\{Z_k\}$ is not
    pseudorandom.
  \item
    Assume $\{Z_k\}_{k\in \N}$ is unpredictable. Then for every probabilistic 
    polynomial time algorithm $A$ and polynomial $p$ it holds that
    $\P_i[A(1_k, F_i(Z_k)) = B_{i+1} (Z_k )]-1/2$ is less than 
    $1/p(|Z_k|)$ for $k$ large enough.

    Assume by way of contradiction that $\{Z_k\}_{k\in \N}$ is not 
    pseudorandom. Then there exists a probabilistic polynomial time algorithm 
    $M$ and a polynomial $q$ such that  $|\P[M(Z_k)=1]-\P[M(U_k)=1]|>1/q(|Z_k|)$
    infinitely many times. Assume WLOG that $\P[M(Z_k)=1]-\P[M(U_k)=1]>1/q(|Z_k|)$
    infinitely many times (otherwise take $N=1-M$). 

    Given a string $X$ of length $|Z_k|$, let $M^i(X)$ be the result of 
    running $M$ on string consisting of the first $i$ bits of $X$ concatenated
    with $|Z_k|-i$ random bits. Then $M^0(X)=M(U_k)$ and $M^{|Z_k|}(X)=M(X)$.

    Let $A$ be the following algorithm: on input $1_k$ and $X$,
    $A$ runs $M$ on $X$ concatenated with a string $R$ of
    $|Z_k|-|X|$ random bits. $A$ returns $R_0$ (the first bit of $R$) if 
    $M$ returns one, and a random bit otherwise. Then:
    \begin{align*}
      \lefteqn{\P_i[A(1_k,F_i(Z_k)]=B_{i+1}(Z_k)]} 
      \\ &= \P_{i,R}[M(F_i(Z_k)R)=1|R_0=B_{i+1}(Z_k)]\P[R_0=B_{i+1}(Z_k)] + \half\P_{i,R}[M(F_i(Z_k)R)=0]
      \\ &= \half\P_{i,R}[M(F_i(Z_k)R)=1|R_0=B_{i+1}(Z_k)] + \half\P_{i,R}[M(F_i(Z_k)R)=0]
      \\ &= \half\P_{i,R}[M^{i+1}(Z_k)] + \half(1-\P_{i,R}[M(F_i(Z_k)R)=1])
      \\ &= \half\P_{i,R}[M^{i+1}(Z_k)] + \half(1-\P_{i,R}[M^i(Z_k)])
      \\ &= \half \left(1+ \P_{i,R}[M^{i+1}(Z_k)] -\P_{i,R}[M^i(Z_k)]\right)
      \\ &= \half \left(1+ {1\over |Z_k|}\sum_i\P_{R}[M^{i+1}(Z_k)] -\P_{i,R}[M^i(Z_k)]\right)
      \\ &= \half \left(1+ {1\over |Z_k|}\left(\P[M(Z_k)] -\P[M(U_k)]\right)\right)
    \end{align*}
    This is greater than $\half+\half{1\over |Z_k|q(|Z_k|)}$ for infinitely
    many values of $k$, in contradiction to the hypothesis that 
    $\{Z_k\}_{k\in\N}$ is pseudorandom.
   
    
  \end{itemize}

\item
  
  As suggested in the guideline, let $\gamma$ be less than one. Otherwise the
  construction is trivial, as condition two always holds for any two sets that
  fulfil the first requirement. Less following the guideline, set 
  $m(k)=\gamma k/2$ and $l(k)=2^{\gamma^2 m(k)/6}$. 
  
  Given $k$, we select the sets $S_1,\ldots,S_{l(k)}$ one
  after the other, by finding each time, through exhaustive search, a set
  that fulfils the first condition, and the second condition with respect to the 
  sets already chosen. We prove that such a set always exists, and that
  this process can be executed in time that is exponential in $k$. This shows
  the required construction.
  
  \begin{claim}
    At iteration $i$, there exists a set $S_i$ such that
    \begin{itemize}
    \item $S_i\in [k]$ and $|S_i|=m(k)$.
    \item For all $j<i$, it holds that $|S_i\cap S_j|\leq \gamma m(k)$.
    \end{itemize}
  \end{claim}
  \begin{proof}
    We'll prove the claim by showing that a randomly picked set satisfies both
    conditions with positive probability.
    
    Pick $S_i=\{x_1,\ldots,x_k\}$ uniformly at random from the subsets of size 
    $m(k)$ of $[k]$.
    Then with probability one the first condition holds. 
    
    For a particular $j$ (less than $i$) the probability that some $x\in S_i$ is also
    in $S_j$ is $|S_j|/k=\gamma /2$. Now, the events $x_p\in S_j$ and $x_q\in S_j$ are 
    not independent, since $x_p\neq x_q$ when $p \neq q$. However, the conditional 
    probability 
    $\P[x_{q+1}\in S_j|x_1,\ldots,x_q]$ is {\bf at most} 
    $|S_j|/(k-q)\leq m(k)/(k-q)=\gamma /(2-2q/k)$. Hence we can assume that the
    events $x_q\in S_j$ are independent, each with expectation $\gamma /(2-2q/k)$.
    Then the expectation of the size of the intersection is
    \begin{align*}
      \E[|S_i \cap S_j| &\leq \sum_{q=1}^{m(k)} \E[1_{x_q\in S_j}]
      \\  &=\sum_{q=1}^{m(k)}{\gamma \over 2 - 2q/k}
      \\  &\gamma/2=\sum_{q=1}^{m(k)}{1 \over 1 - q/k}
      \\  &\leq \gamma/2\sum_{q=1}^{m(k)/2}{1 \over 1 - m(k)/2/k}+\sum_{q=m(k)/2}^{m(k)}{1 \over 1 - m(k)/k}
      \\  &\leq \gamma/2\sum_{q=1}^{m(k)/2}{1 \over 1 - \gamma/4}+\sum_{q=m(k)/2}^{m(k)}{1 \over 1 - \gamma/2}
      \\  &\leq \gamma/2\sum_{q=1}^{m(k)/2}{4 \over 3}+\sum_{q=m(k)/2}^{m(k)}2
      \\  &\leq {5\over 6}\gamma m(k).
    \end{align*}
    We can now use the Hoeffding inequality:
    \begin{align*}
      \P[|S_i\cap S_j|>\gamma m(k)] &= \P[|S_i\cap S_j|- 5\gamma m(k)/6>\gamma m(k)/6]
      \\ &\leq e^{-2\gamma^2 m(k)/6^2}
      \\ &\leq 2^{-\gamma^2 m(k)/6}.
    \end{align*}
    By union bound, this means that $S_i$ will not intersect any of the previous
    $s_j$'s since these are fewer (by at least one) than 
    $l(k)=2^{\gamma^2 m(k)/6}$. 
  \end{proof}
  
  \begin{claim}
    Each set $S_i$ can be found in time that is exponential in $k$.  
  \end{claim}
  Since the number of sets $S_i$ is exponential, this would prove that the entire 
  process takes exponential time, since the set of exponential functions is 
  closed under composition.
  
  \begin{proof}
    We test each subset of  $[k]$ of size $m(k)$. The number of such sets is less 
    than $2^k$ (the total number of subsets of $[k]$). For each subset we consider
    the size of its intersection with at most $l(k)$ sets, where each such 
    comparison takes polynomial time $p(k)$. Hence to total time is less than 
    $2^kl(k)p(k)$, which is certainly exponential in $k$.
  \end{proof}
  


\end{enumerate}
\end{document}


