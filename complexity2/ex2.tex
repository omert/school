\documentclass[11pt]{article} \usepackage{amssymb}
\usepackage{amsfonts} \usepackage{amsmath} \usepackage{bm}
\usepackage{latexsym} \usepackage{epsfig}

\setlength{\textwidth}{6.5 in} \setlength{\textheight}{8.25in}
\setlength{\oddsidemargin}{0in} \setlength{\topmargin}{0in}
\addtolength{\textheight}{.8in} \addtolength{\voffset}{-.5in}

\newtheorem{theorem}{Theorem}[section]
\newtheorem{lemma}[theorem]{Lemma}
\newtheorem{proposition}[theorem]{Proposition}
\newtheorem{corollary}[theorem]{Corollary}
\newtheorem{fact}[theorem]{Fact}
\newtheorem{definition}[theorem]{Definition}
\newtheorem{remark}[theorem]{Remark}
\newtheorem{conjecture}[theorem]{Conjecture}
\newtheorem{claim}[theorem]{Claim}
\newtheorem{example}[theorem]{Example}
\newenvironment{proof}{\noindent \textbf{Proof:}}{$\Box$}

\newcommand{\ignore}[1]{}

\newcommand{\enote}[1]{} \newcommand{\knote}[1]{}
\newcommand{\rnote}[1]{}



% \newcommand{\enote}[1]{{\bf [[Elchanan:} {\emph{#1}}{\bf ]]}}
% \newcommand{\knote}[1]{{\bf [[Krzysztof:} {\emph{#1}}{\bf ]]}}
% \newcommand{\rnote}[1]{{\bf [[Ryan:} {\emph{#1}}{\bf ]]}}



\DeclareMathOperator{\Support}{Supp} \DeclareMathOperator{\Opt}{Opt}
\DeclareMathOperator{\Ordo}{\mathcal{O}}
\newcommand{\MaxkCSP}{\textsc{Max $k$-CSP}}
\newcommand{\MaxkCSPq}{\textsc{Max $k$-CSP$_{q}$}}
\newcommand{\MaxCSP}[1]{\textsc{Max CSP}(#1)} \renewcommand{\Pr}{{\bf
    P}} \renewcommand{\P}{{\bf P}} \newcommand{\Px}{\mathop{\bf P\/}}
\newcommand{\E}{{\bf E}} \newcommand{\Cov}{{\bf Cov}}
\newcommand{\Var}{{\bf Var}} \newcommand{\Varx}{\mathop{\bf Var\/}}

\newcommand{\bits}{\{-1,1\}}

\newcommand{\nsmaja}{\textstyle{\frac{2}{\pi}} \arcsin \rho}

\newcommand{\Inf}{\mathrm{Inf}} \newcommand{\I}{\mathrm{I}}
\newcommand{\J}{\mathrm{J}}

\newcommand{\eps}{\epsilon} \newcommand{\lam}{\lambda}

% \newcommand{\trunc}{\ell_{2,[-1,1]}}
\newcommand{\trunc}{\zeta} \newcommand{\truncprod}{\chi}

\newcommand{\N}{\mathbb N} \newcommand{\R}{\mathbb R}
\newcommand{\Z}{\mathbb Z} \newcommand{\CalE}{{\mathcal{E}}}
\newcommand{\CalC}{{\mathcal{C}}} \newcommand{\CalM}{{\mathcal{M}}}
\newcommand{\CalR}{{\mathcal{R}}} \newcommand{\CalS}{{\mathcal{S}}}
\newcommand{\CalV}{{\mathcal{V}}}
\newcommand{\CalX}{{\boldsymbol{\mathcal{X}}}}
\newcommand{\CalG}{{\boldsymbol{\mathcal{G}}}}
\newcommand{\CalH}{{\boldsymbol{\mathcal{H}}}}
\newcommand{\CalY}{{\boldsymbol{\mathcal{Y}}}}
\newcommand{\CalZ}{{\boldsymbol{\mathcal{Z}}}}
\newcommand{\CalW}{{\boldsymbol{\mathcal{W}}}}
\newcommand{\CalF}{{\mathcal{Z}}}
% \newcommand{\boldG}{{\boldsymbol G}}
% \newcommand{\boldQ}{{\boldsymbol Q}}
% \newcommand{\boldP}{{\boldsymbol P}}
% \newcommand{\boldR}{{\boldsymbol R}}
% \newcommand{\boldS}{{\boldsymbol S}}
% \newcommand{\boldX}{{\boldsymbol X}}
% \newcommand{\boldB}{{\boldsymbol B}}
% \newcommand{\boldY}{{\boldsymbol Y}}
% \newcommand{\boldZ}{{\boldsymbol Z}}
% \newcommand{\boldV}{{\boldsymbol V}}
\newcommand{\boldi}{{\boldsymbol i}} \newcommand{\boldj}{{\boldsymbol
    j}} \newcommand{\boldk}{{\boldsymbol k}}
\newcommand{\boldr}{{\boldsymbol r}}
\newcommand{\boldsigma}{{\boldsymbol \sigma}}
\newcommand{\boldupsilon}{{\boldsymbol \upsilon}}
\newcommand{\hone}{{\boldsymbol{H1}}}
\newcommand{\htwo}{\boldsymbol{H2}}
\newcommand{\hthree}{\boldsymbol{H3}}
\newcommand{\hfour}{\boldsymbol{H4}}


\newcommand{\sgn}{\mathrm{sgn}} \newcommand{\Maj}{\mathrm{Maj}}
\newcommand{\Acyc}{\mathrm{Acyc}}
\newcommand{\UniqMax}{\mathrm{UniqMax}}
\newcommand{\Thr}{\mathrm{Thr}} \newcommand{\littlesum}{{\textstyle
    \sum}}

\newcommand{\half}{{\textstyle \frac12}}
\newcommand{\third}{{\textstyle \frac13}}
\newcommand{\fourth}{{\textstyle \frac14}}

\newcommand{\Stab}{\mathbb{S}}
\newcommand{\StabThr}[2]{\Gamma_{#1}(#2)}
\newcommand{\StabThrmin}[2]{{\underline{\Gamma}}_{#1}(#2)}
\newcommand{\StabThrmax}[2]{{\overline{\Gamma}}_{#1}(#2)}
\newcommand{\TestFcn}{\Psi}

\renewcommand{\phi}{\varphi}

\begin{document}
\title{Complexity - Exercise 1}

 \author{Omer Tamuz, 035696574}
\maketitle


\begin{enumerate}
\item
Given $f$ as described, let $f(x)_i$ be the $i$'th bit of $f(x)$. 
Let $f'(x,i)$ with $1\leq i \leq l(x)$ be defined by
$$f'(x,i)=f(x)_i$$
that is, $f'$ returns the $i$'th bit of $f(x)$.

Clearly, $f'$ can be calculated by an exponential time Turing machine, given
a machine that calculates $f$ in this time.

\begin{claim}
  For every polynomial size family of circuits $\{C'_{n'}\}_{n'\in \N}$ 
  it holds for large enough $n$ that
$$\P_{x,i}[C'_{n'}(x,i)\neq f'(x,i)]\geq \delta(n')/\l(n'),$$
where $x$ is distributed uniformly over strings of length $n$ so that
$n+\log l(n)=n'$, and $i$ is
distributed uniformly over the integers between $1$ and $\l(n)$.
\end{claim}
\begin{proof}
Assume to the contrary that there exists a family of polynomial circuits
$\{C'_{n'}\}$ that 
calculates $f'$ with failure probability lower than $\delta(n')/l(n')$ for infinitely many
values of $n'$. 
For $n$ such that  $n'=n+\log l(n)$, let 
$C_{n}(x):=C'_{n'}(x,1)C'_{n'}(x,2)\cdots C'_{n'}(x,l(n))$. Clearly, if all the calls to $C'$ succeed then
$C_n(x)=f(x)$. The probability that $C_n(x)$ fails
is equal to the probability that at least one of the calls to $C'$ fails. 
The worst case would be the one in which for every $x$ at most one of the calls
failed - this would maximize the proportion of $x$'s for which $C$ fails.
In this case, since the proportion of failing $(x,i)$'s is at most
$\delta(n')/l(n')$, and since for every $x$ there are $l(n)$ possible
values of $i$, then $C$ fails with probability at most $\delta(n')l(n)/l(n')$.

For any $\epsilon>0$, it holds that for large enough $n$, $n'$ is between $n$ 
and $(1+\epsilon)n$. Hence, since $l$ is a polynomial, we have that
$\lim_{n\to\infty}l(n)/l(n')=1$. Since $\delta(n')$ is less than $\delta(n)$
then we have that
\begin{equation*}
  \P[C_n(x)\neq f(x)] \leq \delta(n),
\end{equation*}
which is a contradiction, and the proof is concluded.
\end{proof}

\item
  Given a binary vector $r$ of length $t(n)$, let 
  $I_r=\{i\mbox{ s.t. } r_i=1\}$. Note that
  \begin{eqnarray*}
    H(x,r)&=& b\left(F\left(x^1,\ldots,x^{t(n)}\right),r\right)
    \\ &=& \bigoplus_{i\in I_r}f(x^i).
  \end{eqnarray*}
  Let $x,r,y$ be such that for $z$, as defined in the hint, it holds that
  $C(z)=G(z)$. Then 
  \begin{eqnarray*}
    C(z)\bigoplus_{i\not\in I_r}f(y^i) &=& G(z)\bigoplus_{i\not\in I_r}f(y^i)
    \\ &=& \bigoplus_{i\in[1,t(n)]}f(z^i)\bigoplus_{i\not\in I_r}f(y^i)
    \\ &=& \bigoplus_{i\in I_r}f(x^i)\bigoplus_{i\not\in I_r}f(y^i)\bigoplus_{i\not\in I_r}f(y^i)
    \\ &=& \bigoplus_{i\in I_r}f(x^i)
  \end{eqnarray*}
  and so if $C(z)=G(z)$ then the ``circuit
  with a magic box'' $C'$ succeeds and $C'(x,r)=H(x,r)$. 
  This implies that the probability
  that $C'(x,r)= H(x,r)$ is greater than or equal to the probability that
  $C(z)= G(z)$, since the latter event is contained in the former.
  
  The probability that  $C'$ will succeed for a random input $(x,r)$ with
  $|r|=n$ is thus
  \begin{eqnarray*}
    \P[\mbox{success}] &=& \P_{(x,r)}\left[\P_y[C'(x,r)= H(x,r)|(x,r)]\right]
    \\ &\geq& \P_{(x,r)}\left[\P_y\left[C(z)= G(z)|(x,r)\right]\right].
  \end{eqnarray*}
  Now, the distribution of $z$, when taken over random $x$ (and $r$), is clearly
  uniform too. Hence
  \begin{eqnarray*}
    \P[\mbox{success}] &\geq & \P_z\left[C(z)= G(z)|(x,r)\right],
  \end{eqnarray*}
  and therefore if there exists a family of circuits that violates Yao's 
  Lemma then there exists a family of ``magic circuits'' that violates the 
  Selective XOR Lemma.

  To arrive at a family of non-magical circuits with the same property, we
  use the averaging argument used in class. Instead of averaging over $(x,r)$
  and then over $y$, we average first over $y$ and then over $x,y$. Since
  then the average (over $y$) probability of success is high, there must exist 
  a single $y$ for which the probability is high. We then hard-wire this
  $y$ and the corresponding $f(y)$ into the circuit, and preserving the
  probability of success, but without the use of magic.
\end{enumerate}
\end{document}


