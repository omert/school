\documentclass[11pt]{article} \usepackage{amssymb}
\usepackage{amsfonts} \usepackage{amsmath} \usepackage{bm}
\usepackage{latexsym} \usepackage{epsfig}

\setlength{\textwidth}{6.5 in} \setlength{\textheight}{8.25in}
\setlength{\oddsidemargin}{0in} \setlength{\topmargin}{0in}
\addtolength{\textheight}{.8in} \addtolength{\voffset}{-.5in}

\newtheorem{theorem}{Theorem}[section]
\newtheorem{lemma}[theorem]{Lemma}
\newtheorem{proposition}[theorem]{Proposition}
\newtheorem{corollary}[theorem]{Corollary}
\newtheorem{fact}[theorem]{Fact}
\newtheorem{definition}[theorem]{Definition}
\newtheorem{remark}[theorem]{Remark}
\newtheorem{conjecture}[theorem]{Conjecture}
\newtheorem{claim}[theorem]{Claim}
\newtheorem{example}[theorem]{Example}
\newenvironment{proof}{\noindent \textbf{Proof:}}{$\Box$}

\newcommand{\ignore}[1]{}

\newcommand{\enote}[1]{} \newcommand{\knote}[1]{}
\newcommand{\rnote}[1]{}



% \newcommand{\enote}[1]{{\bf [[Elchanan:} {\emph{#1}}{\bf ]]}}
% \newcommand{\knote}[1]{{\bf [[Krzysztof:} {\emph{#1}}{\bf ]]}}
% \newcommand{\rnote}[1]{{\bf [[Ryan:} {\emph{#1}}{\bf ]]}}



\DeclareMathOperator{\Support}{Supp} \DeclareMathOperator{\Opt}{Opt}
\DeclareMathOperator{\Ordo}{\mathcal{O}}
\newcommand{\MaxkCSP}{\textsc{Max $k$-CSP}}
\newcommand{\MaxkCSPq}{\textsc{Max $k$-CSP$_{q}$}}
\newcommand{\MaxCSP}[1]{\textsc{Max CSP}(#1)} \renewcommand{\Pr}{{\bf
    P}} \renewcommand{\P}{{\bf P}} \newcommand{\Px}{\mathop{\bf P\/}}
\newcommand{\E}{{\bf E}} \newcommand{\Cov}{{\bf Cov}}
\newcommand{\Var}{{\bf Var}} \newcommand{\Varx}{\mathop{\bf Var\/}}

\newcommand{\bits}{\{-1,1\}}

\newcommand{\nsmaja}{\textstyle{\frac{2}{\pi}} \arcsin \rho}

\newcommand{\Inf}{\mathrm{Inf}} \newcommand{\I}{\mathrm{I}}
\newcommand{\J}{\mathrm{J}}

\newcommand{\eps}{\epsilon} \newcommand{\lam}{\lambda}

% \newcommand{\trunc}{\ell_{2,[-1,1]}}
\newcommand{\trunc}{\zeta} \newcommand{\truncprod}{\chi}

\newcommand{\N}{\mathbb N} \newcommand{\R}{\mathbb R}
\newcommand{\Z}{\mathbb Z} \newcommand{\CalE}{{\mathcal{E}}}
\newcommand{\CalC}{{\mathcal{C}}} \newcommand{\CalM}{{\mathcal{M}}}
\newcommand{\CalR}{{\mathcal{R}}} \newcommand{\CalS}{{\mathcal{S}}}
\newcommand{\CalV}{{\mathcal{V}}}
\newcommand{\CalX}{{\boldsymbol{\mathcal{X}}}}
\newcommand{\CalG}{{\boldsymbol{\mathcal{G}}}}
\newcommand{\CalH}{{\boldsymbol{\mathcal{H}}}}
\newcommand{\CalY}{{\boldsymbol{\mathcal{Y}}}}
\newcommand{\CalZ}{{\boldsymbol{\mathcal{Z}}}}
\newcommand{\CalW}{{\boldsymbol{\mathcal{W}}}}
\newcommand{\CalF}{{\mathcal{Z}}}
% \newcommand{\boldG}{{\boldsymbol G}}
% \newcommand{\boldQ}{{\boldsymbol Q}}
% \newcommand{\boldP}{{\boldsymbol P}}
% \newcommand{\boldR}{{\boldsymbol R}}
% \newcommand{\boldS}{{\boldsymbol S}}
% \newcommand{\boldX}{{\boldsymbol X}}
% \newcommand{\boldB}{{\boldsymbol B}}
% \newcommand{\boldY}{{\boldsymbol Y}}
% \newcommand{\boldZ}{{\boldsymbol Z}}
% \newcommand{\boldV}{{\boldsymbol V}}
\newcommand{\boldi}{{\boldsymbol i}} \newcommand{\boldj}{{\boldsymbol
    j}} \newcommand{\boldk}{{\boldsymbol k}}
\newcommand{\boldr}{{\boldsymbol r}}
\newcommand{\boldsigma}{{\boldsymbol \sigma}}
\newcommand{\boldupsilon}{{\boldsymbol \upsilon}}
\newcommand{\hone}{{\boldsymbol{H1}}}
\newcommand{\htwo}{\boldsymbol{H2}}
\newcommand{\hthree}{\boldsymbol{H3}}
\newcommand{\hfour}{\boldsymbol{H4}}


\newcommand{\sgn}{\mathrm{sgn}} \newcommand{\Maj}{\mathrm{Maj}}
\newcommand{\Acyc}{\mathrm{Acyc}}
\newcommand{\UniqMax}{\mathrm{UniqMax}}
\newcommand{\Thr}{\mathrm{Thr}} \newcommand{\littlesum}{{\textstyle
    \sum}}

\newcommand{\half}{{\textstyle \frac12}}
\newcommand{\third}{{\textstyle \frac13}}
\newcommand{\fourth}{{\textstyle \frac14}}

\newcommand{\Stab}{\mathbb{S}}
\newcommand{\StabThr}[2]{\Gamma_{#1}(#2)}
\newcommand{\StabThrmin}[2]{{\underline{\Gamma}}_{#1}(#2)}
\newcommand{\StabThrmax}[2]{{\overline{\Gamma}}_{#1}(#2)}
\newcommand{\TestFcn}{\Psi}

\renewcommand{\phi}{\varphi}

\begin{document}
\title{Sublinear Algorithms - Exercises 1 and 2}

 \author{Omer Tamuz, 035696574}
\maketitle

\section{Exercise 1}
\begin{enumerate}
  \item 
    The algorithm is the following: We randomly sample the vector 
    $k(\epsilon)$ times from the uniform distribution, 
    and output the fraction of non-zero coordinates in 
    our sample.

    By the Chernoff-Hoeffding bound, the probability that the result is wrong
    by more than $\epsilon$ is at most $2e^{-2k(\epsilon)\epsilon^2}$. 
    Hence, to achieve 2/3 success probability we must have:
    \begin{eqnarray*}
      2e^{-2k(\epsilon)\epsilon^2} &\leq& 1/3
      \\ k(\epsilon) &\geq& {1\over 2\epsilon^2}\log{6}
    \end{eqnarray*}

    \item 

      The algorithm here is essentially identical to the one from the previous
      question: we choose at random, from the uniform distribution, 
      $k(\epsilon)$ pairs $i,j$ such that $i\neq j$. We then output the fraction
      that are inverted. By the same analysis as above we get with probability
      2/3 a result within $\epsilon$ when 
      $k(\epsilon)={1\over 2\epsilon^2}\log 6$.
\end{enumerate}
\section{Exercise 2}
\begin{enumerate}
  \item 

    In Kruskal's algorithm we order the edges by ascending weight, consider
    each one in turn and add it to the tree if it does not form a cycle.

    Each time we add an edge to the forest we reduce the number of 
    connected components by one. Therefore the number of edges of weight
    $i$ that were added to the tree is precisely $c_{i-1}-c_i$, with $c_0=n$ and
    $c_W=1$ (since the graph is connected).
    The weight of the tree is then:
    \begin{eqnarray*}
      \mbox{MST}(G)&=&\sum_{i=1}^Wi(c_{i-1}-c_i)
      \\ &=& \sum_{i=1}^Wic_{i-1}-\sum_{i=1}^Wic_i
      \\ &=& c_0+\sum_{i=1}^{W-1}(i+1)c_i-\sum_{i=1}^{W-1}ic_i+Wc_W
      \\ &=& n-W+\sum_{i=1}^{W-1}c_i
    \end{eqnarray*}
  \item
    Let $M$ be a maximum matching and $L$ be a matching such that
    $|L|<\half|M|$. Let $V_L$ be the set of vertices that the edges of $L$
    are adjacent to. Then because $L$ is a matching $|V_L|=2|L|$ and
    $|V_L|<|M|$. Hence it is impossible that each edge of $M$ is adjacent
    to some vertex in $V_L$, since there are more edges in $M$ than vertices
    in $V_L$, and each vertex is adjacent to at most one edge. 
    Therefore there 
    exists an edge $e$ in $M\setminus L$ such that $L\cup \{e\}$ is a matching,
    and so $L$ is not maximal. 

    We have shown that if $L$ is of size less than
    half the size of a maximum matching then it is not maximal. Hence if it is 
    maximal then its size is at least half the size of a maximal matching.


\end{enumerate}
\end{document}


