%&latex
\documentclass[handout]{beamer}

\usepackage{beamerthemesplit}
\setbeamertemplate{footline}[frame number]
\mode<presentation>
{
  \usetheme{Warsaw}
  \setbeamercovered{transparent}
}

\setlength{\unitlength}{\textwidth}
\renewcommand{\figurename}{}

\usepackage{ulem}

\DeclareMathOperator{\Support}{Supp} \DeclareMathOperator{\Opt}{Opt}
\DeclareMathOperator{\Ordo}{\mathcal{O}}
\newcommand{\MaxkCSP}{\textsc{Max $k$-CSP}}
\newcommand{\MaxkCSPq}{\textsc{Max $k$-CSP$_{q}$}}
\newcommand{\MaxCSP}[1]{\textsc{Max CSP}(#1)} \renewcommand{\Pr}{{\bf
    P}} \renewcommand{\P}{{\bf P}} \newcommand{\Px}{\mathop{\bf P\/}}
\newcommand{\E}{\mbox{E}} \newcommand{\Cov}{\mbox{Cov}}
\newcommand{\Var}{\mbox{Var}} \newcommand{\Varx}{\mathop{\bf Var\/}}

\newcommand{\bits}{\{-1,1\}}

\newcommand{\nsmaja}{\textstyle{\frac{2}{\pi}} \arcsin \rho}

\newcommand{\Inf}{\mathrm{Inf}} \newcommand{\I}{\mathrm{I}}
\newcommand{\J}{\mathrm{J}}

\newcommand{\eps}{\epsilon} \newcommand{\lam}{\lambda}

% \newcommand{\trunc}{\ell_{2,[-1,1]}}
\newcommand{\trunc}{\zeta} \newcommand{\truncprod}{\chi}

\newcommand{\N}{\mathbb N} \newcommand{\R}{\mathbb R}
\newcommand{\Z}{\mathbb Z} \newcommand{\CalE}{{\mathcal{E}}}
\newcommand{\CalC}{{\mathcal{C}}} \newcommand{\CalM}{{\mathcal{M}}}
\newcommand{\CalR}{{\mathcal{R}}} \newcommand{\CalS}{{\mathcal{S}}}
\newcommand{\CalV}{{\mathcal{V}}}
\newcommand{\CalX}{{\boldsymbol{\mathcal{X}}}}
\newcommand{\CalG}{{\boldsymbol{\mathcal{G}}}}
\newcommand{\CalH}{{\boldsymbol{\mathcal{H}}}}
\newcommand{\CalY}{{\boldsymbol{\mathcal{Y}}}}
\newcommand{\CalZ}{{\boldsymbol{\mathcal{Z}}}}
\newcommand{\CalW}{{\boldsymbol{\mathcal{W}}}}
\newcommand{\CalF}{{\mathcal{Z}}}
% \newcommand{\boldG}{{\boldsymbol G}}
% \newcommand{\boldQ}{{\boldsymbol Q}}
% \newcommand{\boldP}{{\boldsymbol P}}
% \newcommand{\boldR}{{\boldsymbol R}}
% \newcommand{\boldS}{{\boldsymbol S}}
% \newcommand{\boldX}{{\boldsymbol X}}
% \newcommand{\boldB}{{\boldsymbol B}}
% \newcommand{\boldY}{{\boldsymbol Y}}
% \newcommand{\boldZ}{{\boldsymbol Z}}
% \newcommand{\boldV}{{\boldsymbol V}}
\newcommand{\boldi}{{\boldsymbol i}} \newcommand{\boldj}{{\boldsymbol
    j}} \newcommand{\boldk}{{\boldsymbol k}}
\newcommand{\boldr}{{\boldsymbol r}}
\newcommand{\boldsigma}{{\boldsymbol \sigma}}
\newcommand{\boldupsilon}{{\boldsymbol \upsilon}}
\newcommand{\hone}{{\boldsymbol{H1}}}
\newcommand{\htwo}{\boldsymbol{H2}}
\newcommand{\hthree}{\boldsymbol{H3}}
\newcommand{\hfour}{\boldsymbol{H4}}


\newcommand{\sgn}{\mathrm{sgn}} \newcommand{\Maj}{\mathrm{Maj}}
\newcommand{\Acyc}{\mathrm{Acyc}}
\newcommand{\UniqMax}{\mathrm{UniqMax}}
\newcommand{\Thr}{\mathrm{Thr}} \newcommand{\littlesum}{{\textstyle
    \sum}}

\newcommand{\half}{{\textstyle \frac12}}
\newcommand{\third}{{\textstyle \frac13}}
\newcommand{\fourth}{{\textstyle \frac14}}
\newcommand{\fifth}{{\textstyle \frac15}}

\newcommand{\Stab}{\mathbb{S}}
\newcommand{\StabThr}[2]{\Gamma_{#1}(#2)}
\newcommand{\StabThrmin}[2]{{\underline{\Gamma}}_{#1}(#2)}
\newcommand{\StabThrmax}[2]{{\overline{\Gamma}}_{#1}(#2)}
\newcommand{\TestFcn}{\Psi}

\title{Testing Symmetric Properties of Distributions} \subtitle{ Paul
  Valiant, 2008}

% Use the \inst command to identify several affiliations.
\author{Omer Tamuz, Tamar Zondiner}
\date{\today}

\begin{document}

%\newcounter{realtotalframenumber}{\value{totalframenumber}}
\frame{\titlepage}

\begin{frame}
  \frametitle{Definitions}
  \framesubtitle{Properties}

  \begin{block}<1->{A property $\pi$} A property of a distribution is a
    function $\pi:D_n\rightarrow \mathbb{R}$, where $D_n$ is the set
    of probability distributions on $[n]$.
  \end{block}

  \begin{block}<2->{A binary property $\pi_a^b$} A property $\pi$ and
    pair of real numbers $a<b$ induce a binary property
    $\pi_a^b:D_n\to\{"yes","no",\varnothing\}$ defined by:
    \begin{equation*}
      \pi_a^b(p)=
      \left\{\begin{matrix}
          "yes" & \mbox{if }\pi(p)>b \\ 
          "no" & \mbox{if }\pi(p)<a\\ 
          \varnothing  & \mbox{otherwise} 
        \end{matrix}\right.
    \end{equation*}
  \end{block}
\end{frame}


\begin{frame}
  \frametitle{Definitions}
  \framesubtitle{ A tester} Let $\pi_a^b$ be a binary property on
  $D_n$.
  \begin{block}<2->{A tester} 

    An algorithm $T$ is a {\bf ``$\pi_a^b-$tester with sample
      complexity $k(\cdot)$''} if, given a sample of size $k(n)$ from
    a distribution $p \in D_n$, algorithm $T$ will:
    \begin{itemize}
    \item<3-> accept with probability greater than $\frac{2}{3}$ if
      $\pi_a^b(p)="yes"$, and
    \item<4-> reject with probability greater than $\frac{2}{3}$ if
      $\pi_a^b(p)="no"$, and
    \end{itemize}
  \end{block}
  \begin{block}<5->{} The tester's behavior is unspecified when
    $\pi_a^b(p)=\phi$, i.e. when $a\leq\pi(p)\leq b$.
  \end{block}

 
\end{frame}
 
\begin{frame}
  \frametitle{Definitions}
  \framesubtitle{Symmetry, ($\epsilon,\delta)-$weak continuity}

  \begin{block}<1->{A Symmetric Property}A property $\pi$ is symmetric if
    for all distributions $p$ and all permutations $\sigma$ we have
    $\pi(p)=\pi(p\circ\sigma)$.
  \end{block}

  \begin{block}<2->{An ($\epsilon, \delta$)-weakly continuous property}
    A property $\pi$ is $(\epsilon,\delta)$-weakly continuous if for
    all distributions $p^+,p^-$ satisfying $|p^+-p^-|\le\delta$ we
    have $|\pi(p^+)-\pi(p^-)|\le \epsilon$.
  \end{block}
  \onslide<3->{$|x-y|$ denotes the $L_1$ distance.}
\end{frame}

\begin{frame}
  \frametitle{Example} \framesubtitle{Distance from the uniform
    distribution}
  \begin{theorem}
    Distance from the uniform distribution is a symmetric and
    $(\delta, \delta)$-weakly continuous property.
  \end{theorem}
  \begin{proof}
    \begin{itemize}
    \item<1-> Let $U_n$ be the uniform distribution on $[n]$.
    \item<2-> Let $\pi(p)=|U_n-p|$ for $p\in D_n$.
    \item<3-> Let $p^+,p^-\in D_n$ be such that $|p^+-p^-|<\delta$.
    \item<4-> Assume WLOG that $\pi(p^+) \geq \pi(p^-)$.
    \end{itemize}
    \onslide<5->{
      \begin{align*}
        |\pi(p^+)-\pi(p^-)|&=|U_n-p^+|-|U_n-p^-|
        \\ &\leq |U_n-p^-|+|p^+-p^-|-|U_n-p^-|
        \\ &= |p^+-p^-| \leq \delta
      \end{align*}
    }
  \end{proof}
\end{frame}

\begin{frame}
  \frametitle{Example} \framesubtitle{Entropy}
  \begin{theorem}<1->

    The entropy is a symmetric and $\left(1,{1\over 2\log
        n}\right)$-weakly continuous property.
  \end{theorem}
  \begin{proof}<2->
    Easy.
  \end{proof}
\end{frame}


\begin{frame}
  \frametitle{Non-Examples} \framesubtitle{}
  \begin{block}{}
    Distance from a distribution that is not the uniform distribution
    is not symmetric.
  \end{block}
\end{frame}


\begin{frame}
  \frametitle{The Canonical Tester} \framesubtitle{Canonical Tester
    $T^\theta$ for $\pi_a^b$}
  \begin{block}{The Canonical Tester($\theta$)}
    \begin{enumerate}
    \item<2-> Insert the constraint $\sum_ip(i)=1$.
    \item<3-> For each $i$ such that $s(i)>\theta$ insert the constraint
      $p(i)=\frac{s(i)}{k}$. Otherwise insert the constraint $p(i)\in
      [0,\frac{\theta}{k}]$.
    \item<4-> Let $P$ be the set of solutions to these constraints.
    \item<5-> If the set $\pi_a^b(P)$ (the image of elements of $P$
      under $\pi_a^b$) contains only ``$yes$'' and $\varnothing$ return
      ``$yes$''. If it contains only ``$no$'' and $\varnothing$ return
      ``$no$''. Otherwise answer arbitrarily.
    \end{enumerate}
  \end{block}
\end{frame}
\begin{frame}
  \frametitle{The Canonical Tester} \framesubtitle{Canonical Tester
    $T^\theta$ for $\pi_a^b$}
  \begin{block}{}
    \begin{itemize}
    \item<1-> It seems plausible that the canonical tester behaves
      correctly for the high frequency elements.
    \item<2-> The tester effectively discards all information
      regarding the low frequency elements.
    \item<3-> If we can show that no tester can extract information from
      these elements then it will follow that the canonical tester is
      almost optimal.
    \item<4-> In this talk we prove a weaker statement: no tester can
      extract information from a sample where all the elements have
      low frequency.
    \end{itemize}
  \end{block}
\end{frame}

\begin{frame}
  \frametitle{The Canonical Testing Theorem We Wish For}
  \framesubtitle{}
  \begin{block}<1->{Not True Theorem}
    Given a symmetric $(\epsilon,\delta)$-weakly continuous property
    $\pi:D_n\rightarrow \mathbb{R}$ and two thresholds $a<b$, such
    that the Canonical Tester $T^\theta$ for $\theta=600\log
    n/\delta^2$ on $\pi_a^b$ fails to distinguish between $\pi>b$ and
    $\pi<a$ in $k$ samples, then no tester can distinguish between
    $\pi>b$ and $\pi<a$ in $k$ samples.
  \end{block}
  \onslide<2->{ Sadly, this is not true. }
\end{frame}
\begin{frame}
  \frametitle{Canonical Testing Theorem}
  \framesubtitle{}
  \begin{theorem}
    Given a symmetric $(\epsilon,\delta)$-weakly continuous property
    $\pi:D_n\rightarrow \mathbb{R}$ and two thresholds $a<b$, such
    that the Canonical Tester $T^\theta$ for $\theta=600\log
    n/\delta^2$ on $\pi_a^b$ fails to distinguish between
    $\pi>b+\epsilon$ and $\pi<a-\epsilon$ in $k$ samples, then no
    tester can distinguish between $\pi>b-\epsilon$ and
    $\pi<a+\epsilon$ in $k\cdot \frac{\delta^3}{n^{o(1)}}$ samples.
  \end{theorem}
  \onslide<2->{ Essentially, the Canonical Tester is optimal up to
    small additive constants and a small $(n^{o(1)})$ factor in the
    number of samples $k$.}
\end{frame}


\begin{frame}
  \frametitle{Low Frequency Blindness} \framesubtitle{}

  \onslide<1->{ The crux is to prove that the canonical tester does
    the ``right thing'' (i.e., nothing!) for the low frequency
    elements.  }
  
  \begin{block}<2->{Low Frequency Blindness Theorem}
    Let $\pi$ be a symmetric property on distributions on $[n]$ that
    is $(\epsilon,\delta)$-weakly continuous.

    Let $p^+,p^-$ be two distributions that are identical for any
    index occurring with probability at least $\frac{\theta}{k}$ in
    either distribution, where $\theta=\frac{600\log n}{\delta^2}$.

    If $\pi(p^+)>b$ and $\pi(p^-)<a$ , then no tester can distinguish
    between $\pi>b-\epsilon$ and $\pi<a+\epsilon$ in $k\cdot
    \frac{\delta^3}{n^{o(1)}}$ samples.
  \end{block}

  \onslide<3->{ If we could show that such $p^+$ and $p^-$ exist
    whenever the canonical tester fails than this would imply the
    canonical testing theorem.}
  
\end{frame}



\begin{frame}
  \frametitle{Low Frequency Blindness $\Rightarrow$ Canonical Testing
    Theorem} \framesubtitle{}
  \begin{lemma}<1->
    Given a distribution $p$ and a parameter $\theta$, if we draw $k$
    random samples from $p$ then with probability at least
    $1-\frac{4}{n}$ the set $P$ constructed by the Canonical Tester
    will include a distribution $\hat{p}$ such that $|p-\hat p
    |\le 24\sqrt{\frac{\log n}{\theta}}$.
  \end{lemma}
  \onslide<2->{ If $\theta = 600\log n / \delta^2$ then this reads
    $|p-\hat p |\le \delta$.}
  \begin{proof}<3->
    ``The proof is elementary: use Chernoff bounds on each index $i$
    and then apply the union bound to combine the bounds.''
  \end{proof}
\end{frame}


\begin{frame}
  \frametitle{Low Frequency Blindness $\Rightarrow$ Canonical Testing
    Theorem} \framesubtitle{}

  \onslide<1->{ {\bf Reminder:} the canonical testing theorem states
    that if the canonical tester fails with $k$ samples then any
    slightly weaker tester also fails.  }
  
  \begin{block}{Proof: Canonical Testing Theorem}
    \begin{itemize}
    \item<2-> Assume canonical tester says ``no'' with probability $1/3$
      to some $p$ for which $\pi(p)>b+\epsilon$ (so it should have
      said yes).
    \item<3-> $\Rightarrow$ with probability $1/3$ there exists $p^-\ \in
      P$ such that $\pi(p^-)<a$.
    \item<4-> By the lemma, $P$ contains some $p^+$ such that
      $|p-p^+|<\delta$ with probability $1-4/n$. $\pi(p^+)>b$ by continuity.
    \item<5-> $\Rightarrow$ there exists a single $P$ with both of these
      properties.
    \item<6-> $\Rightarrow$ there exist $p^-$ and $p^+$ with the same
      $\theta$-high-frequency elements such that $\pi(p^-)<a$ and $\pi(p^+)>b$.
    \item<7-> $\Rightarrow$ the theorem follows by application of low
      frequency blindness.
    \end{itemize}
  \end{block}
\end{frame}

\begin{frame}
  \frametitle{Coffee Break}
  \framesubtitle{Coffee Break}
  \begin{block}{Coffee Break}
    Coffee Break
  \end{block}
\end{frame}

\begin{frame}
  \frametitle{Fingerprints} \framesubtitle{Definition}

  \begin{block}<1->{Histogram}
    The histogram $h$ of a vector $v=(v_1,\ldots,v_k)$ is a vector such
    that $h_i$ is the number of components of $v$ with value $i$.
  \end{block}
  \begin{block}<2->{Fingerprint}
    A fingerprint $f$ of a vector $v$ is the histogram of the
    histogram of $v$.
  \end{block}
\end{frame}

\begin{frame}
  \frametitle{Fingerprints} \framesubtitle{Example}

  \begin{block}{Example}
    Let $v=(3,1,2,2,5,1,2)$. Then:
    \begin{itemize}
    \item<1-> Its histogram is $h=(2,3,1,0,1)$.
    \item<2-> Its fingerprint is $f=(2,1,1)$.
    \item<3-> We omit the zero component of $f$.
    \end{itemize}
  \end{block}
  \begin{block}<4->{} A tester for a symmetric distribution $\pi$ may
    consider just the fingerprint of the sample and discard the rest
    of the information.
  \end{block}
\end{frame}


\begin{frame}
  \frametitle{Low Frequency Blindness} \framesubtitle{}
  \begin{theorem}
    Let $\pi$ be a symmetric property on distributions on $[n]$ that
    is $(\epsilon,\delta)$-weakly continuous.

    Let $p^+,p^-$ be two distributions that are identical for any
    index occurring with probability at least $\frac{\theta}{k}$ in
    either distribution, where $\theta=\frac{600\log n}{\delta^2}$.

    If $\pi(p^+)>b$ and $\pi(p^-)<a$ , then no tester can distinguish
    between $\pi>b-\epsilon$ and $\pi<a+\epsilon$ in $k\cdot
    \frac{\delta^3}{n^{o(1)}}$ samples.
  \end{theorem}
\end{frame}


\begin{frame}
  \frametitle{Low Frequency Blindness} \framesubtitle{(simplified)}
  \onslide<1->{ We'll limit our analysis to distributions with low
    frequencies.  Suppose {\bf all} elements have probability
    $<\frac{\theta}{k}$ where $\theta=\frac{600\log n}{\delta^2}$.  }
  \begin{theorem}<2->
    Let $\pi$ be a symmetric property on distributions on $[n]$ that
    is $(\epsilon,\delta)$-weakly continuous.

    Let $p^+,p^-$ be two distributions for which all indices occur
    with probability at most $\frac{\theta}{k}$, where
    $\theta=\frac{600\log n}{\delta^2}$.

    If $\pi(p^+)>b$ and $\pi(p^-)<a$ , then no tester can distinguish
    between $\pi>b-\epsilon$ and $\pi<a+\epsilon$ in $k\cdot
    \frac{\delta^3}{n^{o(1)}}$ samples.
  \end{theorem}
\end{frame}


\begin{frame}
  \frametitle{Proof Sketch} \framesubtitle{}

  \begin{itemize}
  \item<1-> Let $p^+$ and $p^-$ be {\bf low frequency distributions}
    such that $\pi(p^+)>b$ and $\pi(p^-)<a$.
  \item<2-> We construct $\hat{p}^+$ and $\hat{p}^-$ such that
    \begin{itemize}
    \item<3-> $|\hat{p}^\pm-p^\pm|<\delta$, and therefore
      $\pi(\hat{p}^+)>b-\epsilon$ and
      $\pi(\hat{p}^-)<a+\epsilon$.
    \item<4->$\hat{p}^+$ and $\hat{p}^-$ have similar {\bf Poisson
        moments} for sample size $\hat{k}=k\frac{\delta^3}{n^{o(1)}}$.
    \end{itemize}
  \item<5-> For any sample size for which two distributions have similar
    Poisson moments, they also have similar {\bf fingerprints}. This
    is the {\bf ``Wishful Thinking Theorem''}.
  \item<6-> We now have two distributions with similar fingerprints;
    one has the property and the other doesn't. It is therefore
    impossible to test for $\pi_a^b$ with $\hat{k}$ samples.
  \end{itemize}

\end{frame}

\begin{frame}
  \frametitle{Wishful Thinking Theorem} \framesubtitle{Poisson
    Moments}

  \begin{block}{}
    
    \begin{itemize}
    \item<1-> Let $p$ be a distribution on $[n]$.
    \item<2-> For sample size $k$, the expectation of the number of
      appearances of $i$ in the sample (=$h_i$) is $k_i:=k\cdot p_i$.
    \item<3-> Let $\mbox{poi}_x(y)$ be the probability for $y$ drawn
      from the Poisson distribution with parameter $x$.
    \end{itemize}
  \end{block}
  \begin{block}<4->{Definition} Let
    $\lambda_a:=\sum_i\mbox{poi}_{k_i}(a)$. Then
    $\lambda=\{\lambda_a\}_{a=1}^\infty$ are the {\bf Poisson moments
      of $p$ for sample size $k$}.
  \end{block}
  \onslide<5->{ $\lambda_a$ is a good approximation to the expected
    value of $f_a$.}
\end{frame}



\begin{frame}
  \frametitle{Wishful Thinking Theorem} \framesubtitle{Intuition}

  \begin{block}{}
    \begin{itemize}
    \item<1-> Each component of the fingerprint is a sum of many
      indicators. For example, $f_3$ is the sum of the indicators of
      the events $h_i=3$.
    \item<2-> These indicators are only weakly dependent when the
      frequencies are low, so let's ignore these dependencies
      altogether.
    \item<3-> Assume $f_a$ is distributed Poisson (independently) with mean
      (and variance) $\lambda_a$.
    \item<4-> If for $p^+$ and $p^-$ and each $a$ we have that
      $|\lambda^-_a-\lambda^+_a|$ is smaller than $\sqrt{\lambda^+_a}$
      then we expect the distributions' fingerprints to be
      indistinguishable.
    \item<5-> If $\pi(p^+)>b$ and $\pi(p^-)<a$ then no tester can test
      $\pi_a^b$.
    \end{itemize}
  \end{block}
\end{frame}


\begin{frame}
  \frametitle{Wishful Thinking Theorem} \framesubtitle{Statement}

  \begin{block}{Wishful Thinking Theorem}
    Given an integer $k>0$, let $p^+$ and $p^-$ be two distributions,
    all of whose frequencies are at most $\frac{1}{500k}$. Let
    $\lambda^+$ and $\lambda^-$ be their Poisson moments for sample
    size $k$. If it is the case that
    \begin{equation*}
      \sum_a\frac{|\lambda^+_a-\lambda^-_a|}{\sqrt{1+\max\{\lambda^+_a,\lambda^-_a\}}}<\frac{1}{25}
    \end{equation*}
    then it is impossible to test any symmetric property that is true
    for $p^+$ and false for $p^-$ in $k$ samples.
  \end{block}

\end{frame}

\begin{frame}
  \frametitle{Wishful Thinking Theorem} \framesubtitle{Overview}

  \begin{itemize}
  \item<1-> Given two distributions $p^+$ and $p^-$, we approximate
    each of their fingerprint distributions $f_a$ by an independent
    Poisson distribution with parameter $\lambda_a$. I.e., $f$ is
    distributed similarly to the multivariate Poisson distribution
    $\mbox{Poi}(\lambda)$.
  \item<2-> We can then use a bound for the distance between multivariate
    Poisson distributions to show that the approximate distributions
    are close.
  \item<3-> We deduce a bound on the distance between the
    actual fingerprints' distributions.
  \item<4-> Finally, we conclude that since the fingerprints are
    indistinguishable (even though the distributions might not be),
    then the property can't be tested.
  \end{itemize}

\end{frame}

\begin{frame}
  \frametitle{Poissonization} \framesubtitle{}

  \begin{block}{Poissonization}
    A $k$-Poissonized tester $T$ is a function that correctly
    classifies a property on a distribution $p$ with probability
    $7/12$ on input samples generated in the following way:
    \begin{itemize}
    \item<2-> Draw $k'\leftarrow \mbox{poi}_k$.
    \item<3-> Return $k'$ samples from $p$.
    \end{itemize}
  \end{block}

  \begin{block}<4->{Lemma}
    If there exists a $k$-sample tester $T$ for a property $\pi_a^b$
    then there exists a $k$-Poissonized tester $T'$ for $\pi_a^b$.
  \end{block}
  
\end{frame}

\begin{frame}
  \frametitle{Poissonization} \framesubtitle{}

  
  \begin{itemize}
  \item<1-> After Poissonization, the histogram component $h_i$ is
    distributed $\mbox{poi}_{k_i}$, and the different $h_i$s are independent.
  \item<2-> Hence, by additivity of expectations (and variances), the
    expectation (and variance) of $f_a$ is $\sum_i\mbox{poi}_{k_i}(a)$
    which equals $\lambda_a$.

  \item<3-> However, the different $f_a$s aren't independent.
 \end{itemize}
  
\end{frame}


\begin{frame}
  \frametitle{Generalized Multinomial Distribution}
  \begin{block}{Definition: $M^\rho$, the generalized multinomial
      distribution($\rho$)}
  
    \begin{itemize}
    \item<1-> Let $\rho$ be a matrix with $n$ rows, such that row
      $\rho_i$ represents a distribution.
    \item<2-> From each such row, draw one column according to the
      distribution.
    \item<3-> Return a row vector recording the total number of
      samples falling into each column (the histogram of the samples).
    \end{itemize}
  \end{block}

  \begin{block}<4->{Lemma} The distribution of fingerprints of
    $\mbox{poi}(k)$ samples from $p$ (the distribution of $f$ after
    Poissonization) is the generalized multinomial distribution,
    $M^\rho$, when using $\rho_i(a)=\mbox{poi}_{k_i}(a)$ to define the
    rows $\rho_i$.
  \end{block}
\end{frame}

\begin{frame}
  \frametitle{Roos's Theorem}
  \begin{block}{Roos's theorem}
    
    Given a matrix $\rho$, letting $\lambda_a=\sum_i\rho_i(a)$ be the
    vector of column sums, we have
    \begin{equation*}
      |M^\rho-\mbox{Poi}(\lambda)|\le
      8.8\sum_a\frac{\sum_i\rho_i(a)^2}{\sum_i \rho_i(a)}.
    \end{equation*}
  \end{block}
  So, the multivariate Poisson distribution is a good approximation
  for the fingerprints, if $\rho$ is small enough.
\end{frame}
\begin{frame}
  \frametitle{Roos's Theorem}

  \begin{block}<1->{Bounding $\rho$ using the low-frequencies}
    Suppose that for some $0<\eps\le\frac{1}{2}$ it holds that $p_i\le
    \frac{\eps}{k}$. Then
    $\rho_i(a)=\mbox{poi}_{k_i}(a)=\frac{e^{-k_i}k_i^a}{a!}=\frac{e^{-k\cdot
        p_i}(k\cdot p_i)^a}{a!}\le (k\cdot p_i) ^a\le \eps^a$.
  \end{block}

  \onslide<2->{Thus:}
  \begin{block}<3->{}
    \begin{equation*}
      \sum_a\frac{\sum_i\rho_i(a)^2}{\sum_i \rho_i(a)}\le \sum_a\max_i \rho_i(a)\le \sum_a \eps^a\le 2\eps
    \end{equation*}
    and by Roos's theorem:
    \begin{equation*}
      |M^\rho-\mbox{Poi}(\lambda)|\le 2\cdot 8.8\eps.       
    \end{equation*}
  \end{block}
\end{frame}


\begin{frame}
  \frametitle{Multivariate Poisson Statistical Distance}
    \begin{block}<1->{Bounding the statistical distance between $\lambda^+$ and $\lambda^-$}
      The statistical distance between two multivariate Poisson
      distributions with parameters $\lambda^+, \lambda^-$ is bounded
      by
      \begin{equation*}
        |\mbox{Poi}(\lambda^+)-\mbox{Poi}(\lambda^-)|\le 2\sum_a\frac{|\lambda^+_a-\lambda^-_a|}{\sqrt{1+\max\{\lambda^+_a,\lambda_a^-\}}}.
      \end{equation*}
    \end{block}
    \onslide<2->{Hence, by the theorem's hypothesis:}
    \begin{block}<3->{}
      \begin{equation*}
        |\mbox{Poi}(\lambda^+)-\mbox{Poi}(\lambda^-)|\le \frac{2}{25}.
      \end{equation*}
    \end{block}
\end{frame}

\begin{frame}
  \frametitle{Wishful Thinking Theorem} \framesubtitle{(reminder)}

  \begin{block}{Wishful Thinking Theorem}
    Given an integer $k>0$, let $p^+$ and $p^-$ be two distributions,
    all of whose frequencies are at most $\frac{1}{500k}$. Let
    $\lambda^+$ and $\lambda^-$ be their Poisson moments for sample
    size $k$. If it is the case that
    \begin{equation*}
      \sum_a\frac{|\lambda^+_a-\lambda^-_a|}{\sqrt{1+\max\{\lambda^+_a,\lambda^-_a\}}}<\frac{1}{25}
    \end{equation*}
    then it is impossible to test any symmetric property that is true
    for $p^+$ and false for $p^-$ in $k$ samples.
  \end{block}

\end{frame}


\begin{frame}
  \frametitle{Wishful Thinking Theorem} \framesubtitle{Proof of
    Wishful Thinking Theorem}  
    
    \begin{proof}
    \begin{itemize}
    \item<1-> $f^\pm\sim M^{\rho^\pm}$.
    \item<2-> Combining Roos's theorem with the bound on $\rho$, and
      assuming that $p_i^\pm\le\frac{1}{500k}$, we get that $|M^{\rho^\pm}$ -
      $\mbox{Poi}(\lambda^\pm)|\le \frac{2\cdot 8.8}{500}<\frac{1}{25}$.
    \item<3-> The theorem's hypothesis implies
      $|\mbox{Poi}(\lambda^+)-\mbox{Poi}(\lambda^-)|\le\frac{2}{25}$.
    \item<4-> Using the triangle inequality, we get that the
      statistical distance between the distributions of fingerprints
      of $\mbox{Poi}(k)$ samples from $p^+$ versus $p^-$ is at most
      $\frac{4}{25}<\frac{1}{6}$.
    \item<5-> A $k$-tester (poissonized) must have a gap$>\frac{1}{6}$
      (succeed with probability $\frac{7}{12}$). This is impossible if
      $|p^+-p^-|<1/6$.
    \item<6-> If a $k$-Poissonized tester doesn't exist, then neither
      does a $k$-tester.
    \end{itemize}
    \onslide<7->{ $\Rightarrow $ it is impossible to test any
      symmetric property that is true for $p^+$ and false for $p^-$ in
      $k$ samples.}
    \end{proof}
\end{frame}

\begin{frame}
  Questions?
\end{frame}

\begin{frame}
  Thanks!
\end{frame}


\end{document} 
