\documentclass[11pt]{article} \usepackage{amssymb}
\usepackage{amsfonts} \usepackage{amsmath} \usepackage{bm}
\usepackage{latexsym} \usepackage{epsfig}

\setlength{\textwidth}{6.5 in} \setlength{\textheight}{8.25in}
\setlength{\oddsidemargin}{0in} \setlength{\topmargin}{0in}
\addtolength{\textheight}{.8in} \addtolength{\voffset}{-.5in}

\newtheorem{theorem}{Theorem}[section]
\newtheorem{lemma}[theorem]{Lemma}
\newtheorem{proposition}[theorem]{Proposition}
\newtheorem{corollary}[theorem]{Corollary}
\newtheorem{fact}[theorem]{Fact}
\newtheorem{definition}[theorem]{Definition}
\newtheorem{remark}[theorem]{Remark}
\newtheorem{conjecture}[theorem]{Conjecture}
\newtheorem{claim}[theorem]{Claim}
\newtheorem{example}[theorem]{Example}
\newenvironment{proof}{\noindent \textbf{Proof:}}{$\Box$}

\newcommand{\ignore}[1]{}

\newcommand{\enote}[1]{} \newcommand{\knote}[1]{}
\newcommand{\rnote}[1]{}



% \newcommand{\enote}[1]{{\bf [[Elchanan:} {\emph{#1}}{\bf ]]}}
% \newcommand{\knote}[1]{{\bf [[Krzysztof:} {\emph{#1}}{\bf ]]}}
% \newcommand{\rnote}[1]{{\bf [[Ryan:} {\emph{#1}}{\bf ]]}}



\DeclareMathOperator{\Support}{Supp} \DeclareMathOperator{\Opt}{Opt}
\DeclareMathOperator{\Ordo}{\mathcal{O}}
\newcommand{\MaxkCSP}{\textsc{Max $k$-CSP}}
\newcommand{\MaxkCSPq}{\textsc{Max $k$-CSP$_{q}$}}
\newcommand{\MaxCSP}[1]{\textsc{Max CSP}(#1)} \renewcommand{\Pr}{{\bf
    P}} \renewcommand{\P}{{\bf P}} \newcommand{\Px}{\mathop{\bf P\/}}
\newcommand{\E}{{\bf E}} \newcommand{\Cov}{{\bf Cov}}
\newcommand{\Var}{{\bf Var}} \newcommand{\Varx}{\mathop{\bf Var\/}}

\newcommand{\bits}{\{-1,1\}}

\newcommand{\nsmaja}{\textstyle{\frac{2}{\pi}} \arcsin \rho}

\newcommand{\Inf}{\mathrm{Inf}} \newcommand{\I}{\mathrm{I}}
\newcommand{\J}{\mathrm{J}}

\newcommand{\eps}{\epsilon} \newcommand{\lam}{\lambda}

% \newcommand{\trunc}{\ell_{2,[-1,1]}}
\newcommand{\trunc}{\zeta} \newcommand{\truncprod}{\chi}

\newcommand{\N}{\mathbb N} \newcommand{\R}{\mathbb R}
\newcommand{\Z}{\mathbb Z} \newcommand{\CalE}{{\mathcal{E}}}
\newcommand{\CalC}{{\mathcal{C}}} \newcommand{\CalM}{{\mathcal{M}}}
\newcommand{\CalR}{{\mathcal{R}}} \newcommand{\CalS}{{\mathcal{S}}}
\newcommand{\CalV}{{\mathcal{V}}}
\newcommand{\CalX}{{\boldsymbol{\mathcal{X}}}}
\newcommand{\CalG}{{\boldsymbol{\mathcal{G}}}}
\newcommand{\CalH}{{\boldsymbol{\mathcal{H}}}}
\newcommand{\CalY}{{\boldsymbol{\mathcal{Y}}}}
\newcommand{\CalZ}{{\boldsymbol{\mathcal{Z}}}}
\newcommand{\CalW}{{\boldsymbol{\mathcal{W}}}}
\newcommand{\CalF}{{\mathcal{Z}}}
% \newcommand{\boldG}{{\boldsymbol G}}
% \newcommand{\boldQ}{{\boldsymbol Q}}
% \newcommand{\boldP}{{\boldsymbol P}}
% \newcommand{\boldR}{{\boldsymbol R}}
% \newcommand{\boldS}{{\boldsymbol S}}
% \newcommand{\boldX}{{\boldsymbol X}}
% \newcommand{\boldB}{{\boldsymbol B}}
% \newcommand{\boldY}{{\boldsymbol Y}}
% \newcommand{\boldZ}{{\boldsymbol Z}}
% \newcommand{\boldV}{{\boldsymbol V}}
\newcommand{\boldi}{{\boldsymbol i}} \newcommand{\boldj}{{\boldsymbol
    j}} \newcommand{\boldk}{{\boldsymbol k}}
\newcommand{\boldr}{{\boldsymbol r}}
\newcommand{\boldsigma}{{\boldsymbol \sigma}}
\newcommand{\boldupsilon}{{\boldsymbol \upsilon}}
\newcommand{\hone}{{\boldsymbol{H1}}}
\newcommand{\htwo}{\boldsymbol{H2}}
\newcommand{\hthree}{\boldsymbol{H3}}
\newcommand{\hfour}{\boldsymbol{H4}}


\newcommand{\sgn}{\mathrm{sgn}} \newcommand{\Maj}{\mathrm{Maj}}
\newcommand{\Acyc}{\mathrm{Acyc}}
\newcommand{\UniqMax}{\mathrm{UniqMax}}
\newcommand{\Thr}{\mathrm{Thr}} \newcommand{\littlesum}{{\textstyle
    \sum}}

\newcommand{\half}{{\textstyle \frac12}}
\newcommand{\third}{{\textstyle \frac13}}
\newcommand{\fourth}{{\textstyle \frac14}}

\newcommand{\Stab}{\mathbb{S}}
\newcommand{\StabThr}[2]{\Gamma_{#1}(#2)}
\newcommand{\StabThrmin}[2]{{\underline{\Gamma}}_{#1}(#2)}
\newcommand{\StabThrmax}[2]{{\overline{\Gamma}}_{#1}(#2)}
\newcommand{\TestFcn}{\Psi}

\renewcommand{\phi}{\varphi}

\begin{document}
\title{Sublinear Algorithms - Exercise 6}

 \author{Omer Tamuz, 035696574}
\maketitle
\begin{enumerate}
\item

  \begin{lemma}
    \label{lemma:cut}
    There exists a universal constant $c$ such that for every $n>0$
    every planar graph with $n$ vertices and maximum degree $\leq d$
    has a $(cd/\sqrt{n},n/2)$-partition.
  \end{lemma}
  \begin{proof}
    By the planar separator theorem there exists a universal constant
    $c$ such that every planar graph with $n$ vertices has a set $S$
    of size at most $c\sqrt{n}$ such that each connected component of
    $G\setminus S$ is of size at most $n/2$.

    Consider the partition of $G$ into these connected components and
    $S$. Any cross-edge has to be adjacent to a vertex in $S$, and so
    there are at most $cd\sqrt{n}$ edges between the sets of the
    partition.
  \end{proof}

  \begin{lemma}
    Let $c$ be the universal constant from lemma~\ref{lemma:cut} and
    $\gamma = {\sqrt{2}\over \sqrt{2}-1}$.
    
    Every planar graph of size $n$ with maximum degree $\leq d$ admits
    an $(\eps, k^*(\eps, d, n))$-partition, for every $\eps > 0$, where
    \begin{equation*}
      k^*(\eps, d, n)=\left({\gamma cd \over \eps+\gamma {cd \over \sqrt{2n}}}\right)^2
    \end{equation*}
    when $n>c^2d^2/\eps^2$, and otherwise
    \begin{equation*}
      k^*(\eps, d, n)=c^2d^2/\eps^2.
    \end{equation*}
  \end{lemma}
  Before proving the lemma we note that $k^*(\eps-cd/\sqrt{n}, d,
  n/2)=k^*(\eps, d, n)$, and that $k^*$ increases monotonously with $n$.

  
  \begin{proof}
    Let $G=(V,E)$ and $n=|V|$. If $n \leq c^2d^2/\eps^2$ the statement is
    trivially true, since then $n \leq k^*(\eps, d, n)$. Otherwise,
    assume the statement is true for all graphs with at most $n-1$
    vertices. Partition $G$ using $S$, the guaranteed separator of
    size at most $c\sqrt{n}$, and let $P_1,\ldots P_l$ be the sets of
    this partition.

    Let
    \begin{equation*}
      \eps'=\eps-{cd \over \sqrt{n}}.
    \end{equation*}
    Note that we are considering the case where $\sqrt{n} > cd/\eps$,
    and therefore $\eps'>0$.
    
    By the inductive hypothesis, each $P_i$ has an $(\eps', k^*(\eps',
    d, |P_i|))$-partition. If we indeed partition each $P_i$ using
    this partition, then the number of cross-edges $|E_x|$ in the
    resulting partition of $G$ satisfies:
    \begin{align*}
      |E_x| &\leq cd\sqrt{n} + \sum_{i=1}^l\eps'|P_i|
      \\ &\leq cd\sqrt{n}+\eps\left(1-{cd \over \eps \sqrt{n}}\right)\sum_{i=1}^l|P_i|
      \\ &\leq cd\sqrt{n}+\eps\left(1-{cd  \over \eps \sqrt{n}}\right)n
      \\ &= \eps n
    \end{align*}
    Furthermore, we note that $k^*(\eps, d, n)$ is increasing with
    $n$, and that therefore $k^*(\eps', d, |P_i|) \leq k^*(\eps', d,
    n/2)$. Hence:
    \begin{align*}
      k^*(\eps', d, |P_i|) &\leq k^*(\eps', d, n/2)
      \\ &= k^*\left(\eps-{cd \over \sqrt{n}}, d, n/2\right)
      \\ &= k^*(\eps, d, n)
    \end{align*}
    where the last step follows directly from the definition of $k^*$.
    We have therefore shown that each set of the partition is of size
    at most $k^*(\eps, d, n)$.
  \end{proof}
  \begin{theorem}
    Every planar graph with maximum degree at most $d$ admits an
    $(\eps, (\gamma cd/\eps)^2)$-partition, for every $\eps>0$.
  \end{theorem}
  \begin{proof}
    This follows from the previous theorem and the fact that
    $k^*(\eps, d, n) \leq (\gamma cd/\eps)^2$ for every $\eps>0$ and $n$.
  \end{proof}

\item
  Every planar graph has average degree at most six. Hence for any
  $\eps > 0$, at most $\eps$ of the vertices have degree larger than
  $6 \over \eps$.
\end{enumerate}
\end{document}


