\documentclass[11pt]{article} \usepackage{amssymb}
\usepackage{amsfonts} \usepackage{amsmath} \usepackage{amsthm} \usepackage{bm}
\usepackage{latexsym} \usepackage{epsfig}

\setlength{\textwidth}{6.5 in} \setlength{\textheight}{8.25in}
\setlength{\oddsidemargin}{0in} \setlength{\topmargin}{0in}
\addtolength{\textheight}{.8in} \addtolength{\voffset}{-.5in}

\newtheorem*{theorem*}{Theorem}
\newtheorem{theorem}{Theorem}[section]
\newtheorem{lemma}[theorem]{Lemma}
\newtheorem*{proposition*}{Proposition}
\newtheorem*{claim*}{Claim}
\newtheorem*{definition*}{Definition}

\newtheorem{proposition}[theorem]{Proposition}
\newtheorem{claim}[theorem]{Claim}
\newtheorem{corollary}[theorem]{Corollary}
\newtheorem{fact}[theorem]{Fact}
\newtheorem{definition}[theorem]{Definition}
\newtheorem{remark}[theorem]{Remark}
\newtheorem{conjecture}[theorem]{Conjecture}
\newtheorem{example}[theorem]{Example}
%\newenvironment{proof}{\noindent \textbf{Proof:}}{$\Box$}

\newcommand{\ignore}[1]{}

\newcommand{\enote}[1]{} \newcommand{\knote}[1]{}
\newcommand{\rnote}[1]{}


\DeclareMathOperator{\Support}{Supp} \DeclareMathOperator{\Opt}{Opt}
\DeclareMathOperator{\Ordo}{\mathcal{O}}
\newcommand{\MaxkCSP}{\textsc{Max $k$-CSP}}
\newcommand{\MaxkCSPq}{\textsc{Max $k$-CSP$_{q}$}}
\newcommand{\MaxCSP}[1]{\textsc{Max CSP}(#1)} \renewcommand{\Pr}{{\bf
    P}} \renewcommand{\P}{{\bf P}} \newcommand{\Px}{\mathop{\bf P\/}}
\newcommand{\E}{{\bf E}} \newcommand{\Cov}{{\bf Cov}}
\newcommand{\Var}{{\bf Var}} \newcommand{\Varx}{\mathop{\bf Var\/}}

\newcommand{\bits}{\{-1,1\}}

\newcommand{\nsmaja}{\textstyle{\frac{2}{\pi}} \arcsin \rho}

\newcommand{\Inf}{\mathrm{Inf}} \newcommand{\I}{\mathrm{I}}
\newcommand{\J}{\mathrm{J}}

\newcommand{\eps}{\epsilon} \newcommand{\lam}{\lambda}

% \newcommand{\trunc}{\ell_{2,[-1,1]}}
\newcommand{\trunc}{\zeta} \newcommand{\truncprod}{\chi}

\newcommand{\N}{\mathbb N} \newcommand{\R}{\mathbb R}
\newcommand{\Z}{\mathbb Z} \newcommand{\CalE}{{\mathcal{E}}}
\newcommand{\CalC}{{\mathcal{C}}} \newcommand{\CalM}{{\mathcal{M}}}
\newcommand{\CalR}{{\mathcal{R}}} \newcommand{\CalS}{{\mathcal{S}}}
\newcommand{\CalV}{{\mathcal{V}}}
\newcommand{\CalX}{{\boldsymbol{\mathcal{X}}}}
\newcommand{\CalG}{{\boldsymbol{\mathcal{G}}}}
\newcommand{\CalH}{{\boldsymbol{\mathcal{H}}}}
\newcommand{\CalY}{{\boldsymbol{\mathcal{Y}}}}
\newcommand{\CalZ}{{\boldsymbol{\mathcal{Z}}}}
\newcommand{\CalW}{{\boldsymbol{\mathcal{W}}}}
\newcommand{\CalF}{{\mathcal{Z}}}
% \newcommand{\boldG}{{\boldsymbol G}}
% \newcommand{\boldQ}{{\boldsymbol Q}}
% \newcommand{\boldP}{{\boldsymbol P}}
% \newcommand{\boldR}{{\boldsymbol R}}
% \newcommand{\boldS}{{\boldsymbol S}}
% \newcommand{\boldX}{{\boldsymbol X}}
% \newcommand{\boldB}{{\boldsymbol B}}
% \newcommand{\boldY}{{\boldsymbol Y}}
% \newcommand{\boldZ}{{\boldsymbol Z}}
% \newcommand{\boldV}{{\boldsymbol V}}
\newcommand{\boldi}{{\boldsymbol i}} \newcommand{\boldj}{{\boldsymbol
    j}} \newcommand{\boldk}{{\boldsymbol k}}
\newcommand{\boldr}{{\boldsymbol r}}
\newcommand{\boldsigma}{{\boldsymbol \sigma}}
\newcommand{\boldupsilon}{{\boldsymbol \upsilon}}
\newcommand{\hone}{{\boldsymbol{H1}}}
\newcommand{\htwo}{\boldsymbol{H2}}
\newcommand{\hthree}{\boldsymbol{H3}}
\newcommand{\hfour}{\boldsymbol{H4}}


\newcommand{\sgn}{\mathrm{sgn}} \newcommand{\Maj}{\mathrm{Maj}}
\newcommand{\Acyc}{\mathrm{Acyc}}
\newcommand{\UniqMax}{\mathrm{UniqMax}}
\newcommand{\Thr}{\mathrm{Thr}} \newcommand{\littlesum}{{\textstyle
    \sum}}

\newcommand{\half}{{\textstyle \frac12}}
\newcommand{\third}{{\textstyle \frac13}}
\newcommand{\fourth}{{\textstyle \frac14}}
\newcommand{\fifth}{{\textstyle \frac15}}

\newcommand{\Stab}{\mathbb{S}}
\newcommand{\StabThr}[2]{\Gamma_{#1}(#2)}
\newcommand{\StabThrmin}[2]{{\underline{\Gamma}}_{#1}(#2)}
\newcommand{\StabThrmax}[2]{{\overline{\Gamma}}_{#1}(#2)}
\newcommand{\TestFcn}{\Psi}

\renewcommand{\phi}{\varphi}

\renewcommand{\theenumi}{\Alph{enumi}}
\renewcommand{\labelenumi}{\theenumi.}
\renewcommand{\theenumii}{\arabic{enumii}}
\renewcommand{\labelenumii}{\theenumii.}
\renewcommand{\theenumiii}{\alph{enumiii}}
\renewcommand{\labelenumiii}{\theenumiii)}

\begin{document}
\title{Game Theory Exam Solution}
\author{Omer Tamuz}
\date{\today}
\maketitle

\begin{enumerate}
\item
  \begin{enumerate}
  \item
    \begin{enumerate}
    \item Denote by $R$ the convex set defined in the question. Let
      $\vec{n}={\sqrt{3}\over 3} (1,1,1)$. Let $L$ be the plane defined by
      $L=\{\vec{x}\: : \: (\vec{x}-\vec{n}) \cdot \vec{n} = 0\}$.

      Note that since $(\vec{0}-\vec{n}) \cdot \vec{n} = -|\vec{n}|^2 = -1 <
      0$ then the origin is not in $L$.

      Consider $\vec{x}=(x_1, x_2, x_3) \in R$. Then
      \begin{align}
        (\vec{x}-\vec{n}) \cdot \vec{n} &= \vec{x} \cdot \vec{n} - 1
        \\ &= -1+{\sqrt{3}\over 3}\sum_{i=1}^3x_i
        \\ &\geq -1 + \sqrt{3}\left(\prod_{i=1}^3x_i\right)^{1/3} \label{eq:geo}
        \\ &\geq -1 + \sqrt{3} \label{eq:in-r}
        \\ &> 0,
      \end{align}
      where \eqref{eq:geo} stems from the fact that for positive
      numbers the geometric mean is at most as large at the arithmetic
      mean, and \eqref{eq:in-r} is a consequence of the fact
      that $x\in R$.

      Thus we have that $(\vec{0}-\vec{n}) \cdot \vec{n} < 0$ and
      $(\vec{x}-\vec{n}) \cdot \vec{n} > 0$ for all $x\in R$, and thus
      $L$ is a separating plane, as required.

    \item The statement is not true. Let $R=\{(x,0,0) : x \geq 0\}$ be
      the closed convex subset of $\R^3$ that contains the
      origin. Then clearly the origin is a boundary point of $R$, any
      plane that passes through the origin and does not include the
      $x$-axis is a supporting hyperplane, and there is more than one
      such plane.
    \end{enumerate}
  \item
    \begin{enumerate}
    \item
      \begin{theorem*}
        Let $G$ be a zero sum two player game. Then there exists a $V
        \in \R$, a mixed strategy $S_1$ for player 1 and a mixed
        strategy $S_2$ for player 2 such that if player 2 plays $S_2$
        then the highest possible payoff for player 1 is $V$ and if
        player 1 plays $S_1$ then the highest possible payoff for
        player 2 is $-V$.
      \end{theorem*}
      \begin{proof}        
        Let $S_1$ be a mixed strategy for player 1 and $S_2$ be a
        mixed strategy for player 2, such that $(S_1,S_2)$ is a Nash
        equilibrium point of the game. The existence of such a pair is
        guaranteed by the Nash equilibrium theorem.

        Let $V$ be player 1's expected payoff at this Nash. Then,
        since $G$ is a zero sum game then player 2's expected payoff
        is $-V$. Since $(S_1,S_2)$ is a Nash equilibrium then the
        expected payoff for player 1 when playing any other strategy
        is at most $V$, and the same holds for player 2 and $-V$. Thus
        $V$, $S_1$ and $S_2$ satisfy the requirements of the theorem.
      \end{proof}
    \item 
      \begin{theorem*}
        Let $G$ be a zero sum two player game with payoffs $a_{ij}$
        for strategies $1 \leq i \leq n$ and $1 \leq j \leq m$. Then
        \begin{align*}
          \max_{1 \leq i \leq n}\left\{\min_{1 \leq j \leq m}
              a_{ij}\right\}
          \leq \min_{1 \leq j \leq m}\left\{\max_{1 \leq i \leq n} a_{ij}\right\}
        \end{align*}
      \end{theorem*}
      \begin{proof}
        Denote $x_i= \min_{1 \leq j \leq m} a_{ij}$ and $y_j = \max_{1
          \leq i \leq n} a_{ij}$. For any $i$ and $j$ it holds that
        $x_i \leq a_{ij} \leq y_i$, by definition. Hence $\forall i,j$
        it holds that $x_i \leq y_j$, and in particular $\max_ix_i
        \leq min_jy_j$, which is an equivalent statement to that of
        the theorem.        
      \end{proof}
    \end{enumerate}
  \end{enumerate}
\item
  \begin{enumerate}
  \item
   {\bf Two person general sum game.}

      A two person general sum game includes two players. Each player
      $i \in \{1,2\}$ must choose a strategy $s_i \in S_i$, for some
      set of strategies $S_i$. When player 1 chooses $s_1$ and player
      2 chooses $s_2$ the payoff to player 1 is $A_{s_1,s_2}$ and to
      player 2 is $B_{s_1,s_2}$.  $A$ and $B$ have real entries but
      are not otherwise constrained. A mixed strategy $\sigma_i$ is a
      distribution over $S_i$. The payoff to player 1 for strategies
      $\sigma_1$ and $\sigma_2$ is the expected payoff when $s_i$ is
      picked from $\sigma_i$.

      There are a number of solution concepts for such a game. $s_1$
      is a {\em dominant strategy} for player 1 if for any strategy
      played by player 2 the maximum payoff is achieved by playing
      $s_1$:
      \begin{align*}        
        \forall t_1 \in S_1,s_2 \in S_2: A_{s_1,s_2} \geq A_{t_1,s_2}.
      \end{align*}
      If a player has a dominant strategy then it is hard to imagine
      why she would not use it. The prisoner's dilemma game (see
      below) has ``defect'' as a dominant strategy.

      A pure Nash equilibrium is a pair of pure (i.e., not mixed)
      strategies $(s_1,s_2)$ such that player 1 cannot get a better
      payoff by deviating from $s_1$, assuming the player 2 plays
      $s_2$, and vice versa.  The battle of the sexes game (also see
      below) has two such equilibria.

      A mixed Nash equilibrium is a pair of mixed strategies for which
      the same holds. Such an equilibrium is guaranteed to exist in
      games where the sets of strategies are finite. Every pure
      equilibrium is also a mixed equilibrium. The battle of the sexes
      game also has a mixed equilibrium that is not pure.

      A correlated equilibrium is a joint distribution $\mathcal{C}$
      over $S_1 \times S_2$. Such that if $\omega = (s_1,s_2)$ are
      drawn from $\mathcal{C}$ then the expected payoff for player 1,
      given that she is told to play $s_1$ is maximized by playing
      $s_1$. The battle of the sexes game has an additional
      interesting correlated equilibrium. See below.

      \vspace{0.1in}
      {\bf Bargaining problem.}

      A two player bargaining problem has a set of solutions $S \in
      \R^2$ such that if $(u,v)\in S$ is chosen then the payoff to
      player 1 is $u$ and the payoff to player 2 is $v$. $S$ is
      assumed to be convex, closed and comprehensive. The latter
      condition means that $S \cap \{(u,v):\: u \geq u^*, v \geq
      v^*\}$ is compact, where $u^*$ is player 1's ``reserve price''
      or the price she get by playing an maximin strategy.

      An example of such a set is the set of expected payoffs in the
      correlated equilibria of a two player game.

      The problem in a bargaining problem is to choose a point in the
      set. Nash's solution is to choose the maximizer in $S$ of
      $(u-u^*)\cdot(v-v^*)$. He showed that this is the unique
      solution that is invariant to affine transformations of the
      utilities, Pareto efficient, symmetric, and satisfies the
      ``independence of irrelevant alternatives'' (IIA) criterion.

      A different approach, which dispenses with the IIA criterion, is
      the Kalia-Smorodinsky solution. This solution chooses a point 
      $(u,v) \in S$ that maximizes player 1's (or player 2's) payoff while
      keeping the ratio $u/v$ equal to $\max_S u / \max_S v$. 

  \item
    \begin{enumerate}
    \item {\bf Prisoners' dilemma}

      Since the players in this game have a dominant strategy, this
      game has one obvious pure Nash equilibrium. It occurs when both
      players play the second strategy. Their payoff is then -5, and
      they can only decrease it to -10 be deviating.

      This game has no additional Nash or correlated equilibria.
      
      \vspace{0.1in}
      {\bf Battle of the sexes}

      This game has two pure equilibria and a mixed equilibrium. Two
      two pure equilibria occur when both players choose the same
      strategy: either both choose the first or both choose the
      second. Their payoffs are then either 10 or 3, and can only
      decrease to 0 by deviating.

      In a mixed equilibrium the expected payoff to each player is the
      same for playing all strategies in the support of her mixed
      strategy. We'll use this to find the mixed equilibrium for this
      game. Let the equilibrium strategies be such that the column
      player plays strategy 1 with probability $p$, and the row
      player plays strategy 1 with probability $q$. Then the expected
      payoff to the column player is $3q$ for playing strategy 1 and
      $10(1-q)$ for playing strategy 2. The solution of the equation
      $3q=10(1-q)$ is $q=10/13$. Likewise the row player's expected
      payoffs are $10p$ and $3(1-p)$, and so $p=3/13$. The players'
      payoffs are hence both equal to $30/13$.

      This game has a particularly interesting (see below) correlated
      equilibrium: simply toss a coin and have both players choose
      strategy 1 or both choose strategy 2, with equal
      probability. The payoffs are then $13/2$, for both players.

    \item {\bf Prisoners' dilemma}

      Since there is only one Nash in this game then the set of
      correlated equilibria is a singleton, with payoff $-5$ for both
      players.
      
      \vspace{0.1in}
      {\bf Battle of the sexes}

      By Papademitriou's theorem the set of correlated equilibria is
      the convex hull of the Nash equilibria. Hence the set of
      achievable utilities is the triangle in $R^2$ with vertices at
      $(10,3)$, $(3,10)$ and $(30/13, 30/13)$.

    \item {\bf Prisoners' dilemma}

      The single trivial solution is $(-5,-5)$.
      
      \vspace{0.1in}
      {\bf Battle of the sexes}

      In this case each player can guarantee an expected payoff of
      $30/13$ by playing the mixed Nash strategy. Hence
      $u^*=v^*=30/13$. Since Nash bargaining is Pareto efficient the
      solution must lie on the line segment connecting $(10,3)$ and
      $(3,10)$. Since Nash Bargaining is symmetric, and since the
      feasible set is symmetric, then the solution must lie in the
      middle of the line segment, and in fact be achieved by the
      correlated equilibrium we mentioned earlier, with payoffs
      $(13/2,13/2)$.

    \end{enumerate}
  \end{enumerate}
\item
  \begin{enumerate}
  \item 
    {\bf Definitions and statement of the Shapley-Bondareva Theorem.}
    \begin{definition*}
      Let $N$ be the set of players. A {\em balanced collection}
      $\mathcal{S} = \{S_1,\ldots,S_k\}$ is balanced if $\forall i:
      S_i \subseteq N$ and for there exist $\{\lambda_i\}_{i=1}^k$
      such that for every $1 \leq j \leq n$ it holds that $\sum_{S_i:j
        \in S_i}\lambda_i=1$.
    \end{definition*}
    \begin{definition*}
      The {\em core} of a cooperative game $v$ for a set of players
      $N$ is the set of payoff vectors $x \in R^{|N|}$ such that for
      all $S \subseteq N$ it holds that $v(S) \leq \sum_{i \in S}x_i$.
    \end{definition*}
    Denote by $C(v)$ the core of the game $v$.
    \begin{theorem*}[Shapley - Bondareva]
      $C(v) \neq \emptyset$ iff for every balanced collection
      $\{S_1,\ldots,S_k\}$ with balancing weights $\{\lambda_i\}$ it
      holds that $\sum_{i=1}^k\lambda_iv(S_i) \leq v(N)$.
    \end{theorem*}

    {\bf Proof of the first direction of the Shapley-Bondareva theorem.}
    \begin{theorem*}[Shapley - Bondareva]
      If $C(v) \neq \emptyset$ then for every balanced collection
      $\{S_1,\ldots,S_k\}$ with balancing weights $\{\lambda_i\}$ it
      holds that $\sum_{i=1}^k\lambda_iv(S_i) \leq v(N)$.
    \end{theorem*}
    \begin{proof}
      Let $x$ be in the core of $v$, and let $\{S_1,\ldots,S_k\}$ be a
      balanced collection with balancing weights $\{\lambda_i\}$. Then
      $\sum_{j \in S_i}x_j \geq v(S_i)$ and so $\lambda_i\sum_{j \in
        S_i}x_j \geq \lambda_iv(S_i)$. We now sum over $i$:
      \begin{align*}
        \sum_{i=1}^k\lambda_iv(S_i) &\leq \sum_{i=1}^k\lambda_i\sum_{j
          \in S_i}x_j
        \\ &= \sum_{i=1}^k\sum_{j \in S_i}\lambda_ix_j
        \\ &= \sum_{i=1}^k\sum_{j}{\bf 1}_{j\in S_i}\lambda_ix_j
        \\ &= \sum_{j}x_j\sum_{i=1}^k{\bf 1}_{j\in S_i}\lambda_i
        \\ &= \sum_{j}x_j\sum_{i: j \in S_i}^k\lambda_i
        \\ &= \sum_{j}x_j
        \\ &\leq v(N)
      \end{align*}
    \end{proof}
  \item We denote by $R$, $G$ and $B$ the Red, Green and Blue
      parties. 
    \begin{enumerate}
    \item 
      The characteristic function is:
      \begin{align*}
        v(\{R\}) = v(\{G\}) = v(\{B\}) = v(\{B, G\}) = 0
        \\ v(\{R, B\}) = v(\{R, G\}) = v(\{R, G, B\}) = 1
      \end{align*}

    \item
      A payoff vector in the core has to satisfy the following
      (in)equalities:
      \begin{align}
        x_R,x_G,x_B &\geq 0 \label{eq0}
        \\ x_R+x_G+x_B &= 1 \label{eq1}
        \\ x_R+x_G &\geq 1 \label{eq2}
        \\ x_R+x_B &\geq 1. \label{eq3}
      \end{align}
      Clearly \eqref{eq1} and \eqref{eq0} imply that \eqref{eq2} and
      \eqref{eq3} must also be equalities. Then, by subtracting
      \eqref{eq2} and \eqref{eq3} from \eqref{eq1} we have that the
      only solution is $x_R=1$, $x_G=x_B=0$.

    \item
      There are six possible permutations of the players. In each one
      exactly one player, when added in, changes the coalition's
      payoff from zero to one. Hence each party's Shapley value is
      $1/6$ times the number of permutations in which it first creates
      the $2/3$ majority. The following table lists the permutations
      and the respective party that creates the majority:
      \begin{table}
        \centering
        \begin{tabular}{l|l}
          Permutation & Party\\
          \hline
          RGB & G \\
          RBG & B \\
          GRB & R \\
          GBR & R \\
          BRG & R \\
          BGR & R
        \end{tabular}
      \end{table}
      Hence the Red Party's Shapley value is $2/3$ and the Green's and
      Blue's are both $1/6$.
    \end{enumerate}
    
  \end{enumerate}
\end{enumerate}

\end{document}


















