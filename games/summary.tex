\documentclass[11pt]{article} \usepackage{amssymb}
\usepackage{amsfonts} \usepackage{amsmath} \usepackage{amsthm} \usepackage{bm}
\usepackage{latexsym} \usepackage{epsfig}

\setlength{\textwidth}{6.5 in} \setlength{\textheight}{8.25in}
\setlength{\oddsidemargin}{0in} \setlength{\topmargin}{0in}
\addtolength{\textheight}{.8in} \addtolength{\voffset}{-.5in}

\newtheorem*{theorem*}{Theorem}
\newtheorem{theorem}{Theorem}[section]
\newtheorem{lemma}[theorem]{Lemma}
\newtheorem*{proposition*}{Proposition}
\newtheorem*{claim*}{Claim}
\newtheorem*{definition*}{Definition}

\newtheorem{proposition}[theorem]{Proposition}
\newtheorem{claim}[theorem]{Claim}
\newtheorem{corollary}[theorem]{Corollary}
\newtheorem{fact}[theorem]{Fact}
\newtheorem{definition}[theorem]{Definition}
\newtheorem{remark}[theorem]{Remark}
\newtheorem{conjecture}[theorem]{Conjecture}
\newtheorem{example}[theorem]{Example}
%\newenvironment{proof}{\noindent \textbf{Proof:}}{$\Box$}

\newcommand{\ignore}[1]{}

\newcommand{\enote}[1]{} \newcommand{\knote}[1]{}
\newcommand{\rnote}[1]{}


\DeclareMathOperator{\Support}{Supp} \DeclareMathOperator{\Opt}{Opt}
\DeclareMathOperator{\Ordo}{\mathcal{O}}
\newcommand{\MaxkCSP}{\textsc{Max $k$-CSP}}
\newcommand{\MaxkCSPq}{\textsc{Max $k$-CSP$_{q}$}}
\newcommand{\MaxCSP}[1]{\textsc{Max CSP}(#1)} \renewcommand{\Pr}{{\bf
    P}} \renewcommand{\P}{{\bf P}} \newcommand{\Px}{\mathop{\bf P\/}}
\newcommand{\E}{{\bf E}} \newcommand{\Cov}{{\bf Cov}}
\newcommand{\Var}{{\bf Var}} \newcommand{\Varx}{\mathop{\bf Var\/}}

\newcommand{\bits}{\{-1,1\}}

\newcommand{\nsmaja}{\textstyle{\frac{2}{\pi}} \arcsin \rho}

\newcommand{\Inf}{\mathrm{Inf}} \newcommand{\I}{\mathrm{I}}
\newcommand{\J}{\mathrm{J}}

\newcommand{\eps}{\epsilon} \newcommand{\lam}{\lambda}

% \newcommand{\trunc}{\ell_{2,[-1,1]}}
\newcommand{\trunc}{\zeta} \newcommand{\truncprod}{\chi}

\newcommand{\N}{\mathbb N} \newcommand{\R}{\mathbb R}
\newcommand{\Z}{\mathbb Z} \newcommand{\CalE}{{\mathcal{E}}}
\newcommand{\CalC}{{\mathcal{C}}} \newcommand{\CalM}{{\mathcal{M}}}
\newcommand{\CalR}{{\mathcal{R}}} \newcommand{\CalS}{{\mathcal{S}}}
\newcommand{\CalV}{{\mathcal{V}}}
\newcommand{\CalX}{{\boldsymbol{\mathcal{X}}}}
\newcommand{\CalG}{{\boldsymbol{\mathcal{G}}}}
\newcommand{\CalH}{{\boldsymbol{\mathcal{H}}}}
\newcommand{\CalY}{{\boldsymbol{\mathcal{Y}}}}
\newcommand{\CalZ}{{\boldsymbol{\mathcal{Z}}}}
\newcommand{\CalW}{{\boldsymbol{\mathcal{W}}}}
\newcommand{\CalF}{{\mathcal{Z}}}
% \newcommand{\boldG}{{\boldsymbol G}}
% \newcommand{\boldQ}{{\boldsymbol Q}}
% \newcommand{\boldP}{{\boldsymbol P}}
% \newcommand{\boldR}{{\boldsymbol R}}
% \newcommand{\boldS}{{\boldsymbol S}}
% \newcommand{\boldX}{{\boldsymbol X}}
% \newcommand{\boldB}{{\boldsymbol B}}
% \newcommand{\boldY}{{\boldsymbol Y}}
% \newcommand{\boldZ}{{\boldsymbol Z}}
% \newcommand{\boldV}{{\boldsymbol V}}
\newcommand{\boldi}{{\boldsymbol i}} \newcommand{\boldj}{{\boldsymbol
    j}} \newcommand{\boldk}{{\boldsymbol k}}
\newcommand{\boldr}{{\boldsymbol r}}
\newcommand{\boldsigma}{{\boldsymbol \sigma}}
\newcommand{\boldupsilon}{{\boldsymbol \upsilon}}
\newcommand{\hone}{{\boldsymbol{H1}}}
\newcommand{\htwo}{\boldsymbol{H2}}
\newcommand{\hthree}{\boldsymbol{H3}}
\newcommand{\hfour}{\boldsymbol{H4}}


\newcommand{\sgn}{\mathrm{sgn}} \newcommand{\Maj}{\mathrm{Maj}}
\newcommand{\Acyc}{\mathrm{Acyc}}
\newcommand{\UniqMax}{\mathrm{UniqMax}}
\newcommand{\Thr}{\mathrm{Thr}} \newcommand{\littlesum}{{\textstyle
    \sum}}

\newcommand{\half}{{\textstyle \frac12}}
\newcommand{\third}{{\textstyle \frac13}}
\newcommand{\fourth}{{\textstyle \frac14}}
\newcommand{\fifth}{{\textstyle \frac15}}

\newcommand{\Stab}{\mathbb{S}}
\newcommand{\StabThr}[2]{\Gamma_{#1}(#2)}
\newcommand{\StabThrmin}[2]{{\underline{\Gamma}}_{#1}(#2)}
\newcommand{\StabThrmax}[2]{{\overline{\Gamma}}_{#1}(#2)}
\newcommand{\TestFcn}{\Psi}
\newcommand{\interior}{\mbox{int}}

\renewcommand{\phi}{\varphi}

\renewcommand{\theenumi}{\Alph{enumi}}
\renewcommand{\labelenumi}{\theenumi.}
\renewcommand{\theenumii}{\arabic{enumii}}
\renewcommand{\labelenumii}{\theenumii.}
\renewcommand{\theenumiii}{\alph{enumiii}}
\renewcommand{\labelenumiii}{\theenumiii)}

\begin{document}

\title{Summary of Pages 372-378 of Y. Kannai's ``The Core and
  Balancedness''}

\author{Omer Tamuz}
\date{\today}

\maketitle

Let $N=\{1,\ldots,n\}$ be the set of players. A coalition in a subset
of $N$. We denote by $x=(x_1,\ldots,x_n) \in \R^n$ the payoff vector
to the players.

A non-transferable $n$ person non-transferable utility game is defined
by a function $V:2^N \to 2^{\R^N}$ (from coalitions to sets in $\R^n$)
that has the following properties:
\begin{itemize}
\item $V(\emptyset) = \emptyset$.
\item $V(S)$ is non empty and closed for all $S$.
\item Let $y_i \leq x_i$ for all $i$. Then $x \in V(S)$ implies $y \in
  V(S)$, for any $S$.
\end{itemize}
The third condition implies that if a coalition $S$ can achieve a certain
payoff vector $x$ then it can also achieve any payoff vector $y$ that
is strictly smaller than it, for every member of $S$.

Given a transferable utility game $v$ we can translate it into a
transferable utility game $V$ by setting $V(S)=\{x \in \R^n \mbox{
  s.t. } \sum_{i \in S}x_i \leq v(S)\}$. We shall in fact assume that
$V$ is always constructed in a similar way, so that there exists a
closed and compact set $F$ such that
\begin{equation}
  \label{eq:F}
  V(N)=\{x \in \R^n \mbox{
    s.t. } (\exists y \in F)(\forall i):\:x_i \leq y_i\}.  
\end{equation}
We shall further make the simplifying assumption that
\begin{align}
  \label{eq:zero}
  V(\{i\}) = \{x \in \R^n :\: x_i \leq 0\}.
\end{align}

We call a payoff vector $x$ {\em individually rational} if for all $i$
and $y \in V(\{i\})$ it holds that $x_i \geq y_i$; a player will not
join a coalition which gives her a lower payoff than she a achieve by
herself. Clearly individually rational vectors exist iff there exists
an $x \in F$ such that $x_i \geq 0$ for all $i$. We will assume
henceforth that this is indeed the case.

If $x$ is in the interior of $V(S)$ then $S$ can improve upon it, in
the sense that there exists an $y \in V(S)$ such that $x_i < y_i$ for
all $i$. Hence $V(N) \setminus \cap_{S \subseteq N} \mbox{int} V(S)$ is the
set of feasible payoff vectors that cannot be improved upon by any
coalition. This set is referred to as the {\em core} of the game.

In analogy to the case of transferable utilities we call a
non-transferable utility game $V$ {\em balanced} if $\cap_{i=1}^k(S_i)
\subseteq V(N)$ for every balanced collection $(S_1,\ldots,S_k)$ of
subsets of $N$. This definition is due to
Scarf~\cite{scarf:1967}. Note that if $v$ is balanced then the
corresponding $V$ (as defined above) is also balanced.

We devote the rest of this summary to prove the following theorem,
following Shapely~\cite{shapley:1973} and Kannai~\cite{kannai:1970}:

\begin{theorem}
  \label{thm:0}
  Every balanced game has a non-empty core.
\end{theorem}

To prove this theorem we introduce two lemmas, generalizing Sperner's
lemma and the Knaster, Kuratowski and Mazurkiewicz theorem.

Denote the unit vectors of $R^n$ by $\{e^i\}_{i=1}^n$. For $S
\subseteq N$, $S \neq \emptyset$ let $A^S$ equal the complex hull of
$\{e^i\}_{i \in S}$. Let $m_S = \frac{1}{|S|}\sum_{i \in S}e^i$ be the
barycenter of $A^S$. Let $V(\Sigma)$ denote the set of vertices of a
simplicial subdivision $\Sigma$ of $A^N$. A {\em labelling} of
$\Sigma$ is a function from $V(\Sigma)$ to the non-empty subsets of
$N$.

Our first lemma is the following:
\begin{theorem}
  \label{thm:sperner}
  Let $\Sigma$ be a simplicial subdivision of $N$, and let $f$ be a
  labelling of $V(\Sigma)$. If for every non-empty subset $S$ of $N$
  and $q$ it holds that $f(q) \subseteq S$ whenever $q \in V(\Sigma)
  \cap A^S$, then there exists a simplex $\sigma$ in $\Sigma$, of some
  dimension, such that $\{f(q):\: q \in \sigma\}$ is {\em balanced}.
\end{theorem}

\begin{proof}
  Let $\tilde{f}:A^N \to A^N$ be a linear extension of the function
  that maps each $q \in V(\Sigma)$ to $\frac{1}{|f(q)|}\sum_{i \in
    f(q)}e^i$.
  
  Let $B=\cup_{S \subset N}A^S$ be the relative boundary of $A^N$. Let
  $g:B \to B$ be the antipodal map with $m_N=(1/n,\ldots,1/n)$ as the
  center. It is easy to convince oneself that for any $x \in S$
  (expect $S=N$) it holds that $g(x) \not \in A^S$.

  Let $h : A^N \setminus \{m_n\} \to B$ be the usual radial
  deformation of $A^N$ (expect its center) to its boundary $B$. Note
  that $h$ is continuous and that for every $x \in B$ it holds that
  $h(x) = x$.

  Let $\phi : A^N \to B$ be defined by $\phi = g \circ h \circ
  \tilde{f}$. We claim that $\phi$ does not have a fixed point. Assume
  the contrary, so that there exists $x\in B$ such that
  $\phi(x)=x$. Then, since $B=\cup_{S \subset N}A^S$ then there must
  exist an $S \neq N$ such that $x \in A^S$. There must then also
  exist a simplex $\sigma = \{q_1, \ldots, q_{|S|}\} \in \Sigma$ such
  that all $q_i$'s are in $A^S$ and $x$ is in the convex hull of
  $\sigma$. By the theorem hypothesis $f(q_i) \in S$ for all $i$, and
  so $\tilde{f}(q_i) \in A^S$, and furthermore $\tilde{f}(x) \in A^S
  \subset B$. Since $h(x) = x$ for $x \in B$ then $\phi(x) =
  g(h(\tilde{f}(x)))= g(\tilde{f}(x))$. But by the definition of $g$,
  and since $\tilde{f}(x) \in A^S$ we have that $g(\tilde{f}(x)) \not
  \in A^S$, and so it cannot be that $\phi(x)=x$.

  We have shown that $\phi$ has no fixed point. By the Brouwer fixed
  point theorem this can only be if $\phi$ is not continuous or if
  $\phi$ is not well defined. Since $\phi$ is a composition of
  continuous functions it must then be the case that it is not well
  defined. This can happen only if $m_N \in \tilde{f}(A^N)$, since
  $h(m_N)$ is not defined. Hence we conclude that $m_N \in
  \tilde{f}(A^N)$.

  Let then $x$ be such that $\tilde{f}(x) = m_N$, or equivalently
  $\tilde{f}(x)_i=1/n$. Then there must exist a simplex $\sigma =
  \{q_1, \ldots, q_k\} \in \Sigma$ that contains $x$.  Let $(\alpha_1,
  \ldots, \alpha_k)$ be the coefficient of $x$'s representation as a
  convex combination of $\{q_1,\ldots,q_k\}$, so that
  $x=\sum_{i=1}^k\alpha_iq_i$. We claim that $\lambda_j =
  n\alpha_j/|f(q_j)|$ are balancing weights of the collection
  $\{f(q_j)\}_{j=1}^k$:
  \begin{align*}
    \sum_{j :\: i \in S_j}\lambda_j &=
    \sum_{j :\: i \in S_j}\frac{n\alpha_j}{|f(q_j)|}
    \\ &= n\sum_{j=1}^k\alpha_j\frac{1}{|f(q_j)|}{\bf 1}_{i \in f(q_j)}
    \\ &= n\sum_{j=1}^k\alpha_j\frac{1}{|f(q_j)|}\sum_{l \in f(q_j)}e^l_i
    \\ &= n\sum_{j=1}^k\alpha_jf(q_j)_i
    \\ &= n\tilde{f}(x)_i
    \\ &= 1.
  \end{align*}
\end{proof}

We now prove a second lemma:
\begin{theorem}
  \label{thm:KKMS}
  For every non-empty $S \subseteq N$ let $C(S)$ be a closed subset of
  $A^N$ such that $A^T \subseteq \cup_{S \subseteq T}C(S)$ for every $T$.
  Then there exists a balanced collection $\{S_1, \ldots S_k\}$ such
  that $\cap_{i=1}^kC(S_i) \neq \emptyset$.
\end{theorem}
\begin{proof}
  Let $\{\Sigma^{(m)}\}{m=1}^\infty$ be a sequence of simplicial
  partitions of $A^N$, such that the limit, as $m$ tends to infinity,
  of the maximum diameter of a simplex in $\Sigma^{(m)}$ is
  zero. Given $q \in T(\Sigma^{(m)})$ let $T(q)$ be the minimal
  coalition such that $q \in A^{T(q)}$. By the theorem hypothesis $q
  \in C(S)$ for some $S \subseteq T(q)$. Let $f^{(m)}:V(\Sigma^{(m)}
  \to 2^N$ assign to each $q$ such a set $S$. Note that $f^{(m)}$ is
  in fact a labelling that satisfies the conditions of
  theorem~\ref{thm:sperner}, and so by that theorem for every $m \in
  N$ and $q \in V(\Sigma^{(m)}$ there exists a simplex $\sigma^{(m)}$
  such that the collection $\mathcal{S}^{(m)}$ of the sets
  $f^{(m)}(q)$ for $q \in \sigma^{(m)}$ is balanced. Since both $A^N$
  and the set of balanced collections are compact then there must
  exist a subsequence $\{\Sigma^{(m_i)}\}$ that $\sigma^{(m_i)}$
  converges to some $q_0 \in A^N$ and $S^{(m_i)}$ converges to some
  balanced collection $\mathcal{S}_0=(S_1,\ldots,S_k)$. Since, for all
  $j$, $f^{(m)}(q_j^{(m_i)}) \in C(S_j)$ then $q_0$, which is the
  limit of $q_j^{(m_i)}$, is also in $C(S_j)$ for all $j$, since the
  $C(S_j)$'s are closed. Hence $q_0 \in \cap_{i=j}^kC(S_j)$, and the
  set is not empty.
\end{proof}

We are now ready to prove our main theorem~\ref{thm:0}:

\begin{proof}
  By our assumption~\eqref{eq:F} there exists a large constant $M$
  such that for all $x \in V(N)$ it holds for all $i \in N$ that $x_i
  \leq M$. Assume for now that this also holds for all other $S \neq
  N$, i.e. that $x_i < M$ for all $i \in S$.

  For $x \in \R^n$, let $t(x) = \max\{t: x+t(1,\ldots, 1) \in \cup_S
  V(S)\}$. Since $V(S)$ is closed for all $S$ and by our assumption
  bounded, then this maximum always exists, and $t(x)$ is in fact a
  continuous function. Let $y(x) = x+t(x)(1, \ldots, 1)$. Clearly $y$
  is also continuous.

  Define $\tilde{e}^i = -nMe^i$, and let $\tilde{A}^S$ be the convex
  hull of the corresponding $\tilde{e}^i$'s, in analogy to the
  definition of $A^S$.  Let $C_S = \{x \in \tilde{A}^N:\: y(x) \in
  V(S)\}$. By the continuity of $y$ and the fact that $V(S)$ is closed
  we have that $C_S$ is also closed. We would like to show that the
  $C_S$'s and $\tilde{A}^T$'s satisfy the conditions of
  theorem~\ref{thm:KKMS}. We will do this by showing that if there
  exists an $x \in C_S \cap \tilde{A}^T$ then $S \subseteq T$. For
  $T=N$ the statement is trivial. Hence we assume henceforth that $T
  \neq N$.

  Since $x \in \tilde{A}^T$ then it is in the convex combination of
  $-nMe^i$, and so for some coordinate $j \in T$ we have that $x_j <
  -M$, and so $t(x) > M$, since $x+M(1,\ldots,1) \in V(\{j\})$. Then,
  by our assumption we have that $x_i <0$ for all $i \in S$, and since
  $x \in \tilde{A}^T$ then $x_i \neq 0$ implies $i \in T$. Hence it
  must be the case that $S \subseteq T$.

  Let $\{S_1, \ldots, S_k\}$ be a balanced collection, and let $x \in
  \tilde{A}^N$ be a point in each of the $C_{S_i}$'s. The existence of
  these is guaranteed by theorem~\ref{thm:KKMS}. Therefore $y(x)$ is
  in each of the $V(S_i)$'s but not to any of the interior of any
  $V(S)$.  Therefore, since the game is balanced, $y$ must be in
  $V(N)$ and hence in the core. We have thus shown that the core is
  not empty under our assumption that $x_i < M$ for every $x \in V(S)$
  and $i \in S$.

  To drop this assumption we recall that it does indeed hold for $S=V$
  for some $M$. Define a ``cropped'' game $V_M$ by $V_M(S)=V(S) \cap
  \{x :\: x_i \leq M \mbox{ for } i \in S\}$. For this game our
  assumption holds, and so let $y^{(M)}$ be a member of $V_M$'s
  core. Since $y$ is in $V_M(N)$ it is also in $V(N)$, and by our
  assumption~\ref{eq:zero} we have that $y^{(M)}_i \geq 0$. Hence
  there exists a sequence of $M$'s such that if we let $M$ tend to
  infinity along this sequence then $y^{(M)}$ will tend to some
  $\bar{y} \in V(N)$. $\bar{y}$ must be in the core of $V$, since
  otherwise for large enough $M$ there will exist some point $z \in
  V(S)$ that will dominate $\bar{y}$ for some coalition $S$. But $z$
  must be in some $V_{M_0}(S)$, and it cannot dominate $y^{(M_0)}$.
  
\end{proof}


\bibliographystyle{abbrv} \bibliography{summary}
\end{document}


















