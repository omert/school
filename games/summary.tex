\documentclass[11pt]{article} \usepackage{amssymb}
\usepackage{amsfonts} \usepackage{amsmath} \usepackage{amsthm} \usepackage{bm}
\usepackage{latexsym} \usepackage{epsfig}

\setlength{\textwidth}{6.5 in} \setlength{\textheight}{8.25in}
\setlength{\oddsidemargin}{0in} \setlength{\topmargin}{0in}
\addtolength{\textheight}{.8in} \addtolength{\voffset}{-.5in}

\newtheorem*{theorem*}{Theorem}
\newtheorem{theorem}{Theorem}[section]
\newtheorem{lemma}[theorem]{Lemma}
\newtheorem*{proposition*}{Proposition}
\newtheorem*{claim*}{Claim}
\newtheorem*{definition*}{Definition}

\newtheorem{proposition}[theorem]{Proposition}
\newtheorem{claim}[theorem]{Claim}
\newtheorem{corollary}[theorem]{Corollary}
\newtheorem{fact}[theorem]{Fact}
\newtheorem{definition}[theorem]{Definition}
\newtheorem{remark}[theorem]{Remark}
\newtheorem{conjecture}[theorem]{Conjecture}
\newtheorem{example}[theorem]{Example}
%\newenvironment{proof}{\noindent \textbf{Proof:}}{$\Box$}

\newcommand{\ignore}[1]{}

\newcommand{\enote}[1]{} \newcommand{\knote}[1]{}
\newcommand{\rnote}[1]{}


\DeclareMathOperator{\Support}{Supp} \DeclareMathOperator{\Opt}{Opt}
\DeclareMathOperator{\Ordo}{\mathcal{O}}
\newcommand{\MaxkCSP}{\textsc{Max $k$-CSP}}
\newcommand{\MaxkCSPq}{\textsc{Max $k$-CSP$_{q}$}}
\newcommand{\MaxCSP}[1]{\textsc{Max CSP}(#1)} \renewcommand{\Pr}{{\bf
    P}} \renewcommand{\P}{{\bf P}} \newcommand{\Px}{\mathop{\bf P\/}}
\newcommand{\E}{{\bf E}} \newcommand{\Cov}{{\bf Cov}}
\newcommand{\Var}{{\bf Var}} \newcommand{\Varx}{\mathop{\bf Var\/}}

\newcommand{\bits}{\{-1,1\}}

\newcommand{\nsmaja}{\textstyle{\frac{2}{\pi}} \arcsin \rho}

\newcommand{\Inf}{\mathrm{Inf}} \newcommand{\I}{\mathrm{I}}
\newcommand{\J}{\mathrm{J}}

\newcommand{\eps}{\epsilon} \newcommand{\lam}{\lambda}

% \newcommand{\trunc}{\ell_{2,[-1,1]}}
\newcommand{\trunc}{\zeta} \newcommand{\truncprod}{\chi}

\newcommand{\N}{\mathbb N} \newcommand{\R}{\mathbb R}
\newcommand{\Z}{\mathbb Z} \newcommand{\CalE}{{\mathcal{E}}}
\newcommand{\CalC}{{\mathcal{C}}} \newcommand{\CalM}{{\mathcal{M}}}
\newcommand{\CalR}{{\mathcal{R}}} \newcommand{\CalS}{{\mathcal{S}}}
\newcommand{\CalV}{{\mathcal{V}}}
\newcommand{\CalX}{{\boldsymbol{\mathcal{X}}}}
\newcommand{\CalG}{{\boldsymbol{\mathcal{G}}}}
\newcommand{\CalH}{{\boldsymbol{\mathcal{H}}}}
\newcommand{\CalY}{{\boldsymbol{\mathcal{Y}}}}
\newcommand{\CalZ}{{\boldsymbol{\mathcal{Z}}}}
\newcommand{\CalW}{{\boldsymbol{\mathcal{W}}}}
\newcommand{\CalF}{{\mathcal{Z}}}
% \newcommand{\boldG}{{\boldsymbol G}}
% \newcommand{\boldQ}{{\boldsymbol Q}}
% \newcommand{\boldP}{{\boldsymbol P}}
% \newcommand{\boldR}{{\boldsymbol R}}
% \newcommand{\boldS}{{\boldsymbol S}}
% \newcommand{\boldX}{{\boldsymbol X}}
% \newcommand{\boldB}{{\boldsymbol B}}
% \newcommand{\boldY}{{\boldsymbol Y}}
% \newcommand{\boldZ}{{\boldsymbol Z}}
% \newcommand{\boldV}{{\boldsymbol V}}
\newcommand{\boldi}{{\boldsymbol i}} \newcommand{\boldj}{{\boldsymbol
    j}} \newcommand{\boldk}{{\boldsymbol k}}
\newcommand{\boldr}{{\boldsymbol r}}
\newcommand{\boldsigma}{{\boldsymbol \sigma}}
\newcommand{\boldupsilon}{{\boldsymbol \upsilon}}
\newcommand{\hone}{{\boldsymbol{H1}}}
\newcommand{\htwo}{\boldsymbol{H2}}
\newcommand{\hthree}{\boldsymbol{H3}}
\newcommand{\hfour}{\boldsymbol{H4}}


\newcommand{\sgn}{\mathrm{sgn}} \newcommand{\Maj}{\mathrm{Maj}}
\newcommand{\Acyc}{\mathrm{Acyc}}
\newcommand{\UniqMax}{\mathrm{UniqMax}}
\newcommand{\Thr}{\mathrm{Thr}} \newcommand{\littlesum}{{\textstyle
    \sum}}

\newcommand{\half}{{\textstyle \frac12}}
\newcommand{\third}{{\textstyle \frac13}}
\newcommand{\fourth}{{\textstyle \frac14}}
\newcommand{\fifth}{{\textstyle \frac15}}

\newcommand{\Stab}{\mathbb{S}}
\newcommand{\StabThr}[2]{\Gamma_{#1}(#2)}
\newcommand{\StabThrmin}[2]{{\underline{\Gamma}}_{#1}(#2)}
\newcommand{\StabThrmax}[2]{{\overline{\Gamma}}_{#1}(#2)}
\newcommand{\TestFcn}{\Psi}
\newcommand{\interior}{\mbox{int}}

\renewcommand{\phi}{\varphi}

\renewcommand{\theenumi}{\Alph{enumi}}
\renewcommand{\labelenumi}{\theenumi.}
\renewcommand{\theenumii}{\arabic{enumii}}
\renewcommand{\labelenumii}{\theenumii.}
\renewcommand{\theenumiii}{\alph{enumiii}}
\renewcommand{\labelenumiii}{\theenumiii)}

\begin{document}

\title{Summary of pp.372-378 of Y. Kannai's ``The Core and
  Balancedness``}

\author{Omer Tamuz}
\date{\today}

\maketitle

Let $N=\{1,\ldots,n\}$ be the set of players. A coalition in a subset
of $N$. We denote by $x=(x_1,\ldots,x_n) \in \R^n$ the payoff vector
to the players.

A non-transferable $n$ person non-transferable utility game is defined
by a function $V:2^N \to 2^{\R^N}$ (from coalitions to sets in $\R^n$)
that has the following properties:
\begin{itemize}
\item $V(\emptyset) = \emptyset$.
\item $V(S)$ is non empty and closed for all $S$.
\item Let $y_i \leq x_i$ for all $i$. Then $x \in V(S)$ implies $y \in
  V(S)$, for any $S$.
\end{itemize}
The third condition implies that if a coalition $S$ can achieve a certain
payoff vector $x$ then it can also achieve any payoff vector $y$ that
is strictly smaller than it, for every member of $S$.

Given a transferable utility game $v$ we can translate it into a
transferable utility game $V$ by setting $V(S)=\{x \in \R^n \mbox{
  s.t. } \sum_{i \in S}x_i \leq v(S)\}$. We shall in fact assume that
$V$ is always constructed in a similar way, so that there exists a
closed and compact set $F$ such that $V(S)=\{x \in \R^n \mbox{
  s.t. } (\exists y \in F)(\forall i):\:x_i \leq y_i\}$. We shall
further make the simplifying assumption that $V(\{i\}) = \{x \in \R^n
:\: x_i \leq 0\}$.

We call a payoff vector $x$ {\em individually rational} if for all $i$
and $y \in V(\{i\})$ it holds that $x_i \geq y_i$; a player will not
join a coalition which gives her a lower payoff than she a achieve by
herself. Clearly individually rational vectors exist iff there exists
an $x \in F$ such that $x_i \geq 0$ for all $i$. We will assume
henceforth that this is indeed the case.

If $x$ is in the interior of $V(S)$ then $S$ can improve upon it, in
the sense that there exists an $y \in V(S)$ such that $x_i < y_i$ for
all $i$. Hence $V(N) \setminus \cap{S \subseteq N} int V(S)$ is the
set of feasible payoff vectors that cannot be improved upon by any
coalition. This set is referred to as the {\em core} of the game.

In analogy to the case of transferable utilities we call a
non-transferable utility game $V$ {\em balanced} if $\cap_{i=1}^k(S_i)
\subseteq V(N)$ for every balanced collection $(S_1,\ldots,S_k)$ of
subsets of $N$. This definition is due to
Scarf~\cite{scarf:1967}. Note that if $v$ is balanced then the
corresponding $V$ (as defined above) is also balanced.

We devote the rest of this summary to prove the following theorem,
following Shapely~\cite{shapley:1973} and Kannai~\cite{kannai:1970}:

\begin{theorem}
  Every balanced game has a non-empty core.
\end{theorem}

To prove this theorem we introduce two lemmas, generalizing Sperner's
lemma and the Knaster, Kuratowski and Mazurkiewicz theorem.

Denote the unit vectors of $R^n$ by $\{e^i\}_{i=1}^n$. For $S
\subseteq N$, $S \neq \emptyset$ let $A^S$ equal the complex hull of
$\{e^i\}_{i \in S}$. Let $m_S = \frac{1}{|S|}\sum_{i \in S}e^i$ be the
barycenter of $A^S$. 
 


\bibliographystyle{abbrv} \bibliography{summary}
\end{document}


















