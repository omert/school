\documentclass[11pt]{article} \usepackage{amssymb}
\usepackage{amsfonts} \usepackage{amsmath} \usepackage{bm}
\usepackage{latexsym} \usepackage{epsfig}

\setlength{\textwidth}{6.5 in} \setlength{\textheight}{8.25in}
\setlength{\oddsidemargin}{0in} \setlength{\topmargin}{0in}
\addtolength{\textheight}{.8in} \addtolength{\voffset}{-.5in}

\newtheorem{theorem}{Theorem}[section]
\newtheorem{lemma}[theorem]{Lemma}
\newtheorem{proposition}[theorem]{Proposition}
\newtheorem{corollary}[theorem]{Corollary}
\newtheorem{fact}[theorem]{Fact}
\newtheorem{definition}[theorem]{Definition}
\newtheorem{remark}[theorem]{Remark}
\newtheorem{conjecture}[theorem]{Conjecture}
\newtheorem{example}[theorem]{Example}
\newenvironment{proof}{\noindent \textbf{Proof:}}{$\Box$}

\newcommand{\ignore}[1]{}

\newcommand{\enote}[1]{} \newcommand{\knote}[1]{}
\newcommand{\rnote}[1]{}



% \newcommand{\enote}[1]{{\bf [[Elchanan:} {\emph{#1}}{\bf ]]}}
% \newcommand{\knote}[1]{{\bf [[Krzysztof:} {\emph{#1}}{\bf ]]}}
% \newcommand{\rnote}[1]{{\bf [[Ryan:} {\emph{#1}}{\bf ]]}}



\DeclareMathOperator{\Support}{Supp} \DeclareMathOperator{\Opt}{Opt}
\DeclareMathOperator{\Ordo}{\mathcal{O}}
\newcommand{\MaxkCSP}{\textsc{Max $k$-CSP}}
\newcommand{\MaxkCSPq}{\textsc{Max $k$-CSP$_{q}$}}
\newcommand{\MaxCSP}[1]{\textsc{Max CSP}(#1)} \renewcommand{\Pr}{{\bf
    P}} \renewcommand{\P}{{\bf P}} \newcommand{\Px}{\mathop{\bf P\/}}
\newcommand{\E}{{\bf E}} \newcommand{\Cov}{{\bf Cov}}
\newcommand{\Var}{{\bf Var}} \newcommand{\Varx}{\mathop{\bf Var\/}}

\newcommand{\bits}{\{-1,1\}}

\newcommand{\nsmaja}{\textstyle{\frac{2}{\pi}} \arcsin \rho}

\newcommand{\Inf}{\mathrm{Inf}} \newcommand{\I}{\mathrm{I}}
\newcommand{\J}{\mathrm{J}}

\newcommand{\eps}{\epsilon} \newcommand{\lam}{\lambda}

% \newcommand{\trunc}{\ell_{2,[-1,1]}}
\newcommand{\trunc}{\zeta} \newcommand{\truncprod}{\chi}

\newcommand{\N}{\mathbb N} \newcommand{\R}{\mathbb R}
\newcommand{\Z}{\mathbb Z} \newcommand{\CalE}{{\mathcal{E}}}
\newcommand{\CalC}{{\mathcal{C}}} \newcommand{\CalM}{{\mathcal{M}}}
\newcommand{\CalR}{{\mathcal{R}}} \newcommand{\CalS}{{\mathcal{S}}}
\newcommand{\CalV}{{\mathcal{V}}}
\newcommand{\CalX}{{\boldsymbol{\mathcal{X}}}}
\newcommand{\CalG}{{\boldsymbol{\mathcal{G}}}}
\newcommand{\CalH}{{\boldsymbol{\mathcal{H}}}}
\newcommand{\CalY}{{\boldsymbol{\mathcal{Y}}}}
\newcommand{\CalZ}{{\boldsymbol{\mathcal{Z}}}}
\newcommand{\CalW}{{\boldsymbol{\mathcal{W}}}}
\newcommand{\CalF}{{\mathcal{Z}}}
% \newcommand{\boldG}{{\boldsymbol G}}
% \newcommand{\boldQ}{{\boldsymbol Q}}
% \newcommand{\boldP}{{\boldsymbol P}}
% \newcommand{\boldR}{{\boldsymbol R}}
% \newcommand{\boldS}{{\boldsymbol S}}
% \newcommand{\boldX}{{\boldsymbol X}}
% \newcommand{\boldB}{{\boldsymbol B}}
% \newcommand{\boldY}{{\boldsymbol Y}}
% \newcommand{\boldZ}{{\boldsymbol Z}}
% \newcommand{\boldV}{{\boldsymbol V}}
\newcommand{\boldi}{{\boldsymbol i}} \newcommand{\boldj}{{\boldsymbol
    j}} \newcommand{\boldk}{{\boldsymbol k}}
\newcommand{\boldr}{{\boldsymbol r}}
\newcommand{\boldsigma}{{\boldsymbol \sigma}}
\newcommand{\boldupsilon}{{\boldsymbol \upsilon}}
\newcommand{\hone}{{\boldsymbol{H1}}}
\newcommand{\htwo}{\boldsymbol{H2}}
\newcommand{\hthree}{\boldsymbol{H3}}
\newcommand{\hfour}{\boldsymbol{H4}}


\newcommand{\sgn}{\mathrm{sgn}} \newcommand{\Maj}{\mathrm{Maj}}
\newcommand{\Acyc}{\mathrm{Acyc}}
\newcommand{\UniqMax}{\mathrm{UniqMax}}
\newcommand{\Thr}{\mathrm{Thr}} \newcommand{\littlesum}{{\textstyle
    \sum}}

\newcommand{\half}{{\textstyle \frac12}}
\newcommand{\third}{{\textstyle \frac13}}
\newcommand{\fourth}{{\textstyle \frac14}}

\newcommand{\Stab}{\mathbb{S}}
\newcommand{\StabThr}[2]{\Gamma_{#1}(#2)}
\newcommand{\StabThrmin}[2]{{\underline{\Gamma}}_{#1}(#2)}
\newcommand{\StabThrmax}[2]{{\overline{\Gamma}}_{#1}(#2)}
\newcommand{\TestFcn}{\Psi}

\renewcommand{\phi}{\varphi}

\begin{document}
\title{Information Theory - Exercise 4}

 \author{Omer Tamuz, 035696574}
\maketitle


\begin{enumerate}
\item

Let $\mathcal{H}=H_1,\ldots,H_k$ be $k$ copies of $H$ that cover $G$. 
Let $v$ be a node of $G$, randomly picked from the uniform distribution. 
Let $V=(v_1,\ldots,v_k)$ 
be the nodes of $\mathcal{H}$ corresponding to $v$. Let $s:V\to\{1,2,3\}$ be s.t.
$s(v_i)$ is the part of $H_i$ that $v_i$ belongs to. 
Let $S(v)=(s(v_1),\ldots,s(v_k))$. 
Now, if for two nodes of $G$, $u$ and $v$, it holds that $S(u)=S(v)$, then
the edge between them is not covered by $\mathcal{H}$. Therefore, for $\mathcal{H}$
to cover $G$, the
number of different $S(v)$ must equal to the number of nodes, and in particular
\begin{eqnarray*}
  H(S(v))=H(v)=\log m.
\end{eqnarray*}

The entropy of each $s(v_i)$ is $\half\log 2+\fourth\log 4+\fourth \log 4 = 2/3$.
Therefore:
\begin{eqnarray*}
\log m = H(S(v)) &=& H(s(v_1),\ldots,s(v_k))  \\
&\leq& H(s(v_1))+\cdots+H(s(v_k))\\
&=& kH(s(v_1))\\
&=& (3/2)k.
\end{eqnarray*}

Hence $k\geq (2/3)\log m$.
\item

Let $S\subset\{1,2,3\}^n$, and
 $E=\left\{\{u,v\}:\:u,v\in S, |u-v|=1\right\}$, where $|w|$ is the number of
non-zero coordinates in $w$.
We will prove that $|E|\leq {1\over \log 3}|S|\log|S|$.

Let $e=\{u,v\}$ belong to $E_i$ iff $e$ belongs to $E$ and 
$u$ and $v$ differ in coordinate $i$. Then $E$ is the disjoint union of the $E_i$'s,
and $|E|=\sum_i|E_i|$.

Let $X=(X_1,\ldots,X_n)\in S$ be picked from the uniform distribution over
$S$.
\begin{lemma}
  $H(X_i|X_1,\ldots,X_{i-1},X_{i+1},\ldots,X_n)\geq {\log 3|E_i|\over|S|}$
\end{lemma}
\begin{proof}
Assume w.l.o.g that $i=1$. Then let $K_1$ be the number of values 
of $X_2,\ldots,X_n$ for which there is only one possible value of $X_1$,
and likewise define $K_2$ and $K_3$. Then $|S|=K_1+2K_2+3K_3$ and
 $|E_1|=K_2+3K_3$. The entropy of
$X_1$ is:
\begin{eqnarray*}
  H(X_1)&=&{K_1\over |S|}\cdot 0 +{2K_2\over|S|}\log 2 + {3K_3\over |S|}\log 3\\
&=& {\log 3\over|S|}\left({2\over \log 3}K_2+3K_3\right)\\ 
&\geq& {\log 3|E_1|\over|S|}
\end{eqnarray*}
\end{proof}
\begin{theorem}
  $|E|\leq |S|\log|S|$
\end{theorem}
\begin{proof}
  \begin{eqnarray*}
    \log|S|&=&H(X_1,\ldots,X_n)\\
    &\geq& H(X_1)+H(X_2|X_1)+H(X_3|X_1,X_2)+\cdots+H(X_n|X_1,\ldots,X_{n-1})\\
    &\geq& H(X_1|X_2,\ldots,X_n)+H(X_2|X_1,X_3,\ldots,X_n)+\cdots+H(X_n|X_1,\ldots,X_{n-1})\\
    &\geq& {\log3|E_1|\over|S|}+\cdots+{\log3|E_n|\over|S|}\\
    &=&{\log3|E|\over|S|}
  \end{eqnarray*}
and hence
\begin{equation*}
  |E|\leq{1\over \log 3}|S|\log|S|
\end{equation*}
\end{proof}
\item
The statement is true.
Let $\mathcal{H}= H(X_1,X_2,X_3)+H(X_1,X_2,X_4)+H(X_1,X_3,X_4)+H(X_2,X_3,X_4)$.
Then, by Shirer's inequality, applied to each one of the addends:
  \begin{eqnarray*}
\lefteqn{\mathcal{H}\leq
  {1\over 2}\left[H(X_1,X_2)+H(X_1,X_3)+H(X_2,X_3)\right]+}\\
&&{1\over 2}\left[H(X_1,X_2)+H(X_1,X_4)+H(X_2,X_4)\right]+\\
&&{1\over 2}\left[H(X_1,X_3)+H(X_1,X_4)+H(X_3,X_4)\right]+\\
&&{1\over 2}\left[H(X_2,X_3)+H(X_2,X_4)+H(X_3,X_4)\right]\\
&=&H(X_1,X_2)+H(X_1,X_3)+H(X_1,X_4)+H(X_2,X_3)+H(X_2,X_4)+H(X_3,X_4) 
  \end{eqnarray*}
\item
  \begin{eqnarray*}
    \lefteqn{H(X_1,X_2,X_3)+H(X_2,X_3,X_4)+H(X_3,X_4,X_1)+H(X_4,X_1,X_2)}\\
&&= H(X_1,X_2)+H(X_3|X_1,X_2)+H(X_2,X_3)+H(X_4|X_2,X_3)\\
&&\quad+H(X_3,X_4)+H(X_1|X_3,X_4)+H(X_4,X_1)+H(X_2|X_4,X_1)\\
&&= H(X_1,X_2)+H(X_2,X_3)+H(X_3,X_4)+H(X_4,X_1)\\
&&\quad+\half H(X_3|X_1,X_2)+\half H(X_4|X_2,X_3)+\half H(X_1|X_3,X_4)+\half H(X_2|X_4,X_1)\\
&&\quad+\half H(X_3|X_1,X_2)+\half H(X_4|X_2,X_3)+\half H(X_1|X_3,X_4)+\half H(X_2|X_,4X_1)\\
&&\leq H(X_1,X_2)+H(X_2,X_3)+H(X_3,X_4)+H(X_4,X_1)\\
&&\quad+\half H(X_3|X_1,X_2)+\half H(X_4|X_2,X_3)+\half H(X_1|X_3,X_4)+\half H(X_2|X_4,X_1)\\
&&\quad+\half H(X_3|X_2)+\half H(X_4|X_3)+\half H(X_1|X_4)+\half H(X_2|X_1)\\
&&= H(X_1,X_2)+H(X_2,X_3)+H(X_3,X_4)+H(X_4,X_1)\\
&&\quad+\half (H(X_1|X_4)+H(X_2|X_4,X_1))+\half (H(X_2|X_1)+H(X_3|X_1,X_2))\\
&&\quad+\half (H(X_3|X_2)+H(X_4|X_2,X_3))+\half (H(X_4|X_3)+H(X_1|X_3,X_4))\\
&&= H(X_1,X_2)+H(X_2,X_3)+H(X_3,X_4)+H(X_4,X_1)\\
&&\quad+\half H(X_1,X_2|X_4)+\half H(X_2,X_3|X_1)\\
&&\quad+\half H(X_3,X_4|X_2)+\half H(X_4,X_1|X_3)\\
&&\leq H(X_1,X_2)+H(X_2,X_3)+H(X_3,X_4)+H(X_4,X_1)\\
&&\quad+\half H(X_1,X_2)+\half H(X_2,X_3)\\
&&\quad+\half H(X_3,X_4)+\half H(X_4,X_1)\\
&&= 1.5\cdot\left(H(X_1,X_2)+H(X_2,X_3)+H(X_3,X_4)+H(X_4,X_1)\right)
\end{eqnarray*}
\item
The statement is false. Let $X_1$ and $X_3$ be independent random bits,
each one with probability $\half$. Let $X_2=X_1$ and $X_4=X_3$. Then
\begin{eqnarray*}
  H(X_1,X_2,X_3)+H(X_1,X_2,X_4)+H(X_1,X_3,X_4)+H(X_2,X_3,X_4) = 2 + 2 + 2 + 2=8
\end{eqnarray*}
whereas
\begin{eqnarray*}
  3\cdot[H(X_1,X_2)+H(X_2,X_1)]=3\cdot[1+1] = 6.
\end{eqnarray*}
\end{enumerate}
\end{document}



