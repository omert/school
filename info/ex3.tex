\documentclass[11pt]{article} \usepackage{amssymb}
\usepackage{amsfonts} \usepackage{amsmath} \usepackage{bm}
\usepackage{latexsym} \usepackage{epsfig}

\setlength{\textwidth}{6.5 in} \setlength{\textheight}{8.25in}
\setlength{\oddsidemargin}{0in} \setlength{\topmargin}{0in}
\addtolength{\textheight}{.8in} \addtolength{\voffset}{-.5in}

\newtheorem{theorem}{Theorem}[section]
\newtheorem{lemma}[theorem]{Lemma}
\newtheorem{proposition}[theorem]{Proposition}
\newtheorem{corollary}[theorem]{Corollary}
\newtheorem{fact}[theorem]{Fact}
\newtheorem{definition}[theorem]{Definition}
\newtheorem{remark}[theorem]{Remark}
\newtheorem{conjecture}[theorem]{Conjecture}
\newtheorem{example}[theorem]{Example}
\newenvironment{proof}{\noindent \textbf{Proof:}}{$\Box$}

\newcommand{\ignore}[1]{}

\newcommand{\enote}[1]{} \newcommand{\knote}[1]{}
\newcommand{\rnote}[1]{}



% \newcommand{\enote}[1]{{\bf [[Elchanan:} {\emph{#1}}{\bf ]]}}
% \newcommand{\knote}[1]{{\bf [[Krzysztof:} {\emph{#1}}{\bf ]]}}
% \newcommand{\rnote}[1]{{\bf [[Ryan:} {\emph{#1}}{\bf ]]}}



\DeclareMathOperator{\Support}{Supp} \DeclareMathOperator{\Opt}{Opt}
\DeclareMathOperator{\Ordo}{\mathcal{O}}
\newcommand{\MaxkCSP}{\textsc{Max $k$-CSP}}
\newcommand{\MaxkCSPq}{\textsc{Max $k$-CSP$_{q}$}}
\newcommand{\MaxCSP}[1]{\textsc{Max CSP}(#1)} \renewcommand{\Pr}{{\bf
    P}} \renewcommand{\P}{{\bf P}} \newcommand{\Px}{\mathop{\bf P\/}}
\newcommand{\E}{{\bf E}} \newcommand{\Cov}{{\bf Cov}}
\newcommand{\Var}{{\bf Var}} \newcommand{\Varx}{\mathop{\bf Var\/}}

\newcommand{\bits}{\{-1,1\}}

\newcommand{\nsmaja}{\textstyle{\frac{2}{\pi}} \arcsin \rho}

\newcommand{\Inf}{\mathrm{Inf}} \newcommand{\I}{\mathrm{I}}
\newcommand{\J}{\mathrm{J}}

\newcommand{\eps}{\epsilon} \newcommand{\lam}{\lambda}

% \newcommand{\trunc}{\ell_{2,[-1,1]}}
\newcommand{\trunc}{\zeta} \newcommand{\truncprod}{\chi}

\newcommand{\N}{\mathbb N} \newcommand{\R}{\mathbb R}
\newcommand{\Z}{\mathbb Z} \newcommand{\CalE}{{\mathcal{E}}}
\newcommand{\CalC}{{\mathcal{C}}} \newcommand{\CalM}{{\mathcal{M}}}
\newcommand{\CalR}{{\mathcal{R}}} \newcommand{\CalS}{{\mathcal{S}}}
\newcommand{\CalV}{{\mathcal{V}}}
\newcommand{\CalX}{{\boldsymbol{\mathcal{X}}}}
\newcommand{\CalG}{{\boldsymbol{\mathcal{G}}}}
\newcommand{\CalH}{{\boldsymbol{\mathcal{H}}}}
\newcommand{\CalY}{{\boldsymbol{\mathcal{Y}}}}
\newcommand{\CalZ}{{\boldsymbol{\mathcal{Z}}}}
\newcommand{\CalW}{{\boldsymbol{\mathcal{W}}}}
\newcommand{\CalF}{{\mathcal{Z}}}
% \newcommand{\boldG}{{\boldsymbol G}}
% \newcommand{\boldQ}{{\boldsymbol Q}}
% \newcommand{\boldP}{{\boldsymbol P}}
% \newcommand{\boldR}{{\boldsymbol R}}
% \newcommand{\boldS}{{\boldsymbol S}}
% \newcommand{\boldX}{{\boldsymbol X}}
% \newcommand{\boldB}{{\boldsymbol B}}
% \newcommand{\boldY}{{\boldsymbol Y}}
% \newcommand{\boldZ}{{\boldsymbol Z}}
% \newcommand{\boldV}{{\boldsymbol V}}
\newcommand{\boldi}{{\boldsymbol i}} \newcommand{\boldj}{{\boldsymbol
    j}} \newcommand{\boldk}{{\boldsymbol k}}
\newcommand{\boldr}{{\boldsymbol r}}
\newcommand{\boldsigma}{{\boldsymbol \sigma}}
\newcommand{\boldupsilon}{{\boldsymbol \upsilon}}
\newcommand{\hone}{{\boldsymbol{H1}}}
\newcommand{\htwo}{\boldsymbol{H2}}
\newcommand{\hthree}{\boldsymbol{H3}}
\newcommand{\hfour}{\boldsymbol{H4}}


\newcommand{\sgn}{\mathrm{sgn}} \newcommand{\Maj}{\mathrm{Maj}}
\newcommand{\Acyc}{\mathrm{Acyc}}
\newcommand{\UniqMax}{\mathrm{UniqMax}}
\newcommand{\Thr}{\mathrm{Thr}} \newcommand{\littlesum}{{\textstyle
    \sum}}

\newcommand{\half}{{\textstyle \frac12}}
\newcommand{\third}{{\textstyle \frac13}}
\newcommand{\fourth}{{\textstyle \frac14}}

\newcommand{\Stab}{\mathbb{S}}
\newcommand{\StabThr}[2]{\Gamma_{#1}(#2)}
\newcommand{\StabThrmin}[2]{{\underline{\Gamma}}_{#1}(#2)}
\newcommand{\StabThrmax}[2]{{\overline{\Gamma}}_{#1}(#2)}
\newcommand{\TestFcn}{\Psi}

\renewcommand{\phi}{\varphi}

\begin{document}
\title{Information Theory - Exercise 3}

 \author{Omer Tamuz, 035696574}
\maketitle


\begin{enumerate}
\item
Let $C$ be a binary channel that, given $X$, outputs $Y=X$ with
probability $1-p$ and $Y=1-X$ with probability $p$. Then 
\begin{eqnarray*}
  I(X;Y)&=&H(Y)-H(Y|X)
     \\ &=&H(Y) - \E_XH(Y|X=x) 
     \\ &=&H(Y)-\E_XH(p,1-p)
     \\ &=&H(Y)-H(p,1-p)
\end{eqnarray*}
Let $X$ be one with probability $q$. Then 
\begin{equation*}
P(Y=1)=q(1-p)+(1-q)p=q-qp+p-qp=q+p-2qp  
\end{equation*}
 and
 \begin{equation*}
 P(Y=0)=qp+(1-q)(1-p)=1-(q+p-2qp).
 \end{equation*}
Since $Y$ is binary, $H(Y)$ can be at most one. This is indeed achieved when 
$q+p-2qp=\half$, so
\begin{equation*}
\forall p \: \mbox{s.t.}\: 0\leq p \leq 1, \: q= \lim_{s\to p}{\half-s\over 1-2s}  \:\Longrightarrow\: P(Y=1)=P(Y=0)=\half \:\Longrightarrow\: H(Y)=1
\end{equation*}
(the limit is required for when $p=\half$). 

Hence:
\begin{equation*}
  \sup_{\mbox{dist}(X)}I(X;Y)=1-H(p,1-p)
\end{equation*}
\item
The mutual information between the input and output of this channel is:
\begin{eqnarray*}
  I(X;Y)&=&H(X)-H(X|Y)
   \\   &=&H(X)-\E_YH(X|Y=y)
   \\   &=&H(X)-P(Y=*)H(X|Y=*)+P(Y\not=*)H(X|Y\not=*)
   \\   &=&H(X)-pH(X)+(1-p)0
   \\   &=&(1-p)H(X)
\end{eqnarray*}
The information capacity of the channel is therefore:
\begin{eqnarray*}
  \sup_{\mbox{dist}(X)}I(X;Y)=\sup_{\mbox{dist}(X)}(1-p)H(X)=1-p,
\end{eqnarray*}
since $H(X)$ is at most one.

\item

Let $Z$ be an r.v. which equals the index of the bit in $X$ that was
dropped. Then $H(Z)=\log n$. 
\begin{eqnarray*}
  H(X|Y)&\leq& H(Z,X|Y) 
\quad\mbox{(adding an r.v. increases entropy)}
\\      &=&H(Y,Z,X|Y)   
\quad\mbox{($Y,Z,X$ is a one-to-one function of $Z,X$)}
\\      &=&H(Y,Z,X_Z|Y) 
\quad\mbox{($Y,Z,X_Z$ is a one-to-one function of $Y,Z,X$)}
\\      &=&H(Z, X_Z|Y)
\quad\mbox{($Y$ is constant given $Y$)}
\\      &\leq&H(Z|Y)+H(X_Z|Y)
\\      &\leq& \log n + 1
\quad\mbox{($H(X_Z)$ is at most 1, $X_Z$ being a binary r.v.)}
\end{eqnarray*}

The information capacity of the channel is therefore:
\begin{eqnarray*}
  \sup_{\mbox{dist}(X)}I(X;Y)&=&\sup_{\mbox{dist}(X)}H(X)-H(X|Y)
\\                        &\geq&\sup_{\mbox{dist}(X)}H(X)-\log n - 1,
\\                        &\geq&n-\log n - 1,
\end{eqnarray*}
since $H(X)$ is at most $n$.

For $n=2$: if $P(X_1=1,X_2=1)=P(X_1=0,X_2=0)=\half$ and $P(X_1\not=X_2)=0$,
then clearly $H(Y|X)=0$ and $H(Y)=1$. These entail the lowest possible value
for the former, and the highest possible value for the latter.

The information capacity of the channel is therefore:

\begin{eqnarray*}
  \sup_{\mbox{dist}(X)}I(X;Y)=\sup_{\mbox{dist}(X)}H(Y)-H(Y|X)=1-0=1
\end{eqnarray*}

\item
  \begin{eqnarray*}
    I(X;Y)=I(X_1,X_2;Y)
  \end{eqnarray*}
\end{enumerate}
\end{document}



