\documentclass[11pt]{article} \usepackage{amssymb}
\usepackage{amsfonts} \usepackage{amsmath} \usepackage{bm}
\usepackage{latexsym} \usepackage{epsfig}
\usepackage{algorithm}
\usepackage{algorithmic}

\setlength{\textwidth}{6.5 in} \setlength{\textheight}{8.25in}
\setlength{\oddsidemargin}{0in} \setlength{\topmargin}{0in}
\addtolength{\textheight}{.8in} \addtolength{\voffset}{-.5in}

\newtheorem{theorem}{Theorem}[section]
\newtheorem{lemma}[theorem]{Lemma}
\newtheorem{proposition}[theorem]{Proposition}
\newtheorem{corollary}[theorem]{Corollary}
\newtheorem{fact}[theorem]{Fact}
\newtheorem{definition}[theorem]{Definition}
\newtheorem{remark}[theorem]{Remark}
\newtheorem{conjecture}[theorem]{Conjecture}
\newtheorem{claim}[theorem]{Claim}
\newtheorem{example}[theorem]{Example}
\newenvironment{proof}{\noindent \textbf{Proof:}}{$\Box$}

\newcommand{\ignore}[1]{}

\newcommand{\enote}[1]{} \newcommand{\knote}[1]{}
\newcommand{\rnote}[1]{}



% \newcommand{\enote}[1]{{\bf [[Elchanan:} {\emph{#1}}{\bf ]]}}
% \newcommand{\knote}[1]{{\bf [[Krzysztof:} {\emph{#1}}{\bf ]]}}
% \newcommand{\rnote}[1]{{\bf [[Ryan:} {\emph{#1}}{\bf ]]}}



\DeclareMathOperator{\Support}{Supp} \DeclareMathOperator{\Opt}{Opt}
\DeclareMathOperator{\Ordo}{\mathcal{O}}
\newcommand{\MaxkCSP}{\textsc{Max $k$-CSP}}
\newcommand{\MaxkCSPq}{\textsc{Max $k$-CSP$_{q}$}}
\newcommand{\MaxCSP}[1]{\textsc{Max CSP}(#1)} \renewcommand{\Pr}{{\bf
    P}} \renewcommand{\P}{{\bf P}} \newcommand{\Px}{\mathop{\bf P\/}}
\newcommand{\E}{{\bf E}} \newcommand{\Cov}{{\bf Cov}}
\newcommand{\Var}{{\bf Var}} \newcommand{\Varx}{\mathop{\bf Var\/}}

\newcommand{\bits}{\{-1,1\}}

\newcommand{\nsmaja}{\textstyle{\frac{2}{\pi}} \arcsin \rho}

\newcommand{\Inf}{\mathrm{Inf}} \newcommand{\I}{\mathrm{I}}
\newcommand{\J}{\mathrm{J}}

\newcommand{\eps}{\epsilon} \newcommand{\lam}{\lambda}

% \newcommand{\trunc}{\ell_{2,[-1,1]}}
\newcommand{\trunc}{\zeta} \newcommand{\truncprod}{\chi}

\newcommand{\N}{\mathbb N} \newcommand{\R}{\mathbb R}
\newcommand{\Z}{\mathbb Z} \newcommand{\CalE}{{\mathcal{E}}}
\newcommand{\CalC}{{\mathcal{C}}} \newcommand{\CalM}{{\mathcal{M}}}
\newcommand{\CalR}{{\mathcal{R}}} \newcommand{\CalS}{{\mathcal{S}}}
\newcommand{\CalV}{{\mathcal{V}}}
\newcommand{\CalX}{{\boldsymbol{\mathcal{X}}}}
\newcommand{\CalG}{{\boldsymbol{\mathcal{G}}}}
\newcommand{\CalH}{{\boldsymbol{\mathcal{H}}}}
\newcommand{\CalY}{{\boldsymbol{\mathcal{Y}}}}
\newcommand{\CalZ}{{\boldsymbol{\mathcal{Z}}}}
\newcommand{\CalW}{{\boldsymbol{\mathcal{W}}}}
\newcommand{\CalF}{{\mathcal{Z}}}
% \newcommand{\boldG}{{\boldsymbol G}}
% \newcommand{\boldQ}{{\boldsymbol Q}}
% \newcommand{\boldP}{{\boldsymbol P}}
% \newcommand{\boldR}{{\boldsymbol R}}
% \newcommand{\boldS}{{\boldsymbol S}}
% \newcommand{\boldX}{{\boldsymbol X}}
% \newcommand{\boldB}{{\boldsymbol B}}
% \newcommand{\boldY}{{\boldsymbol Y}}
% \newcommand{\boldZ}{{\boldsymbol Z}}
% \newcommand{\boldV}{{\boldsymbol V}}
\newcommand{\boldi}{{\boldsymbol i}} \newcommand{\boldj}{{\boldsymbol
    j}} \newcommand{\boldk}{{\boldsymbol k}}
\newcommand{\boldr}{{\boldsymbol r}}
\newcommand{\boldsigma}{{\boldsymbol \sigma}}
\newcommand{\boldupsilon}{{\boldsymbol \upsilon}}
\newcommand{\hone}{{\boldsymbol{H1}}}
\newcommand{\htwo}{\boldsymbol{H2}}
\newcommand{\hthree}{\boldsymbol{H3}}
\newcommand{\hfour}{\boldsymbol{H4}}


\newcommand{\sgn}{\mathrm{sgn}} \newcommand{\Maj}{\mathrm{Maj}}
\newcommand{\Acyc}{\mathrm{Acyc}}
\newcommand{\UniqMax}{\mathrm{UniqMax}}
\newcommand{\Thr}{\mathrm{Thr}} \newcommand{\littlesum}{{\textstyle
    \sum}}

\newcommand{\half}{{\textstyle \frac12}}
\newcommand{\third}{{\textstyle \frac13}}
\newcommand{\fourth}{{\textstyle \frac14}}

\newcommand{\Stab}{\mathbb{S}}
\newcommand{\StabThr}[2]{\Gamma_{#1}(#2)}
\newcommand{\StabThrmin}[2]{{\underline{\Gamma}}_{#1}(#2)}
\newcommand{\StabThrmax}[2]{{\overline{\Gamma}}_{#1}(#2)}
\newcommand{\TestFcn}{\Psi}

\renewcommand{\phi}{\varphi}

\begin{document}
\title{Complexity - Exercise 1}

 \author{Omer Tamuz, 035696574}
\maketitle


\begin{enumerate}
  \item
  \item
  \item
    
    Let $D$ be a machine that decides whether or not a graph has a clique
    of size $k$, os that $D(G,k)\in \{0,1\}$ where $G$ is a graph.

    The following is an algorithm $S(G,k)$ to find a clique of size $k$ in a 
    graph $G$, if one exists:

    
    Let $G_{N(v)}$ be $G$ with vertex $v$ and its non-neighbors removed. 
    \begin{algorithmic} 
      \IF {$D(G,k)=0$}
        \STATE {\bf return $\varnothing$}
      \ELSE
        \STATE {Find a vertex $v$ from $G$ such that $D(G_{N(v)},k-1)=1$}
          \STATE {\bf return $\{v\}\cup S(G_{N(v)},k-1)$} 
      \ENDIF
    \end{algorithmic}
    
    This algorithm finds a clique of size $k$ if one exists, and returns
    $\varnothing$ otherwise.

    \begin{proof}
      If no clique of size $k$ exists then the algorithm, in the first line, 
      returns $\varnothing$. 
      Otherwise, the algorithm will find a vertex $v$ such that 
      $D(G_{N(v)},k-1)=1$, since this will be true for any vertex in a clique
      of size $k$, and such a vertex must exist.
      
      $S(G,0)$ clearly returns the correct answer. Assume $S(G,k-1)$ does, too.
      Then given a vertex $v$ such that $D(G_{N(v)},k-1)=1$, the graph of its
      neighbors $G_{n(v)}$ will have a clique of size $k-1$, which, when
      united with $v$, will form a clique of size $k$.


    \end{proof}
  \item We will show a reduction from 3SAT to restricted-3SAT. Given a 3SAT
    clause $\phi$ with variables $a_1,\ldots,a_n$, change the occurences
    of $a_i$ to $a_{i,1},a_{i,2},\ldots,a_{i,k}$, and AND the resulting 
    $\phi$ with
    \begin{equation*}
    (\bar{a}_{i,1}\vee a_{i,2})\wedge(\bar{a}_{i,2}\vee a_{i,3})\wedge\cdots\wedge
(\bar{a}_{i,k-1}\vee a_{i,k})\wedge(\bar{a}_{i,k}\vee a_{i,1})  
    \end{equation*}
     for each $i$, which is the same as 

    \begin{equation*}
    (a_{i,1}\to a_{i,2})\wedge(a_{i,2}\to a_{i,3})\wedge\cdots\wedge
(a_{i,k-1}\to a_{i,k})\wedge(a_{i,k}\to a_{i,1}).
    \end{equation*}

    Since the added clause ensures that all the new variants of $a_i$ will have
    the same truth value, it will be satisfied iff the original one was. And
    clearly, each variable appears in it only once, and it is polynomially
    constructible from $\phi$.

    

\end{enumerate}
\end{document}


