\documentclass[11pt]{article} \usepackage{amssymb}
\usepackage{amsfonts} \usepackage{amsmath} \usepackage{bm}
\usepackage{latexsym} \usepackage{epsfig}

\setlength{\textwidth}{6.5 in} \setlength{\textheight}{8.25in}
\setlength{\oddsidemargin}{0in} \setlength{\topmargin}{0in}
\addtolength{\textheight}{.8in} \addtolength{\voffset}{-.5in}

\newtheorem{theorem}{Theorem}[section]
\newtheorem{lemma}[theorem]{Lemma}
\newtheorem{proposition}[theorem]{Proposition}
\newtheorem{corollary}[theorem]{Corollary}
\newtheorem{fact}[theorem]{Fact}
\newtheorem{definition}[theorem]{Definition}
\newtheorem{remark}[theorem]{Remark}
\newtheorem{conjecture}[theorem]{Conjecture}
\newtheorem{claim}[theorem]{Claim}
\newtheorem{example}[theorem]{Example}
\newenvironment{proof}{\noindent \textbf{Proof:}}{$\Box$}

\newcommand{\ignore}[1]{}

\newcommand{\enote}[1]{} \newcommand{\knote}[1]{}
\newcommand{\rnote}[1]{}



% \newcommand{\enote}[1]{{\bf [[Elchanan:} {\emph{#1}}{\bf ]]}}
% \newcommand{\knote}[1]{{\bf [[Krzysztof:} {\emph{#1}}{\bf ]]}}
% \newcommand{\rnote}[1]{{\bf [[Ryan:} {\emph{#1}}{\bf ]]}}



\DeclareMathOperator{\Support}{Supp} \DeclareMathOperator{\Opt}{Opt}
\DeclareMathOperator{\Ordo}{\mathcal{O}}
\newcommand{\MaxkCSP}{\textsc{Max $k$-CSP}}
\newcommand{\MaxkCSPq}{\textsc{Max $k$-CSP$_{q}$}}
\newcommand{\MaxCSP}[1]{\textsc{Max CSP}(#1)} \renewcommand{\Pr}{{\bf
    P}} \renewcommand{\P}{{\bf P}} \newcommand{\Px}{\mathop{\bf P\/}}
\newcommand{\E}{{\bf E}} \newcommand{\Cov}{{\bf Cov}}
\newcommand{\Var}{{\bf Var}} \newcommand{\Varx}{\mathop{\bf Var\/}}

\newcommand{\bits}{\{-1,1\}}

\newcommand{\nsmaja}{\textstyle{\frac{2}{\pi}} \arcsin \rho}

\newcommand{\Inf}{\mathrm{Inf}} \newcommand{\I}{\mathrm{I}}
\newcommand{\J}{\mathrm{J}}

\newcommand{\eps}{\epsilon} \newcommand{\lam}{\lambda}

% \newcommand{\trunc}{\ell_{2,[-1,1]}}
\newcommand{\trunc}{\zeta} \newcommand{\truncprod}{\chi}

\newcommand{\N}{\mathbb N} \newcommand{\R}{\mathbb R}
\newcommand{\Q}{\mathbb Q}
\newcommand{\Z}{\mathbb Z} \newcommand{\CalE}{{\mathcal{E}}}
\newcommand{\CalC}{{\mathcal{C}}} \newcommand{\CalM}{{\mathcal{M}}}
\newcommand{\CalR}{{\mathcal{R}}} \newcommand{\CalS}{{\mathcal{S}}}
\newcommand{\CalV}{{\mathcal{V}}}
\newcommand{\CalX}{{\boldsymbol{\mathcal{X}}}}
\newcommand{\CalG}{{\boldsymbol{\mathcal{G}}}}
\newcommand{\CalH}{{\boldsymbol{\mathcal{H}}}}
\newcommand{\CalY}{{\boldsymbol{\mathcal{Y}}}}
\newcommand{\CalZ}{{\boldsymbol{\mathcal{Z}}}}
\newcommand{\CalW}{{\boldsymbol{\mathcal{W}}}}
\newcommand{\CalF}{{\mathcal{Z}}}
% \newcommand{\boldG}{{\boldsymbol G}}
% \newcommand{\boldQ}{{\boldsymbol Q}}
% \newcommand{\boldP}{{\boldsymbol P}}
% \newcommand{\boldR}{{\boldsymbol R}}
% \newcommand{\boldS}{{\boldsymbol S}}
% \newcommand{\boldX}{{\boldsymbol X}}
% \newcommand{\boldB}{{\boldsymbol B}}
% \newcommand{\boldY}{{\boldsymbol Y}}
% \newcommand{\boldZ}{{\boldsymbol Z}}
% \newcommand{\boldV}{{\boldsymbol V}}
\newcommand{\boldi}{{\boldsymbol i}} \newcommand{\boldj}{{\boldsymbol
    j}} \newcommand{\boldk}{{\boldsymbol k}}
\newcommand{\boldr}{{\boldsymbol r}}
\newcommand{\boldsigma}{{\boldsymbol \sigma}}
\newcommand{\boldupsilon}{{\boldsymbol \upsilon}}
\newcommand{\hone}{{\boldsymbol{H1}}}
\newcommand{\htwo}{\boldsymbol{H2}}
\newcommand{\hthree}{\boldsymbol{H3}}
\newcommand{\hfour}{\boldsymbol{H4}}


\newcommand{\sgn}{\mathrm{sgn}} \newcommand{\Maj}{\mathrm{Maj}}
\newcommand{\Acyc}{\mathrm{Acyc}}
\newcommand{\UniqMax}{\mathrm{UniqMax}}
\newcommand{\Thr}{\mathrm{Thr}} \newcommand{\littlesum}{{\textstyle
    \sum}}

\newcommand{\half}{{\textstyle \frac12}}
\newcommand{\third}{{\textstyle \frac13}}
\newcommand{\fourth}{{\textstyle \frac14}}

\newcommand{\Stab}{\mathbb{S}}
\newcommand{\StabThr}[2]{\Gamma_{#1}(#2)}
\newcommand{\StabThrmin}[2]{{\underline{\Gamma}}_{#1}(#2)}
\newcommand{\StabThrmax}[2]{{\overline{\Gamma}}_{#1}(#2)}
\newcommand{\TestFcn}{\Psi}

\renewcommand{\phi}{\varphi}

\begin{document}
\title{Algebra Through Examples - Exam}

 \author{Omer Tamuz, 035696574}
\maketitle


\begin{enumerate}
\item 
  \begin{enumerate}
  \item 
    Let $a$ be a root of $x^3+2x+1$ in $GF(27)$. Then the elements of $GF(27)^*$
    can be written as follows:

    \begin{tabular}{l| l}
      $a^{0}$ & $1$\\
      $a^{1}$ & $a$\\
      $a^{2}$ & $a^{2}$\\
      $a^{3}$ & $a+2$\\
      $a^{4}$ & $a^{2}+2a$\\
      $a^{5}$ & $2a^{2}+a+2$\\
      $a^{6}$ & $a^{2}+a+1$\\
      $a^{7}$ & $a^{2}+2a+2$\\
      $a^{8}$ & $2a^{2}+2$\\
      $a^{9}$ & $a+1$\\
      $a^{10}$ & $a^{2}+a$\\
      $a^{11}$ & $a^{2}+a+2$\\
      $a^{12}$ & $a^{2}+2$\\
      $a^{13}$ & $2$\\
      $a^{14}$ & $2a$\\
      $a^{15}$ & $2a^{2}$\\
      $a^{16}$ & $2a+1$\\
      $a^{17}$ & $2a^{2}+a$\\
      $a^{18}$ & $a^{2}+2a+1$\\
      $a^{19}$ & $2a^{2}+2a+2$\\
      $a^{20}$ & $2a^{2}+a+1$\\
      $a^{21}$ & $a^{2}+1$\\
      $a^{22}$ & $2a+2$\\
      $a^{23}$ & $2a^{2}+2a$\\
      $a^{24}$ & $2a^{2}+2a+1$\\
      $a^{25}$ & $2a^{2}+1$\\
    \end{tabular}
%    We note (to be used in the next question), that $x^3+2x+1$ is a minimum
%    polynomial of $a$. 
  \item
    We would like to factor $x^{27}-x$. We showed in class that for $p$ 
    prime and $n$ natural the 
    polynomial $x^{p^n}-x$ is the product of all the monic irreducible 
    polynomials of degrees that divide $n$ over $GF(p)$. Hence $x^{27}-x$
    is the product of the irreducible linear and cubic monic polynomials
    over $GF(3)$.

    The linear monics ($x$, $x+1$ and $x+2$) are all
    trivially irreducible. To find the cubic factors of $x^{27}-x$ we list all 
    the {\em reducible} cubics, and infer that the rest are irreducible. 
    We restrict
    ourselves to monic polynomials with non-zero free coefficients, of which
    there are $3 \cdot 3 \cdot 2=18$ --- one needs to choose coefficients 
    for $x^2$ and $x$ in $GF(3)$, and a free coefficient in $\{1,2\}$.
    
    Of the class of cubics described above, the ones reducible to linear
    coefficients are
    \begin{eqnarray*}
      (x + 1)^3 &=& x^3 + 1\\
      (x + 1)^2(x + 2) &=& x^3 + x^2 + 2x + 2\\
      (x + 1)(x + 2)^2 &=& x^3 + 2x^2 + 2x + 1\\
      (x + 2)^3 &=& x^3 + 2.
    \end{eqnarray*}

    To find the ones reducible to an irreducible quadratic and a linear 
    coefficient, we note that the {\em reducible} quadratics are
    \begin{eqnarray*}
      (x+1)^2 &=& x^2+2x+1\\
      (x+1)(x+2) &=& x^2+2\\
      (x+2)^2 &=& x^2+x+1\\
    \end{eqnarray*}
    and so the {\em irreducible} quadratics are  $x^2+1$, $x^2+2x+2$ and 
    $x^2+x+2$. Hence the monic cubics (with a free coefficient) reducible to a 
    quadratic and a linear are
    \begin{eqnarray*}
      (x^2 + 1)(x + 1) &=& x^3 + x^2 + x + 1\\
      (x^2 + 2x + 2)(x + 1) &=& x^3 + x + 2\\
      (x^2 + x + 2)(x + 1) &=& x^3 + 2x^2 + 2\\
      (x^2 + 1)(x + 2) &=& x^3 + 2x^2 + x + 2\\
      (x^2 + 2x + 2)(x + 2) &=& x^3 + x^2 + 1\\
      (x^2 + x + 2)(x + 2) &=& x^3 + x + 1.
    \end{eqnarray*}

    We found ten reducible monic cubics (with a free coefficient), and so
    there remain eight irreducible ones. They are:
    \begin{eqnarray*}
      x^3 + 2x + 1\\
      x^3 + 2x + 2\\
      x^3 + x^2 + 2\\
      x^3 + x^2 + x + 2\\
      x^3 + x^2 + 2x + 1\\
      x^3 + 2x^2 + 1\\
      x^3 + 2x^2 + x + 1\\
      x^3 + 2x^2 + 2x + 2\\
    \end{eqnarray*}

    And therefore
    \begin{eqnarray*}
       x^{27} - x = (x^3 + 2x + 1)(x^3 + 2x + 2)(x^3 + x^2 + 2)(x^3 + x^2 + x + 2)(x^3 + x^2 + 2x + 1)&\\
    (x^3 + 2x^2 + 1)(x^3 + 2x^2 + x + 1)(x^3 + 2x^2 + 2x + 2)(x + 1)(x + 2)x&
    \end{eqnarray*}


  \item
    All the elements of $GF(27)$ are roots of $x^{27}-x$: 0 is clearly a root,
    and the order of $g \in GF(27)^*$ divides 26, and so $g^{26}=1$ or
    $g^{27}=g$. 
    Because a polynomial of degree 27 cannot have more than 27 distinct
    roots (and all are accounted for in $GF(27)$), then $GF(27)$ is precisely
    the set of roots of $x^{27}-x$.

    Therefore (as we've shown in class) the irreducible factors of $x^{27}-x$
    (bar $x$) are the minimal polynomials of the elements of $GF(27)^*$, which are 
    precisely those that can be written in the form $a^i,\:0 \leq i \leq 25$.

    This task can be performed by substituting each $a^i$ into each of the 
    factors of $x^{27}-x$. It is possible, however, to perform this using
    less tedious (but more devious) techniques:

    \begin{enumerate}
    \item 
      Trivially, $x+2$ is the minimum polynomial of $a^0=1$, as
      $x+1$ is of $a^{13}=2$.
    \item
      By definition, $a$ is a root of $x^3+2x+1$, and so the latter, which we've
      shown above to be irreducible, is the former's minimum 
      polynomial. By the ``Freshman's dream'', $(p+q+r)^3=p^3+q^3+r^3$ under
      $GF(27)$. Hence $a^3$ is also  a root of $x^3+2x+1$, as is $a^9$.
    \item
      The reciprocal polynomial of $x^3+2x+1$ is 
      $$x^3(x^{-3}+2x^{-1}+1)=x^3+2x^2+1.$$
      We showed in a homework assignment that this implies that
      $a^{25}=a^{-1}$ is a root
      of $x^3+2x^2+1$, as are $a^{23}=a^{-3}$ and $a^{17}=a^{-9}$. It also implies that 
      $x^3+2x^2+1$ is irreducible (as we've already shown above), and that
      therefore it is their minimum polynomial.
    \item
      By the ``Freshman's'' argument above, a single irreducible cubic has 
      roots $a^5$, $a^{15}$ and $a^{45}=a^{19}$. It therefore has to equal 
      \begin{eqnarray*}
        (x-a^{5})(x-a^{15})(x-a^{19})
            &=& x^3-(a^{5}+a^{15}+a^{19})x^2+(a^{20}+a^{24}+a^{34})x-a^{39}
        \\  &=& x^3-(a^{5}+a^{15}+a^{19})x^2+(a^{20}+a^{24}+a^{8})x+1
        \\  &=& x^3-(2a^2+a+2+2a^2+2a^2+2a+2)x^2+
        \\   &&(2a^2+a+1+2a^2+2a+1+2a^2+2)x+1
        \\  &=& x^3+2x^2+x+1 
      \end{eqnarray*}
      Which is thus the minimum polynomial of $a^{5}$, $a^{15}$ and $a^{19}$. 
      Hence the reciprocal polynomial, $x^3+x^2+2x+1$,
      is the minimum polynomial of $a^{-5}=a^{21}$, $a^{-15}=a^{11}$ and
      $a^{-19}=a^{7}$.
    \item
      We apply the same argument to $a^2$, $a^6$ and $a^{18}$. Their minimum
      polynomial is
      \begin{eqnarray*}
        (x-a^{2})(x-a^{6})(x-a^{18})
            &=& x^3-(a^{2}+a^{6}+a^{18})x^2+(a^{8}+a^{20}+a^{24})x-a^{26}
        \\  &=& x^3-(a^2+a^2+a+1+a^2+2a+1)x^2+
        \\   && (2a^2+2+2a^2+a+1+2a^2+2a+1)x+2
        \\  &=& x^3+x^2+x+2 
      \end{eqnarray*}
      and its reciprocal (times two), $x^3+2x^2+2x+2$, is the minimum
      polynomials of $a^{-2}=a^{24}$, $a^{-6}=a^{20}$ and $a^{-18}=a^{8}$.
    \item
      We have two irreducible polynomials left, $x^3+2x+2$ and $x^3+x^2+2$. 
      They are reciprocals (up to a factor of two), and so one is the
      root of $a^4$, $a^{12}$ and $a^{10}$, while the other is the minimum
      polynomial
      of these elements' inverses, $a^{-4}=a^{22}$, $a^{-12}=a^{14}$ and
      $a^{-10}=a^{16}$. Here we finally resort to (a single) substitution, and
      check whether $a^4$ is a root of $x^3+2x+2$:
      $$\left(a^4\right)^3+2a^4+2=a^{12}+2a^4+2=a^2+2+2(a^2+2a)+2 \neq 0$$
      Since it is not, then it must be a root of $x^3+x^2+2$. So this polynomial
      is also the minimum polynomial of $a^{12}$ and $a^{10}$, whereas
      its reciprocal, $x^3+2x+2$, is the minimum polynomial of $a^{22}$,
      $a^{14}$ and $a^{16}$
    \end{enumerate}
    These results are summarised in the tables below. In the first table 
    we list for each element of $GF(27)^*$ its minimum polynomial. 
    The number in parentheses references the argument above by which it was 
    derived. In the second table we list the roots of each of these
    polynomials.

    {\bf Elements of $GF(27)^*$ and Their Minimum Polynomials}

    \begin{tabular}{l| l r}

      $a^{0}$ & $x-1$ & (i)\\
      $a^{1}$ & $x^3+2x+1$ & (ii)\\
      $a^{2}$ & $x^3+x^2+x+2$ & (v)\\
      $a^{3}$ & $x^3+2x+1$ & (ii)\\
      $a^{4}$ & $x^3+x^2+2$ & (vi)\\
      $a^{5}$ & $x^3+2x^2+x+1$ & (iv)\\
      $a^{6}$ & $x^3+x^2+x+2$ & (v)\\
      $a^{7}$ & $x^3+x^2+2x+1$ & (iv)\\
      $a^{8}$ & $x^3+2x^2+2x+2$ & (v)\\
      $a^{9}$ & $x^3+2x+1$ & (ii)\\
      $a^{10}$ & $x^3+x^2+2$ & (vi)\\
      $a^{11}$ & $x^3+x^2+2x+1$ & (iv)\\
      $a^{12}$ & $x^3+x^2+2$ & (vi)\\
      $a^{13}$ & $x-2$ & (i)\\
      $a^{14}$ & $x^3+2x+2$ & (vi)\\
      $a^{15}$ & $x^3+2x^2+x+1$ & (iv)\\
      $a^{16}$ & $x^3+2x+2$ & (vi)\\
      $a^{17}$ & $x^3+2x^2+1$ & (iii)\\
      $a^{18}$ & $x^3+x^2+x+2$ & (v)\\
      $a^{19}$ & $x^3+2x^2+x+1$ & (iv)\\
      $a^{20}$ & $x^3+2x^2+2x+2$ & (v)\\
      $a^{21}$ & $x^3+x^2+2x+1$ & (iv)\\
      $a^{22}$ & $x^3+2x+2$ & (vi)\\
      $a^{23}$ & $x^3+2x^2+1$ & (iii)\\
      $a^{24}$ & $x^3+2x^2+2x+2$ & (v)\\
      $a^{25}$ & $x^3+2x^2+1$ & (iii)\\
    \end{tabular}

    {\bf Irreducible Monic Polynomials over $GF(3)$ of Orders One and Three, and
      Their Roots}

    \begin{tabular}{l| l }
      \label{table:polys}
      $x-1$ & $a^{0}$\\
      $x-2$ & $a^{13}$\\
      $x^3+2x+1$ & $a^{1}, a^3, a^9$\\
      $x^3+x^2+x+2$ & $a^{2},a^6,a^{18}$\\
      $x^3+x^2+2$ & $a^4, a^{12}, a^{10}$\\
      $x^3+2x^2+x+1$ & $a^5,a^{15}, a^{19}$\\
      $x^3+x^2+2x+1$ & $a^7,a^{11},a^{21}$\\
      $x^3+2x^2+2x+2$ & $a^{8},a^{20},a^{24}$\\
      $x^3+2x+2$ & $a^{14},a^{16},a^{22}$\\
      $x^3+2x^2+1$ & $a^{17},a^{23},a^{25}$\\
    \end{tabular}

  \end{enumerate}
  \item

    The analysis that follows is based on a fact stated in class (but not
    proved): two matrices over a field $F$ are similar iff they have the same
    Jordan form over an algebraic closure of $F$.
    \begin{enumerate}
    \item 
      We list below each of the conjugacy classes of $GL(3,3)$, and note
      the order of the elements and the size of the class (so we answer
      both (a) and (b)).
      \begin{itemize}
      \item The elements of the centre of $GL(3,3)$ are
        \begin{equation*}
          \begin{pmatrix}
            1&0  &0 \\ 
            0&1  &0 \\ 
            0&0  &1 
          \end{pmatrix}
          \mbox{ and }
          \begin{pmatrix}
            2&0  &0 \\ 
            0&2  &0 \\ 
            0&0  &2 
          \end{pmatrix}.
        \end{equation*}
        They are each in their own conjugacy class. Their orders are one
        and two, respectively. 
      \item
        The diagonalisable elements with characteristic polynomial $(x+2)^2(x+1)$
        form a conjugacy class. A representative of this class (in Jordan
        form) is
        \begin{equation*}
          X=
          \begin{pmatrix}
            1&0  &0 \\ 
            0&1  &0 \\ 
            0&0  &2 
          \end{pmatrix},
        \end{equation*}
        and so the order of the elements in this class is two.
        
        Let $A=(a_{ij})$ be an element of $C_G(X)$, so that $AX=XA$:
        \begin{eqnarray*}
          \begin{pmatrix}
            a_{11}&a_{12}  &a_{13} \\ 
            a_{21}&a_{22}  &a_{23} \\ 
            a_{31}&a_{32}  &a_{33} 
          \end{pmatrix}
          \cdot
          \begin{pmatrix}
            1&0  &0 \\ 
            0&1  &0 \\ 
            0&0  &2 
          \end{pmatrix}
          &=&
          \begin{pmatrix}
            1&0  &0 \\ 
            0&1  &0 \\ 
            0&0  &2 
          \end{pmatrix}
          \begin{pmatrix}
            a_{11}&a_{12}  &a_{13} \\ 
            a_{21}&a_{22}  &a_{23} \\ 
            a_{31}&a_{32}  &a_{33} 
          \end{pmatrix}
          \\
          \begin{pmatrix}
            a_{11}&a_{12}  &2a_{13} \\ 
            a_{21}&a_{22}  &2a_{23} \\ 
            a_{31}&a_{32}  &2a_{33} 
          \end{pmatrix}
          &=&
          \begin{pmatrix}
            a_{11}&a_{12}  &a_{13} \\ 
            a_{21}&a_{22}  &a_{23} \\ 
            2a_{31}&2a_{32}  &2a_{33} 
          \end{pmatrix}
        \end{eqnarray*}
        and $a_{13}=a_{23}=a_{31}=a_{32}=0$. Then $a_{33}$ must be non-zero 
        (otherwise the matrix would be singular), and 
        $\begin{pmatrix}a_{11}&a_{12}\\a_{21}&a_{22}\end{pmatrix}$ can be any 
        element of $GL(2,3)$, which is of size $48$. Hence
        $|C_G(x)|=96$ and $|cl(X)|=11232/96=117$. 
      \item
        Similarly, the diagonalisable elements with characteristic polynomial 
        $(x+1)^2(x+2)$ also form a conjugacy class. A representative of this 
        class (again, and henceforth, in Jordan form) is
        \begin{equation*}
          X=
          \begin{pmatrix}
            1&0  &0 \\ 
            0&2  &0 \\ 
            0&0  &2 
          \end{pmatrix},
        \end{equation*}
        and therefore in this case the order of the element is two, too.
        
        In this case the condition for $A=(a_{ij})$ to be an element of $C_G(X)$
        is
        \begin{eqnarray*}
          \begin{pmatrix}
            a_{11}&a_{12}  &a_{13} \\ 
            a_{21}&a_{22}  &a_{23} \\ 
            a_{31}&a_{32}  &a_{33} 
          \end{pmatrix}
          \cdot
          \begin{pmatrix}
            1&0  &0 \\ 
            0&2  &0 \\ 
            0&0  &2 
          \end{pmatrix}
          &=&
          \begin{pmatrix}
            1&0  &0 \\ 
            0&2  &0 \\ 
            0&0  &2 
          \end{pmatrix}
          \begin{pmatrix}
            a_{11}&a_{12}  &a_{13} \\ 
            a_{21}&a_{22}  &a_{23} \\ 
            a_{31}&a_{32}  &a_{33} 
          \end{pmatrix}
          \\
          \begin{pmatrix}
            a_{11}&2a_{12}  &2a_{13} \\ 
            a_{21}&2a_{22}  &2a_{23} \\ 
            a_{31}&2a_{32}  &2a_{33} 
          \end{pmatrix}
          &=&
          \begin{pmatrix}
            a_{11}&a_{12}  &a_{13} \\ 
            2a_{21}&2a_{22}  &2a_{23} \\ 
            2a_{31}&2a_{32}  &2a_{33} 
          \end{pmatrix}
        \end{eqnarray*}
        or $a_{21}=a_{31}=a_{12}=a_{13}=0$. Hence by an identical counting argument
        here too 
        $|C_G(x)|=96$ and $|cl(X)|=11232/96=117$. We have thus covered the diagonalisable
        elements of $GL(3,3)$, since no other combination of non-zero eigenvalues
        exists. 
      \item
        The conjugacy class of the non-diagonalisable elements with 
        characteristic polynomial
        $(x+2)^2(x+1)$ has as representative the element
        \begin{equation*}
          X=
          \begin{pmatrix}
            1&1  &0 \\ 
            0&1  &0 \\ 
            0&0  &2 
          \end{pmatrix}.
        \end{equation*}
        Let $M=\begin{pmatrix}1&1\\0&1\end{pmatrix}$. Then 
        $M^3=I\neq M^2$. Since $X=\begin{pmatrix}M&\\&2\end{pmatrix}$ then the order of $X$ is six, and this is
        also the order of the rest of the elements in this class.
        
        As above, let $A$ be an element of $C_G(X)$. Then:
        
        \begin{eqnarray*}
          \begin{pmatrix}
            a_{11}&a_{12}  &a_{13} \\ 
            a_{21}&a_{22}  &a_{23} \\ 
            a_{31}&a_{32}  &a_{33} 
          \end{pmatrix}
          \cdot
          \begin{pmatrix}
            1&1  &0 \\ 
            0&1  &0 \\ 
            0&0  &2 
          \end{pmatrix}
          &=&
          \begin{pmatrix}
            1&1  &0 \\ 
            0&1  &0 \\ 
            0&0  &2 
          \end{pmatrix}
          \begin{pmatrix}
            a_{11}&a_{12}  &a_{13} \\ 
            a_{21}&a_{22}  &a_{23} \\ 
            a_{31}&a_{32}  &a_{33} 
          \end{pmatrix}
          \\
          \begin{pmatrix}
            a_{11}&a_{11}+a_{12}  &2a_{13} \\ 
            a_{21}&a_{21}+a_{22}  &2a_{23} \\ 
            a_{31}&a_{31}+a_{32}  &2a_{33} 
          \end{pmatrix}
          &=&
          \begin{pmatrix}
            a_{11}+a_{21}&a_{12}+a_{22}  &a_{13}+a_{23} \\ 
            a_{21}&a_{22}  &a_{23} \\ 
            2a_{31}&2a_{32}  &2a_{33} 
          \end{pmatrix}.
        \end{eqnarray*}
        Hence $a_{21}=a_{31}=a_{13}=a_{23}=a_{32}=0$ and $a_{11}=a_{22}$. 
        $A$ can therefore be written as
        \begin{equation*}
          A=
          \begin{pmatrix}
            a &b  &0 \\ 
            0 &a  &0 \\ 
            0 &0  &c 
          \end{pmatrix},
        \end{equation*}
        with $ac\neq 0$, since otherwise the matrix would be singular. There are
        thus four options for the pair $(a,c)$ and three for $b$, and so
        $|C_G(x)|=12$. Hence $|cl(X)|=11232/12=936$. 
        
      \item
        We use an essentially identical analysis for the conjugacy class
        of the non-diagonalisables with characteristic polynomial $(x+1)^2(x+2)$.
        The representative is
        \begin{equation*}
          X=
          \begin{pmatrix}
            2&1  &0 \\ 
            0&2  &0 \\ 
            0&0  &1 
          \end{pmatrix},
        \end{equation*}
        which by an identical argument also has order six. The condition for
        commutativity is:
        
        
        \begin{eqnarray*}
          \begin{pmatrix}
            a_{11}&a_{12}  &a_{13} \\ 
            a_{21}&a_{22}  &a_{23} \\ 
            a_{31}&a_{32}  &a_{33} 
          \end{pmatrix}
          \cdot
          \begin{pmatrix}
            2&1  &0 \\ 
            0&2  &0 \\ 
            0&0  &1
          \end{pmatrix}
          &=&
          \begin{pmatrix}
            2&1  &0 \\ 
            0&2  &0 \\ 
            0&0  &1
          \end{pmatrix}
          \begin{pmatrix}
            a_{11}&a_{12}  &a_{13} \\ 
            a_{21}&a_{22}  &a_{23} \\ 
            a_{31}&a_{32}  &a_{33} 
          \end{pmatrix}
          \\
          \begin{pmatrix}
            2a_{11}&a_{11}+2a_{12}  &a_{13} \\ 
            2a_{21}&a_{21}+2a_{22}  &a_{23} \\ 
            2a_{31}&a_{31}+2a_{32}  &a_{33} 
          \end{pmatrix}
          &=&
          \begin{pmatrix}
            2a_{11}+a_{21}&2a_{12}+a_{22}  &2a_{13}+a_{23} \\ 
            2a_{21}&2a_{22}  &2a_{23} \\ 
            a_{31}&a_{32}  &a_{33} 
          \end{pmatrix}
        \end{eqnarray*}
        which implies the same form for $A$ as above. Hence in this case, too,
        $|C_G(X)|=12$ and $|cl(X)|=11232/12=936$.
      \item
        The first conjugacy class of the non-diagonalisable elements with 
        characteristic polynomial
        $(x+2)^3$ has as representative the element
        \begin{equation*}
          X=
          \begin{pmatrix}
            1&1  &0 \\ 
            0&1  &0 \\ 
            0&0  &1 
          \end{pmatrix}.
        \end{equation*}
        Here the order is three, as can be seen by carrying out the 
        calculations for $X^2$ and $X^3$.

        Let $A$ be an element of $C_G(X)$. Then:
        
        \begin{eqnarray*}
          \begin{pmatrix}
            a_{11}&a_{12}  &a_{13} \\ 
            a_{21}&a_{22}  &a_{23} \\ 
            a_{31}&a_{32}  &a_{33} 
          \end{pmatrix}
          \cdot
          \begin{pmatrix}
            1&1  &0 \\ 
            0&1  &0 \\ 
            0&0  &1 
          \end{pmatrix}
          &=&
          \begin{pmatrix}
            1&1  &0 \\ 
            0&1  &0 \\ 
            0&0  &1 
          \end{pmatrix}
          \begin{pmatrix}
            a_{11}&a_{12}  &a_{13} \\ 
            a_{21}&a_{22}  &a_{23} \\ 
            a_{31}&a_{32}  &a_{33} 
          \end{pmatrix}
          \\
          \begin{pmatrix}
            a_{11}&a_{11}+a_{12}  &a_{13} \\ 
            a_{21}&a_{21}+a_{22}  &a_{23} \\ 
            a_{31}&a_{31}+a_{32}  &a_{33} 
          \end{pmatrix}
          &=&
          \begin{pmatrix}
            a_{11}+a_{21}&a_{12}+a_{22}  &a_{13}+a_{23} \\ 
            a_{21}&a_{22}  &a_{23} \\ 
            a_{31}&a_{32}  &a_{33} 
          \end{pmatrix}.
        \end{eqnarray*}
        Hence $a_{21}=a_{23}=a_{31}=0$, $a_{11}=a_{22}$, and $a_{12}=a_{23}$. 
        $A$ can therefore be written as
        \begin{equation*}
          A=
          \begin{pmatrix}
            a &c  &d \\ 
            0 &a  &0 \\ 
            0 &e  &b 
          \end{pmatrix}.
        \end{equation*}
        Neither $a$ nor $b$ can be zero, since $\det A=a^2b$. Hence there
        are $2^23^3=108$ possible such $A$'s, the size of $C_G(X)$ is 108
        and that of  $cl(X)$ is $11232/108=104$.

        The first conjugacy class of the non-diagonalisable elements with 
        characteristic polynomial
        $(x+1)^3$ has as representative the element
        \begin{equation*}
          Y=
          \begin{pmatrix}
            2&1  &0 \\ 
            0&2  &0 \\ 
            0&0  &2 
          \end{pmatrix}.
        \end{equation*}
        Since $Y=X+I$ then $Y^3=X^3+3X^2+3X+I=2I$ and the order of $Y$ is six.
        Also by the property $Y=X+I$ we have that $C_G(X)=C_G(Y)$, and so
        $|cl(Y)|=|cl(X)|$.
        
      \item
        The second conjugacy class of the non-diagonalisable elements with 
        characteristic polynomial
        $(x+2)^3$ has as representative the element
        \begin{equation*}
          X=
          \begin{pmatrix}
            1&1  &0 \\ 
            0&1  &1 \\ 
            0&0  &1 
          \end{pmatrix}.
        \end{equation*}
        Here too the order is three, as again can be seen by carrying out the 
        calculations for $X^2$ and $X^3$.

        Let $A$ be an element of $C_G(X)$. Then:
        
        \begin{eqnarray*}
          \begin{pmatrix}
            a_{11}&a_{12}  &a_{13} \\ 
            a_{21}&a_{22}  &a_{23} \\ 
            a_{31}&a_{32}  &a_{33} 
          \end{pmatrix}
          \cdot
          \begin{pmatrix}
            1&1  &0 \\ 
            0&1  &1 \\ 
            0&0  &1 
          \end{pmatrix}
          &=&
          \begin{pmatrix}
            1&1  &0 \\ 
            0&1  &1 \\ 
            0&0  &1 
          \end{pmatrix}
          \begin{pmatrix}
            a_{11}&a_{12}  &a_{13} \\ 
            a_{21}&a_{22}  &a_{23} \\ 
            a_{31}&a_{32}  &a_{33} 
          \end{pmatrix}
          \\
          \begin{pmatrix}
            a_{11}&a_{11}+a_{12}  &a_{12}+a_{13} \\ 
            a_{21}&a_{21}+a_{22}  &a_{22}+a_{23} \\ 
            a_{31}&a_{31}+a_{32}  &a_{32}+a_{33} 
          \end{pmatrix}
          &=&
          \begin{pmatrix}
            a_{11}+a_{21}&a_{12}+a_{22}  &a_{13}+a_{23} \\ 
            a_{21}+a_{31}&a_{22}+a_{32}  &a_{23}+a_{33} \\ 
            a_{31}&a_{32}  &a_{33} 
          \end{pmatrix}.
        \end{eqnarray*}
        Hence $a_{21}=a_{23}=a_{31}=a_{32}=0$ $a_{23}=a_{12}$ and 
        $a_{11}=a_{22}=a_{33}$, 
        $A$ can therefore be written as
        \begin{equation*}
          A=
          \begin{pmatrix}
            a &b  &c \\ 
            0 &a  &b \\ 
            0 &0  &a 
          \end{pmatrix}.
        \end{equation*}
        $a$ cannot be zero, since $\det A=a^3$. Hence there
        are $23^2=18$ possible such $A$'s, the size of $C_G(X)$ is 18
        and that of  $cl(X)$ is $11232/108=624$.

        The second conjugacy class of the non-diagonalisable elements with 
        characteristic polynomial
        $(x+1)^3$ has as representative the element
        \begin{equation*}
          Y=
          \begin{pmatrix}
            2&1  &0 \\ 
            0&2  &1 \\ 
            0&0  &2 
          \end{pmatrix}.
        \end{equation*}
        As above, $Y=X+I$ and so order of $Y$ is six,
        $C_G(X)=C_G(Y)$, and so $|cl(Y)|=|cl(X)|$.
        

        We have thus concluded the analysis of all the
        conjugacy classes of elements with characteristic polynomials 
        reducible over $GF(3)$. The rest of the classes will therefore not have
        representitives in Jordan normal form.


      \item
        Let $p(x)=x^3+ax^2+bx+c$ be one of the eight irreducible monic 
        cubic polynomials listed above. Then the following element
        of $GL(3,3)$ has characteristic polynomial $p(x)$:
        \begin{equation*}
          X_p=
          \begin{pmatrix}
            0&1  &0 \\ 
            0&0  &1 \\ 
            -c&-b  &-a 
          \end{pmatrix}.
        \end{equation*}
        Now by the second table in~\ref{table:polys}, $p(x)$ has three 
        distinct roots, and so there is only one possible Jordan form for $X_p$
        over $GF(27)$, where it is diagnolisable.
        Hence the elements with characteristic polynomial $p(x)$
        form a conjugacy class.
        
        Let $X_p'$ be the diagonalised matrix over $GF(27)$ that is similar
        to $X_p$. Then again by the table in~\ref{table:polys}, $X_p'$ has
        the values $a^i$, $a^{3i}$ and $a^{9i}$ on its diagonal, with $1 \leq i \leq 25$ and
        $i \neq 13$. The value of $i$ depends on $p(x)$, by the said table.
        The order of $X_p'$ is the order of $a^i$
        and is thus equal to $26$ when $i$ is odd and to 13 when $i$ is even.
        Equivalently, the order equals
        26 when the free coefficient of $p(x)$ is 1, and 13 otherwise.
        The order of $X_p$ is the order of $X_p'$, 
        since similar matrices have identical orders.

        

        Let $Y=(y_{ij})$ be an element of $C_G(X_p)$. Then:
        
        \begin{eqnarray*}
          \begin{pmatrix}
            y_{11}&y_{12}  &y_{13} \\ 
            y_{21}&y_{22}  &y_{23} \\ 
            y_{31}&y_{32}  &y_{33} 
          \end{pmatrix}
          \cdot
          \begin{pmatrix}
            0&1  &0 \\ 
            0&0  &1 \\ 
            -c&-b  &-a 
          \end{pmatrix}
          &=&
          \begin{pmatrix}
            0&1  &0 \\ 
            0&0  &1 \\ 
            -c&-b  &-a 
          \end{pmatrix}
          \begin{pmatrix}
            y_{11}&y_{12}  &y_{13} \\ 
            y_{21}&y_{22}  &y_{23} \\ 
            y_{31}&y_{32}  &y_{33} 
          \end{pmatrix}
        \end{eqnarray*}
        \begin{eqnarray*}
          \lefteqn{
            \begin{pmatrix}
              -cy_{13}&y_{11}-by_{13}  &y_{12}-ay_{13} \\ 
              -cy_{23}&y_{21}-by_{23}  &y_{22}-ay_{23} \\ 
              -cy_{33}&y_{31}-by_{33}  &y_{32}-ay_{33} 
            \end{pmatrix}
            =}
          \\
          &&
          \begin{pmatrix}
            y_{21}&y_{22}  &y_{23} \\ 
            y_{31}&y_{32}  &y_{33} \\ 
            -cy_{11}-by_{21}-ay_{31}&-cy_{12}-by_{22}-ay_{32}&-cy_{13}-by_{23}-ay_{33}
          \end{pmatrix}.
        \end{eqnarray*}
        We first consider the case where $a=0$. Then by examining the 
        polynomials take we see that $b$ must equal two. Hence
        \begin{eqnarray*}
          \lefteqn{
            \begin{pmatrix}
              -cy_{13}&y_{11}+y_{13}  &y_{12} \\ 
              -cy_{23}&y_{21}+y_{23}  &y_{22} \\ 
              -cy_{33}&y_{31}+y_{33}  &y_{32} 
            \end{pmatrix}
            =}
          \\
          &&
          \begin{pmatrix}
            y_{21}&y_{22}  &y_{23} \\ 
            y_{31}&y_{32}  &y_{33} \\ 
            -cy_{11}+y_{21}&-cy_{12}+y_{22}&-cy_{13}+y_{23}
          \end{pmatrix}.
        \end{eqnarray*}

        We write this equation as a set of nine linear equations on the nine
        variables $y_{ij}$:
        \begin{eqnarray*}
          \begin{pmatrix}
            0& 0& -c& -1& 0& 0& 0& 0& 0 \\
            0& 0& 0& 0& 0& 0& 0& 0& 0 \\
            0& 0& 0& 0& 0& 0& 0& 0& 0 \\
            0& 0& 0& 0& 0& 0& 0& 0& 0 \\
            0& 0& 0& 0& 0& 0& 0& 0& 0 \\
            0& 0& 0& 0& 0& 0& 0& 0& 0 \\
            0& 0& 0& 0& 0& 0& 0& 0& 0 \\
            0& 0& 0& 0& 0& 0& 0& 0& 0 \\
            0& 0& 0& 0& 0& 0& 0& 0& 0 \\
          \end{pmatrix}
        \end{eqnarray*}
        
        Since $p(x)$ has a non-zero free coefficient then $c\in GF(3)^*$, and so
        we can divide by it, which in $GF(3)^*$ is the same as multiplication.

        We now consider several cases. $c$ may take values only in $\{1,2\}$,
        since $p(x)$ has a non-zero free coefficient. We first consider the
        case of $c=1$. Then the commutativity condition above is:
        \begin{eqnarray*}
          \lefteqn{
            \begin{pmatrix}
              -y_{13}&y_{11}-by_{13}  &y_{12}-ay_{13} \\ 
              -y_{23}&y_{21}-by_{23}  &y_{22}-ay_{23} \\ 
              -y_{33}&y_{31}-by_{33}  &y_{32}-ay_{33} 
            \end{pmatrix}
            =}
          \\
          &&
          \begin{pmatrix}
            y_{21}&y_{22}  &y_{23} \\ 
            y_{31}&y_{32}  &y_{33} \\ 
            -y_{11}-by_{21}-ay_{31}&-y_{12}-by_{22}-ay_{32}&-y_{13}-by_{23}-ay_{33}
          \end{pmatrix}.
        \end{eqnarray*}
        
      \end{itemize}
      The information above is summarised in the following table:

      {\bf Conjugacy Classes of $GL(3,3)$}

      \begin{tabular}{l|l|l|l|l}
        Representative& Characteristic Polynomial& Element Order& Class Size& Centralizer Size\\
        \hline
          $\begin{pmatrix}
            1&0  &0 \\ 
            0&1  &0 \\ 
            0&0  &1 
          \end{pmatrix}$
          & $(x+2)^3$ & 1 & 1 &11232\\
        \hline
          $\begin{pmatrix}
            2&0  &0 \\ 
            0&2  &0 \\ 
            0&0  &2 
          \end{pmatrix}$
          & $(x+1)^3$ & 2 & 1&11232\\
        \hline
          $\begin{pmatrix}
            1&0  &0 \\ 
            0&1  &0 \\ 
            0&0  &2 
          \end{pmatrix}$
          & $(x+2)^2(x+1)$ & 2 & 117&96\\
        \hline
          $\begin{pmatrix}
            1&0  &0 \\ 
            0&2  &0 \\ 
            0&0  &2 
          \end{pmatrix}$
          & $(x+1)^2(x+2)$ & 2 & 117&96\\
        \hline
          $\begin{pmatrix}
            1&1  &0 \\ 
            0&1  &0 \\ 
            0&0  &2 
          \end{pmatrix}$
          & $(x+2)^2(x+1)$ & 6 & 936&12\\
        \hline
          $\begin{pmatrix}
            2&1  &0 \\ 
            0&2  &0 \\ 
            0&0  &1 
          \end{pmatrix}$
          & $(x+1)^2(x+2)$ & 6 & 936&12\\
        \hline
          $\begin{pmatrix}
            1&1  &0 \\ 
            0&1  &0 \\ 
            0&0  &1 
          \end{pmatrix}$
          & $(x+2)^3$ & 3 & 104&108\\
        \hline
          $\begin{pmatrix}
            2&1  &0 \\ 
            0&2  &0 \\ 
            0&0  &2 
          \end{pmatrix}$
          & $(x+1)^3$ & 6 & 104&108\\
        \hline
          $\begin{pmatrix}
            1&1  &0 \\ 
            0&1  &1 \\ 
            0&0  &1 
          \end{pmatrix}$
          & $(x+1)^3$ & 3 & 624&18\\
        \hline
          $\begin{pmatrix}
            2&1  &0 \\ 
            0&2  &1 \\ 
            0&0  &2 
          \end{pmatrix}$
          & $(x+2)^3$ & 6 & 624&18\\
      \end{tabular}
    \item
    The size of $|GL(3,3)|$ is $3^3(3^3-1)(3^2-1)(3-1)=11232$. The prime
    factorisation of $11232$ is 
    $27 \cdot 26 \cdot 8 \cdot 2=3^3 \cdot 13 \cdot 2^5$. Therefore $GL(3,3)$ has three Sylow $p$-subgroups:
    a 3-subgroup of order 27, a 13-subgroup of order 13, and a
    $2$-subgroup of order 32. 

    \begin{itemize}
    \item {\bf Sylow 3-subgroups}
      The number of Sylow 3-subgroups, $n_3$, equals to $1\mod 3$ and divides
      $13\cdot 32=416$. It can therefore be one of $\{1,13,52,208\}$. The total number
      of elements of order three is $104+624=728$. The most
      elements these groups may include is $(27-1)n_3+1$, if they all 
      intersect trivially. Hence there cannot be only one or 13 of them, since 
      they would not include enough elements.

    \item {\bf Sylow 13-subgroups}
      
      The number of Sylow 13-subgroups, $n_{13}$, equals to $1\mod 13$ and 
      divides $27\cdot 32=864$. It can therefore be one of $\{1,27,144\}$. Since
      there are four conjugacy classes of elements of order 13, there there
      are a total of $4\cdot 432=1728$ elements of order 13. Since each two
      of these groups intersect trivially, there have to be $1728/12=144$ of
      them, and $n_{13}=144$.

    \item {\bf Sylow 2-subgroups}
      
      The number of Sylow 2-subgroups, $n_2$, equals to $1\mod 2$ and divides
      $27\cdot 13=351$. It can therefore be one of 
      $\{1,3,9,13,27,39,117,351\}$. There are 235 elements of order two.
    \end{itemize}
    \end{enumerate}
  \item
  \item
    We commence by proving some auxiliary claims. 

    We showed in a homework assignment that a group of order $p^2$, where
    $p$ is prime, can only be $C_{p^2}$ or $C_p \times C_p$. We prove here
    two similar claims:
    \begin{proposition}
      \label{prop:byprime}
      Let $p,q$ be distinct primes greater than two, and let $G$ be a group 
      of order $pq$. Then $G$ is Abelian.
    \end{proposition}
    \begin{proof}
      By the Sylow theorems $G$ has subgroups of order $p$ and $q$ --- its 
      Sylow $p$- and $q$-subgroups.
      Being of prime order, these must be cyclic.
      
      The number of Sylow $p$-subgroups, 
      being equal to $1\mod p$ (an even number, since $p$ is larger than two
      and therefore odd) and dividing $q$ (an odd number), 
      has to equal one. Call this group $P$ and let $a$ be a generator.
      
      By the same argument the number of Sylow $q$-subgroups is also one.
      Call this group $Q$ and let $b$ be a generator.
      
      Consider the element $c=a^{-1}bab^{-1}=b^ab^{-1}=a^ba^{-1}$. Since
      $b^a$ is of order $q$ then it must also be a member of $Q$, and then
      so is $c$. By the same argument for $a^b$ we have that $c \in P$. Since
      the two groups have co-prime orders they intersect trivially and $c=1$. 
      Hence $a$ and $b$ commute, the $pq$ elements of the group can be written 
      as $a^ib^j$ for $0\leq i <p$ and $0\leq j < q$, and the group is Abelian.
    \end{proof}
    \begin{corollary}
      \label{prop:2p_groups}
      Let $p$ be a prime greater than two. Let $G$ be a group of order
      $2p$. Then if $G$ is not Abelian then it has a single Sylow $p$-subgroup
      and $p$ Sylow $2$-subgroups.
    \end{corollary}
    \begin{proof}
      Using the same notation as above and the same first argument, we claim
      that $G$ has a single Sylow $p$-subgroup. It would be Abelian if it were 
      to have a single Sylow $2$-subgroup, as well. The only other option for
      $n_2$, considering that it divides $p$, is for it to equal $p$. Hence
      $G$ has $p$ Sylow $2$-subgroups.
    \end{proof}

    We can use these claims to prove the following:
    \begin{proposition}
      \label{prop:2p_unique}
      A group of order $2p$, with $p>2$ prime, is either Abelian or isomorphic 
      to the single non-Abelian
      group of order $2p$. In the latter case multiplication is defined by
      $a^ib=ba^{-1}$ where $b$ is an element of order two and $a$ is a generator
      of the Sylow $p$-subgroup.
    \end{proposition}
    \begin{proof}
      A group $G$ of order $2p$, for $p>2$, includes $p$ elements that 
      belong to the Sylow $p$-group,
      which can be represented as $a^i$ for some $a$, and $p$ elements of order
      two, which can be represented as $a^ib$ for some $b$ of order two. As
      elements of order two they are their own inverses, and so $a^iba^ib=1$ and
      $a^ib=ba^{-i}$. This defines multiplication (and hence the group) 
      unambiguously: $a^ia^j=a^{i+j}$,
      $(a^ib)a^j=a^{i-j}b$, $a^i(a^jb)=a^{i+j}b$ and $(a^ib)(a^jb)=a^{i-j}b$.
    \end{proof}

    Before starting to describe the subgroups of $A_5$, we classify its
    elements.
    The elements of $A_5$ are defined to be the even permutations of size five.
    We represent each element of $A_5$
    as a collection of disjoint cycles. For a permutation on five elements
    to be even, it must either:
    \begin{itemize}
    \item Be the identity permutation. There is one such permutation.
    \item Be two disjoint cycles of size two, e.g. $(12)(34)$. The number of 
      such permutations
      is fifteen (choose a number to leave out and then choose a cycle member
      for the lowest number left). These elements are of order two.
    \item Be a cycle of size three, e.g. $(123)$. There are 
      $2{5 \choose 3}=20$ such cycles. These elements are of order three.
    \item Be a cycle of size five, e.g. $(12345)$. There are $4!=24$ such 
      cycles. These elements are of order five.
    \end{itemize}
    The total number of elements listed above is sixty, and so all are accounted
    for.

    A last fact that we use below is that a permutation can also be viewed
    as a automorphism of $A_5$ - it merely changes the names of the items 
    being permuted. Hence any element of order $p$
    is a ``generic'' element of order $p$, in the sense that any statement
    about it (which doesn't include relations to other elements) is true
    for all other elements of order $p$. For example, if a subgroup $G$ is said
    to ``have an element of order five'' then either it or some isomorphic
    group has the element $(12345)$.

    We are now ready to turn to the actual problem of classifying the subgroups
    of $A_5$.
    The prime factorisation of sixty is $60=2^2\cdot 3\cdot 5$. Hence its non-trivial
    divisors are $2,3,4,5,6,10,12,15,20$ and $30$. We list below the subgroups
    of each of these orders (up to isomorphism), or prove they don't exist.

    \begin{itemize}
    \item {\bf 2, 3, 5}
      
      For a prime divisor $p\in\{2,3,5\}$, a group of order $p$ exists 
      (by the Sylow theorems) and is generated by
      an element of order $p$. Hence for each of the elements of order 
      $p\in\{2,3,5\}$ listed above
      corresponds a subgroup of order $p$.

    \item {\bf 4}
      
      A Sylow $2$-subgroup of $A_5$, of order four,
      exists and is unique up to isomorphism, by the Sylow theorems. It can
      be isomorphic only to $C_2\times C_2$ or to $C_4$, as we showed in class
      for groups of order $p^2$. However, there is no element of order four
      in $A_5$ (we listed the elements and their orders above) and therefore
      the only possibility is $C_2 \times C_2$. 
      
    \item {\bf 6}
      Were there to exist an Abelian subgroup of order six then it would have
      to be $C_6$ and have an element of order six. Since none exists then
      any subgroup of order six is not Abelian.
      
      By proposition~\ref{prop:2p_unique}, only one non-Abelian group of order 
      six exists --- $S_3$. This group has
      a single Sylow $3$-subgroup and three Sylow $2$-subgroups. An example
      of an $S_3$ in $A_5$ is one in which $\sigma=(1 2 3)$ generates the Sylow
      $3$-subgroup, and the elements of the form $\tau_{ij}=(ij)(45)$, with 
      $i,j \in \{1,2,3\}$, generate the Sylow $2$-subgroups. 

      This group includes any possible cycle on $\{1,2,3\}$ and is closed under
      multiplication (it clearly cannot involve any action on $\{4,5\}$ other
      than $(45)$) and so is isomorphic to $S_3$.

    \item {\bf 10}
      By the same argument as above, any Abelian subgroup of order ten would
      have an element of order ten, and since none exist then  
      any subgroup of order ten is not Abelian. Also as above, only one such
      group exists. It has a single Sylow $5$-subgroup and five Sylow 
      $2$-subgroups.

      Let $\sigma=(12345)$ generate the Sylow 
      $5$-subgroup. Let 
      $\tau_i=\left(ii^{\sigma^3}\right)\left(i^{\sigma}i^{\sigma^2}\right)$ 
      be an element of $A_5$. Then there are five possible
      choices for $\tau$, which generate the five Sylow $2$-subgroups. Furthermore, 
      $\tau_i^{-1}\sigma\tau_i=(54321)=\sigma^{-1}$, as 
      required in proposition~\ref{prop:2p_unique}.

    \item {\bf 15}
      
      By proposition~\ref{prop:byprime} above, any subgroup of order $15=3\cdot 5$
      is Abelian, and therefore can only be isomorphic to 
      $C_5\times C_3=C_{15}$. 

      If such a subgroup were to exist it would have an element of order 15. 
      Since no such element exists then $A_5$ has no subgroup of order 15.

    \item {\bf 12}
      
      Let $G$ be a subgroup of order 12. Then $G$ has a Sylow $3$-subgroup,
      of order three, and a Sylow $2$-subgroup, of order four. Since no 
      element of order four exists, then the Sylow $2$-subgroup must be 
      $C_2\times C_2$. $G$ cannot be Abelian, since it would have to have 
      an element of order six.

      The numbers of Sylow subgroups, $n_3$ and $n_4$, cannot therefore both
      be one. By the Sylow theorems, $n_3$ must equal $1\mod 3$ and divide
      four, and so is one or four. $n_4$, which must equal $1\mod 2$ and divide
      three, has to equal one or three. There are not enough element in $G$
      for both $n_3$ and $n_4$ to be greater than one, and so precisely one
      of them equals one.

      Assume first that $n_3=1$. Then WLOG let $(123)$ generate the single,
      normal Sylow $3$-subgroup $P_3$. However, the only elements $\tau$ of
      order two for which $\tau^{-1}P_3\tau=P_3$ are of the form $(ii^{(123)})(45)$ with
      $i\in\{1,2,3\}$. There are only three such elements, and so it is not
      possible that $n_3=1$.

      The only possibility left is that $n_3=4$ and $n_4=1$. This describes
      $A_4$, which is, of course, a subgroup of $A_5$. In $A_4$ the elements that
      generate the Sylow $3$-groups are the three cycles: $(123)$, $(124)$ and 
      $(234)$. The elements of the Sylow $2$-group are the identity and the 
      pairs of transpositions: $(12)(34)$, $(13)(24)$ and $(14)(23)$. 
      
    \item {\bf 20}
      Let $G$ be a subgroup of order 20.
      As above, the Sylow $2$-subgroup would
      be $C_2\times C_2$. 
      Also as above, $G$  cannot be Abelian, since it would have
      to have a element of order ten.

      The number of Sylow $5$-subgroups equals $1\mod 5$ and divides four,
      and so must equal one. Hence the number of Sylow $2$-subgroups cannot
      equal one as well. Since it is odd and divides five then it must equal
      five. Hence $G$ would have an identity element, five elements of order 
      two and four elements of order one - not enough to make 20 elements.
      Hence no subgroup of order 20 exists.

    \item {\bf 30}
      Let $G$ be a subgroup of order 30. Then $G$ has a Sylow 
      $2$-subgroup, a Sylow $3$-subgroup and a Sylow $5$-subgroup. 
      It is impossible that two of them are normal, since their
      Cartesian product would be an Abelian subgroup of $G$ with order 
      greater than five, and again no element of such order exists.

      The number of Sylow $5$-subgroups, $n_5$, equals $1\mod 5$ and 
      divides six. Hence it may equal one or six.

      The number of Sylow $3$-subgroups, $n_3$, equals $1\mod 3$ and 
      divides ten. Hence it may equal one, or ten.

      The number of Sylow $2$-subgroups, $n_2$, equals $1\mod 2$ and 
      divides fifteen. Hence it may equal one, three, five or fifteen.

      Only one of $n_5$, $n_3$ and $n_2$ may be one, by the argument above.

      Assume first that $n_2=1$. Then $n_5=6$ and $n_3=10$. This makes too 
      many elements, and so $n_2=1$ is an impossibility. By the same argument
      it is impossible that all three of $n_5$, $n_3$ and $n_2$ are greater 
      than one.

      Next assume $n_5=1$. Then $n_3=10$ and so $G$ includes all of $A_5$'s 
      elements of order three, but only four element of order five. WLOG,
      let $(12453)$ {\em not} belong to $G$. But $(12453)=(123)(345)$ --- contradiction.

      Finally let $n_3=1$. Then $n_5=6$, and to make the number of elements
      equal 30 we must have $n_2=3$. In this case $G$ includes all the elements
      of order five, but not all the elements of order two. WLOG let 
      $(13)(25)$ {\em not} belong to $G$. But $(12345)(12354)=(13)(25)$ ---
      contradiction. 

      We have ruled out all the possibilities for the number of Sylow subgroups
      of $G$, and so $G$ is fictitious and $A_5$ has no subgroup of order 30.
    \end{itemize}
\end{enumerate}
\end{document}


