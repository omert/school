\documentclass[11pt]{article} \usepackage{amssymb}
\usepackage{amsfonts} \usepackage{amsmath} \usepackage{bm}
\usepackage{latexsym} \usepackage{epsfig}

\setlength{\textwidth}{6.5 in} \setlength{\textheight}{8.25in}
\setlength{\oddsidemargin}{0in} \setlength{\topmargin}{0in}
\addtolength{\textheight}{.8in} \addtolength{\voffset}{-.5in}

\newtheorem{theorem}{Theorem}[section]
\newtheorem{lemma}[theorem]{Lemma}
\newtheorem{proposition}[theorem]{Proposition}
\newtheorem{corollary}[theorem]{Corollary}
\newtheorem{fact}[theorem]{Fact}
\newtheorem{definition}[theorem]{Definition}
\newtheorem{remark}[theorem]{Remark}
\newtheorem{conjecture}[theorem]{Conjecture}
\newtheorem{claim}[theorem]{Claim}
\newtheorem{example}[theorem]{Example}
\newenvironment{proof}{\noindent \textbf{Proof:}}{$\Box$}

\newcommand{\ignore}[1]{}

\newcommand{\enote}[1]{} \newcommand{\knote}[1]{}
\newcommand{\rnote}[1]{}



% \newcommand{\enote}[1]{{\bf [[Elchanan:} {\emph{#1}}{\bf ]]}}
% \newcommand{\knote}[1]{{\bf [[Krzysztof:} {\emph{#1}}{\bf ]]}}
% \newcommand{\rnote}[1]{{\bf [[Ryan:} {\emph{#1}}{\bf ]]}}



\DeclareMathOperator{\Support}{Supp} \DeclareMathOperator{\Opt}{Opt}
\DeclareMathOperator{\Ordo}{\mathcal{O}}
\newcommand{\MaxkCSP}{\textsc{Max $k$-CSP}}
\newcommand{\MaxkCSPq}{\textsc{Max $k$-CSP$_{q}$}}
\newcommand{\MaxCSP}[1]{\textsc{Max CSP}(#1)} \renewcommand{\Pr}{{\bf
    P}} \renewcommand{\P}{{\bf P}} \newcommand{\Px}{\mathop{\bf P\/}}
\newcommand{\E}{{\bf E}} \newcommand{\Cov}{{\bf Cov}}
\newcommand{\Var}{{\bf Var}} \newcommand{\Varx}{\mathop{\bf Var\/}}

\newcommand{\bits}{\{-1,1\}}

\newcommand{\nsmaja}{\textstyle{\frac{2}{\pi}} \arcsin \rho}

\newcommand{\Inf}{\mathrm{Inf}} \newcommand{\I}{\mathrm{I}}
\newcommand{\J}{\mathrm{J}}

\newcommand{\eps}{\epsilon} \newcommand{\lam}{\lambda}

% \newcommand{\trunc}{\ell_{2,[-1,1]}}
\newcommand{\trunc}{\zeta} \newcommand{\truncprod}{\chi}

\newcommand{\N}{\mathbb N} \newcommand{\R}{\mathbb R}
\newcommand{\Q}{\mathbb Q}
\newcommand{\Z}{\mathbb Z} \newcommand{\CalE}{{\mathcal{E}}}
\newcommand{\CalC}{{\mathcal{C}}} \newcommand{\CalM}{{\mathcal{M}}}
\newcommand{\CalR}{{\mathcal{R}}} \newcommand{\CalS}{{\mathcal{S}}}
\newcommand{\CalV}{{\mathcal{V}}}
\newcommand{\CalX}{{\boldsymbol{\mathcal{X}}}}
\newcommand{\CalG}{{\boldsymbol{\mathcal{G}}}}
\newcommand{\CalH}{{\boldsymbol{\mathcal{H}}}}
\newcommand{\CalY}{{\boldsymbol{\mathcal{Y}}}}
\newcommand{\CalZ}{{\boldsymbol{\mathcal{Z}}}}
\newcommand{\CalW}{{\boldsymbol{\mathcal{W}}}}
\newcommand{\CalF}{{\mathcal{Z}}}
% \newcommand{\boldG}{{\boldsymbol G}}
% \newcommand{\boldQ}{{\boldsymbol Q}}
% \newcommand{\boldP}{{\boldsymbol P}}
% \newcommand{\boldR}{{\boldsymbol R}}
% \newcommand{\boldS}{{\boldsymbol S}}
% \newcommand{\boldX}{{\boldsymbol X}}
% \newcommand{\boldB}{{\boldsymbol B}}
% \newcommand{\boldY}{{\boldsymbol Y}}
% \newcommand{\boldZ}{{\boldsymbol Z}}
% \newcommand{\boldV}{{\boldsymbol V}}
\newcommand{\boldi}{{\boldsymbol i}} \newcommand{\boldj}{{\boldsymbol
    j}} \newcommand{\boldk}{{\boldsymbol k}}
\newcommand{\boldr}{{\boldsymbol r}}
\newcommand{\boldsigma}{{\boldsymbol \sigma}}
\newcommand{\boldupsilon}{{\boldsymbol \upsilon}}
\newcommand{\hone}{{\boldsymbol{H1}}}
\newcommand{\htwo}{\boldsymbol{H2}}
\newcommand{\hthree}{\boldsymbol{H3}}
\newcommand{\hfour}{\boldsymbol{H4}}


\newcommand{\sgn}{\mathrm{sgn}} \newcommand{\Maj}{\mathrm{Maj}}
\newcommand{\Acyc}{\mathrm{Acyc}}
\newcommand{\UniqMax}{\mathrm{UniqMax}}
\newcommand{\Thr}{\mathrm{Thr}} \newcommand{\littlesum}{{\textstyle
    \sum}}

\newcommand{\half}{{\textstyle \frac12}}
\newcommand{\third}{{\textstyle \frac13}}
\newcommand{\fourth}{{\textstyle \frac14}}

\newcommand{\Stab}{\mathbb{S}}
\newcommand{\StabThr}[2]{\Gamma_{#1}(#2)}
\newcommand{\StabThrmin}[2]{{\underline{\Gamma}}_{#1}(#2)}
\newcommand{\StabThrmax}[2]{{\overline{\Gamma}}_{#1}(#2)}
\newcommand{\TestFcn}{\Psi}

\renewcommand{\phi}{\varphi}

\begin{document}
\title{Algebra Through Examples - Exam}

 \author{Omer Tamuz, 035696574}
\maketitle


\begin{enumerate}
\item 
  \begin{enumerate}
  \item 
    Let $a$ be a root of $x^3+2x+1$ in $GF(27)$. Then the elements of $GF(27)^*$
    can be written as follows:

    \begin{tabular}{l| l}
      $a^{0}$ & $1$\\
      $a^{1}$ & $a$\\
      $a^{2}$ & $a^{2}$\\
      $a^{3}$ & $a+2$\\
      $a^{4}$ & $a^{2}+2a$\\
      $a^{5}$ & $2a^{2}+a+2$\\
      $a^{6}$ & $a^{2}+a+1$\\
      $a^{7}$ & $a^{2}+2a+2$\\
      $a^{8}$ & $2a^{2}+2$\\
      $a^{9}$ & $a+1$\\
      $a^{10}$ & $a^{2}+a$\\
      $a^{11}$ & $a^{2}+a+2$\\
      $a^{12}$ & $a^{2}+2$\\
      $a^{13}$ & $2$\\
      $a^{14}$ & $2a$\\
      $a^{15}$ & $2a^{2}$\\
      $a^{16}$ & $2a+1$\\
      $a^{17}$ & $2a^{2}+a$\\
      $a^{18}$ & $a^{2}+2a+1$\\
      $a^{19}$ & $2a^{2}+2a+2$\\
      $a^{20}$ & $2a^{2}+a+1$\\
      $a^{21}$ & $a^{2}+1$\\
      $a^{22}$ & $2a+2$\\
      $a^{23}$ & $2a^{2}+2a$\\
      $a^{24}$ & $2a^{2}+2a+1$\\
      $a^{25}$ & $2a^{2}+1$\\
    \end{tabular}
%    We note (to be used in the next question), that $x^3+2x+1$ is a minimum
%    polynomial of $a$. 
  \item
    We would like to factor $x^{27}-x$. We showed in class that for $p$ 
    prime and $n$ natural the 
    polynomial $x^{p^n}-x$ is the product of all the monic irreducible 
    polynomials of degrees that divide $n$ over $GF(p)$. Hence $x^{27}-x$
    is the product of the irreducible linear and cubic monic polynomials
    over $GF(3)$.

    The linear monics ($x$, $x+1$ and $x+2$) are all
    trivially irreducible. To find the cubic factors of $x^{27}-x$ we list all 
    the {\em reducible} cubics, and infer that the rest are irreducible. 
    We restrict
    ourselves to monic polynomials with non-zero free coefficients, of which
    there are $3 \cdot 3 \cdot 2=18$ --- one needs to choose coefficients 
    for $x^2$ and $x$ in $GF(3)$, and a free coefficient in $\{1,2\}$.
    
    Of the class of cubics described above, the ones reducible to linear
    coefficients are
    \begin{eqnarray*}
      (x + 1)^3 &=& x^3 + 1\\
      (x + 1)^2(x + 2) &=& x^3 + x^2 + 2x + 2\\
      (x + 1)(x + 2)^2 &=& x^3 + 2x^2 + 2x + 1\\
      (x + 2)^3 &=& x^3 + 2.
    \end{eqnarray*}

    To find the ones reducible to an irreducible quadratic and a linear 
    coefficient, we note that the {\em reducible} quadratics are
    \begin{eqnarray*}
      (x+1)^2 &=& x^2+2x+1\\
      (x+1)(x+2) &=& x^2+2\\
      (x+2)^2 &=& x^2+x+1\\
    \end{eqnarray*}
    and so the {\em irreducible} quadratics are  $x^2+1$, $x^2+2x+2$ and 
    $x^2+x+2$. Hence the monic cubics (with a free coefficient) reducible to a 
    quadratic and a linear are
    \begin{eqnarray*}
      (x^2 + 1)(x + 1) &=& x^3 + x^2 + x + 1\\
      (x^2 + 2x + 2)(x + 1) &=& x^3 + x + 2\\
      (x^2 + x + 2)(x + 1) &=& x^3 + 2x^2 + 2\\
      (x^2 + 1)(x + 2) &=& x^3 + 2x^2 + x + 2\\
      (x^2 + 2x + 2)(x + 2) &=& x^3 + x^2 + 1\\
      (x^2 + x + 2)(x + 2) &=& x^3 + x + 1.
    \end{eqnarray*}

    We found ten reducible monic cubics (with a free coefficient), and so
    there remain eight irreducible ones. They are:
    \begin{eqnarray*}
      x^3 + 2x + 1\\
      x^3 + 2x + 2\\
      x^3 + x^2 + 2\\
      x^3 + x^2 + x + 2\\
      x^3 + x^2 + 2x + 1\\
      x^3 + 2x^2 + 1\\
      x^3 + 2x^2 + x + 1\\
      x^3 + 2x^2 + 2x + 2\\
    \end{eqnarray*}

    And therefore
    \begin{eqnarray*}
       x^{27} - x = (x^3 + 2x + 1)(x^3 + 2x + 2)(x^3 + x^2 + 2)(x^3 + x^2 + x + 2)(x^3 + x^2 + 2x + 1)&\\
    (x^3 + 2x^2 + 1)(x^3 + 2x^2 + x + 1)(x^3 + 2x^2 + 2x + 2)(x + 1)(x + 2)x&
    \end{eqnarray*}


  \item
    All the elements of $GF(27)$ are roots of $x^{27}-x$: 0 is clearly a root,
    and the order of $g \in GF(27)^*$ divides 26, and so $g^{26}=1$ or
    $g^{27}=g$. 
    Because a polynomial of degree 27 cannot have more than 27 distinct
    roots (and all are accounted for in $GF(27)$), then $GF(27)$ is precisely
    the set of roots of $x^{27}-x$.

    Therefore (as we've shown in class) the irreducible factors of $x^{27}-x$
    (bar $x$) are the minimal polynomials of the elements of $GF(27)^*$, which are 
    precisely those that can be written in the form $a^i,\:0 \leq i \leq 25$.

    This task can be performed by substituting each $a^i$ into each of the 
    factors of $x^{27}-x$. It is possible, however, to perform this using
    less tedious (but more devious) techniques:

    \begin{enumerate}
    \item 
      Trivially, $x+2$ is the minimum polynomial of $a^0=1$, as
      $x+1$ is of $a^{13}=2$.
    \item
      By definition, $a$ is a root of $x^3+2x+1$, and so the latter, which we've
      shown above to be irreducible, is the former's minimum 
      polynomial. By the ``Freshman's dream'', $(p+q+r)^3=p^3+q^3+r^3$ under
      $GF(27)$. Hence $a^3$ is also  a root of $x^3+2x+1$, as is $a^9$.
    \item
      The reciprocal polynomial of $x^3+2x+1$ is 
      $$x^3(x^{-3}+2x^{-1}+1)=x^3+2x^2+1.$$
      We showed in a homework assignment that this implies that
      $a^{25}=a^{-1}$ is a root
      of $x^3+2x^2+1$, as are $a^{23}=a^{-3}$ and $a^{17}=a^{-9}$. It also implies that 
      $x^3+2x^2+1$ is irreducible (as we've already shown above), and that
      therefore it is their minimum polynomial.
    \item
      By the ``Freshman's'' argument above, a single irreducible cubic has 
      roots $a^5$, $a^{15}$ and $a^{45}=a^{19}$. It therefore has to equal 
      \begin{eqnarray*}
        (x-a^{5})(x-a^{15})(x-a^{19})
            &=& x^3-(a^{5}+a^{15}+a^{19})x^2+(a^{20}+a^{24}+a^{34})x-a^{39}
        \\  &=& x^3-(a^{5}+a^{15}+a^{19})x^2+(a^{20}+a^{24}+a^{8})x+1
        \\  &=& x^3-(2a^2+a+2+2a^2+2a^2+2a+2)x^2+
        \\   &&(2a^2+a+1+2a^2+2a+1+2a^2+2)x+1
        \\  &=& x^3+2x^2+x+1 
      \end{eqnarray*}
      Which is thus the minimum polynomial of $a^{5}$, $a^{15}$ and $a^{19}$. 
      Hence the reciprocal polynomial, $x^3+x^2+2x+1$,
      is the minimum polynomial of $a^{-5}=a^{21}$, $a^{-15}=a^{11}$ and
      $a^{-19}=a^{7}$.
    \item
      We apply the same argument to $a^2$, $a^6$ and $a^{18}$. Their minimum
      polynomial is
      \begin{eqnarray*}
        (x-a^{2})(x-a^{6})(x-a^{18})
            &=& x^3-(a^{2}+a^{6}+a^{18})x^2+(a^{8}+a^{20}+a^{24})x-a^{26}
        \\  &=& x^3-(a^2+a^2+a+1+a^2+2a+1)x^2+
        \\   && (2a^2+2+2a^2+a+1+2a^2+2a+1)x+2
        \\  &=& x^3+x^2+x+2 
      \end{eqnarray*}
      and its reciprocal (times two), $x^3+2x^2+2x+2$, is the minimum
      polynomials of $a^{-2}=a^{24}$, $a^{-6}=a^{20}$ and $a^{-18}=a^{8}$.
    \item
      We have two irreducible polynomials left, $x^3+2x+2$ and $x^3+x^2+2$. 
      They are reciprocals (up to a factor of two), and so one is the
      root of $a^4$, $a^{12}$ and $a^{10}$, while the other is the minimum
      polynomial
      of these elements' inverses, $a^{-4}=a^{22}$, $a^{-12}=a^{14}$ and
      $a^{-10}=a^{16}$. Here we finally resort to (a single) substitution, and
      check whether $a^4$ is a root of $x^3+2x+2$:
      $$\left(a^4\right)^3+2a^4+2=a^{12}+2a^4+2=a^2+2+2(a^2+2a)+2 \neq 0$$
      Since it is not, then it must be a root of $x^3+x^2+2$. So this polynomial
      is also the minimum polynomial of $a^{12}$ and $a^{10}$, whereas
      its reciprocal, $x^3+2x+2$, is the minimum polynomial of $a^{22}$,
      $a^{14}$ and $a^{16}$
    \end{enumerate}
    These results are summarised in the table below. The number in parentheses
    references the argument above by which it was derived.

    \begin{tabular}{l| l r}
      $a^{0}$ & $x-1$ & (i)\\
      $a^{1}$ & $x^3+2x+1$ & (ii)\\
      $a^{2}$ & $x^3+x^2+x+2$ & (v)\\
      $a^{3}$ & $x^3+2x+1$ & (ii)\\
      $a^{4}$ & $x^3+x^2+2$ & (vi)\\
      $a^{5}$ & $x^3+2x^2+x+1$ & (iv)\\
      $a^{6}$ & $x^3+x^2+x+2$ & (v)\\
      $a^{7}$ & $x^3+x^2+2x+1$ & (iv)\\
      $a^{8}$ & $x^3+2x^2+2x+2$ & (v)\\
      $a^{9}$ & $x^3+2x+1$ & (ii)\\
      $a^{10}$ & $x^3+x^2+2$ & (vi)\\
      $a^{11}$ & $x^3+x^2+2x+1$ & (iv)\\
      $a^{12}$ & $x^3+x^2+2$ & (vi)\\
      $a^{13}$ & $x-2$ & (i)\\
      $a^{14}$ & $x^3+2x+2$ & (vi)\\
      $a^{15}$ & $x^3+2x^2+x+1$ & (iv)\\
      $a^{16}$ & $x^3+2x+2$ & (vi)\\
      $a^{17}$ & $x^3+2x^2+1$ & (iii)\\
      $a^{18}$ & $x^3+x^2+x+2$ & (v)\\
      $a^{19}$ & $x^3+2x^2+x+1$ & (iv)\\
      $a^{20}$ & $x^3+2x^2+2x+2$ & (v)\\
      $a^{21}$ & $x^3+x^2+2x+1$ & (iv)\\
      $a^{22}$ & $x^3+2x+2$ & (vi)\\
      $a^{23}$ & $x^3+2x^2+1$ & (iii)\\
      $a^{24}$ & $x^3+2x^2+2x+2$ & (v)\\
      $a^{25}$ & $x^3+2x^2+1$ & (iii)\\
    \end{tabular}
  \end{enumerate}
  \item
  \item
  \item
    We commence by proving some auxiliary claims. 
    We showed in a homework assignment that a group of order $p^2$, where
    $p$ is prime, can only be $C_{p^2}$ or $C_p \times C_p$. We prove here
    two similar claims:
    \begin{proposition}
      \label{prop:byprime}
      Let $p,q$ be distinct primes greater than two, and let $G$ be a group 
      of order $pq$. Then $G$ is Abelian.
    \end{proposition}
    \begin{proof}
      By the Sylow theorems $G$ has subgroups of order $p$ and $q$ --- its 
      Sylow $p$- and $q$-subgroups.
      Being of prime order, these must be cyclic.
      
      The number of Sylow $p$-subgroups, 
      being equal to $1\mod p$ (an even number, since $p$ is larger than two
      and therefore odd) and dividing $q$ (an odd number), 
      has to equal one. Call this group $P$ and let $a$ be a generator.
      
      By the same argument the number of Sylow $q$-subgroups is also one.
      Call this group $Q$ and let $b$ be a generator.
      
      Consider the element $c=a^{-1}bab^{-1}=b^ab^{-1}=a^ba^{-1}$. Since
      $b^a$ is of order $q$ then it must also be a member of $Q$, and then
      so is $c$. By the same argument for $a^b$ we have that $c \in P$. Since
      the two groups have co-prime orders they intersect trivially and $c=1$. 
      Hence $a$ and $b$ commute, the $pq$ elements of the group can be written 
      as $a^ib^j$ for $0\leq i <p$ and $0\leq j < q$, and the group is Abelian.
    \end{proof}
    \begin{corollary}
      \label{prop:2p_groups}
      Let $p$ be a prime greater than two. Let $G$ be a group of order
      $2p$. Then if $G$ is not Abelian then it has a single Sylow $p$-subgroup
      and $p$ Sylow $2$-subgroups.
    \end{corollary}
    \begin{proof}
      Using the same notation as above and the same first argument, we claim
      that $G$ has a single Sylow $p$-subgroup. It would be Abelian if it were 
      to have a single Sylow $2$-subgroup, as well. The only other option for
      $n_2$, considering that it divides $p$, is for it to equal $p$. Hence
      $G$ has $p$ Sylow $2$-subgroups.
    \end{proof}

    We can use these to prove the following:
    \begin{proposition}
      \label{prop:2p_unique}
      A group of order is either Abelian or isomorphic to the single non-Abelian
      group of order $2p$. In the latter case multiplication is defined by
      $a^ib=ba^{-1}$ where $b$ is an element of order two and $a$ is a generator
      of the Sylow $p$-subgroup.
    \end{proposition}
    \begin{proof}
      A group $G$ of order $2p$, for $p>2$, includes $p$ elements that 
      belong to the Sylow $p$-group,
      which can be represented as $a^i$ for some $a$, and $p$ elements of order
      two, which can be represented as $a^ib$ for some $b$ of order two. As
      elements of order two they are their own inverses, and so $a^iba^ib=1$ and
      $a^ib=ba^{-i}$. This defines multiplication unambiguously: $a^ia^j=a^{i+j}$,
      $(a^ib)a^j=a^{i-j}b$, $a^i(a^jb)=a^{i+j}b$ and $(a^ib)(a^jb)=a^{i-j}b$.
    \end{proof}
    Before starting to describe the subgroups of $A_5$, we classify its
    elements.
    The elements of $A_5$ are defined to be the even permutations of size five.
    We represent each element of $A_5$
    as a collection of disjoint cycles. For a permutation on five elements
    to be even, it must either:
    \begin{itemize}
    \item Be the identity permutation. There is one such permutation.
    \item Be two disjoint cycles of size two, e.g. $(12)(34)$. The number of 
      such permutations
      is fifteen (choose a number to leave out and then choose a cycle member
      for the lowest number left). These elements are of order two.
    \item Be a cycle of size three, e.g. $(123)$. There are 
      $2{5 \choose 3}=20$ such cycles. These elements are of order three.
    \item Be a cycle of size five, e.g. $(12345)$. There are $4!=24$ such 
      cycles. These elements are of order five.
    \end{itemize}
    The total number of elements listed above is sixty, and so all are accounted
    for.

    We are now ready to turn to the actual problem of classifying the subgroups
    of $A_5$.
    The prime factorisation of sixty is $60=2^2\cdot 3\cdot 5$. Hence its
    divisors are $1,2,3,4,5,6,10,12,15,20,30,60$. We list below the subgroups
    of each of these orders.

    \begin{itemize}
    \item {\bf 1}

      The trivial subgroup containing the identity element is indeed trivially 
      the only subgroup of order one.

    \item {\bf 2, 3, 5}
      
      For a prime divisor $p\in\{2,3,5\}$, a group of order $p$ exists 
      (by the Sylow theorems) and is generated by
      an element of order $p$. Hence for each of the elements of order 
      $p\in\{2,3,5\}$ listed above
      corresponds a subgroup of order $p$.

    \item {\bf 4}
      
      A Sylow $2$-subgroup of $A_5$, of order four,
      exists and is unique up to isomorphism, by the Sylow theorems. It can
      be isomorphic only to $C_2\times C_2$ or to $C_4$, as we showed in class
      for groups of order $p^2$. However, there is no element of order four
      in $A_5$ (we listed the elements and their orders above) and therefore
      the only possibility is $C_2 \times C_2$. 
      
    \item {\bf 6}
      Were there to exist an Abelian subgroup of order six then it would have
      to be $C_6$ and have an element of order six. Since none exists then
      any subgroup of order six is not Abelian.
      
      By proposition~\ref{prop:2p_unique}, only one non-Abelian group of order 
      six exists --- $S_3$. This group has
      a single Sylow $3$-subgroup and three Sylow $2$-subgroups. An example
      of an $S_3$ in $A_5$ is one in which $\sigma=(1 2 3)$ generates the Sylow
      $3$-subgroup, and the elements of the form $\tau_{ij}=(ij)(45)$, with 
      $i,j \in \{1,2,3\}$, generate the Sylow $2$-subgroups. 

      This group includes any possible cycle on $\{1,2,3\}$ and is closed under
      multiplication (it clearly cannot involve any action on $\{4,5\}$ other
      than $(45)$) and so is isomorphic to $S_3$.

    \item {\bf 10}
      By the same argument as above, any Abelian subgroup of order ten would
      have an element of order ten, and since none exist then  
      any subgroup of order ten is not Abelian. Also as above, only one such
      group exists. It has a single Sylow $5$-subgroup and five Sylow 
      $2$-subgroups.

      Let $\sigma=(12345)$ generate the Sylow 
      $5$-subgroup. Let 
      $\tau_i=\left(ii^{\sigma^3}\right)\left(i^{\sigma}i^{\sigma^2}\right)$ 
      be an element of $A_5$. Then there are five possible
      choices for $\tau$ and $\tau_i^{-1}\sigma\tau_i=(54321)=\sigma^{-1}$, as 
      required in proposition~\ref{prop:2p_unique}.

    \item {\bf 15}
      
      By proposition~\ref{prop:byprime} above, any subgroup of order $15=3\cdot 5$
      is Abelian, and therefore can only be isomorphic to 
      $C_5\times C_3=C_{15}$. 

      If such a subgroup were to exist it would have an element of order 15. 
      Since no such element exists then $A_5$ has no subgroup of order 15.

    \item {\bf 12}
      
      Let $G$ be a subgroup of order 12. Then $G$ has a Sylow $3$-subgroup,
      of order three, and a Sylow $2$-subgroup, of order four. Since no 
      element of order four exists, then the Sylow $2$-subgroup must be 
      $C_2\times C_2$. $G$ cannot be Abelian, since it would have to have 
      an element of order six.

      The numbers of Sylow subgroups, $n_3$ and $n_4$, cannot therefore both
      be one. By the Sylow theorems, $n_3$ must equal $1\mod 3$ and divide
      four, and so is one or four. $n_4$, which must equal $1\mod 2$ and divide
      three, has to equal one or three. There are not enough element in $G$
      for both $n_3$ and $n_4$ to be greater than one, and so precisely one
      of them equals one.

      Assume first that $n_3=1$. Then WLOG let $(123)$ generate the single,
      normal Sylow $3$-subgroup $P_3$. However, the only elements $\tau$ of
      order two for which $\tau^{-1}P_3\tau=P_3$ are of the form $(ii^{(123)})(45)$ with
      $i\in\{1,2,3\}$. There are only three such elements, and so it is not
      possible that $n_3=1$.

      The only possibility left is that $n_3=4$ and $n_4=1$. This describes
      $A_4$, which is, of course, a subgroup of $A_5$. In $A_4$ the elements that
      generate the Sylow $3$-groups are the three cycles: $(123)$, $(124)$ and 
      $(234)$. The elements of the Sylow $2$-group are the identity and the 
      pairs of transpositions: $(12)(34)$, $(13)(24)$ and $(14)(23)$. 
      
    \item {\bf 20}
      Let $G$ be a subgroup of order 20.
      As above, the Sylow $2$-subgroup would
      be $C_2\times C_2$. 
      Also as above, $G$  cannot be Abelian, since it would have
      to have a element of order ten.

      The number of Sylow $5$-subgroups equals $1\mod 5$ and divides four,
      and so must equal one. Hence the number of Sylow $2$-subgroups cannot
      equal one as well. Since it is odd and divides five then it must equal
      five. Hence $G$ would have an identity element, five elements of order 
      two and four elements of order one - not enough to make 20 elements.
      Hence no subgroup of order 20 exists.

    \item {\bf 30}
      Let $G$ be a subgroup of order 30. Then $G$ would have a Sylow 
      $2$-subgroup, a Sylow $3$-subgroup and a Sylow $5$-subgroup. 
      It is impossible that two of them are normal, since their
      Cartesian product would be an Abelian subgroup of $G$ with order 
      greater than five, and again no element of such order exists.

      The number of Sylow $5$-subgroups, $n_5$, equals $1\mod 5$ and 
      divides six. Hence it may equal one or six.

      The number of Sylow $3$-subgroups, $n_3$, equals $1\mod 3$ and 
      divides ten. Hence it may equal one, or ten.

      The number of Sylow $2$-subgroups, $n_2$, equals $1\mod 2$ and 
      divides fifteen. Hence it may equal one, three, five or fifteen.

      Assume first that $n_2=1$. Then $n_5=6$ and $n_3=10$. This makes too 
      many elements, and so $n_2=1$ is an impossibility. By the same argument
      it is impossible that all three are greater than one.

      Next assume $n_5=1$. Then $n_3=10$ and $n_2 \in \{3,5,15\}$. The number
      of elements adds up to 30 iff $n_2=5$.

      
    \item {\bf 60}
      
      The only subgroup of order 60 is, trivially, $A_5$ itself.
        
    \end{itemize}
\end{enumerate}
\end{document}


