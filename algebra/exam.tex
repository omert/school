\documentclass[11pt]{article} \usepackage{amssymb}
\usepackage{amsfonts} \usepackage{amsmath} \usepackage{bm}
\usepackage{latexsym} \usepackage{epsfig}

\setlength{\textwidth}{6.5 in} \setlength{\textheight}{8.25in}
\setlength{\oddsidemargin}{0in} \setlength{\topmargin}{0in}
\addtolength{\textheight}{.8in} \addtolength{\voffset}{-.5in}

\newtheorem{theorem}{Theorem}[section]
\newtheorem{lemma}[theorem]{Lemma}
\newtheorem{proposition}[theorem]{Proposition}
\newtheorem{corollary}[theorem]{Corollary}
\newtheorem{fact}[theorem]{Fact}
\newtheorem{definition}[theorem]{Definition}
\newtheorem{remark}[theorem]{Remark}
\newtheorem{conjecture}[theorem]{Conjecture}
\newtheorem{claim}[theorem]{Claim}
\newtheorem{example}[theorem]{Example}
\newenvironment{proof}{\noindent \textbf{Proof:}}{$\Box$}

\newcommand{\ignore}[1]{}

\newcommand{\enote}[1]{} \newcommand{\knote}[1]{}
\newcommand{\rnote}[1]{}



% \newcommand{\enote}[1]{{\bf [[Elchanan:} {\emph{#1}}{\bf ]]}}
% \newcommand{\knote}[1]{{\bf [[Krzysztof:} {\emph{#1}}{\bf ]]}}
% \newcommand{\rnote}[1]{{\bf [[Ryan:} {\emph{#1}}{\bf ]]}}



\DeclareMathOperator{\Support}{Supp} \DeclareMathOperator{\Opt}{Opt}
\DeclareMathOperator{\Ordo}{\mathcal{O}}
\newcommand{\MaxkCSP}{\textsc{Max $k$-CSP}}
\newcommand{\MaxkCSPq}{\textsc{Max $k$-CSP$_{q}$}}
\newcommand{\MaxCSP}[1]{\textsc{Max CSP}(#1)} \renewcommand{\Pr}{{\bf
    P}} \renewcommand{\P}{{\bf P}} \newcommand{\Px}{\mathop{\bf P\/}}
\newcommand{\E}{{\bf E}} \newcommand{\Cov}{{\bf Cov}}
\newcommand{\Var}{{\bf Var}} \newcommand{\Varx}{\mathop{\bf Var\/}}

\newcommand{\bits}{\{-1,1\}}

\newcommand{\nsmaja}{\textstyle{\frac{2}{\pi}} \arcsin \rho}

\newcommand{\Inf}{\mathrm{Inf}} \newcommand{\I}{\mathrm{I}}
\newcommand{\J}{\mathrm{J}}

\newcommand{\eps}{\epsilon} \newcommand{\lam}{\lambda}

% \newcommand{\trunc}{\ell_{2,[-1,1]}}
\newcommand{\trunc}{\zeta} \newcommand{\truncprod}{\chi}

\newcommand{\N}{\mathbb N} \newcommand{\R}{\mathbb R}
\newcommand{\Q}{\mathbb Q}
\newcommand{\Z}{\mathbb Z} \newcommand{\CalE}{{\mathcal{E}}}
\newcommand{\CalC}{{\mathcal{C}}} \newcommand{\CalM}{{\mathcal{M}}}
\newcommand{\CalR}{{\mathcal{R}}} \newcommand{\CalS}{{\mathcal{S}}}
\newcommand{\CalV}{{\mathcal{V}}}
\newcommand{\CalX}{{\boldsymbol{\mathcal{X}}}}
\newcommand{\CalG}{{\boldsymbol{\mathcal{G}}}}
\newcommand{\CalH}{{\boldsymbol{\mathcal{H}}}}
\newcommand{\CalY}{{\boldsymbol{\mathcal{Y}}}}
\newcommand{\CalZ}{{\boldsymbol{\mathcal{Z}}}}
\newcommand{\CalW}{{\boldsymbol{\mathcal{W}}}}
\newcommand{\CalF}{{\mathcal{Z}}}
% \newcommand{\boldG}{{\boldsymbol G}}
% \newcommand{\boldQ}{{\boldsymbol Q}}
% \newcommand{\boldP}{{\boldsymbol P}}
% \newcommand{\boldR}{{\boldsymbol R}}
% \newcommand{\boldS}{{\boldsymbol S}}
% \newcommand{\boldX}{{\boldsymbol X}}
% \newcommand{\boldB}{{\boldsymbol B}}
% \newcommand{\boldY}{{\boldsymbol Y}}
% \newcommand{\boldZ}{{\boldsymbol Z}}
% \newcommand{\boldV}{{\boldsymbol V}}
\newcommand{\boldi}{{\boldsymbol i}} \newcommand{\boldj}{{\boldsymbol
    j}} \newcommand{\boldk}{{\boldsymbol k}}
\newcommand{\boldr}{{\boldsymbol r}}
\newcommand{\boldsigma}{{\boldsymbol \sigma}}
\newcommand{\boldupsilon}{{\boldsymbol \upsilon}}
\newcommand{\hone}{{\boldsymbol{H1}}}
\newcommand{\htwo}{\boldsymbol{H2}}
\newcommand{\hthree}{\boldsymbol{H3}}
\newcommand{\hfour}{\boldsymbol{H4}}


\newcommand{\sgn}{\mathrm{sgn}} \newcommand{\Maj}{\mathrm{Maj}}
\newcommand{\Acyc}{\mathrm{Acyc}}
\newcommand{\UniqMax}{\mathrm{UniqMax}}
\newcommand{\Thr}{\mathrm{Thr}} \newcommand{\littlesum}{{\textstyle
    \sum}}

\newcommand{\half}{{\textstyle \frac12}}
\newcommand{\third}{{\textstyle \frac13}}
\newcommand{\fourth}{{\textstyle \frac14}}

\newcommand{\Stab}{\mathbb{S}}
\newcommand{\StabThr}[2]{\Gamma_{#1}(#2)}
\newcommand{\StabThrmin}[2]{{\underline{\Gamma}}_{#1}(#2)}
\newcommand{\StabThrmax}[2]{{\overline{\Gamma}}_{#1}(#2)}
\newcommand{\TestFcn}{\Psi}

\renewcommand{\phi}{\varphi}

\begin{document}
\title{Algebra Through Examples - Exam}

 \author{Omer Tamuz, 035696574}
\maketitle


\begin{enumerate}
\item 
  \begin{enumerate}
  \item 
    Let $a$ be a root of $x^3+2x+1$ in $GF(27)$. Then the elements of $GF(27)^*$
    can be written as follows:

    \begin{tabular}{l| l}
      $a^{0}$ & $1$\\
      $a^{1}$ & $a$\\
      $a^{2}$ & $a^{2}$\\
      $a^{3}$ & $a+2$\\
      $a^{4}$ & $a^{2}+2a$\\
      $a^{5}$ & $2a^{2}+a+2$\\
      $a^{6}$ & $a^{2}+a+1$\\
      $a^{7}$ & $a^{2}+2a+2$\\
      $a^{8}$ & $2a^{2}+2$\\
      $a^{9}$ & $a+1$\\
      $a^{10}$ & $a^{2}+a$\\
      $a^{11}$ & $a^{2}+a+2$\\
      $a^{12}$ & $a^{2}+2$\\
      $a^{13}$ & $2$\\
      $a^{14}$ & $2a$\\
      $a^{15}$ & $2a^{2}$\\
      $a^{16}$ & $2a+1$\\
      $a^{17}$ & $2a^{2}+a$\\
      $a^{18}$ & $a^{2}+2a+1$\\
      $a^{19}$ & $2a^{2}+2a+2$\\
      $a^{20}$ & $2a^{2}+a+1$\\
      $a^{21}$ & $a^{2}+1$\\
      $a^{22}$ & $2a+2$\\
      $a^{23}$ & $2a^{2}+2a$\\
      $a^{24}$ & $2a^{2}+2a+1$\\
      $a^{25}$ & $2a^{2}+1$\\
    \end{tabular}
%    We note (to be used in the next question), that $x^3+2x+1$ is a minimum
%    polynomial of $a$. 
  \item
    We would like to factor $x^{27}-x$. We showed in class that for $p$ 
    prime and $n$ natural the 
    polynomial $x^{p^n}-x$ is the product of all the monic irreducible 
    polynomials of degrees that divide $n$ over $GF(p)$. Hence $x^{27}-x$
    is the product of the irreducible linear and cubic monic polynomials
    over $GF(3)$.

    The linear monics ($x$, $x+1$ and $x+2$) are all
    trivially irreducible. To find the cubic factors of $x^{27}-x$ we list all 
    the {\em reducible} cubics, and infer that the rest are irreducible. 
    We restrict
    ourselves to monic polynomials with non-zero free coefficients, of which
    there are $3 \cdot 3 \cdot 2=18$ --- one needs to choose coefficients 
    for $x^2$ and $x$ in $GF(3)$, and a free coefficient in $\{1,2\}$.
    
    Of the class of cubics described above, the ones reducible to linear
    coefficients are
    \begin{eqnarray*}
      (x + 1)^3 &=& x^3 + 1\\
      (x + 1)^2(x + 2) &=& x^3 + x^2 + 2x + 2\\
      (x + 1)(x + 2)^2 &=& x^3 + 2x^2 + 2x + 1\\
      (x + 2)^3 &=& x^3 + 2.
    \end{eqnarray*}

    To find the ones reducible to an irreducible quadratic and a linear 
    coefficient, we note that the {\em reducible} quadratics are
    \begin{eqnarray*}
      (x+1)^2 &=& x^2+2x+1\\
      (x+1)(x+2) &=& x^2+2\\
      (x+2)^2 &=& x^2+x+1\\
    \end{eqnarray*}
    and so the {\em irreducible} quadratics are  $x^2+1$, $x^2+2x+2$ and 
    $x^2+x+2$. Hence the monic cubics (with a free coefficient) reducible to a 
    quadratic and a linear are
    \begin{eqnarray*}
      (x^2 + 1)(x + 1) &=& x^3 + x^2 + x + 1\\
      (x^2 + 2x + 2)(x + 1) &=& x^3 + x + 2\\
      (x^2 + x + 2)(x + 1) &=& x^3 + 2x^2 + 2\\
      (x^2 + 1)(x + 2) &=& x^3 + 2x^2 + x + 2\\
      (x^2 + 2x + 2)(x + 2) &=& x^3 + x^2 + 1\\
      (x^2 + x + 2)(x + 2) &=& x^3 + x + 1.
    \end{eqnarray*}

    We found ten reducible monic cubics (with a free coefficient), and so
    there remain eight irreducible ones. They are:
    \begin{eqnarray*}
      x^3 + 2x + 1\\
      x^3 + 2x + 2\\
      x^3 + x^2 + 2\\
      x^3 + x^2 + x + 2\\
      x^3 + x^2 + 2x + 1\\
      x^3 + 2x^2 + 1\\
      x^3 + 2x^2 + x + 1\\
      x^3 + 2x^2 + 2x + 2\\
    \end{eqnarray*}

    And therefore
    \begin{eqnarray*}
       x^{27} - x = (x^3 + 2x + 1)(x^3 + 2x + 2)(x^3 + x^2 + 2)(x^3 + x^2 + x + 2)(x^3 + x^2 + 2x + 1)&\\
    (x^3 + 2x^2 + 1)(x^3 + 2x^2 + x + 1)(x^3 + 2x^2 + 2x + 2)(x + 1)(x + 2)x&
    \end{eqnarray*}


  \item
    All the elements of $GF(27)$ are roots of $x^{27}-x$: 0 is clearly a root,
    and the order of $g \in GF(27)^*$ divides 26, and so $g^{26}=1$ or
    $g^{27}=g$. 
    Because a polynomial of degree 27 cannot have more than 27 distinct
    roots (and all are accounted for in $GF(27)$), then $GF(27)$ is precisely
    the set of roots of $x^{27}-x$.

    Therefore (as we've shown in class) the irreducible factors of $x^{27}-x$
    (bar $x$) are the minimal polynomials of the elements of $GF(27)^*$, which are 
    precisely those that can be written in the form $a^i,\:0 \leq i \leq 25$.

    This task can be performed by substituting each $a^i$ into each of the 
    factors of $x^{27}-x$. It is possible, however, to perform this using
    less tedious (but more devious) techniques:

    \begin{enumerate}
    \item 
      Trivially, $x+2$ is the minimum polynomial of $a^0=1$, as
      $x+1$ is of $a^{13}=2$.
    \item
      By definition, $a$ is a root of $x^3+2x+1$, and so the latter, which we've
      shown above to be irreducible, is the former's minimum 
      polynomial. By the ``Freshman's dream'', $(p+q+r)^3=p^3+q^3+r^3$ under
      $GF(27)$. Hence $a^3$ is also  a root of $x^3+2x+1$, as is $a^9$.
    \item
      The reciprocal polynomial of $x^3+2x+1$ is 
      $$x^3(x^{-3}+2x^{-1}+1)=x^3+2x^2+1.$$
      We showed in a homework assignment that this implies that
      $a^{25}=a^{-1}$ is a root
      of $x^3+2x^2+1$, as are $a^{23}=a^{-3}$ and $a^{17}=a^{-9}$. It also implies that 
      $x^3+2x^2+1$ is irreducible (as we've already shown above), and that
      therefore it is their minimum polynomial.
    \item
      By the ``Freshman's'' argument above, a single irreducible cubic has 
      roots $a^5$, $a^{15}$ and $a^{45}=a^{19}$. It therefore has to equal 
      \begin{eqnarray*}
        (x-a^{5})(x-a^{15})(x-a^{19})
            &=& x^3-(a^{5}+a^{15}+a^{19})x^2+(a^{20}+a^{24}+a^{34})x-a^{39}
        \\  &=& x^3-(a^{5}+a^{15}+a^{19})x^2+(a^{20}+a^{24}+a^{8})x+1
        \\  &=& x^3-(2a^2+a+2+2a^2+2a^2+2a+2)x^2+
        \\   &&(2a^2+a+1+2a^2+2a+1+2a^2+2)x+1
        \\  &=& x^3+2x^2+x+1 
      \end{eqnarray*}
      Which is thus the minimum polynomial of $a^{5}$, $a^{15}$ and $a^{19}$. 
      Hence the reciprocal polynomial, $x^3+x^2+2x+1$,
      is the minimum polynomial of $a^{-5}=a^{21}$, $a^{-15}=a^{11}$ and
      $a^{-19}=a^{7}$.
    \item
      We apply the same argument to $a^2$, $a^6$ and $a^{18}$. Their minimum
      polynomial is
      \begin{eqnarray*}
        (x-a^{2})(x-a^{6})(x-a^{18})
            &=& x^3-(a^{2}+a^{6}+a^{18})x^2+(a^{8}+a^{20}+a^{24})x-a^{26}
        \\  &=& x^3-(a^2+a^2+a+1+a^2+2a+1)x^2+
        \\   && (2a^2+2+2a^2+a+1+2a^2+2a+1)x+2
        \\  &=& x^3+x^2+x+2 
      \end{eqnarray*}
      and its reciprocal (times two), $x^3+2x^2+2x+2$, is the minimum
      polynomials of $a^{-2}=a^{24}$, $a^{-6}=a^{20}$ and $a^{-18}=a^{8}$.
    \item
      We have two irreducible polynomials left, $x^3+2x+2$ and $x^3+x^2+2$. 
      They are reciprocals (up to a factor of two), and so one is the
      root of $a^4$, $a^{12}$ and $a^{10}$, while the other is the minimum
      polynomial
      of these elements' inverses, $a^{-4}=a^{22}$, $a^{-12}=a^{14}$ and
      $a^{-10}=a^{16}$. Here we finally resort to (a single) substitution, and
      check whether $a^4$ is a root of $x^3+2x+2$:
      $$\left(a^4\right)^3+2a^4+2=a^{12}+2a^4+2=a^2+2+2(a^2+2a)+2 \neq 0$$
      Since it is not, then it must be a root of $x^3+x^2+2$. So this polynomial
      is also the minimum polynomial of $a^{12}$ and $a^{10}$, whereas
      its reciprocal, $x^3+2x+2$, is the minimum polynomial of $a^{22}$,
      $a^{14}$ and $a^{16}$
    \end{enumerate}
    These results are summarised in the table below. The number in parentheses
    refrences the argument above by which it was derived.

    \begin{tabular}{l| l r}
      $a^{0}$ & $x-1$ & (i)\\
      $a^{1}$ & $x^3+2x+1$ & (ii)\\
      $a^{2}$ & $x^3+x^2+x+2$ & (v)\\
      $a^{3}$ & $x^3+2x+1$ & (ii)\\
      $a^{4}$ & $x^3+x^2+2$ & (vi)\\
      $a^{5}$ & $x^3+2x^2+x+1$ & (iv)\\
      $a^{6}$ & $x^3+x^2+x+2$ & (v)\\
      $a^{7}$ & $x^3+x^2+2x+1$ & (iv)\\
      $a^{8}$ & $x^3+2x^2+2x+2$ & (v)\\
      $a^{9}$ & $x^3+2x+1$ & (ii)\\
      $a^{10}$ & $x^3+x^2+2$ & (vi)\\
      $a^{11}$ & $x^3+x^2+2x+1$ & (iv)\\
      $a^{12}$ & $x^3+x^2+2$ & (vi)\\
      $a^{13}$ & $x-2$ & (i)\\
      $a^{14}$ & $x^3+2x+2$ & (vi)\\
      $a^{15}$ & $x^3+2x^2+x+1$ & (iv)\\
      $a^{16}$ & $x^3+2x+2$ & (vi)\\
      $a^{17}$ & $x^3+2x^2+1$ & (iii)\\
      $a^{18}$ & $x^3+x^2+x+2$ & (v)\\
      $a^{19}$ & $x^3+2x^2+x+1$ & (iv)\\
      $a^{20}$ & $x^3+2x^2+2x+2$ & (v)\\
      $a^{21}$ & $x^3+x^2+2x+1$ & (iv)\\
      $a^{22}$ & $x^3+2x+2$ & (vi)\\
      $a^{23}$ & $x^3+2x^2+1$ & (iii)\\
      $a^{24}$ & $x^3+2x^2+2x+2$ & (v)\\
      $a^{25}$ & $x^3+2x^2+1$ & (iii)\\
    \end{tabular}
  \end{enumerate}
\end{enumerate}
\end{document}


