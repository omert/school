\documentclass[11pt]{article} \usepackage{amssymb}
\usepackage{amsfonts} \usepackage{amsmath} \usepackage{bm}
\usepackage{latexsym} \usepackage{epsfig}

\setlength{\textwidth}{6.5 in} \setlength{\textheight}{8.25in}
\setlength{\oddsidemargin}{0in} \setlength{\topmargin}{0in}
\addtolength{\textheight}{.8in} \addtolength{\voffset}{-.5in}

\newtheorem{theorem}{Theorem}[section]
\newtheorem{lemma}[theorem]{Lemma}
\newtheorem{proposition}[theorem]{Proposition}
\newtheorem{corollary}[theorem]{Corollary}
\newtheorem{fact}[theorem]{Fact}
\newtheorem{definition}[theorem]{Definition}
\newtheorem{remark}[theorem]{Remark}
\newtheorem{conjecture}[theorem]{Conjecture}
\newtheorem{claim}[theorem]{Claim}
\newtheorem{example}[theorem]{Example}
\newenvironment{proof}{\noindent \textbf{Proof:}}{$\Box$}

\newcommand{\ignore}[1]{}

\newcommand{\enote}[1]{} \newcommand{\knote}[1]{}
\newcommand{\rnote}[1]{}



% \newcommand{\enote}[1]{{\bf [[Elchanan:} {\emph{#1}}{\bf ]]}}
% \newcommand{\knote}[1]{{\bf [[Krzysztof:} {\emph{#1}}{\bf ]]}}
% \newcommand{\rnote}[1]{{\bf [[Ryan:} {\emph{#1}}{\bf ]]}}



\DeclareMathOperator{\Support}{Supp} \DeclareMathOperator{\Opt}{Opt}
\DeclareMathOperator{\Ordo}{\mathcal{O}}
\newcommand{\MaxkCSP}{\textsc{Max $k$-CSP}}
\newcommand{\MaxkCSPq}{\textsc{Max $k$-CSP$_{q}$}}
\newcommand{\MaxCSP}[1]{\textsc{Max CSP}(#1)} \renewcommand{\Pr}{{\bf
    P}} \renewcommand{\P}{{\bf P}} \newcommand{\Px}{\mathop{\bf P\/}}
\newcommand{\E}{{\bf E}} \newcommand{\Cov}{{\bf Cov}}
\newcommand{\Var}{{\bf Var}} \newcommand{\Varx}{\mathop{\bf Var\/}}

\newcommand{\bits}{\{-1,1\}}

\newcommand{\nsmaja}{\textstyle{\frac{2}{\pi}} \arcsin \rho}

\newcommand{\Inf}{\mathrm{Inf}} \newcommand{\I}{\mathrm{I}}
\newcommand{\J}{\mathrm{J}}

\newcommand{\eps}{\epsilon} \newcommand{\lam}{\lambda}

% \newcommand{\trunc}{\ell_{2,[-1,1]}}
\newcommand{\trunc}{\zeta} \newcommand{\truncprod}{\chi}

\newcommand{\N}{\mathbb N} \newcommand{\R}{\mathbb R}
\newcommand{\Q}{\mathbb Q}
\newcommand{\Z}{\mathbb Z} \newcommand{\CalE}{{\mathcal{E}}}
\newcommand{\CalC}{{\mathcal{C}}} \newcommand{\CalM}{{\mathcal{M}}}
\newcommand{\CalR}{{\mathcal{R}}} \newcommand{\CalS}{{\mathcal{S}}}
\newcommand{\CalV}{{\mathcal{V}}}
\newcommand{\CalX}{{\boldsymbol{\mathcal{X}}}}
\newcommand{\CalG}{{\boldsymbol{\mathcal{G}}}}
\newcommand{\CalH}{{\boldsymbol{\mathcal{H}}}}
\newcommand{\CalY}{{\boldsymbol{\mathcal{Y}}}}
\newcommand{\CalZ}{{\boldsymbol{\mathcal{Z}}}}
\newcommand{\CalW}{{\boldsymbol{\mathcal{W}}}}
\newcommand{\CalF}{{\mathcal{Z}}}
% \newcommand{\boldG}{{\boldsymbol G}}
% \newcommand{\boldQ}{{\boldsymbol Q}}
% \newcommand{\boldP}{{\boldsymbol P}}
% \newcommand{\boldR}{{\boldsymbol R}}
% \newcommand{\boldS}{{\boldsymbol S}}
% \newcommand{\boldX}{{\boldsymbol X}}
% \newcommand{\boldB}{{\boldsymbol B}}
% \newcommand{\boldY}{{\boldsymbol Y}}
% \newcommand{\boldZ}{{\boldsymbol Z}}
% \newcommand{\boldV}{{\boldsymbol V}}
\newcommand{\boldi}{{\boldsymbol i}} \newcommand{\boldj}{{\boldsymbol
    j}} \newcommand{\boldk}{{\boldsymbol k}}
\newcommand{\boldr}{{\boldsymbol r}}
\newcommand{\boldsigma}{{\boldsymbol \sigma}}
\newcommand{\boldupsilon}{{\boldsymbol \upsilon}}
\newcommand{\hone}{{\boldsymbol{H1}}}
\newcommand{\htwo}{\boldsymbol{H2}}
\newcommand{\hthree}{\boldsymbol{H3}}
\newcommand{\hfour}{\boldsymbol{H4}}


\newcommand{\sgn}{\mathrm{sgn}} \newcommand{\Maj}{\mathrm{Maj}}
\newcommand{\Acyc}{\mathrm{Acyc}}
\newcommand{\UniqMax}{\mathrm{UniqMax}}
\newcommand{\Thr}{\mathrm{Thr}} \newcommand{\littlesum}{{\textstyle
    \sum}}

\newcommand{\half}{{\textstyle \frac12}}
\newcommand{\third}{{\textstyle \frac13}}
\newcommand{\fourth}{{\textstyle \frac14}}

\newcommand{\Stab}{\mathbb{S}}
\newcommand{\StabThr}[2]{\Gamma_{#1}(#2)}
\newcommand{\StabThrmin}[2]{{\underline{\Gamma}}_{#1}(#2)}
\newcommand{\StabThrmax}[2]{{\overline{\Gamma}}_{#1}(#2)}
\newcommand{\TestFcn}{\Psi}

\renewcommand{\phi}{\varphi}

\begin{document}
\title{Algebra Through Examples - Exercise 4}

 \author{Omer Tamuz, 035696574}
\maketitle


\begin{enumerate}
\item 
  \begin{enumerate}
  \item 

    $x^4+x^3+x^2+x+1$ is equal to one when $x=0$ and when $x=1$, by
    substitution. This is true for any polynomial with an odd number
    of non-zero terms which include the free term.
  \item

    Represent $GF(4)$ as $\{0,1,a,a+1\}$, where $a$ is the
    root of $x^2+x+1$, which is irreducible over $GF(2)$. Then
    \begin{itemize}
    \item The polynomial $x^4+x^3+x^2+x+1$ does not have a root in $GF(4)$
      (by substitution), and so any factoring into two quadratic factors
      would be a factoring into irreducibles. A possible solution is
      $x^4+x^3+x^2+x+1=(x^2+ax+1)(x^2+a^2x+1)$.

    \item 
      Likewise, $x^4+x^3+1=(x^2+a^2x+a)(x^2+ax+a^2)$.

    \end{itemize}

    We showed in class that $x^4+x^3+1$ can be factored completely over 
    $GF(16)$. An identical proof works for $x^4+x^3+x^2+x+1$. 
    
    We note that the elements of $GF(16)^*$ have orders in $\{1,3,5,15\}$
    and the elements of $GF(8)^*$ have orders in $\{1,7\}$.

    Assume one of the polynomials can be factored into a linear factor $x-a$
    and a cubic factor, in $GF(8)$. $a$ is not zero, since zero is clearly
    not a root. Then $a$ has order $7$ (or order
    $1$, but $x-1$ is not a factor, so this is impossible). However $a$ is 
    also a member of $GF(16)$, and so has order $3$, $5$ or $15$ - 
    contradiction.

    Let $x^2+ix+j$ be a quadratic factor of one of the polynomials over $GF(8)$.
    Then, by the argument above, each of $i$ and $j$ would either have to be 
    zero, or have orders 
    both in $\{1,7\}$ and in $\{1,3,5,15\}$, and so have to be one. But then
    the polynomial can be factored in $GF(2)$, which is not true - 
    contradiction.
    
    We have shown that the polynomials don't have linear or quadratic factors,
    and so they cannot be factored in $GF(8)$.
  \end{enumerate}
\item

  Let $F=GF(p^k)$ be a finite field of characteristic $p$. 
  Let $\phi:F\to F$ be defined
  by $\phi(a)=a^p$. We would like to show that $\phi$ is an automorphism. 


  We showed in class that the multiplicative group of a 
  finite field is cyclic. Let $g$ be a generator of $F^*$.
  We know that the order of $F$ is $p^k$. 
  Hence by Euler's theorem $g^{p^k-1}=1$ and so $\left(g^p\right)^k=g$, and 
  therefore $g^p$ is also a generator. 

  We showed in class that given two fields of identical order, and generators
  of each field, then the map defined by mapping one generator to the
  other is a field homomoprishm, and in particular an isomorphism. 
  Hence $\phi$ is also an isomorphism, and since
  its domain is the same as its image then it is an automorphism.

  Let $G$ be a subfield of $F$. We showed in class that $G$ is the set of
  roots of $x^{p^m}-x$ where $m$ divides $k$. Then $G$ is the set of
  fixed points of $\phi^m$, since $\phi^m(x)=x^{p^m}$.

\item

  By definition $i$ and $j$ are both roots of $x^2+1$. 

  Let $q=a+bi+cj+dk$ be a root of $x^2+1$. Then
  \begin{eqnarray*}
    1 &=&(a+bi+cj+dk)^2
    \\ &=&(a+bi+cj+dk)(a+bi+cj+dk)
    \\ &=&a^2-b^2-c^2-d^2+2abi+2acj+2adk
  \end{eqnarray*}

  Hence $q$ is a root iff $a=0$ and $b^2+c^2+d^2=1$. Hence there clearly are
  infinitely many roots to $x^2+1$.

\item
  Let $f(x)=1+\sum_{i=1}^mf_ix^i$. Then:
  \begin{enumerate}
  \item 
    \begin{eqnarray*}
      g(x)&=&x^mf(x^{-1})
      \\ &=& x^m\left(1+\sum_{i=1}^mf_ix^{-i}\right)
      \\ &=& x^m+\sum_{i=1}^mf_ix^{-i+m}
      \\ &=& x^m+\sum_{k=0}^{m-1}f_{m-k}x^k \;\;\;\; (k=-i+m).
    \end{eqnarray*}
    Hence we've showed that $g(x)$ is also a polynomial of degree $m$.

  \item
    Since $f$ has a non-zero constant term, then $\alpha\neq 0$. Hence
    $g(\alpha^{-1})=(\alpha^{-1})^mf((\alpha^{-1})^{-1})=\alpha^{-m}f(\alpha)=0$, and so
    $\alpha^{-1}$ is a root of $g(x)$.

  \item

    Assume $f(x)$ is {\em not} irreducible. Then for some $p(x)$ and $q(x)$, 
    both of degree at least one, it holds
    that $f(x)=p(x)q(x)$. Let $n$ be the degree of $p$, and $m-n$ the degree of 
    $q$. Since $zero$ is not a root of $f(x)$ it is also not a root of $p(x)$
    or $q(x)$, and so they both have non-zero constant terms. So by
    the proof above, $p'(x)=x^np(x^{-1})$ is polynomial of degree $n$ and 
    $q'(x)=x^{m-n}q(x^{-1})$ is a polnomial of degree $m-n$. Hence:
    \begin{eqnarray*}
      g(x)&=& x^mf(x^{-1})
      \\ &=& x^mp(x^{-1})q(x^{-1})
      \\ &=& x^np(x^{-1})x^{m-n}q(x^{-1})
      \\ &=& p'(x)q'(x),
    \end{eqnarray*}

    and so $g'$ is {\em not} irreducible. 

    We've shown that if $f$ is not irreducible then $g$ is not irreducible. 
    Hence if $g$ is irreducible then $f$ is irreducible. Since $f(x)$ is
    the reciprocal of $f(x)$, then this proof would also work with the roles
    of $g(x)$ and $f(x)$ reversed, and therefore we've shown that $f(x)$
    is irreducible iff $g(x)$ is irreducible.

  \item
    
    
  
  \end{enumerate}
\end{enumerate}
\end{document}


