\documentclass[11pt]{article} \usepackage{amssymb}
\usepackage{amsfonts} \usepackage{amsmath} \usepackage{bm}
\usepackage{latexsym} \usepackage{epsfig}

\setlength{\textwidth}{6.5 in} \setlength{\textheight}{8.25in}
\setlength{\oddsidemargin}{0in} \setlength{\topmargin}{0in}
\addtolength{\textheight}{.8in} \addtolength{\voffset}{-.5in}

\newtheorem{theorem}{Theorem}[section]
\newtheorem{lemma}[theorem]{Lemma}
\newtheorem{proposition}[theorem]{Proposition}
\newtheorem{corollary}[theorem]{Corollary}
\newtheorem{fact}[theorem]{Fact}
\newtheorem{definition}[theorem]{Definition}
\newtheorem{remark}[theorem]{Remark}
\newtheorem{conjecture}[theorem]{Conjecture}
\newtheorem{claim}[theorem]{Claim}
\newtheorem{example}[theorem]{Example}
\newenvironment{proof}{\noindent \textbf{Proof:}}{$\Box$}

\newcommand{\ignore}[1]{}

\newcommand{\enote}[1]{} \newcommand{\knote}[1]{}
\newcommand{\rnote}[1]{}



% \newcommand{\enote}[1]{{\bf [[Elchanan:} {\emph{#1}}{\bf ]]}}
% \newcommand{\knote}[1]{{\bf [[Krzysztof:} {\emph{#1}}{\bf ]]}}
% \newcommand{\rnote}[1]{{\bf [[Ryan:} {\emph{#1}}{\bf ]]}}



\DeclareMathOperator{\Support}{Supp} \DeclareMathOperator{\Opt}{Opt}
\DeclareMathOperator{\Ordo}{\mathcal{O}}
\newcommand{\MaxkCSP}{\textsc{Max $k$-CSP}}
\newcommand{\MaxkCSPq}{\textsc{Max $k$-CSP$_{q}$}}
\newcommand{\MaxCSP}[1]{\textsc{Max CSP}(#1)} \renewcommand{\Pr}{{\bf
    P}} \renewcommand{\P}{{\bf P}} \newcommand{\Px}{\mathop{\bf P\/}}
\newcommand{\E}{{\bf E}} \newcommand{\Cov}{{\bf Cov}}
\newcommand{\Var}{{\bf Var}} \newcommand{\Varx}{\mathop{\bf Var\/}}

\newcommand{\bits}{\{-1,1\}}

\newcommand{\nsmaja}{\textstyle{\frac{2}{\pi}} \arcsin \rho}

\newcommand{\Inf}{\mathrm{Inf}} \newcommand{\I}{\mathrm{I}}
\newcommand{\J}{\mathrm{J}}

\newcommand{\eps}{\epsilon} \newcommand{\lam}{\lambda}

% \newcommand{\trunc}{\ell_{2,[-1,1]}}
\newcommand{\trunc}{\zeta} \newcommand{\truncprod}{\chi}

\newcommand{\N}{\mathbb N} \newcommand{\R}{\mathbb R}
\newcommand{\Q}{\mathbb Q}
\newcommand{\Z}{\mathbb Z} \newcommand{\CalE}{{\mathcal{E}}}
\newcommand{\CalC}{{\mathcal{C}}} \newcommand{\CalM}{{\mathcal{M}}}
\newcommand{\CalR}{{\mathcal{R}}} \newcommand{\CalS}{{\mathcal{S}}}
\newcommand{\CalV}{{\mathcal{V}}}
\newcommand{\CalX}{{\boldsymbol{\mathcal{X}}}}
\newcommand{\CalG}{{\boldsymbol{\mathcal{G}}}}
\newcommand{\CalH}{{\boldsymbol{\mathcal{H}}}}
\newcommand{\CalY}{{\boldsymbol{\mathcal{Y}}}}
\newcommand{\CalZ}{{\boldsymbol{\mathcal{Z}}}}
\newcommand{\CalW}{{\boldsymbol{\mathcal{W}}}}
\newcommand{\CalF}{{\mathcal{Z}}}
% \newcommand{\boldG}{{\boldsymbol G}}
% \newcommand{\boldQ}{{\boldsymbol Q}}
% \newcommand{\boldP}{{\boldsymbol P}}
% \newcommand{\boldR}{{\boldsymbol R}}
% \newcommand{\boldS}{{\boldsymbol S}}
% \newcommand{\boldX}{{\boldsymbol X}}
% \newcommand{\boldB}{{\boldsymbol B}}
% \newcommand{\boldY}{{\boldsymbol Y}}
% \newcommand{\boldZ}{{\boldsymbol Z}}
% \newcommand{\boldV}{{\boldsymbol V}}
\newcommand{\boldi}{{\boldsymbol i}} \newcommand{\boldj}{{\boldsymbol
    j}} \newcommand{\boldk}{{\boldsymbol k}}
\newcommand{\boldr}{{\boldsymbol r}}
\newcommand{\boldsigma}{{\boldsymbol \sigma}}
\newcommand{\boldupsilon}{{\boldsymbol \upsilon}}
\newcommand{\hone}{{\boldsymbol{H1}}}
\newcommand{\htwo}{\boldsymbol{H2}}
\newcommand{\hthree}{\boldsymbol{H3}}
\newcommand{\hfour}{\boldsymbol{H4}}


\newcommand{\sgn}{\mathrm{sgn}} \newcommand{\Maj}{\mathrm{Maj}}
\newcommand{\Acyc}{\mathrm{Acyc}}
\newcommand{\UniqMax}{\mathrm{UniqMax}}
\newcommand{\Thr}{\mathrm{Thr}} \newcommand{\littlesum}{{\textstyle
    \sum}}

\newcommand{\half}{{\textstyle \frac12}}
\newcommand{\third}{{\textstyle \frac13}}
\newcommand{\fourth}{{\textstyle \frac14}}

\newcommand{\Stab}{\mathbb{S}}
\newcommand{\StabThr}[2]{\Gamma_{#1}(#2)}
\newcommand{\StabThrmin}[2]{{\underline{\Gamma}}_{#1}(#2)}
\newcommand{\StabThrmax}[2]{{\overline{\Gamma}}_{#1}(#2)}
\newcommand{\TestFcn}{\Psi}

\renewcommand{\phi}{\varphi}

\begin{document}
\title{Algebra Through Examples - Exercise 5}

 \author{Omer Tamuz, 035696574}
\maketitle


\begin{enumerate}
\item 
  By the notation defined in class, we interpret the first four bits of
  an 8 bit vector as $b$, an element of $GF(16)$, and the last four
  bits as $c$, also an element of the same field. We denote by $a$ a root of
  $x^4+x^3+1$ over $GF(2)$, a generator
  of $GF(16)$. Given $b$ and $c$, let $b_i=a^{i-1}$ and $b_j=a^{j-1}$ be the 
  roots of
  $x^2+bx+cb^{-1}+b^2$. Then we showed in class that $i$ and $j$ are the 
  positions of the errors, if there are two errors. If there is one error, then
  $c=b^3$, and if $b=a^{i-1}$ then the error is in position $i$. 
  \begin{enumerate}
  \item 
    For this vector we have $b=a$, $c=1+a+a^2+a^3=a^6$. Then there are at least
    two errors (since $b^3=a^3 \neq a^6=c$). If there are two then they are the roots 
    of $p(x)$, where 
    \begin{eqnarray*}
      p(x)&=& x^2+ax+a^6a^{-1}+a^2
      \\ &=&  x^2+ax+a^5+a^2
      \\ &=& x^2+ax+a^3+a+1+a^2
      \\ &=& x^2+ax+a^6.
    \end{eqnarray*}
    However, this polynomial has no roots in $GF(16)$, and hence there are more
    than two errors.
  \item
    In this case $b=c=a^2$, again there are at least two errors and
    \begin{eqnarray*}
      p(x)&=& x^2+a^2x+a^2a^{-2}+a^4
      \\ &=&  x^2+a^2x+1+a^3+1
      \\ &=& x^2+a^2x+a^3
      \\ &=& (x+a^8)(x+a^{10})
    \end{eqnarray*}
    Hence the errors are in positions 8 and 10.
  \end{enumerate}
\item
  Let $G$ be a group of order 30. The prime factorization of 30 is 
  $30=3 \cdot 5 \cdot 2$. 
  Hence $G$ has Sylow $p$-subgroups of order 3, 5 and 2. 

  For the number of Sylow $p$-subgroups, $n_p$, it holds that $n_p=1\mod p$ and
  $n_p$ divides $m=30/p$. Hence $n_3=1 \mod 3$ and $n_3$ divides 10. The only
  possible values are 1 or 10. If $n_3=1$ then $P_3$ is a normal subgroup 
  of $G$ and $G$ is simple. If $n_3=10$ then, since the 10 Sylow 3-subgroups
  intersect trivially (since any other element is a generator), and there
  are 20 elements of order 3. Now, $n_5=1\mod 5$ and divides 6. If $n_5=1$ then
  $G$ is simple again. Otherwise $n_5=6$, and by the same argument as above
  we have 24 elements of order 5. But this is impossible (there aren't enough
  elements in $G$ - $20+24>30$) and so $G$ is simple.

\item
  Let $G$ be a group of order $p^2$ for some prime $p$. Then $n$, the number of
  conjugacy classes of $G$ divides $p^2$, and so can only be $p$ or
  $p^2$ (it cannot be one, since for any group of size more that two the number
  of conjugacy classes is more than one). Assume $n=p$. Then 

\item
  
\end{enumerate}
\end{document}


