\documentclass[11pt]{article} \usepackage{amssymb}
\usepackage{amsfonts} \usepackage{amsmath} \usepackage{bm}
\usepackage{latexsym} \usepackage{epsfig}

\setlength{\textwidth}{6.5 in} \setlength{\textheight}{8.25in}
\setlength{\oddsidemargin}{0in} \setlength{\topmargin}{0in}
\addtolength{\textheight}{.8in} \addtolength{\voffset}{-.5in}

\newtheorem{theorem}{Theorem}[section]
\newtheorem{lemma}[theorem]{Lemma}
\newtheorem{proposition}[theorem]{Proposition}
\newtheorem{corollary}[theorem]{Corollary}
\newtheorem{fact}[theorem]{Fact}
\newtheorem{definition}[theorem]{Definition}
\newtheorem{remark}[theorem]{Remark}
\newtheorem{conjecture}[theorem]{Conjecture}
\newtheorem{claim}[theorem]{Claim}
\newtheorem{example}[theorem]{Example}
\newenvironment{proof}{\noindent \textbf{Proof:}}{$\Box$}

\newcommand{\ignore}[1]{}

\newcommand{\enote}[1]{} \newcommand{\knote}[1]{}
\newcommand{\rnote}[1]{}



% \newcommand{\enote}[1]{{\bf [[Elchanan:} {\emph{#1}}{\bf ]]}}
% \newcommand{\knote}[1]{{\bf [[Krzysztof:} {\emph{#1}}{\bf ]]}}
% \newcommand{\rnote}[1]{{\bf [[Ryan:} {\emph{#1}}{\bf ]]}}



\DeclareMathOperator{\Support}{Supp} \DeclareMathOperator{\Opt}{Opt}
\DeclareMathOperator{\Ordo}{\mathcal{O}}
\newcommand{\MaxkCSP}{\textsc{Max $k$-CSP}}
\newcommand{\MaxkCSPq}{\textsc{Max $k$-CSP$_{q}$}}
\newcommand{\MaxCSP}[1]{\textsc{Max CSP}(#1)} \renewcommand{\Pr}{{\bf
    P}} \renewcommand{\P}{{\bf P}} \newcommand{\Px}{\mathop{\bf P\/}}
\newcommand{\E}{{\bf E}} \newcommand{\Cov}{{\bf Cov}}
\newcommand{\Var}{{\bf Var}} \newcommand{\Varx}{\mathop{\bf Var\/}}

\newcommand{\bits}{\{-1,1\}}

\newcommand{\nsmaja}{\textstyle{\frac{2}{\pi}} \arcsin \rho}

\newcommand{\Inf}{\mathrm{Inf}} \newcommand{\I}{\mathrm{I}}
\newcommand{\J}{\mathrm{J}}

\newcommand{\eps}{\epsilon} \newcommand{\lam}{\lambda}

% \newcommand{\trunc}{\ell_{2,[-1,1]}}
\newcommand{\trunc}{\zeta} \newcommand{\truncprod}{\chi}

\newcommand{\N}{\mathbb N} \newcommand{\R}{\mathbb R}
\newcommand{\Q}{\mathbb Q}
\newcommand{\Z}{\mathbb Z} \newcommand{\CalE}{{\mathcal{E}}}
\newcommand{\CalC}{{\mathcal{C}}} \newcommand{\CalM}{{\mathcal{M}}}
\newcommand{\CalR}{{\mathcal{R}}} \newcommand{\CalS}{{\mathcal{S}}}
\newcommand{\CalV}{{\mathcal{V}}}
\newcommand{\CalX}{{\boldsymbol{\mathcal{X}}}}
\newcommand{\CalG}{{\boldsymbol{\mathcal{G}}}}
\newcommand{\CalH}{{\boldsymbol{\mathcal{H}}}}
\newcommand{\CalY}{{\boldsymbol{\mathcal{Y}}}}
\newcommand{\CalZ}{{\boldsymbol{\mathcal{Z}}}}
\newcommand{\CalW}{{\boldsymbol{\mathcal{W}}}}
\newcommand{\CalF}{{\mathcal{Z}}}
% \newcommand{\boldG}{{\boldsymbol G}}
% \newcommand{\boldQ}{{\boldsymbol Q}}
% \newcommand{\boldP}{{\boldsymbol P}}
% \newcommand{\boldR}{{\boldsymbol R}}
% \newcommand{\boldS}{{\boldsymbol S}}
% \newcommand{\boldX}{{\boldsymbol X}}
% \newcommand{\boldB}{{\boldsymbol B}}
% \newcommand{\boldY}{{\boldsymbol Y}}
% \newcommand{\boldZ}{{\boldsymbol Z}}
% \newcommand{\boldV}{{\boldsymbol V}}
\newcommand{\boldi}{{\boldsymbol i}} \newcommand{\boldj}{{\boldsymbol
    j}} \newcommand{\boldk}{{\boldsymbol k}}
\newcommand{\boldr}{{\boldsymbol r}}
\newcommand{\boldsigma}{{\boldsymbol \sigma}}
\newcommand{\boldupsilon}{{\boldsymbol \upsilon}}
\newcommand{\hone}{{\boldsymbol{H1}}}
\newcommand{\htwo}{\boldsymbol{H2}}
\newcommand{\hthree}{\boldsymbol{H3}}
\newcommand{\hfour}{\boldsymbol{H4}}


\newcommand{\sgn}{\mathrm{sgn}} \newcommand{\Maj}{\mathrm{Maj}}
\newcommand{\Acyc}{\mathrm{Acyc}}
\newcommand{\UniqMax}{\mathrm{UniqMax}}
\newcommand{\Thr}{\mathrm{Thr}} \newcommand{\littlesum}{{\textstyle
    \sum}}

\newcommand{\half}{{\textstyle \frac12}}
\newcommand{\third}{{\textstyle \frac13}}
\newcommand{\fourth}{{\textstyle \frac14}}

\newcommand{\Stab}{\mathbb{S}}
\newcommand{\StabThr}[2]{\Gamma_{#1}(#2)}
\newcommand{\StabThrmin}[2]{{\underline{\Gamma}}_{#1}(#2)}
\newcommand{\StabThrmax}[2]{{\overline{\Gamma}}_{#1}(#2)}
\newcommand{\TestFcn}{\Psi}

\renewcommand{\phi}{\varphi}

\begin{document}
\title{Algebra Through Examples - Exercise 3}

 \author{Omer Tamuz, 035696574}
\maketitle


\begin{enumerate}
\item 
  Since $\xi_m$ is a root of $x^m-1$, it is algebraic, has
  a minimum polynomial $g_m(x)$, and $|\Q(\xi_m):\Q|=\deg g_x(x)$. 
  Since $g_m(x)$ is minimal then it we showed in class that it divides
  $x^m-1$ and
  there exists some polynomial $q_m(x)$ such that $x^m-1=g_m(x)q_m(x)$. Since both
  $x^m-1$ and $g_m(x)$ are monic, then so is $q_m(x)$. Furthermore, both $g_m(x)$ and
  $q_m(x)$ have integer coefficients. 

  Assume $m>2$, since for $m=1,2$ we know that $\Q(\xi_m)=Q$ and so 
  $|\Q(\xi_m):Q|=1$.
  Assume $|\Q(\xi_m):\Q|=2$. Then $\deg g_x(x)=2$ and there exist $b,c \in \Z$ such that
  $g_m(x)=x^2+bx+c$, and $\xi_m=-b/2\pm \sqrt{b^2/4-c}$. Since for $m>2$ 
  we know that $\xi_m$
  is not real, then $b^2/4-c$ must be negative. Then, since $|\xi_m|=1$, we
  know that $1=(-b/2)^2+(c-b^2/4)=c$. Since the real part of $\xi_m$, $-b/2$, must be
  strictly between minus one and one, then $b$ must be in $\{0,1,-1\}$. 

  \begin{itemize}
  \item For $m=3$ we have $g_m(x)=x^2+x+1$, and so indeed 
    $|\Q(\xi_3):\Q|=2$, since $x^2+x+1$ is irreducible.
  \item For $m=4$ we have $g_m(x)=x^2+1$, and again indeed
    $|\Q(\xi_4):\Q|=2$.
  \item For $m>4$ the only option left for $b$ is -1, so $g_m(x)=x^2-x+1$, 
    and $\xi_m$ is
    $\half+i\sqrt{3/4}$. This number is a sixth (and no less) root of unity, and hence
    $m=6$ is the final value of $m$ for which  $|\Q(\xi_m):\Q|=2$.
  \end{itemize}

  \item 
    In all the answers below, $\mbox{span}A$ refers to the vector space
    over $\Q$ spanned by the members of $A$. All the fields trivially contain
    1 and -1, second roots of unity.

    \begin{itemize}
    \item $\Q(i)$, $\Q(\sqrt{-2})$, $\Q(\sqrt{-3})$, $\Q(\sqrt{-5})$

      Let $a\in\{1, \sqrt{2}, \sqrt{3}, \sqrt{5}\}$. 
      We know that $|\Q(ia):\Q|=2$. Assume a primitive root of unity of 
      order $m$, $\xi_m$, is in $\Q(ia)$. Then $\Q(ia)$ must include $\Q(\xi_m)$ and 
      $2=|\Q(ia):\Q|=|\Q(ia):\Q(\xi_m)||\Q(\xi_m):\Q|$. Since both
      $|\Q(ia):\Q(\xi_m)|$ and 
      $|\Q(\xi_m):\Q|$ are positive integers, this is possible only if one of
      them is one and the other is two. If $|\Q(\xi_m):\Q|=1$  then
      $\Q=\Q(\xi_m)$, which is only possible if $\xi_m$ is 1 or -1. And indeed
      $\Q(ia)$ includes both 1 and -1. Otherwise $|\Q(\xi_m):\Q|=2$
      and so $m\in\{3,4,6\}$ (as we showed above). $\Q(\xi_m)$ is hence either
      $\Q(i)$ or $\Q(\alpha)$, with $\alpha=\half+\half i\sqrt{3}$, 
      $\alpha^6=1$. In this case, because  $|\Q(ia):\Q(\xi_m)|=1$, then 
      $\Q(ia)=\Q(\xi_m)$.

      $\Q(i)$ does not include $\alpha$, since the imaginary part of
      $\alpha$ is irrational, and the imaginary part of any $x\in\Q(i)$ is
      rational. Hence the roots of unity in $\Q(i)$ are 1, -1, $i$ and $-i$.

      Since $\Q(\sqrt{-3})$ includes $\alpha$ then it equals $\Q(\alpha)$, and its roots
      of unity are the six powers of $\alpha$.
      
      $\Q(\sqrt{-2})$ and $\Q(\sqrt{-5})$ are equal to neither $\Q(i)$ nor
      $\Q(\alpha)$, and hence the roots of unity that they include are only
      1 and -1. They are not equal to $\Q(i)$ since the imaginary part of their
      elements are irrational. They are not equal to $\Q(\sqrt{-3})$ because
      $\sqrt{3}/\sqrt{2}$ and $\sqrt{3}/\sqrt{5}$ are irrational, so $\alpha$
      is not an element of $\Q(\sqrt{2})$ or $\Q(\sqrt{5})$, respectively.
      
      

    \item $\Q(\sqrt{2})$, $\Q(\sqrt{3})$

      Both of these fields are subsets of $\R$, and hence include no
      roots of unity other than 1 and -1.


    \end{itemize}
  \item Let $F$ be an infinite field, and let $F^*=F \setminus \{0\}$ be
    its multiplicative group. Assume by way of contradiction that
    there exists some $a\in F^*$ such that $F^*=\{a^k | k \in \Z\}$. Then, since 
    $F^*$ includes $F$'s unity, there must exist some $k$ such that
    $a^k=1$. But then
    for every $m$ such that $|m|>k$ there exist some $q$ and $0\leq r<k$ 
    such that $m=qk+r$, so that  $a^m=a^{qk+r}=a^{qk}a^r=\left(a^{k}\right)^qa^r=1a^r=a^r$.
    Hence $|F^*| \leq k$, and we have a contradiction.
  \item
    \begin{itemize}
    \item $x^3-x-1$

      This polynomial is irreducible, since it has a single real root which
      is strictly between 1 and 2 (proof omitted), and hence irrational. 
      Call its real root
      $\alpha$ and its complex roots $\beta$ and $-\beta$. 

      Since $\alpha$ is irrational and $\beta$ is complex, then 
      $|\Q(\alpha,\beta):\Q|=2$. Hence the Galois group of this polynomial
      includes the identity and complex conjugation only, and is isomorphic
      to $C_2$.
      
    \item $x^3-10$
      
      The same proof of the previous item applies to this polynomial, with the
      minor modification that the real root is strictly between 2 and three.

    \item $x^3-10$ over $\Q(\sqrt{2})$

      Since $\sqrt[3]{10}$ is not a rational multiple of $\sqrt{2}$ then
      $x^3-10$ is irreducible also over $\Q(\sqrt{2})$ then the same
      also holds is this case.

    \end{itemize}
    In the answers above we used these facts, which were shown in class:
    \begin{enumerate}
    \item If a 
      polynomial with integer coefficients has a rational root then that root
      is also an integer.
    \item Complex conjugation is an element of the Galois group of any 
      polynomial, and is different than the identity map for any polynomial
      with complex roots.
    \end{enumerate}
    
      
  \item
    Let $K=\Q(\sqrt[4]{5}, i)$.
    The roots of $x^4-5$ are 
    $\{\sqrt[4]{5},-\sqrt[4]{5}, i\sqrt[4]{5},-i\sqrt[4]{5}\}$.
    Hence the splitting field of $x^4-5$ is $K$.

    \begin{itemize}
    \item $x^4-5$ over $\Q$

      Since the first 
      power of $\sqrt[4]{5}$ which is rational is 4, then 
      $\{1,\sqrt[4]{5}, \sqrt[4]{5}^2, \sqrt[4]{5}^3\}$ are all linearly
      independent over $\Q$, $|\Q(\sqrt[4]{5}):\Q|=4$ and 
      $|K:\Q|=8$.

      The Galois group of $K:\Q$ is therefore of order 8, has an element of
      order two (complex conjugation) and an element of maximal order four (map
      each of the four roots to the next cyclically), and so can only be 
      isomorphic to $D_8$. This proof is identical to the one done in class
      for $x^4-2$.
      
    \item $x^4-5$ over $\Q(\sqrt{5})$
      
      $|K:\Q(\sqrt{5})|=|K:\Q(\sqrt{5},i)|\cdot|\Q(\sqrt{5},i):\Q(\sqrt{5})|$.
      Since both expressions on the right are equal to two then the Galois
      group is either $C_4$ or $C_2\times C_2$. Since it has an element of order four 
      (the same cyclical map) then it must be $C_4$.

    \item $x^4-5$ over $\Q(\sqrt{-5})$

    \end{itemize}  

  \item
    Let $F$ be a subfield of a field $K$, and $a$ an element of $K$.
    {\bf First Direction}

    Let $a$ be algebraic over $F$. Then there exists some polynomial in $F[x]$
    which has $a$ as a root. Let $g(x)$ be the minimum polynomial of $a$, and
    let $n=\deg g(x)$. 
    We showed in class that $F(a)=\mbox{span}\{1, a, a^2, \ldots, a^{n-1}\}$. 
    
    Let $h(a)$ be in $F[a]$. 
    Then there exists a polynomial
    $r(x)$ with $\deg r(x) < n$, and a polynomial $q(x)$ such that 
    $h(x)=q(x)g(x)+r(x)$. Since $g(a)=0$, we have 
    $h(a)=r(a)=\sum_{0\leq i <n}c_ia^i$, for some $c_i$'s in $F$. Since
    $\sum_{0\leq i <n}c_ia^i \in F(a)$, we have shown that $F[a] \subseteq F(a)$. 
    Conversely, given $k \in F(a)$, there exist $c_i$'s such that $k=\sum_{0\leq i <n}c_ia^i$,
    and so $k=r(a)$, where $r(x)=\sum_{0\leq i <n}c_ix^i \in F[x]$. 
    Hence $F(a) \subseteq F[a]$.

    {\bf Second Direction}

    Let $a$ be such that $F[a]=F(a)$. Then there exists a polynomial 
    $g(x) \in F[x]$ such that $g(a)=a^{-1}$. Let $g(x)=\sum_{0\leq i < n}c_ix^i$. Then
    \begin{eqnarray*}
      g(a)=a^{-1}&=&\sum_{0\leq i < n}c_ia^i
      \\ 1 &=& \sum_{1\leq i < n + 1}c_ia^i
      \\ 0 &=& \sum_{1\leq i < n + 1}c_ia^i - 1
    \end{eqnarray*}
    
    Hence $h(x)=\sum_{1\leq i < n + 1}c_ia^i - 1$ is a polynomial in $F[x]$ which
    has $a$ as a root, and hence $a$ is algebraic.

\end{enumerate}
\end{document}


