\documentclass[11pt]{article} \usepackage{amssymb}
\usepackage{amsfonts} \usepackage{amsmath} \usepackage{bm}
\usepackage{latexsym} \usepackage{epsfig}

\setlength{\textwidth}{6.5 in} \setlength{\textheight}{8.25in}
\setlength{\oddsidemargin}{0in} \setlength{\topmargin}{0in}
\addtolength{\textheight}{.8in} \addtolength{\voffset}{-.5in}

\newtheorem{theorem}{Theorem}[section]
\newtheorem{lemma}[theorem]{Lemma}
\newtheorem{proposition}[theorem]{Proposition}
\newtheorem{corollary}[theorem]{Corollary}
\newtheorem{fact}[theorem]{Fact}
\newtheorem{definition}[theorem]{Definition}
\newtheorem{remark}[theorem]{Remark}
\newtheorem{conjecture}[theorem]{Conjecture}
\newtheorem{claim}[theorem]{Claim}
\newtheorem{example}[theorem]{Example}
\newenvironment{proof}{\noindent \textbf{Proof:}}{$\Box$}

\newcommand{\ignore}[1]{}

\newcommand{\enote}[1]{} \newcommand{\knote}[1]{}
\newcommand{\rnote}[1]{}



% \newcommand{\enote}[1]{{\bf [[Elchanan:} {\emph{#1}}{\bf ]]}}
% \newcommand{\knote}[1]{{\bf [[Krzysztof:} {\emph{#1}}{\bf ]]}}
% \newcommand{\rnote}[1]{{\bf [[Ryan:} {\emph{#1}}{\bf ]]}}



\DeclareMathOperator{\Support}{Supp} \DeclareMathOperator{\Opt}{Opt}
\DeclareMathOperator{\Ordo}{\mathcal{O}}
\newcommand{\MaxkCSP}{\textsc{Max $k$-CSP}}
\newcommand{\MaxkCSPq}{\textsc{Max $k$-CSP$_{q}$}}
\newcommand{\MaxCSP}[1]{\textsc{Max CSP}(#1)} \renewcommand{\Pr}{{\bf
    P}} \renewcommand{\P}{{\bf P}} \newcommand{\Px}{\mathop{\bf P\/}}
\newcommand{\E}{{\bf E}} \newcommand{\Cov}{{\bf Cov}}
\newcommand{\Var}{{\bf Var}} \newcommand{\Varx}{\mathop{\bf Var\/}}

\newcommand{\bits}{\{-1,1\}}

\newcommand{\nsmaja}{\textstyle{\frac{2}{\pi}} \arcsin \rho}

\newcommand{\Inf}{\mathrm{Inf}} \newcommand{\I}{\mathrm{I}}
\newcommand{\J}{\mathrm{J}}

\newcommand{\eps}{\epsilon} \newcommand{\lam}{\lambda}

% \newcommand{\trunc}{\ell_{2,[-1,1]}}
\newcommand{\trunc}{\zeta} \newcommand{\truncprod}{\chi}

\newcommand{\N}{\mathbb N} \newcommand{\R}{\mathbb R}
\newcommand{\Z}{\mathbb Z} \newcommand{\CalE}{{\mathcal{E}}}
\newcommand{\CalC}{{\mathcal{C}}} \newcommand{\CalM}{{\mathcal{M}}}
\newcommand{\CalR}{{\mathcal{R}}} \newcommand{\CalS}{{\mathcal{S}}}
\newcommand{\CalV}{{\mathcal{V}}}
\newcommand{\CalX}{{\boldsymbol{\mathcal{X}}}}
\newcommand{\CalG}{{\boldsymbol{\mathcal{G}}}}
\newcommand{\CalH}{{\boldsymbol{\mathcal{H}}}}
\newcommand{\CalY}{{\boldsymbol{\mathcal{Y}}}}
\newcommand{\CalZ}{{\boldsymbol{\mathcal{Z}}}}
\newcommand{\CalW}{{\boldsymbol{\mathcal{W}}}}
\newcommand{\CalF}{{\mathcal{Z}}}
% \newcommand{\boldG}{{\boldsymbol G}}
% \newcommand{\boldQ}{{\boldsymbol Q}}
% \newcommand{\boldP}{{\boldsymbol P}}
% \newcommand{\boldR}{{\boldsymbol R}}
% \newcommand{\boldS}{{\boldsymbol S}}
% \newcommand{\boldX}{{\boldsymbol X}}
% \newcommand{\boldB}{{\boldsymbol B}}
% \newcommand{\boldY}{{\boldsymbol Y}}
% \newcommand{\boldZ}{{\boldsymbol Z}}
% \newcommand{\boldV}{{\boldsymbol V}}
\newcommand{\boldi}{{\boldsymbol i}} \newcommand{\boldj}{{\boldsymbol
    j}} \newcommand{\boldk}{{\boldsymbol k}}
\newcommand{\boldr}{{\boldsymbol r}}
\newcommand{\boldsigma}{{\boldsymbol \sigma}}
\newcommand{\boldupsilon}{{\boldsymbol \upsilon}}
\newcommand{\hone}{{\boldsymbol{H1}}}
\newcommand{\htwo}{\boldsymbol{H2}}
\newcommand{\hthree}{\boldsymbol{H3}}
\newcommand{\hfour}{\boldsymbol{H4}}


\newcommand{\sgn}{\mathrm{sgn}} \newcommand{\Maj}{\mathrm{Maj}}
\newcommand{\Acyc}{\mathrm{Acyc}}
\newcommand{\UniqMax}{\mathrm{UniqMax}}
\newcommand{\Thr}{\mathrm{Thr}} \newcommand{\littlesum}{{\textstyle
    \sum}}

\newcommand{\half}{{\textstyle \frac12}}
\newcommand{\third}{{\textstyle \frac13}}
\newcommand{\fourth}{{\textstyle \frac14}}

\newcommand{\Stab}{\mathbb{S}}
\newcommand{\StabThr}[2]{\Gamma_{#1}(#2)}
\newcommand{\StabThrmin}[2]{{\underline{\Gamma}}_{#1}(#2)}
\newcommand{\StabThrmax}[2]{{\overline{\Gamma}}_{#1}(#2)}
\newcommand{\TestFcn}{\Psi}

\renewcommand{\phi}{\varphi}

\begin{document}
\title{Algebra Through Examples - Exercise 2}

 \author{Omer Tamuz, 035696574}
\maketitle


\begin{enumerate}
  \item An element $x$ of $\Z[\sqrt{-5}]$ is of the form $x=a+b\sqrt{-5}$, with $a,b\in\Z$. 

    Let $\delta(x)=|x|^2=a^2+5b^2$. Since $a,b$ are in $\Z$, then so is $\delta(x)$. Also,
    it is easy to see that $\delta(xy)=\delta(x)\delta(y)$, and that 
    $\delta(x)=1\leftrightarrow x\in\{-1,1\}$.

    Let
    $x=a+b\sqrt{-5}$ be invertible, and let $y=c+d\sqrt{-5}$ be its inverse, 
    so that $xy=1$. 
    

    Then, since $\delta(xy)=\delta(x)\delta(y)$,
    we have $\delta(y)=1/\delta(z)$. Since $\delta(x),\delta(y)\in\Z$, we have 
    $\delta(x)=\delta(y)=1$. And since the only members of $\Z[\sqrt{-5}]$
    for which $\delta$ is one are one and minus one, the proof is concluded.
    
  \item Assume $x$ is a gcd of $6$ and $2+2\sqrt{-5}$ in $\Z[\sqrt{-5}]$, so that there exist
    $y,z$ such that $xy=6$
    and $xz=2+2\sqrt{-5}$, and all other common divisors also divide $x$.


    Using the definition of $\delta$ from above, we note that any for divisor $a$
    of $6$ and $2+2\sqrt{-5}$ it will hold that 
    $\delta(a)\leq \min(\delta(6),\delta(2+2\sqrt{-5}))=\min(36,24)=24$. The number of elements of 
    $\Z[\sqrt{-5}]$ which are common divisors and for which this condition holds is 
    finite, and in fact
    rather small: $1,2,1+\sqrt{-5}$ and $1-\sqrt{-5}$. Since none of them can be divided
    by all the rest no gcd exists. Hence $\Z[\sqrt{-5}]$ is not Euclidean.

  \item Let $R$ be a PID.  To show that $R$ is a UFD we need to show that
    it each element can be written as a product of irreducibles, and then
    show that this product is unique.

    \begin{itemize}
    \item {\bf Existence of a Decomposition into Irreducibles}

      Let $R^*$ be the set of elements of $R$ that cannot be written as products
      of irreducible elements.
      Assume by contradiction that $R^*$ is non-empty and let $x$ be a member
      of $R^*$.
      Then $x$ is reducible and $x=a_1a_2$, where both $a_1$ and $a_2$ are not 
      units. Furthermore, at least one of $a_1$ and $a_2$ is also in $R^*$.

      We can therefore define tha function $f:R^*\to R^*$ to be a function
      that assigns to every $x\in R^*$ one of its factors in $R^*$, 
      where the factor is part of a non-unit decomposition of $x$.

      We now define a series of elements of $a_j\in R$, with $a_0=x$, 
      and $a_{j+1}=f(a_j)$. We claim that the series of ideals $I_j=a_jR$ is strictly
      increasing. Since $a_j=a_{j+1}b$ for some $b$, then $I_j$ is clearly
      contained in $I_{j+1}$. However, $a_{j+1}$ is not a member of $I_j$ (but
      is a member of $I_{j+1}$): we have $a_j=a_{j+1}b$ for some non-unit $b$. If 
      $a_{j+1}$ were to be in $I_j$, then 
      there would be a $c\in R$ such that $a_{j+1}bc=a_{j+1}$, and hence $b$ would be
      a unit.

      However, since $R$ is a PID, then the series $\{I_j\}$ stabilizes, and 
      hence cannot be strictly increasing, and we have a contradiction. Hence
      $R^*$ is empty, and every element of $R$ can be written as a product
      of irreducibles.
      
    \item {\bf Decomposition into Irreducibles is unique}

      Let $x$ be an element of $R$, so that 
      $x=\Pi_{i=1}^Mp_i$, with $p_i$ irreducible non-units.
      Let $x$ also equal $\Pi_{j=1}^Nq_j$, with $q_j$ irreducible non-units.
      In a PID all irreducibles are
      prime, and hence, because $p_1$ divides $x=\Pi_jq_j=q_1\Pi_{j>1}q_j$,
      it either divides $q_1$ or $\Pi_{j>1}q_j$. 
      Since $q_1$ is irreducible, then either $p_1=q_1$ (up to multiplication by
      a unit), or else $p$ divides
      $\Pi_{j>1}q_j$, and by induction $p_1=q_j$ for some $j$. WLOG assume $p_1=q_1$. Then
      we can apply the above procedure by induction to 
      $x_1=\Pi_{i>1}p_i=\Pi_{j>1}q_j$ and show that
      $\forall i:p_i=q_i$, up to multiplication by a unit. For all $i\leq M,N$.  
      
      If $M<N$ then $x_M$ is equal to both one (what is left of the product
      of the $p$'s) and to $\Pi_{j>M}q_j$, which can't be one, since the $q_j$'s 
      are not units. By symmetry $N<M$ is also impossible.
      
      We have therefore shown that $M=N$ and that for every $i$, $p_i=q_i$, and
      hence $R$ is a UFD.
    \end{itemize}

  \item
    \begin{enumerate}
      \item To show that $\phi$ is an automorphism, we first show that it
        is an isomorphism,
        conserving addition and multiplication:
        \begin{enumerate}
          \item {\em Addition:} 
            \begin{eqnarray*}
              \phi(x)+\phi(y)&=& \phi(a_x+b_x\sqrt{10})+\phi(a_y+b_y\sqrt{10})
              \\ &=& a_x-b_x\sqrt{10}+a_y-b_y\sqrt{10}
              \\ &=& a_x+a_y-(b_x+b_y)\sqrt{10}
              \\ &=& \phi(a_x+a_y+(b_x+b_y)\sqrt{10})
              \\ &=& \phi(x+y)
            \end{eqnarray*}
          \item {\em Multiplication:} 
            \begin{eqnarray*}
              \phi(x)\cdot\phi(y)
              &=& \phi(a_x+b_x\sqrt{10})\cdot\phi(a_y+b_y\sqrt{10})
              \\ &=& (a_x-b_x\sqrt{10})\cdot(a_y-b_y\sqrt{10})
              \\ &=& a_xa_y+10b_xb_y-(a_xb_y+a_yb_x)\sqrt{10}
              \\ &=& \phi(a_xa_y+10b_xb_y+(a_xb_y+a_yb_x)\sqrt{10})
              \\ &=& \phi((a_x+b_x\sqrt{10})\cdot(a_y+b_y\sqrt{10}))
              \\ &=& \phi(x\cdot y)
            \end{eqnarray*}
        \end{enumerate}
        We then show that it's a automorphism by noting that it is a bijection,
        since it is its own inverse.
      \item 
        Let $U=\{x=a+b\sqrt{10}:a^2-10b^2=\pm1\}$. Let $I$ be the set of invertible
        element of $\Z[\sqrt{10}]$. We would like to show that $U=I$.
        \begin{enumerate}
          \item $I \subseteq U$.

            Let $x=a_x+b_x\sqrt{10}\in I$ be a unit, so that there 
            exists a $y=a_y+b_y\sqrt{10}$ such that $xy=1$. 

            By the previous question $1=\phi(1)=\phi(xy)=\phi(x)\phi(y)$, and 
            so $\phi(x)$ is also invertible, with $\phi(y)$ its inverse. Then
            \begin{eqnarray*}
              1=(xy)(\phi(x)\phi(y))=(x\phi(x))(y\phi(y))
              =(a_x^2-10b_x^2)(a_y^2-10b_y^2)
            \end{eqnarray*}
            Since $a_x,b_x,a_y,b_y\in\Z$, this is only possible if 
            $a_x^2-10b_x^2=\pm 1$ and $a_y^2-10b_y^2=\pm 1$, and in particular when $x\in U$.
          \item $U \subseteq I$.

            Let $x=a+b\sqrt{10}\in U$ 
            be such that $a^2-10b^2=\pm1$. Let $y=\pm (a- b\sqrt{10})$. Then
            $xy=(a+b\sqrt{10})\cdot\pm(a- b\sqrt{10})=
            \pm (a^2- 10b^2)=\pm 1$, and hence
            $x$ is invertible and in $I$.
          \end{enumerate}
          and hence $U=I$.  
        \item We will show that $\Z[\sqrt{10}]$ is not a UFD by showing that
          it has irreducibles that are not prime, in contradiction to what
          was stated in class.
          \begin{itemize}
          \item {\bf 2 is irreducible in $\Z[\sqrt{10}]$}
            
            Let $xy=2$. Then $xy\phi(xy)=2\phi(2)$, and so
            $x\phi(x)y\phi(y)=4$. Assume by contradiction that neither $x$ or 
            $y$ are units. Then $x\phi(x)=\pm 2$.
            Let $a=a+b\sqrt{10}$. Then $a^2-10b^2=\pm 2$, and in particular  
            $a^2=\pm 2\mod 10$. However, no number squared is equal to 2 or 8 mod
            10.
          \item {\bf 2 is not prime in $\Z[\sqrt{10}]$}
            $2=\sqrt{10}\sqrt{10}$. But 2 clearly does not divide $\sqrt{10}$,
            and hence it is not prime.

          \end{itemize}
    \end{enumerate}
  \item Let $R$ be a Euclidean domain with Euclidean norm $d$. Then by
    definition $\forall x,y\in R:d(x)\leq d(xy)$. 

    Let $u$
    be a unit in $R$, and let $x$ be some other element in $R$. Then
    there exists an element $v$ such that $uv=1$. Futhermore, 
    $d(u)\leq d(uv) \leq d(uvx) = d(1x) = d(x)$. Hence, $d(u)$ is minimal.
    
      
\end{enumerate}
\end{document}


