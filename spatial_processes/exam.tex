\documentclass[11pt]{article} \usepackage{amssymb}
\usepackage{amsfonts} \usepackage{amsmath} \usepackage{bm}
\usepackage{latexsym} \usepackage{epsfig}

\setlength{\textwidth}{6.5 in} \setlength{\textheight}{8.25in}
\setlength{\oddsidemargin}{0in} \setlength{\topmargin}{0in}
\addtolength{\textheight}{.8in} \addtolength{\voffset}{-.5in}

\newtheorem{theorem}{Theorem}[section]
\newtheorem{lemma}[theorem]{Lemma}
\newtheorem{proposition}[theorem]{Proposition}
\newtheorem{corollary}[theorem]{Corollary}
\newtheorem{fact}[theorem]{Fact}
\newtheorem{definition}[theorem]{Definition}
\newtheorem{remark}[theorem]{Remark}
\newtheorem{conjecture}[theorem]{Conjecture}
\newtheorem{claim}[theorem]{Claim}
\newtheorem{example}[theorem]{Example}
\newenvironment{proof}{\noindent \textbf{Proof:}}{$\Box$}

\newcommand{\ignore}[1]{}

\newcommand{\enote}[1]{} \newcommand{\knote}[1]{}
\newcommand{\rnote}[1]{}



% \newcommand{\enote}[1]{{\bf [[Elchanan:} {\emph{#1}}{\bf ]]}}
% \newcommand{\knote}[1]{{\bf [[Krzysztof:} {\emph{#1}}{\bf ]]}}
% \newcommand{\rnote}[1]{{\bf [[Ryan:} {\emph{#1}}{\bf ]]}}



\DeclareMathOperator{\Support}{Supp} \DeclareMathOperator{\Opt}{Opt}
\DeclareMathOperator{\Ordo}{\mathcal{O}}
\newcommand{\MaxkCSP}{\textsc{Max $k$-CSP}}
\newcommand{\MaxkCSPq}{\textsc{Max $k$-CSP$_{q}$}}
\newcommand{\MaxCSP}[1]{\textsc{Max CSP}(#1)} \renewcommand{\Pr}{{\bf
    P}} \renewcommand{\P}{{\bf P}} \newcommand{\Px}{\mathop{\bf P\/}}
\newcommand{\E}{{\bf E}} \newcommand{\Cov}{{\bf Cov}}
\newcommand{\Var}{{\bf Var}} \newcommand{\Varx}{\mathop{\bf Var\/}}

\newcommand{\bits}{\{-1,1\}}

\newcommand{\nsmaja}{\textstyle{\frac{2}{\pi}} \arcsin \rho}

\newcommand{\Inf}{\mathrm{Inf}} \newcommand{\I}{\mathrm{I}}
\newcommand{\J}{\mathrm{J}}

\newcommand{\eps}{\epsilon} \newcommand{\lam}{\lambda}

% \newcommand{\trunc}{\ell_{2,[-1,1]}}
\newcommand{\trunc}{\zeta} \newcommand{\truncprod}{\chi}

\newcommand{\N}{\mathbb N} \newcommand{\R}{\mathbb R}
\newcommand{\Z}{\mathbb Z} \newcommand{\CalE}{{\mathcal{E}}}
\newcommand{\CalC}{{\mathcal{C}}} \newcommand{\CalM}{{\mathcal{M}}}
\newcommand{\CalR}{{\mathcal{R}}} \newcommand{\CalS}{{\mathcal{S}}}
\newcommand{\CalV}{{\mathcal{V}}}
\newcommand{\CalX}{{\boldsymbol{\mathcal{X}}}}
\newcommand{\CalG}{{\boldsymbol{\mathcal{G}}}}
\newcommand{\CalH}{{\boldsymbol{\mathcal{H}}}}
\newcommand{\CalY}{{\boldsymbol{\mathcal{Y}}}}
\newcommand{\CalZ}{{\boldsymbol{\mathcal{Z}}}}
\newcommand{\CalW}{{\boldsymbol{\mathcal{W}}}}
\newcommand{\CalF}{{\mathcal{Z}}}
% \newcommand{\boldG}{{\boldsymbol G}}
% \newcommand{\boldQ}{{\boldsymbol Q}}
% \newcommand{\boldP}{{\boldsymbol P}}
% \newcommand{\boldR}{{\boldsymbol R}}
% \newcommand{\boldS}{{\boldsymbol S}}
% \newcommand{\boldX}{{\boldsymbol X}}
% \newcommand{\boldB}{{\boldsymbol B}}
% \newcommand{\boldY}{{\boldsymbol Y}}
% \newcommand{\boldZ}{{\boldsymbol Z}}
% \newcommand{\boldV}{{\boldsymbol V}}
\newcommand{\boldi}{{\boldsymbol i}} \newcommand{\boldj}{{\boldsymbol
    j}} \newcommand{\boldk}{{\boldsymbol k}}
\newcommand{\boldr}{{\boldsymbol r}}
\newcommand{\boldsigma}{{\boldsymbol \sigma}}
\newcommand{\boldupsilon}{{\boldsymbol \upsilon}}
\newcommand{\hone}{{\boldsymbol{H1}}}
\newcommand{\htwo}{\boldsymbol{H2}}
\newcommand{\hthree}{\boldsymbol{H3}}
\newcommand{\hfour}{\boldsymbol{H4}}


\newcommand{\sgn}{\mathrm{sgn}} \newcommand{\Maj}{\mathrm{Maj}}
\newcommand{\Acyc}{\mathrm{Acyc}}
\newcommand{\UniqMax}{\mathrm{UniqMax}}
\newcommand{\Thr}{\mathrm{Thr}} \newcommand{\littlesum}{{\textstyle
    \sum}}

\newcommand{\half}{{\textstyle \frac12}}
\newcommand{\third}{{\textstyle \frac13}}
\newcommand{\fourth}{{\textstyle \frac14}}

\newcommand{\Stab}{\mathbb{S}}
\newcommand{\StabThr}[2]{\Gamma_{#1}(#2)}
\newcommand{\StabThrmin}[2]{{\underline{\Gamma}}_{#1}(#2)}
\newcommand{\StabThrmax}[2]{{\overline{\Gamma}}_{#1}(#2)}
\newcommand{\TestFcn}{\Psi}

\renewcommand{\phi}{\varphi}

\begin{document}
\title{Random Spatial Processes - Homework}

 \author{Omer Tamuz, 035696574}
\maketitle


\begin{enumerate}
  \item
  \begin{enumerate}
  \item Let $G_X(\eta)=\E[\eta^X]$ where $X$ is distributed geometrically with
    parameter $p$. We showed in class that when $p \leq \half$ then the 
    extinction probability is one,
    since the expcted number of offsprings is less than or equal to one.
    For $p>\half$ we showed that the extinction probability $\eta$ is
    the smallest solution of $\eta=G_X(\eta)$:
    \begin{eqnarray*}
      \eta &=& G_X(\eta)
      \\ &=& \E[\eta^X]
      \\ &=& \sum_{i=0}^\infty \eta^i\P[X=i]
      \\ &=& \sum_{i=0}^\infty \eta^i(1-p)p^i
      \\ &=& (1-p)\sum_{i=0}^\infty (\eta p)^i
      \\ &=& {1-p \over 1 - \eta p}
    \end{eqnarray*}
    This gives a quadratic equation on $\eta$: $p\eta^2-\eta+1-p=0$.
    \begin{eqnarray*}
      0 &=& p\eta^2-\eta+1-p
      \\ 0 &=& \eta^2-\eta/p+{1-p \over p}
      \\ 0 &=& (\eta-1)(\eta-(1/p-1)).
    \end{eqnarray*}
    Since $p>\half$ then $\eta=1/p-1$.
  \item
    In the paper refered to in the question it is shown that a branching
    process's geneological tree contains an infinite binary tree with
    probability $\pi$ where $1-\pi$ is the smallest root in $[0,1]$ of
    $$G_X(s)+(1-s)G'_X(s)=s.$$
    In our case $G_X(s)=(1-p)/(1-sp)$ and $G'_X(s)=p(1-p)/(1-sp)^2$ 
    and so $1-\pi_p$ is the smallest root in $[0,1]$ of
    \begin{eqnarray*}
      {1-p \over 1-sp}+{(1-s)p(1-p)\over (1-sp)^2}=s
    \end{eqnarray*}
    The solutions to this equation are $s_0=1$,
    $$s_1={1\over p}-{1\over 2}-{1\over 2}\sqrt{5-4/p}$$
    and
    $$s_2={1\over p}-{1\over 2}+{1\over 2}\sqrt{5-4/p}.$$
    
    The following results are a consequence of simple analysis, which we
    omit:

    When $p<4/5$ then $s_1$ and $s_2$ are complex and so $\pi_p=1-s_0=0$ and no
    infinite binary tree exists.

    When $p=4/5$ then $s_1=s_2=5/4-1/2=3/4$ and so we have a phase transition,
    and a binary tree exists with probability $\pi_p=1-s_1=1/4$.

    When $4/5<p<1$ then $0<s_1<3/4<s_2$ and so $\pi_p=1-s_1$.

    When $p=1$ then $s_1=0$ and $\pi_p=1-s_0=1$.
  \item
    Let $\theta_m$ be the probability that the geneological tree of a 
    branching process with Binomial($n,p$) offspring distribution 
    has an $(n-1)$-ary tree of depth $m$, rooted at the time 
    zero individual. This happens if this individual has $n-1$ or $n$
    offsprings and if at least $n-1$ of these offsprings have such trees of 
    depth $m-1$:
    $$\theta_m=p_{n-1}\theta_{m-1}^{n-1}+p_n\left[\theta_{m-1}^n+
      \theta_{m-1}^{n-1}(1-\theta_{m-1})n\right]$$
    The probability $p_n$ for $n$ offsprings is $p^n$. The probability $p_{n-1}$ 
    for $n-1$ offsprings is $np^{n-1}(1-p)$.
    Hence $\theta$, the probability of an infinite $(n-1)$-ary tree, is a root
    of the map
    $$f_p(x)=np^{n-1}(1-p)x^{n-1}+p^n\left[ x^n+x^{n-1}(1-x)n \right]-x.$$
    This is clearly a continuous function, and so we shall show that it has a 
    root in $(0,1)$ by showing that it takes both negative and positive values
    in this range:

    $f_p(0)=0$ and $\lim_{x\to 0}f_p(x)/x=-1$
    so for $\epsilon>0$ small enough $f_p(\epsilon)/\epsilon<0$ and in particular 
    $f_p(\epsilon)<0$. On the other hand,
    \begin{eqnarray*}
      f_p(1)&=&np^{n-1}(1-p)1^{n-1}+p^n\left[ 1^n+1^{n-1}(1-1)n \right]-1
      \\ &=& np^{n-1}(1-p)+p^n-1
    \end{eqnarray*}
    If we consider $g(p)=f_p(1)$ 
    then $g(1)=0$ and $\lim_{p\to 1}g(p)/p^{n-1}=n>0$ . Hence for
    $p$ close enough to one we have that $g(p)=f_p(1)>0$. 
  \end{enumerate}
  \item
    \begin{enumerate}
    \item 
      Add fiducial $B_{1,1}^0$ ($=A_0$) and $X_{1,1}^0$ distributed as $X_{i,j}^n$
      above. 
      Consider $B_{i,j}^n$ an offspring of $B_{k,l}^{n-1}$ if the latter contains
      the former. Then this problem is a representation of a branching process,
      with offspring
      distribution $B(9,p)$ (and conditioned on $X_{1,1}^0=1$). 
      The event $A_0\neq \emptyset$ is equivalent to
      the branching process not dying out, and therefore has positive
      probability when $p>1/9$.
      
    \item
      \begin{claim}
        If the branching process contains an 8-ary infinite tree rooted
        at $B_{1,1}^0$ then $A_\infty$ has a left-to-right crossing.
      \end{claim}
      Note that by 1(c) this statement implies that $p_c<1$, since there exists
      a $q<1$ such that there is a positive probability for an 8-are tree,
      and by the above $p_c<q$.

      \begin{proof}
        Let the branching process contain an 8-ary infinite tree rooted at 
        $B_{1,1}^0$.
        Let $f_1:[0,1]\to [0,1]^2$ be a function with $f_1(0)=(0,y_0)$ and 
        $f_1(1)=(0,y_1)$ for some $y_0,y_1$, and $f_1$ only passes through
        $B_{i,j}^1$'s that are part of the 8-ary binary tree rooted at the root.
        Since there are at least eight such $B_{i,j}^1$'s, it is easy to see 
        that such an $f_1$ exists.

        We define $f_n$ recursively to only pass through the same $B_{i,j}^{n-1}$'s
        that $f_{n-1}$ passed through, and in addition only pass through
        the $B_{i,j}^n$'s that are
        part of the 8-ary tree rooted at $B_{1,1}^0$. This can be done by slightly 
        distorting the image of $f_{n-1}$:

        Let $C$ and $D$ be
        two neighboring 
        $B_{i,j}^{n-1}$'s such that $f_{n-1}$ crosses from $C$ to $D$. Since they each have at 
        least eight offsprings that are included in the tree, then $C$ and $D$
        have adjacent offsprings $C'$ and $D'$ (in the sense that they share an 
        edge).
        Since $f_{n-1}$ crosses from $C$ to $D$, then if necessary we distort $f_n$ so 
        that it will cross from
        $C'$ to $D'$. We perfrom this distortion while keeping constant the $x$
        in which the function crosses from $C$ to $D$.
        
        Having made sure that $f_n$ crosses from $C$ to $D$ in legal offsprings,
        we may need to distort it some more so that {\em while in} $C$ it 
        passes only
        through offsprings belonging to the tree. It is easy to see that this
        can also be done, again because $C$ has at least eight different 
        offsprings, and so any two points on the edge of $C$ that belong to
        offsprings can be connected through offsprings.

        Finally, we may again need to shift $f_n$ slightly so that its starting
        and end points are in the tree. This can also always be done (again
        because we have eight offsprings) within
        the $B_{i,j}^{n-1}$ that $f_{n-1}$ started (or ended) at, while keeping the condition
        that $f_n(0)\in\{0\}\times [0,1]$ and $f_n(1)\in\{1\}\times [0,1]$.
        
        We've shown that the image of $f_n$ is in $A_n$. Furthermore, 
        since all the distortions we performed are within boxes of dimensions
        $3^{-n}\times 3^{-n}$, then the $L_\infty$ distance between $f_{n-1}$ and
        $f_n$ is bounded by $O(3^{-n})$. Hence the series of functions
        converges to a function $f_\infty$ such that the image of $f_\infty$ is in
        $A_\infty$ and both $f_\infty(0)\in\{0\}\times [0,1]$ and 
        $f_\infty(1)\in\{1\}\times [0,1]$.
        
      \end{proof}
    \end{enumerate}
  \item
    \begin{enumerate}
    \item 
      Let $G(n,p)$ be a random E-R graph with $p={\log n-a_n\over n}$ as defined
      in the question. Let $X=\sum_{v\in[n]}{\bf 1}_{C(v)=\{v\}}$, also as defined
      in the question. Then the expectation of $X$ is:
      \begin{eqnarray*}
        \E[X] &=& \E\left[\sum_{v\in[n]}{\bf 1}_{C(v)=\{v\}}\right]
        \\ &=& n\P[C(v_0)=\{v_0\}]
        \\ &=& n\P[C(v_0)=\{v_0\}]
        \\ &=& n(1-p)^{n-1}.
        \\ &=& n\left(1-{\log n-a_n\over n}\right)^{n-1}.
      \end{eqnarray*}
      The limit of $\E[X]$ as $n$ tends to infinity is 
      $n\exp(a_n-\log n)=\exp(a_n)=\infty$.
      The expectation of $X^2$ is:
      \begin{eqnarray*}
        \E[X^2] &=& \E\left[\left(\sum_{v\in[n]}{\bf 1}_{C(v)=\{v\}}\right)^2\right]
        \\ &=& \E\left[\sum_{u,v\in[n]}{\bf 1}_{C(v)=\{v\}}{\bf 1}_{C(u)=\{u\}}\right]
        \\ &=& \sum_{u,v\in[n]}\P[C(v)=\{v\} \mbox{ and } C(u)=\{u\}]
        \\ &=& \sum_{u\neq v\in[n]}\P[C(v)=\{v\} \mbox{ and } C(u)=\{u\}]+\sum_{v\in[n]}\P[C(v)=\{v\}]
        \\ &=& n(n-1)\P[C(v)=\{v\} \mbox{ and } C(u)=\{u\}]+\E[X]
        \\ &=& n(n-1)(1-p)^{2(n-1)-1}+\E[X]
        \\ &=& {n-1\over n(1-p)}\E[X]^2+\E[X]
      \end{eqnarray*}
      Hence
      \begin{eqnarray*}
        \Var[X]&=&\E[X^2]-\E[X]^2
        \\ &=& {n-1\over n(1-p)}\E[X]^2+\E[X] - \E[X]^2
        \\ &=& {np-1\over n-np}\E[X]^2+\E[X] 
      \end{eqnarray*}
      We can now bound the probability that the graph is not connected:
      \begin{eqnarray*}
        \P[\mbox{graph not connected}] &\geq& 
        \P[\mbox{there exists an isolated vertex}]
        \\ &=& \P[X>0]
        \\ &\geq& \P\left[|X-\E[X]| < \E[X]/2\right]
        \\ &\geq& 1-{\Var[X] \over \E[X]^2/4}
        \\ &=& 1-{4(np-1)\over n-np}+{4 \over \E[X]}
        \\ &=& 1-{4(\log n-a_n-1)\over n-\log n + a_n}+{4 \over \E[X]}.
      \end{eqnarray*}
      Since $a_n$ is bounded by $\log n$ then the second term vanishes as
      $n$ tends to infinity, as does the last term. Hence the probability
      that the graph is connected tends to zero as $n$ tends to infinity.
    \end{enumerate}
  \item
    \begin{enumerate}
    \item 
      Let $T_k$ be the amount of time it takes the random walk to cover
      $k+1$ vertices after having covered $k$ vertices. Then clearly
      $C_n=\sum_{k=1}^{n-1}T_n$ (we start from $k=1$ since at time zero the
      walk have already covered one vertex).
      
      $T_k$ is distributed geometrically with parameter 
      $p_k=(n-k)/(n-1)$: The
      probability of adding a new vertex to the set of those covered, after 
      having already covered $k$ vertices, is the probability of choosing
      one of the $n-k$ unvisited vertices out of the $n-1$ possible ones. 
      Therefore
      \begin{eqnarray*}
        \E[C_n]&=&\E\left[\sum_{k=1}^{n-1}T_k\right]
        \\ &=&\sum_{k=1}^{n-1}\E[T_k]
        \\ &=&\sum_{k=1}^{n-1}{1\over p_k}
        \\ &=&\sum_{k=1}^{n-1}{n-1\over (n-k)}
        \\ &=&(n-1)\sum_{k=1}^{n-1}{1\over (n-k)}
        \\ &=& (n-1)\left({1\over n-1}+{1\over n-2}+\cdots+1\right)
      \end{eqnarray*}
    \item
      Let $P_k$ be the probability that vertex $x_0$ was not visited until
      time $k$. Then $P_0=(n-1)/n$ and for $k>0$ we have that 
      $P_k=P_{k-1}(n-2)/(n-1)$, since if $x_0$ was not visited up to time 
      $P_{k-1}$ then in particular $X_{k-1}\neq x_0$ and the probability to visit
      it at time $k$ is $(n-2)/(n-1)$. We can therefore deduce that
      $$P_k={(n-1)\over n}\left({n-2\over n-1}\right)^k.$$
      We use this in the  following calculation:
      \begin{eqnarray*}
        {\E[D_{[un]}]\over n} &=& {1\over n}\E\left[\sum_{x\in K_n}{\bf 1}_{x \not \in \{X_0,\ldots,X_{[un]}\}}\right]
        \\ &=& {1\over n}\sum_{x\in K_n}\E\left[{\bf 1}_{x \not \in \{X_0,\ldots,X_{[un]}\}}\right]
        \\ &=& {1\over n}n\P[x_0 \not \in \{X_0,\ldots,X_{[un]}\}]
        \\ &=& P_{[un]}
        \\ &=& {(n-1)\over n}\left({n-2\over n-1}\right)^{[un]}.
        \end{eqnarray*}
        Then
        \begin{eqnarray*}
          {(n-1)\over n}\left({n-2\over n-1}\right)^{un+1} &\leq {\E[D_{[un]}]\over n} \leq& {(n-1)\over n}\left({n-2\over n-1}\right)^{un-1}
          \\ \lim_{n \to \infty}{(n-1)\over n}\left({n-2\over n-1}\right)^{un+1} &\leq \lim_{n \to \infty}{\E[D_{[un]}]\over n} \leq& \lim_{n \to \infty}{(n-1)\over n}\left({n-2\over n-1}\right)^{un-1}
          \\ \lim_{n \to \infty}{n\over n+1}\left({n-1\over n}\right)^{u(n+1)+1} &\leq \lim_{n \to \infty}{\E[D_{[un]}]\over n} \leq& \lim_{n \to \infty}{n\over n+1}\left({n-1\over n}\right)^{u(n+1)-1}
          \\ \lim_{n \to \infty}{n\over n+1}\left(1-{1\over n}\right)^{n(u+u/n+1/n)} &\leq \lim_{n \to \infty}{\E[D_{[un]}]\over n} \leq& \lim_{n \to \infty}{n\over n+1}\left(1-{1\over n}\right)^{n(u+u/n-1/n)}
          \\ \lim_{n \to \infty}\exp(-u-u/n-1/n) &\leq \lim_{n \to \infty}{\E[D_{[un]}]\over n} \leq& \lim_{n \to \infty}\exp(-u-u/n+1/n)
          \\ \exp(-u) &\leq \lim_{n \to \infty}{\E[D_{[un]}]\over n} \leq& \exp(-u)
        \end{eqnarray*}
      \item
        The times $T_k$, as defined above, are independent. Hence
        \begin{eqnarray*}
          \Var[C_n] &=& \sum_{k=1}^{n-1}\Var[T_k]
          \\ &=& \sum_{k=1}^{n-1}{1-p_k \over p_k^2}
          \\ &=& \sum_{k=1}^{n-1}{1-(n-k)/(n-1) \over (n-k)^2/(n-1)^2}
          \\ &=& \sum_{k=1}^{n-1}{(n-1)^2-(n-k)(n-1) \over (n-k)^2}
          \\ &=& (n-1)\sum_{k=1}^{n-1}{k-1 \over (n-k)^2}
          \\ &\leq& n^2\sum_{k=1}^{n-1}{1 \over (n-k)^2}
          \\ &\leq& n^2\sum_{k=1}^n{1 \over k^2}
          \\ &\leq& n^2{\pi^2 \over 6}
        \end{eqnarray*}
        Hence $\Var[C_n/(n\log n)] \leq 10/\log^2n$ and thus goes to zero as
        $n$ tends to infinity. Hence it converges 
        is the limit of 



        \begin{eqnarray*}
          \E\left[\left({C_n\over n\log n}-1\right)^2\right] 
          &=& \E\left[\left({C_n\over n\log n}\right)^2-{2C_n\over n\log n}+1\right] 
          \\ &=& \left({1\over n\log n}\right)^2\E\left[C_n^2\right]-{2\over n\log n}\E[C_n]+1
        \end{eqnarray*}
        We first calculate $\E[C_n^2]$:
        \begin{eqnarray*}
          \E[C_n^2] &=& \E\left[\left(\sum_{k=1}^{n-1}T_k\right)^2\right]
          \\ &=& \E\left[\sum_{k,l=1}^{n-1}T_kT_l\right]
          \\ &=& \E\left[\sum_{k,l=1,\:k \neq l}^{n-1}T_kT_l+\sum_{k=1}^{n-1}T_k^2\right]
          \\ &=& \sum_{k,l=1,\:k \neq l}^{n-1}\E[T_k]\E[T_l]+\sum_{k=1}^{n-1}\Var[T_k]+\E[T_k]^2
          \\ &=& \sum_{k,l=1,\:k \neq l}^{n-1}{1\over p_kp_l}+\sum_{k=1}^{n-1}{2-p_k\over p_k^2}
        \end{eqnarray*}


    \end{enumerate}


\end{enumerate}
\end{document}


