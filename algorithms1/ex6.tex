\documentclass[11pt]{article} \usepackage{amssymb}
\usepackage{amsfonts} \usepackage{amsmath} \usepackage{bm}
\usepackage{latexsym} \usepackage{epsfig}
\usepackage{algorithm}
\usepackage{algorithmic}

\setlength{\textwidth}{6.5 in} \setlength{\textheight}{8.25in}
\setlength{\oddsidemargin}{0in} \setlength{\topmargin}{0in}
\addtolength{\textheight}{.8in} \addtolength{\voffset}{-.5in}

\newtheorem{theorem}{Theorem}[section]
\newtheorem{lemma}[theorem]{Lemma}
\newtheorem{proposition}[theorem]{Proposition}
\newtheorem{corollary}[theorem]{Corollary}
\newtheorem{fact}[theorem]{Fact}
\newtheorem{definition}[theorem]{Definition}
\newtheorem{remark}[theorem]{Remark}
\newtheorem{conjecture}[theorem]{Conjecture}
\newtheorem{claim}[theorem]{Claim}
\newtheorem{example}[theorem]{Example}
\newenvironment{proof}{\noindent \textbf{Proof:}}{$\Box$}

\newcommand{\ignore}[1]{}

\newcommand{\enote}[1]{} \newcommand{\knote}[1]{}
\newcommand{\rnote}[1]{}



% \newcommand{\enote}[1]{{\bf [[Elchanan:} {\emph{#1}}{\bf ]]}}
% \newcommand{\knote}[1]{{\bf [[Krzysztof:} {\emph{#1}}{\bf ]]}}
% \newcommand{\rnote}[1]{{\bf [[Ryan:} {\emph{#1}}{\bf ]]}}



\DeclareMathOperator{\Support}{Supp} \DeclareMathOperator{\Opt}{Opt}
\DeclareMathOperator{\Ordo}{\mathcal{O}}
\newcommand{\MaxkCSP}{\textsc{Max $k$-CSP}}
\newcommand{\MaxkCSPq}{\textsc{Max $k$-CSP$_{q}$}}
\newcommand{\MaxCSP}[1]{\textsc{Max CSP}(#1)} \renewcommand{\Pr}{{\bf
    P}} \renewcommand{\P}{{\bf P}} \newcommand{\Px}{\mathop{\bf P\/}}
\newcommand{\E}{{\bf E}} \newcommand{\Cov}{{\bf Cov}}
\newcommand{\Var}{{\bf Var}} \newcommand{\Varx}{\mathop{\bf Var\/}}

\newcommand{\bits}{\{-1,1\}}

\newcommand{\nsmaja}{\textstyle{\frac{2}{\pi}} \arcsin \rho}

\newcommand{\Inf}{\mathrm{Inf}} \newcommand{\I}{\mathrm{I}}
\newcommand{\J}{\mathrm{J}}

\newcommand{\eps}{\epsilon} \newcommand{\lam}{\lambda}

% \newcommand{\trunc}{\ell_{2,[-1,1]}}
\newcommand{\trunc}{\zeta} \newcommand{\truncprod}{\chi}

\newcommand{\N}{\mathbb N} \newcommand{\R}{\mathbb R}
\newcommand{\Z}{\mathbb Z} \newcommand{\CalE}{{\mathcal{E}}}
\newcommand{\CalC}{{\mathcal{C}}} \newcommand{\CalM}{{\mathcal{M}}}
\newcommand{\CalR}{{\mathcal{R}}} \newcommand{\CalS}{{\mathcal{S}}}
\newcommand{\CalV}{{\mathcal{V}}}
\newcommand{\CalX}{{\boldsymbol{\mathcal{X}}}}
\newcommand{\CalG}{{\boldsymbol{\mathcal{G}}}}
\newcommand{\CalH}{{\boldsymbol{\mathcal{H}}}}
\newcommand{\CalY}{{\boldsymbol{\mathcal{Y}}}}
\newcommand{\CalZ}{{\boldsymbol{\mathcal{Z}}}}
\newcommand{\CalW}{{\boldsymbol{\mathcal{W}}}}
\newcommand{\CalF}{{\mathcal{Z}}}
% \newcommand{\boldG}{{\boldsymbol G}}
% \newcommand{\boldQ}{{\boldsymbol Q}}
% \newcommand{\boldP}{{\boldsymbol P}}
% \newcommand{\boldR}{{\boldsymbol R}}
% \newcommand{\boldS}{{\boldsymbol S}}
% \newcommand{\boldX}{{\boldsymbol X}}
% \newcommand{\boldB}{{\boldsymbol B}}
% \newcommand{\boldY}{{\boldsymbol Y}}
% \newcommand{\boldZ}{{\boldsymbol Z}}
% \newcommand{\boldV}{{\boldsymbol V}}
\newcommand{\boldi}{{\boldsymbol i}} \newcommand{\boldj}{{\boldsymbol
    j}} \newcommand{\boldk}{{\boldsymbol k}}
\newcommand{\boldr}{{\boldsymbol r}}
\newcommand{\boldsigma}{{\boldsymbol \sigma}}
\newcommand{\boldupsilon}{{\boldsymbol \upsilon}}
\newcommand{\hone}{{\boldsymbol{H1}}}
\newcommand{\htwo}{\boldsymbol{H2}}
\newcommand{\hthree}{\boldsymbol{H3}}
\newcommand{\hfour}{\boldsymbol{H4}}


\newcommand{\sgn}{\mathrm{sgn}} \newcommand{\Maj}{\mathrm{Maj}}
\newcommand{\Acyc}{\mathrm{Acyc}}
\newcommand{\UniqMax}{\mathrm{UniqMax}}
\newcommand{\Thr}{\mathrm{Thr}} \newcommand{\littlesum}{{\textstyle
    \sum}}

\newcommand{\half}{{\textstyle \frac12}}
\newcommand{\third}{{\textstyle \frac13}}
\newcommand{\fourth}{{\textstyle \frac14}}

\newcommand{\Stab}{\mathbb{S}}
\newcommand{\StabThr}[2]{\Gamma_{#1}(#2)}
\newcommand{\StabThrmin}[2]{{\underline{\Gamma}}_{#1}(#2)}
\newcommand{\StabThrmax}[2]{{\overline{\Gamma}}_{#1}(#2)}
\newcommand{\TestFcn}{\Psi}

\renewcommand{\phi}{\varphi}

\begin{document}
\title{Algorithms - Exercise 6}

 \author{Omer Tamuz, 035696574}
\maketitle

\begin{enumerate}
\item 

  \begin{enumerate}
  \item 
    Similarly to the proof for the bound on $C^*$ that we saw in class, we
    order the vertices of $G$ in descending value of their coefficients in the 
    eigenvector with the largest eigenvalue. Then, as noted in class, no
    vertex can have more than $\lambda_1$ neighbors that precede it in the list.
    
    We now start coloring from the beginning of the list, i.e. from the vertex
    with the highest coeffecient in the eigenvector of $\lambda_1$. Each vertex
    that we shall get to will have at most $\lambda$ neighbors that are already 
    colored, and so we'll always have an available color for it, given that
    $\lambda_1+1$ colors are available.

  \item 
    Let $G$ be $K_{d+1}$, the complete graph on $d+1$ vertices.
    $G$ can clearly not be colored with 
    less than $d+1$ colors. Since $G$ is regular with degree $d$ then 
    $\lambda_1=d$ and it cannot be
    colored with less than $\lambda_1+1$ colors
  \end{enumerate}

\item
  
  Let $G=(U,V,E)$ be a bipartite graph with sides $U$ and $V$. Let $x$ be an 
  eigenvector of its adjacency matrix with nonzero eigenvalue $\lambda$. 
  Let $y$ be $x$, with all the coordinates belonging to vertices of $U$ 
  multiplied by $-1$. We showed in class that $y$ is also an eigenvector, with
  eigenvalue $-\lambda$. Since $\lambda\neq 0$ then $\lambda\neq -\lambda$
  and $x$ and $y$ are orthogonal (being eigenvectors of a symmetric matrix with
  different eigenvalues). Hence
  \begin{eqnarray*}
    x^ty&=&0
    \\ \sum_{i\in U \cup V}x_iy_i&=&0
    \\ \sum_{i\in U}x_iy_i + \sum_{i\in V}x_iy_i&=&0
    \\ \sum_{i\in U}x_i(-x_i) + \sum_{i\in V}x_ix_i&=&0
    \\ \sum_{i\in U}x_i^2 &=& \sum_{i\in V}x_i^2
  \end{eqnarray*}
  

\end{enumerate}

\end{document}


