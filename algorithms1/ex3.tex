\documentclass[11pt]{article} \usepackage{amssymb}
\usepackage{amsfonts} \usepackage{amsmath} \usepackage{bm}
\usepackage{latexsym} \usepackage{epsfig}
\usepackage{algorithm}
\usepackage{algorithmic}

\setlength{\textwidth}{6.5 in} \setlength{\textheight}{8.25in}
\setlength{\oddsidemargin}{0in} \setlength{\topmargin}{0in}
\addtolength{\textheight}{.8in} \addtolength{\voffset}{-.5in}

\newtheorem{theorem}{Theorem}[section]
\newtheorem{lemma}[theorem]{Lemma}
\newtheorem{proposition}[theorem]{Proposition}
\newtheorem{corollary}[theorem]{Corollary}
\newtheorem{fact}[theorem]{Fact}
\newtheorem{definition}[theorem]{Definition}
\newtheorem{remark}[theorem]{Remark}
\newtheorem{conjecture}[theorem]{Conjecture}
\newtheorem{claim}[theorem]{Claim}
\newtheorem{example}[theorem]{Example}
\newenvironment{proof}{\noindent \textbf{Proof:}}{$\Box$}

\newcommand{\ignore}[1]{}

\newcommand{\enote}[1]{} \newcommand{\knote}[1]{}
\newcommand{\rnote}[1]{}



% \newcommand{\enote}[1]{{\bf [[Elchanan:} {\emph{#1}}{\bf ]]}}
% \newcommand{\knote}[1]{{\bf [[Krzysztof:} {\emph{#1}}{\bf ]]}}
% \newcommand{\rnote}[1]{{\bf [[Ryan:} {\emph{#1}}{\bf ]]}}



\DeclareMathOperator{\Support}{Supp} \DeclareMathOperator{\Opt}{Opt}
\DeclareMathOperator{\Ordo}{\mathcal{O}}
\newcommand{\MaxkCSP}{\textsc{Max $k$-CSP}}
\newcommand{\MaxkCSPq}{\textsc{Max $k$-CSP$_{q}$}}
\newcommand{\MaxCSP}[1]{\textsc{Max CSP}(#1)} \renewcommand{\Pr}{{\bf
    P}} \renewcommand{\P}{{\bf P}} \newcommand{\Px}{\mathop{\bf P\/}}
\newcommand{\E}{{\bf E}} \newcommand{\Cov}{{\bf Cov}}
\newcommand{\Var}{{\bf Var}} \newcommand{\Varx}{\mathop{\bf Var\/}}

\newcommand{\bits}{\{-1,1\}}

\newcommand{\nsmaja}{\textstyle{\frac{2}{\pi}} \arcsin \rho}

\newcommand{\Inf}{\mathrm{Inf}} \newcommand{\I}{\mathrm{I}}
\newcommand{\J}{\mathrm{J}}

\newcommand{\eps}{\epsilon} \newcommand{\lam}{\lambda}

% \newcommand{\trunc}{\ell_{2,[-1,1]}}
\newcommand{\trunc}{\zeta} \newcommand{\truncprod}{\chi}

\newcommand{\N}{\mathbb N} \newcommand{\R}{\mathbb R}
\newcommand{\Z}{\mathbb Z} \newcommand{\CalE}{{\mathcal{E}}}
\newcommand{\CalC}{{\mathcal{C}}} \newcommand{\CalM}{{\mathcal{M}}}
\newcommand{\CalR}{{\mathcal{R}}} \newcommand{\CalS}{{\mathcal{S}}}
\newcommand{\CalV}{{\mathcal{V}}}
\newcommand{\CalX}{{\boldsymbol{\mathcal{X}}}}
\newcommand{\CalG}{{\boldsymbol{\mathcal{G}}}}
\newcommand{\CalH}{{\boldsymbol{\mathcal{H}}}}
\newcommand{\CalY}{{\boldsymbol{\mathcal{Y}}}}
\newcommand{\CalZ}{{\boldsymbol{\mathcal{Z}}}}
\newcommand{\CalW}{{\boldsymbol{\mathcal{W}}}}
\newcommand{\CalF}{{\mathcal{Z}}}
% \newcommand{\boldG}{{\boldsymbol G}}
% \newcommand{\boldQ}{{\boldsymbol Q}}
% \newcommand{\boldP}{{\boldsymbol P}}
% \newcommand{\boldR}{{\boldsymbol R}}
% \newcommand{\boldS}{{\boldsymbol S}}
% \newcommand{\boldX}{{\boldsymbol X}}
% \newcommand{\boldB}{{\boldsymbol B}}
% \newcommand{\boldY}{{\boldsymbol Y}}
% \newcommand{\boldZ}{{\boldsymbol Z}}
% \newcommand{\boldV}{{\boldsymbol V}}
\newcommand{\boldi}{{\boldsymbol i}} \newcommand{\boldj}{{\boldsymbol
    j}} \newcommand{\boldk}{{\boldsymbol k}}
\newcommand{\boldr}{{\boldsymbol r}}
\newcommand{\boldsigma}{{\boldsymbol \sigma}}
\newcommand{\boldupsilon}{{\boldsymbol \upsilon}}
\newcommand{\hone}{{\boldsymbol{H1}}}
\newcommand{\htwo}{\boldsymbol{H2}}
\newcommand{\hthree}{\boldsymbol{H3}}
\newcommand{\hfour}{\boldsymbol{H4}}


\newcommand{\sgn}{\mathrm{sgn}} \newcommand{\Maj}{\mathrm{Maj}}
\newcommand{\Acyc}{\mathrm{Acyc}}
\newcommand{\UniqMax}{\mathrm{UniqMax}}
\newcommand{\Thr}{\mathrm{Thr}} \newcommand{\littlesum}{{\textstyle
    \sum}}

\newcommand{\half}{{\textstyle \frac12}}
\newcommand{\third}{{\textstyle \frac13}}
\newcommand{\fourth}{{\textstyle \frac14}}

\newcommand{\Stab}{\mathbb{S}}
\newcommand{\StabThr}[2]{\Gamma_{#1}(#2)}
\newcommand{\StabThrmin}[2]{{\underline{\Gamma}}_{#1}(#2)}
\newcommand{\StabThrmax}[2]{{\overline{\Gamma}}_{#1}(#2)}
\newcommand{\TestFcn}{\Psi}

\renewcommand{\phi}{\varphi}

\begin{document}
\title{Algorithms - Exercise 3}

 \author{Omer Tamuz, 035696574}
\maketitle
Let an interval graph $G=(V,E)$, where $v_i=[a_i,b_i]$, be ordered by the 
{\em starting points} of the intervals, so that $i<j\to a_i\leq a_j$.
\begin{definition}
Let {\bf INT-COL} be the following algorithm: at iteration $k$, for $i$
from one to $|V|$, color an uncolored $v_i$ with color $k$ if it doesn't 
intersect any
other interval which was previously colored $k$. In pseudo-code:

    \begin{algorithmic}
      \STATE {$k \to 0$}
      \WHILE {any vertices are not colored}
        \STATE {$k \to k+1$}
        \FOR {$i=1$ to $|V|$}
          \IF {$v_i$ is uncolored and does not intersect any interval colored with color $k$}
            \STATE {color $v_i$ with color $k$}
          \ENDIF
        \ENDFOR
      \ENDWHILE  
    \end{algorithmic}
\end{definition}


  Let $G_k$ be a subgraph of $G$, induced by the vertices colored with colors 
  $\geq k$. That is, $G_k$ is induced by all vertices which are uncolored at the
  beginning of iteration $k$.
\begin{proposition}
  For any $a_i$ such that $v_i$ is in $G_k$, INT-COL colors 
  in iteration $k$ at least one interval containing 
  $a_i$.
\end{proposition}
\begin{proof}
  Assume by way of contradiction that for some $v_i$ in $G_k$, no interval
  colored in iteration $k$ covers $a_i$. Then none of the intervals
  which were colored before $v_i$ was examined (in that iteration) 
  cover $a_i$, and
  therefore none of them intersect $v_i$, and $v_i$ would be colored -
  contradiction.
\end{proof}

\begin{proposition}
  INT-COL colors at least one vertex in each clique in $G_k$.
\end{proposition}
\begin{proof}
  Since the interval covered by all the members of a clique starts with the 
  starting
  point of the last member of the clique, this follows easily from the 
  previous proposition.
\end{proof}

\begin{proposition}
  INT-COL colors exactly one vertex in each clique in $G_k$.
\end{proposition}
\begin{proof}
  Since no two intervals that cover the same point can be colored by the same
  color, this proposition follows trivially from the previous.
\end{proof}

\begin{proposition}
  INT-COL colors $G$ with a number of colors equal to the size of the maximal
  clique in $G$. 
\end{proposition}
\begin{proof}
  We will use induction on the size $c$ of the maximal clique in $G$. The base
  of the induction $c=0$ is trivial. Assume the algorithm works for all 
  maximal clique sizes less than $c$. Then, given a graph with a maximal clique
  size $c$, INT-COL will, in the first iteration, color exactly one vertex
  in each clique with color $k=1$. The graph of the uncolored vertices, $G_2$,
  will therefore have a clique of maximal size $c-1$, and will be colored with 
  $c-1$ colors. Hence, $G$ would be colored with $c$ colors.
\end{proof}
\end{document}


