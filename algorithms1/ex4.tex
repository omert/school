\documentclass[11pt]{article} \usepackage{amssymb}
\usepackage{amsfonts} \usepackage{amsmath} \usepackage{bm}
\usepackage{latexsym} \usepackage{epsfig}
\usepackage{algorithm}
\usepackage{algorithmic}

\setlength{\textwidth}{6.5 in} \setlength{\textheight}{8.25in}
\setlength{\oddsidemargin}{0in} \setlength{\topmargin}{0in}
\addtolength{\textheight}{.8in} \addtolength{\voffset}{-.5in}

\newtheorem{theorem}{Theorem}[section]
\newtheorem{lemma}[theorem]{Lemma}
\newtheorem{proposition}[theorem]{Proposition}
\newtheorem{corollary}[theorem]{Corollary}
\newtheorem{fact}[theorem]{Fact}
\newtheorem{definition}[theorem]{Definition}
\newtheorem{remark}[theorem]{Remark}
\newtheorem{conjecture}[theorem]{Conjecture}
\newtheorem{claim}[theorem]{Claim}
\newtheorem{example}[theorem]{Example}
\newenvironment{proof}{\noindent \textbf{Proof:}}{$\Box$}

\newcommand{\ignore}[1]{}

\newcommand{\enote}[1]{} \newcommand{\knote}[1]{}
\newcommand{\rnote}[1]{}



% \newcommand{\enote}[1]{{\bf [[Elchanan:} {\emph{#1}}{\bf ]]}}
% \newcommand{\knote}[1]{{\bf [[Krzysztof:} {\emph{#1}}{\bf ]]}}
% \newcommand{\rnote}[1]{{\bf [[Ryan:} {\emph{#1}}{\bf ]]}}



\DeclareMathOperator{\Support}{Supp} \DeclareMathOperator{\Opt}{Opt}
\DeclareMathOperator{\Ordo}{\mathcal{O}}
\newcommand{\MaxkCSP}{\textsc{Max $k$-CSP}}
\newcommand{\MaxkCSPq}{\textsc{Max $k$-CSP$_{q}$}}
\newcommand{\MaxCSP}[1]{\textsc{Max CSP}(#1)} \renewcommand{\Pr}{{\bf
    P}} \renewcommand{\P}{{\bf P}} \newcommand{\Px}{\mathop{\bf P\/}}
\newcommand{\E}{{\bf E}} \newcommand{\Cov}{{\bf Cov}}
\newcommand{\Var}{{\bf Var}} \newcommand{\Varx}{\mathop{\bf Var\/}}

\newcommand{\bits}{\{-1,1\}}

\newcommand{\nsmaja}{\textstyle{\frac{2}{\pi}} \arcsin \rho}

\newcommand{\Inf}{\mathrm{Inf}} \newcommand{\I}{\mathrm{I}}
\newcommand{\J}{\mathrm{J}}

\newcommand{\eps}{\epsilon} \newcommand{\lam}{\lambda}

% \newcommand{\trunc}{\ell_{2,[-1,1]}}
\newcommand{\trunc}{\zeta} \newcommand{\truncprod}{\chi}

\newcommand{\N}{\mathbb N} \newcommand{\R}{\mathbb R}
\newcommand{\Z}{\mathbb Z} \newcommand{\CalE}{{\mathcal{E}}}
\newcommand{\CalC}{{\mathcal{C}}} \newcommand{\CalM}{{\mathcal{M}}}
\newcommand{\CalR}{{\mathcal{R}}} \newcommand{\CalS}{{\mathcal{S}}}
\newcommand{\CalV}{{\mathcal{V}}}
\newcommand{\CalX}{{\boldsymbol{\mathcal{X}}}}
\newcommand{\CalG}{{\boldsymbol{\mathcal{G}}}}
\newcommand{\CalH}{{\boldsymbol{\mathcal{H}}}}
\newcommand{\CalY}{{\boldsymbol{\mathcal{Y}}}}
\newcommand{\CalZ}{{\boldsymbol{\mathcal{Z}}}}
\newcommand{\CalW}{{\boldsymbol{\mathcal{W}}}}
\newcommand{\CalF}{{\mathcal{Z}}}
% \newcommand{\boldG}{{\boldsymbol G}}
% \newcommand{\boldQ}{{\boldsymbol Q}}
% \newcommand{\boldP}{{\boldsymbol P}}
% \newcommand{\boldR}{{\boldsymbol R}}
% \newcommand{\boldS}{{\boldsymbol S}}
% \newcommand{\boldX}{{\boldsymbol X}}
% \newcommand{\boldB}{{\boldsymbol B}}
% \newcommand{\boldY}{{\boldsymbol Y}}
% \newcommand{\boldZ}{{\boldsymbol Z}}
% \newcommand{\boldV}{{\boldsymbol V}}
\newcommand{\boldi}{{\boldsymbol i}} \newcommand{\boldj}{{\boldsymbol
    j}} \newcommand{\boldk}{{\boldsymbol k}}
\newcommand{\boldr}{{\boldsymbol r}}
\newcommand{\boldsigma}{{\boldsymbol \sigma}}
\newcommand{\boldupsilon}{{\boldsymbol \upsilon}}
\newcommand{\hone}{{\boldsymbol{H1}}}
\newcommand{\htwo}{\boldsymbol{H2}}
\newcommand{\hthree}{\boldsymbol{H3}}
\newcommand{\hfour}{\boldsymbol{H4}}


\newcommand{\sgn}{\mathrm{sgn}} \newcommand{\Maj}{\mathrm{Maj}}
\newcommand{\Acyc}{\mathrm{Acyc}}
\newcommand{\UniqMax}{\mathrm{UniqMax}}
\newcommand{\Thr}{\mathrm{Thr}} \newcommand{\littlesum}{{\textstyle
    \sum}}

\newcommand{\half}{{\textstyle \frac12}}
\newcommand{\third}{{\textstyle \frac13}}
\newcommand{\fourth}{{\textstyle \frac14}}

\newcommand{\Stab}{\mathbb{S}}
\newcommand{\StabThr}[2]{\Gamma_{#1}(#2)}
\newcommand{\StabThrmin}[2]{{\underline{\Gamma}}_{#1}(#2)}
\newcommand{\StabThrmax}[2]{{\overline{\Gamma}}_{#1}(#2)}
\newcommand{\TestFcn}{\Psi}

\renewcommand{\phi}{\varphi}

\begin{document}
\title{Algorithms - Exercise 4}

 \author{Omer Tamuz, 035696574}
\maketitle

\begin{enumerate}


\item 
  Let $G=(V,E)=(V_L,V_R,E)$ be a bipartite graph.
  
  {\bf First Direction}
  
  Assume $G$ has a perfect matching. Let $H$ be a set of vertices, and let 
  $N(H)$ be their neighbors. 
  
  Let $H_L=H\cap V_L$ be the vertices in $H$ on the left side of the graph, 
  and $H_R=H\cap V_R$ the
  vertices in $H$ on the right. Then
  \begin{itemize}
  \item $H_L$ and $H_R$ are disjoint, since the graph is bipartite.
  \item $|H_L| \leq |N(H_L)|$, since the unique match of every vertex in $H_L$
    is in $N(H_L)$. Likewise $|H_R| \leq |N(H_R)|$.
  \item $N(H_L)$ and $N(H_R)$ are disjoint, since the vertices of the first are
    on the right and the vertices of the second are on the left. 
  \end{itemize}
  
  Hence, using these observations in the order they are listed, we can write: 
  \begin{eqnarray*}
    |H|&=&|H_L\cup H_R| 
    \\ &=& |H_L \uplus H_R| 
    \\ &=& |H_L|+|H_R| 
    \\ &\leq& |N(H_L)| + |N(H_R)|
    \\ &=& |N(H_L) \uplus N(H_R)|
    \\ &=& |N(H)|
  \end{eqnarray*}
  
  {\bf Second Direction}
  
  Assume that for every set of vertices $H$, it holds that $|H| \leq |N(H)|$. Let
  $C$, $S$, $R$ be the Gallai-Edmond decomposition of the graph. Then, since
  $N(C)=S$ then $|C| \leq |S|$. Since the number of unmatched edges is 
  $|C|-|S|$ it must be that $|C|=|S|$ and that we have a perfect matching.

\item
  Let $C$, $S$, $R$ be the Gallai-Edmond decomposition of a $d$-regular
  bipartite graph. Let $N_{XY}$ be the number of edges between 
  $X,Y\in \{C,S,R\}$. Then we know that
  \begin{itemize}
  \item $N_{CC}=N_{CR}=0$, by the definition of the Gallai-Edmond decomposition.
  \item $N_{CS}=d|C|$, since these are all the edges that originate in $C$.
  \item $N_{RR}+N_{RS}=d|R|$, since these are all the edges that originate
    in a vertex in $|R|$.
  \item The total number of edges is $\half d(|C|+|S|+|R|)$. 
  \end{itemize}
  
  Hence $N_{SS}=\half d(|C|+|S|+|R|) - d|C|-d|R|=\half d(-|C|+|S|-|R|)$. Since this number
  is non-negative, we know that $|S|-|C|\geq |R|$. However $|S|-|C|$ is at most
  zero, and therefore the only possibility is that $|R|=0$, and $|C|=|S|$. 
  Since the number of unmathced vertices is $|C|-|S|$, then we know that there
  exists a perfect matching. 

  
  A one-regular bipartite graph is clearly a perfect matching. 
  Since, as we just showed, every $d$-regular bipartite graph has a perfect
  matching, then we can remove a matching, 
  to arrive at a $d-1$-regular graph. Applying
  this repeatedly, we decompose the graph into $d$ perfect matchings.

\item
  Let $G$ be a bipartite graph of maximal degree $\Delta$. We can complete it
  to a $\Delta$-regular bipartite graph by repeatedly adding edges to vertices
  of degree less than $\Delta$. The resulting graph is the disjoint union
  of $\Delta$ perfect matchings, and so the original graph is the disjoint
  union of $\Delta$ matchings, where each matching is the intersection of
  a matching in the regular graph with the set of edges of the original graph.
\end{enumerate}

\end{document}


