\documentclass[11pt]{article} \usepackage{amssymb}
\usepackage{amsfonts} \usepackage{amsmath} \usepackage{bm}
\usepackage{latexsym} \usepackage{epsfig}
\usepackage{algorithm}
\usepackage{algorithmic}

\setlength{\textwidth}{6.5 in} \setlength{\textheight}{8.25in}
\setlength{\oddsidemargin}{0in} \setlength{\topmargin}{0in}
\addtolength{\textheight}{.8in} \addtolength{\voffset}{-.5in}

\newtheorem{theorem}{Theorem}[section]
\newtheorem{lemma}[theorem]{Lemma}
\newtheorem{proposition}[theorem]{Proposition}
\newtheorem{corollary}[theorem]{Corollary}
\newtheorem{fact}[theorem]{Fact}
\newtheorem{definition}[theorem]{Definition}
\newtheorem{remark}[theorem]{Remark}
\newtheorem{conjecture}[theorem]{Conjecture}
\newtheorem{claim}[theorem]{Claim}
\newtheorem{example}[theorem]{Example}
\newenvironment{proof}{\noindent \textbf{Proof:}}{$\Box$}

\newcommand{\ignore}[1]{}

\newcommand{\enote}[1]{} \newcommand{\knote}[1]{}
\newcommand{\rnote}[1]{}



% \newcommand{\enote}[1]{{\bf [[Elchanan:} {\emph{#1}}{\bf ]]}}
% \newcommand{\knote}[1]{{\bf [[Krzysztof:} {\emph{#1}}{\bf ]]}}
% \newcommand{\rnote}[1]{{\bf [[Ryan:} {\emph{#1}}{\bf ]]}}



\DeclareMathOperator{\Support}{Supp} \DeclareMathOperator{\Opt}{Opt}
\DeclareMathOperator{\Ordo}{\mathcal{O}}
\newcommand{\MaxkCSP}{\textsc{Max $k$-CSP}}
\newcommand{\MaxkCSPq}{\textsc{Max $k$-CSP$_{q}$}}
\newcommand{\MaxCSP}[1]{\textsc{Max CSP}(#1)} \renewcommand{\Pr}{{\bf
    P}} \renewcommand{\P}{{\bf P}} \newcommand{\Px}{\mathop{\bf P\/}}
\newcommand{\E}{{\bf E}} \newcommand{\Cov}{{\bf Cov}}
\newcommand{\Var}{{\bf Var}} \newcommand{\Varx}{\mathop{\bf Var\/}}

\newcommand{\bits}{\{-1,1\}}

\newcommand{\nsmaja}{\textstyle{\frac{2}{\pi}} \arcsin \rho}

\newcommand{\Inf}{\mathrm{Inf}} \newcommand{\I}{\mathrm{I}}
\newcommand{\J}{\mathrm{J}}

\newcommand{\eps}{\epsilon} \newcommand{\lam}{\lambda}

% \newcommand{\trunc}{\ell_{2,[-1,1]}}
\newcommand{\trunc}{\zeta} \newcommand{\truncprod}{\chi}

\newcommand{\N}{\mathbb N} \newcommand{\R}{\mathbb R}
\newcommand{\Z}{\mathbb Z} \newcommand{\CalE}{{\mathcal{E}}}
\newcommand{\CalC}{{\mathcal{C}}} \newcommand{\CalM}{{\mathcal{M}}}
\newcommand{\CalR}{{\mathcal{R}}} \newcommand{\CalS}{{\mathcal{S}}}
\newcommand{\CalV}{{\mathcal{V}}}
\newcommand{\CalX}{{\boldsymbol{\mathcal{X}}}}
\newcommand{\CalG}{{\boldsymbol{\mathcal{G}}}}
\newcommand{\CalH}{{\boldsymbol{\mathcal{H}}}}
\newcommand{\CalY}{{\boldsymbol{\mathcal{Y}}}}
\newcommand{\CalZ}{{\boldsymbol{\mathcal{Z}}}}
\newcommand{\CalW}{{\boldsymbol{\mathcal{W}}}}
\newcommand{\CalF}{{\mathcal{Z}}}
% \newcommand{\boldG}{{\boldsymbol G}}
% \newcommand{\boldQ}{{\boldsymbol Q}}
% \newcommand{\boldP}{{\boldsymbol P}}
% \newcommand{\boldR}{{\boldsymbol R}}
% \newcommand{\boldS}{{\boldsymbol S}}
% \newcommand{\boldX}{{\boldsymbol X}}
% \newcommand{\boldB}{{\boldsymbol B}}
% \newcommand{\boldY}{{\boldsymbol Y}}
% \newcommand{\boldZ}{{\boldsymbol Z}}
% \newcommand{\boldV}{{\boldsymbol V}}
\newcommand{\boldi}{{\boldsymbol i}} \newcommand{\boldj}{{\boldsymbol
    j}} \newcommand{\boldk}{{\boldsymbol k}}
\newcommand{\boldr}{{\boldsymbol r}}
\newcommand{\boldsigma}{{\boldsymbol \sigma}}
\newcommand{\boldupsilon}{{\boldsymbol \upsilon}}
\newcommand{\hone}{{\boldsymbol{H1}}}
\newcommand{\htwo}{\boldsymbol{H2}}
\newcommand{\hthree}{\boldsymbol{H3}}
\newcommand{\hfour}{\boldsymbol{H4}}


\newcommand{\sgn}{\mathrm{sgn}} \newcommand{\Maj}{\mathrm{Maj}}
\newcommand{\Acyc}{\mathrm{Acyc}}
\newcommand{\UniqMax}{\mathrm{UniqMax}}
\newcommand{\Thr}{\mathrm{Thr}} \newcommand{\littlesum}{{\textstyle
    \sum}}

\newcommand{\half}{{\textstyle \frac12}}
\newcommand{\third}{{\textstyle \frac13}}
\newcommand{\fourth}{{\textstyle \frac14}}

\newcommand{\Stab}{\mathbb{S}}
\newcommand{\StabThr}[2]{\Gamma_{#1}(#2)}
\newcommand{\StabThrmin}[2]{{\underline{\Gamma}}_{#1}(#2)}
\newcommand{\StabThrmax}[2]{{\overline{\Gamma}}_{#1}(#2)}
\newcommand{\TestFcn}{\Psi}

\renewcommand{\phi}{\varphi}

\begin{document}
\title{Algorithms - Exercise 5}

 \author{Omer Tamuz, 035696574}
\maketitle

\begin{enumerate}
\item 
  
  Let $a$ and $b$ be two $n$ digit numbers in base $d$. 
  Assume $n$ is a power of two (otherwise add at most $n-1$ leading zeros).
  Let $a_h$ be the $n/2$ high order digits, and $a_l$ be the $n/2$ low order
  digits. Define $b_h$ and $b_l$ likewise. Let $M=2^{n/2}$. 

  We would like to calculate $a\cdot b$. Let 
  $x=(a_l+a_h)\cdot (b_l+b_h)=a_l \cdot b_l+a_l\cdot b_h+a_h\cdot b_l + a_h\cdot b_h$, 
  $y=a_h \cdot b_h$ and $z=a_l\cdot b_l$. Then: 
  \begin{eqnarray*}
    a\cdot b&=& (a_hM+a_l)\cdot(b_hM+b_l)
    \\ &=& a_h \cdot b_hM^2+(a_h\cdot b_l+a_l\cdot b_h)M+a_l\cdot b_l
    \\ &=& yM^2+(x-y-z)M+z
  \end{eqnarray*}


  Let $T(n)$ be the amount of time it takes to multiply to $n$ digit numbers.
  Then
  \begin{itemize}
  \item The addition of two $n$ digit numbers is $O(n)$, by the naive algorithm.
  \item Multiplication by $M$ is linear, since all one has to do is add $n/2$
  trailing zeros.
  \item $x$, $y$ and $z$ are products of $n/2$ digit numbers. They each take 
    $T(n/2) + O(n)$  to calculate, since $x$ requires some additions, too.
  \item Once $x$, $y$ and $z$ are known then only additions and multiplications
    by $M$ are required.
  \item Hence  $T(n)=3T(n/2)+O(n)$ and $T(n)$ is $O(n^{\log 3})$.

  \end{itemize}


  \item Let $G$ be a graph. We would like to determine if this graph has
    a 9 vertex clique. 

    We first find all the triangles in the graph, by naively checking every
    triplet of vertices. We then construct a new graph, $G'$, where the 
    vertices of $G'$ correspond to the triangles of $G$. Two vertices $u,v$ 
    in $G'$ are connected iff every vertex in triangle $u$ in $G$ is connected
    to every vertex in triangle $v$ in $G$. 

    Next, we apply to $G'$
    the algorithm we saw in class for determining if a graph
    has triangles. Since a triangle in $G'$ corresponds to three triangles in
    $G$ where all the vertices are connected, then $G'$ has a triangle iff
    $G$ has a clique of size 9.

    $G'$ has $O(n^3)$ vertices, and so its construction will take 
    $O(n^6)$. Determining if it has a triangle, using Fast Matrix 
    Multiplication, will take $O(n^{3\omega})$. Since $3\omega \geq 6$ 
    then the total running time will be $O(n^{3\omega})$.
    
   
\end{enumerate}

\end{document}


