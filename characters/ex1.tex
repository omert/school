\documentclass[11pt]{article} \usepackage{amssymb}
\usepackage{amsfonts} \usepackage{amsmath} \usepackage{bm}
\usepackage{latexsym} \usepackage{epsfig}

\setlength{\textwidth}{6.5 in} \setlength{\textheight}{8.25in}
\setlength{\oddsidemargin}{0in} \setlength{\topmargin}{0in}
\addtolength{\textheight}{.8in} \addtolength{\voffset}{-.5in}

\newtheorem{theorem}{Theorem}[section]
\newtheorem{lemma}[theorem]{Lemma}
\newtheorem{proposition}[theorem]{Proposition}
\newtheorem{corollary}[theorem]{Corollary}
\newtheorem{fact}[theorem]{Fact}
\newtheorem{definition}[theorem]{Definition}
\newtheorem{remark}[theorem]{Remark}
\newtheorem{conjecture}[theorem]{Conjecture}
\newtheorem{claim}[theorem]{Claim}
\newtheorem{example}[theorem]{Example}
\newenvironment{proof}{\noindent \textbf{Proof:}}{$\Box$}

\newcommand{\ignore}[1]{}

\newcommand{\enote}[1]{} \newcommand{\knote}[1]{}
\newcommand{\rnote}[1]{}



% \newcommand{\enote}[1]{{\bf [[Elchanan:} {\emph{#1}}{\bf ]]}}
% \newcommand{\knote}[1]{{\bf [[Krzysztof:} {\emph{#1}}{\bf ]]}}
% \newcommand{\rnote}[1]{{\bf [[Ryan:} {\emph{#1}}{\bf ]]}}



\DeclareMathOperator{\Support}{Supp} \DeclareMathOperator{\Opt}{Opt}
\DeclareMathOperator{\Ordo}{\mathcal{O}}
\newcommand{\MaxkCSP}{\textsc{Max $k$-CSP}}
\newcommand{\MaxkCSPq}{\textsc{Max $k$-CSP$_{q}$}}
\newcommand{\MaxCSP}[1]{\textsc{Max CSP}(#1)} \renewcommand{\Pr}{{\bf
    P}} \renewcommand{\P}{{\bf P}} \newcommand{\Px}{\mathop{\bf P\/}}
\newcommand{\E}{{\bf E}} \newcommand{\Cov}{{\bf Cov}}
\newcommand{\Var}{{\bf Var}} \newcommand{\Varx}{\mathop{\bf Var\/}}

\newcommand{\bits}{\{-1,1\}}

\newcommand{\nsmaja}{\textstyle{\frac{2}{\pi}} \arcsin \rho}

\newcommand{\Inf}{\mathrm{Inf}} \newcommand{\I}{\mathrm{I}}
\newcommand{\J}{\mathrm{J}}

\newcommand{\eps}{\epsilon} \newcommand{\lam}{\lambda}

% \newcommand{\trunc}{\ell_{2,[-1,1]}}
\newcommand{\trunc}{\zeta} \newcommand{\truncprod}{\chi}

\newcommand{\N}{\mathbb N} \newcommand{\R}{\mathbb R}
\newcommand{\Z}{\mathbb Z} \newcommand{\CalE}{{\mathcal{E}}}
\newcommand{\CalC}{{\mathcal{C}}} \newcommand{\CalM}{{\mathcal{M}}}
\newcommand{\CalR}{{\mathcal{R}}} \newcommand{\CalS}{{\mathcal{S}}}
\newcommand{\CalV}{{\mathcal{V}}}
\newcommand{\cF}{\mathcal{F}}
\newcommand{\CalX}{{\boldsymbol{\mathcal{X}}}}
\newcommand{\CalG}{{\boldsymbol{\mathcal{G}}}}
\newcommand{\CalH}{{\boldsymbol{\mathcal{H}}}}
\newcommand{\CalY}{{\boldsymbol{\mathcal{Y}}}}
\newcommand{\CalZ}{{\boldsymbol{\mathcal{Z}}}}
\newcommand{\CalW}{{\boldsymbol{\mathcal{W}}}}
\newcommand{\CalF}{{\mathcal{Z}}}
% \newcommand{\boldG}{{\boldsymbol G}}
% \newcommand{\boldQ}{{\boldsymbol Q}}
% \newcommand{\boldP}{{\boldsymbol P}}
% \newcommand{\boldR}{{\boldsymbol R}}
% \newcommand{\boldS}{{\boldsymbol S}}
% \newcommand{\boldX}{{\boldsymbol X}}
% \newcommand{\boldB}{{\boldsymbol B}}
% \newcommand{\boldY}{{\boldsymbol Y}}
% \newcommand{\boldZ}{{\boldsymbol Z}}
% \newcommand{\boldV}{{\boldsymbol V}}
\newcommand{\boldi}{{\boldsymbol i}} \newcommand{\boldj}{{\boldsymbol
    j}} \newcommand{\boldk}{{\boldsymbol k}}
\newcommand{\boldr}{{\boldsymbol r}}
\newcommand{\boldsigma}{{\boldsymbol \sigma}}
\newcommand{\boldupsilon}{{\boldsymbol \upsilon}}
\newcommand{\hone}{{\boldsymbol{H1}}}
\newcommand{\htwo}{\boldsymbol{H2}}
\newcommand{\hthree}{\boldsymbol{H3}}
\newcommand{\hfour}{\boldsymbol{H4}}


\newcommand{\sgn}{\mathrm{sgn}} \newcommand{\Maj}{\mathrm{Maj}}
\newcommand{\Acyc}{\mathrm{Acyc}}
\newcommand{\UniqMax}{\mathrm{UniqMax}}
\newcommand{\Thr}{\mathrm{Thr}} \newcommand{\littlesum}{{\textstyle
    \sum}}

\newcommand{\half}{{\textstyle \frac12}}
\newcommand{\third}{{\textstyle \frac13}}
\newcommand{\fourth}{{\textstyle \frac14}}

\newcommand{\Stab}{\mathbb{S}}
\newcommand{\StabThr}[2]{\Gamma_{#1}(#2)}
\newcommand{\StabThrmin}[2]{{\underline{\Gamma}}_{#1}(#2)}
\newcommand{\StabThrmax}[2]{{\overline{\Gamma}}_{#1}(#2)}
\newcommand{\TestFcn}{\Psi}


\renewcommand{\phi}{\varphi}

\begin{document}
\title{Character Theory - Exercise 1}

 \author{Omer Tamuz, 035696574}
\maketitle


\begin{enumerate}
\item Let $x = \sum_{0 \leq k < n}x_k\omega^k$ be an element of $FG$
  where $G$ is the cyclic group of order $n$, $\omega$ is a
  generator of $G$, and $x_k$ is in $F$ for all $k$.

  Let $\cF:FG \to FG$ be the transformation defined by $\cF(x)_\ell =
  \sum_{0 \leq k < n}\omega^{-\ell k}x_k$. It is easy to see that
  $\cF$ is linear transformation, i.e., it preserves $FG$-addition.

  \begin{claim}
    $\cF(xy)_\ell = \cF(x)_\ell\cF(y)_\ell$.
  \end{claim}
  \begin{proof}
    \begin{align*}
      \cF(xy)_\ell &= \sum_{0 \leq k < n}\omega^{-\ell k}(xy)_k \\
      &= \sum_{0 \leq k < n}\omega^{-\ell k}\sum_{0 \leq m < n}x_my_{k-m \mod n} \\
      &= \sum_{0 \leq k < n}\omega^{-\ell (k+m)}\sum_{0 \leq m < n}x_my_{k} \\
      &= \sum_{0 \leq m < n}\omega^{-\ell m}x_m\sum_{0 \leq k < n}\omega^{-\ell k}y_{k} \\
      &= \cF(x)_\ell\cF(y)_\ell.
    \end{align*}
  \end{proof}
  
  \begin{lemma}
    If $x$ is the $FG$-multiplicative identity then $\cF(x)_\ell=1$
    for all $\ell$.
  \end{lemma}
  \begin{proof}
    Let $x$ be the $FG$-multiplicative identity so that $x_0 = 1$ and
    $x_k=0$ for $k \neq 0$. Then
    \begin{align*}
      \cF(x)_\ell &= \sum_{0 \leq k < n}x_k\omega^k \\
      &= \omega^0 \\
      &= 1.
    \end{align*}
  \end{proof}
  
  \begin{theorem}
    $x$ is invertible iff $\cF(x)_\ell$ is an invertible element of
    $F$ for all $\ell$.
  \end{theorem}
  \begin{proof}
    Let $x$ be invertible so that $xy=1$. Then $\cF(xy)_\ell=1$ for
    all $\ell$. Hence $\cF(x)_\ell\cF(x)_\ell = \cF(xy)_\ell = 1$, and
    so $\cF(x)_\ell = \cF(y)_\ell^{-1}$.

    Let $x$ be such that $\cF(x)_\ell$ be invertible for all
    $\ell$. Let $y \in FG$ be such that $y_k = \sum_{0 \leq \ell <
      n}\omega^{\ell k}\cF(x)_\ell^{-1}$.  Then
    \begin{align*}
      \cF(y)_\ell &= \sum_{0 \leq k < n}\omega^{-\ell k}y_k \\
      &= \sum_{0 \leq k < n}\omega^{-\ell k}\sum_{0 \leq m <
        n}\omega^{m k}\cF(x)_m^{-1} \\
      &= \sum_{0 \leq m < n}\cF(x)_m^{-1}\sum_{0 \leq k < n}\omega^{-\ell k}\omega^{m k} \\
      &= \sum_{0 \leq m < n}\cF(x)_m^{-1}\sum_{0 \leq k <
        n}\left(\omega^{m-\ell}\right)^k \\
      &= n\cF(x)_\ell^{-1}
    \end{align*}
    Assuming $n \neq 0$ in $F$ then the theorem is proved. In the case
    $n=0$ (i.e., the case that the characteristic of $F$ divides $|G|$) then 
    I'm not sure what to do...
  \end{proof}

\item Let $x=1+(12)$ and assume by way of contradiction that there
  exists a $y \in QS_3$ such that $xy=1$. Then
  \begin{align*}
    1 &= xy \\
    &= [1 + (12)] \cdot
    [y_11+y_{(12)}(12)+y_{(13)}(13)+y_{(23)}(23)+y_{123}(123)+y_{132}(132)]
    \\
    &=
    [y_11+y_{(12)}(12)+y_{(13)}(13)+y_{(23)}(23)+y_{123}(231)+y_{132}(213)]+
    [y_1(12)+y_{(12)}1+y_{(13)}(132)+y_{(23)}(132)+y_{123}(23)+y_{132}(13)]
  \end{align*}
  but this is impossible since it implies both $y_1+y_{(12)}=1$ (the
  coefficient of 1) and $y_1+y_{(12)}=0$ (the coefficient of
  $(12)$). The exact same contradiction is to be had if we define $y$
  to satisfy $yx=1$.

\item
  \begin{enumerate}
  \item 
    \begin{align*}
      \left(\sum_{g \in G}g\right)^2 &= \sum_{g \in G}g\sum_{h \in G}h
      \\
      &= \sum_{g \in G}\sum_{h \in G}gh \\
      &= \sum_{g \in G}\sum_{k \in G}k \\
      &= |G|\sum_{k \in G}k \\
      &= 0.
    \end{align*}

  \item Let $x=\sum_{g \in G}g$. Let $Z$ be the set of elements $z \in
    FG^\circ$ such that $xz=0$. Then by the above $Z$ includes
    $x$. Furthermore it is easy to see that $Z$ is a submodule of
    $FG^\circ$, and that $Z$ is neither trivial (it includes $x$) nor
    equal to $FG^\circ$ (it does not include the $FG$ multiplicative
    identity).

    Assume by way of contradiction that $FG$ is semi-simple. Then
    there exists a submodule $Y$ such that $FG^\circ = Y \oplus Z$, so
    that for every $a \in FG^\circ$ there exist unique $y \in Y$ and
    $z \in Z$ such that $a=y+z$. If we set $a=1$ then we arrive at
    \begin{align*}
      1 &= y + z \\
      x &= xy \\
      y &= 1
    \end{align*}
    and so the $FG$ multiplicative identity is in $Y$. However, $Y$
    must be closed to multiplication by $FG$, and so this means that
    $Y=FG^\circ$, and so the sum of $Z$ and $Y$ is no longer direct -
    contradiction.
  \end{enumerate}
\item
  
\end{enumerate}
\end{document}


