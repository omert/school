\documentclass[11pt]{article} \usepackage{amssymb}
\usepackage{amsfonts} \usepackage{amsmath} \usepackage{bm}
\usepackage{latexsym} \usepackage{epsfig}

\setlength{\textwidth}{6.5 in} \setlength{\textheight}{8.25in}
\setlength{\oddsidemargin}{0in} \setlength{\topmargin}{0in}
\addtolength{\textheight}{.8in} \addtolength{\voffset}{-.5in}

\newtheorem{theorem}{Theorem}[section]
\newtheorem{lemma}[theorem]{Lemma}
\newtheorem{proposition}[theorem]{Proposition}
\newtheorem{corollary}[theorem]{Corollary}
\newtheorem{fact}[theorem]{Fact}
\newtheorem{definition}[theorem]{Definition}
\newtheorem{remark}[theorem]{Remark}
\newtheorem{conjecture}[theorem]{Conjecture}
\newtheorem{claim}[theorem]{Claim}
\newtheorem{example}[theorem]{Example}
\newenvironment{proof}{\noindent \textbf{Proof:}}{$\Box$}

\newcommand{\ignore}[1]{}

\newcommand{\enote}[1]{} \newcommand{\knote}[1]{}
\newcommand{\rnote}[1]{}



% \newcommand{\enote}[1]{{\bf [[Elchanan:} {\emph{#1}}{\bf ]]}}
% \newcommand{\knote}[1]{{\bf [[Krzysztof:} {\emph{#1}}{\bf ]]}}
% \newcommand{\rnote}[1]{{\bf [[Ryan:} {\emph{#1}}{\bf ]]}}



\DeclareMathOperator{\Support}{Supp} \DeclareMathOperator{\Opt}{Opt}
\DeclareMathOperator{\Ordo}{\mathcal{O}}
\newcommand{\MaxkCSP}{\textsc{Max $k$-CSP}}
\newcommand{\MaxkCSPq}{\textsc{Max $k$-CSP$_{q}$}}
\newcommand{\MaxCSP}[1]{\textsc{Max CSP}(#1)} \renewcommand{\Pr}{{\bf
    P}} \renewcommand{\P}{{\bf P}} \newcommand{\Px}{\mathop{\bf P\/}}
\newcommand{\E}{{\bf E}} \newcommand{\Cov}{{\bf Cov}}
\newcommand{\Var}{{\bf Var}} \newcommand{\Varx}{\mathop{\bf Var\/}}

\newcommand{\bits}{\{-1,1\}}

\newcommand{\nsmaja}{\textstyle{\frac{2}{\pi}} \arcsin \rho}

\newcommand{\Inf}{\mathrm{Inf}} \newcommand{\I}{\mathrm{I}}
\newcommand{\J}{\mathrm{J}}

\newcommand{\eps}{\epsilon} \newcommand{\lam}{\lambda}

% \newcommand{\trunc}{\ell_{2,[-1,1]}}
\newcommand{\trunc}{\zeta} \newcommand{\truncprod}{\chi}

\newcommand{\N}{\mathbb N} \newcommand{\R}{\mathbb R}
\newcommand{\Z}{\mathbb Z} \newcommand{\CalE}{{\mathcal{E}}}
\newcommand{\CalC}{{\mathcal{C}}} \newcommand{\CalM}{{\mathcal{M}}}
\newcommand{\CalR}{{\mathcal{R}}} \newcommand{\CalS}{{\mathcal{S}}}
\newcommand{\CalV}{{\mathcal{V}}}
\newcommand{\cF}{\mathcal{F}}
\newcommand{\CalX}{{\boldsymbol{\mathcal{X}}}}
\newcommand{\CalG}{{\boldsymbol{\mathcal{G}}}}
\newcommand{\CalH}{{\boldsymbol{\mathcal{H}}}}
\newcommand{\CalY}{{\boldsymbol{\mathcal{Y}}}}
\newcommand{\CalZ}{{\boldsymbol{\mathcal{Z}}}}
\newcommand{\CalW}{{\boldsymbol{\mathcal{W}}}}
\newcommand{\CalF}{{\mathcal{Z}}}
% \newcommand{\boldG}{{\boldsymbol G}}
% \newcommand{\boldQ}{{\boldsymbol Q}}
% \newcommand{\boldP}{{\boldsymbol P}}
% \newcommand{\boldR}{{\boldsymbol R}}
% \newcommand{\boldS}{{\boldsymbol S}}
% \newcommand{\boldX}{{\boldsymbol X}}
% \newcommand{\boldB}{{\boldsymbol B}}
% \newcommand{\boldY}{{\boldsymbol Y}}
% \newcommand{\boldZ}{{\boldsymbol Z}}
% \newcommand{\boldV}{{\boldsymbol V}}
\newcommand{\boldi}{{\boldsymbol i}} \newcommand{\boldj}{{\boldsymbol
    j}} \newcommand{\boldk}{{\boldsymbol k}}
\newcommand{\boldr}{{\boldsymbol r}}
\newcommand{\boldsigma}{{\boldsymbol \sigma}}
\newcommand{\boldupsilon}{{\boldsymbol \upsilon}}
\newcommand{\hone}{{\boldsymbol{H1}}}
\newcommand{\htwo}{\boldsymbol{H2}}
\newcommand{\hthree}{\boldsymbol{H3}}
\newcommand{\hfour}{\boldsymbol{H4}}


\newcommand{\sgn}{\mathrm{sgn}} \newcommand{\Maj}{\mathrm{Maj}}
\newcommand{\Acyc}{\mathrm{Acyc}}
\newcommand{\UniqMax}{\mathrm{UniqMax}}
\newcommand{\Thr}{\mathrm{Thr}} \newcommand{\littlesum}{{\textstyle
    \sum}}

\newcommand{\half}{{\textstyle \frac12}}
\newcommand{\third}{{\textstyle \frac13}}
\newcommand{\fourth}{{\textstyle \frac14}}

\newcommand{\Stab}{\mathbb{S}}
\newcommand{\StabThr}[2]{\Gamma_{#1}(#2)}
\newcommand{\StabThrmin}[2]{{\underline{\Gamma}}_{#1}(#2)}
\newcommand{\StabThrmax}[2]{{\overline{\Gamma}}_{#1}(#2)}
\newcommand{\TestFcn}{\Psi}


\renewcommand{\phi}{\varphi}

\begin{document}
\title{Character Theory - Exercise 2}

 \author{Omer Tamuz, 035696574}
\maketitle

It is easy to verify that the conjugacy classes of $S_4$ are $(1)$,
$(12)$, $(123)$, $(12)(34)$ and $(1234)$, with sizes 1, 6, 8, 3 and
6. Thus the top of the character table is
\begin{table}[h]
\begin{center}
{\small
\begin{tabular}{ l | c | c | c | c | c }
  & $(1)$ & $(12)$ & $(123)$ & $(12)(34)$ & $(1234)$ \\
\hline
$|C|$ & 1 & 6 & 8 & 3 & 6\\
\end{tabular}
}
\end{center}
\end{table}

The constant character $\chi_1$ appears in any group. The signum
character $\chi_2$ is afforded by the representation that equals -1 for
odd permutations and 1 for even ones.
\begin{table}[h]
\begin{center}
{\small
\begin{tabular}{ l | c | c | c | c | c }
  & $(1)$ & $(12)$ & $(123)$ & $(12)(34)$ & $(1234)$ \\
\hline
$|C|$ & 1 & 6 & 8 & 3 & 6\\
\hline
$\chi_1$ & 1 & 1 & 1 & 1 & 1\\
$\chi_2$ & 1 & -1 & 1 & 1 & -1\\
\end{tabular}
}
\end{center}
\end{table}

A natural representation of $S_4$ is with $4 \times 4$ permutation
matrices. The character $\xi$ afforded by this representation is then
simply the number of fixed points of the permutation, so that
$\xi(1)=4$, $\xi(12) = 2$, $\xi(123) = 2$ and
$\xi((12)(34))=\xi(1234)=0$. Since this character's inner product with
$\chi_1$ is not zero ($(\xi,\chi_1) = 1$) then it is not
irreducible. However, if we calculate its projection on the space
orthogonal to $\chi_1$ then we get $\xi' = \xi - (\xi,\chi_1)\chi_1$
with $(\xi',\xi')=1$. Since only irreducible characters have this
property then $\xi'$ is irreducible and we can set $\chi_3=\xi'$.
\begin{table}[h]
\begin{center}
{\small
\begin{tabular}{ l | c | c | c | c | c }
  & $(1)$ & $(12)$ & $(123)$ & $(12)(34)$ & $(1234)$ \\
\hline
$|C|$ & 1 & 6 & 8 & 3 & 6\\
\hline
$\chi_1$ & 1 & 1 & 1 & 1 & 1\\
$\chi_2$ & 1 & -1 & 1 & 1 & -1\\
$\chi_3$ & 3 & 1 & 0 & -1 & -1\\
\end{tabular}
}
\end{center}
\end{table}

\pagebreak
Since $\sum_i\chi_i(1)^2=24$ then $\chi_4(1)^2+\chi_5(1)^2 =
13$. Since $\chi_i(1)$ is the order of the representation and hence an
integer for all $i$ then the only possibility is
$\{\chi_4(1),\chi_5(1)\}=\{2,3\}$. Hence we can choose as follows.
\begin{table}[h]
\begin{center}
{\small
\begin{tabular}{ l | c | c | c | c | c }
  & $(1)$ & $(12)$ & $(123)$ & $(12)(34)$ & $(1234)$ \\
\hline
$|C|$ & 1 & 6 & 8 & 3 & 6\\
\hline
$\chi_1$ & 1 & 1 & 1 & 1 & 1\\
$\chi_2$ & 1 & -1 & 1 & 1 & -1\\
$\chi_3$ & 3 & 1 & 0 & -1 & -1\\
$\chi_4$ & 2 & & & &\\
$\chi_5$ & 3 & & & &\\
\end{tabular}
}
\end{center}
\end{table}


We can fill in the rest of the gaps using the orthogonality
relations. First we will give names to the missing boxes:
\begin{table}[h]
\begin{center}
{\small
\begin{tabular}{ l | c | c | c | c | c }
  & $(1)$ & $(12)$ & $(123)$ & $(12)(34)$ & $(1234)$ \\
\hline
$|C|$ & 1 & 6 & 8 & 3 & 6\\
\hline
$\chi_1$ & 1 & 1 & 1 & 1 & 1\\
$\chi_2$ & 1 & -1 & 1 & 1 & -1\\
$\chi_3$ & 3 & 1 & 0 & -1 & -1\\
$\chi_4$ & 2 & a & b & c & d\\
$\chi_5$ & 3 & x & y & z & w\\
\end{tabular}
}
\end{center}
\end{table}

Taking the inner product of $\chi_4$ with each of the known characters
 we get that

\begin{align*}
  \begin{pmatrix}
    6 & 8 & 3 & 6 \\
    -6 & 8 & 3 & -6\\
    6 & 0 & -3 & -6\\
  \end{pmatrix}
  \begin{pmatrix}
    a\\
    b\\
    c\\
    d\\
  \end{pmatrix}
  =
  \begin{pmatrix}
    -2\\
    -2\\
    -6\\
  \end{pmatrix}
\end{align*}

Using Gauss elimination we get that
\begin{align*}
  \begin{pmatrix}
    1 & 0 & 0 & 1 \\
    0 & 2 & 0 & -3\\
    0 & 0 & 1 & 4\\
  \end{pmatrix}
  \begin{pmatrix}
    a\\
    b\\
    c\\
    d\\
  \end{pmatrix}
  =
  \begin{pmatrix}
    0\\
    -2\\
    2\\
  \end{pmatrix}
\end{align*}
so we can now rewrite our table as 

\begin{table}[h]
\begin{center}
{\small
\begin{tabular}{ l | c | c | c | c | c }
  & $(1)$ & $(12)$ & $(123)$ & $(12)(34)$ & $(1234)$ \\
\hline
$|C|$ & 1 & 6 & 8 & 3 & 6\\
\hline
$\chi_1$ & 1 & 1 & 1 & 1 & 1\\
$\chi_2$ & 1 & -1 & 1 & 1 & -1\\
$\chi_3$ & 3 & 1 & 0 & -1 & -1\\
$\chi_4$ & 2 & a & $-\frac{3}{2}a-1$ & $4a+2$ & -a\\
$\chi_5$ & 3 & x & y & z & w\\
\end{tabular}
}
\end{center}
\end{table}

\pagebreak
Applying the first orthogonality relation $(\chi_4,\chi_4)=1$ we get
that $a=0$ so that the table now looks like this:

\begin{table}[h]
\begin{center}
{\small
\begin{tabular}{ l | c | c | c | c | c }
  & $(1)$ & $(12)$ & $(123)$ & $(12)(34)$ & $(1234)$ \\
\hline
$|C|$ & 1 & 6 & 8 & 3 & 6\\
\hline
$\chi_1$ & 1 & 1 & 1 & 1 & 1\\
$\chi_2$ & 1 & -1 & 1 & 1 & -1\\
$\chi_3$ & 3 & 1 & 0 & -1 & -1\\
$\chi_4$ & 2 & 0 & $-1$ & $2$ & 0\\
$\chi_5$ & 3 & x & y & z & w\\
\end{tabular}
}
\end{center}
\end{table}

To finish filling in the gaps we can use the second orthogonality
relation, taking the inner product of each column with the first
column. We then get $3x+3=0$, $3y=0$, $3z=3$ and $3w=-3$. Hence the
full character table is:


\begin{table}[h]
\begin{center}
{\small
\begin{tabular}{ l | c | c | c | c | c }
  & $(1)$ & $(12)$ & $(123)$ & $(12)(34)$ & $(1234)$ \\
\hline
$|C|$ & 1 & 6 & 8 & 3 & 6\\
\hline
$\chi_1$ & 1 & 1 & 1 & 1 & 1\\
$\chi_2$ & 1 & -1 & 1 & 1 & -1\\
$\chi_3$ & 3 & 1 & 0 & -1 & -1\\
$\chi_4$ & 2 & 0 & $-1$ & $2$ & 0\\
$\chi_5$ & 3 & 1 & 0 & 1 & -1\\
\end{tabular}
}
\end{center}
\end{table}


\end{document}


