\documentclass[11pt]{article} \usepackage{amssymb}
\usepackage{amsfonts} \usepackage{amsmath} \usepackage{bm}
\usepackage{latexsym} \usepackage{epsfig}

\setlength{\textwidth}{6.5 in} \setlength{\textheight}{8.25in}
\setlength{\oddsidemargin}{0in} \setlength{\topmargin}{0in}
\addtolength{\textheight}{.8in} \addtolength{\voffset}{-.5in}

\newtheorem{theorem}{Theorem}[section]
\newtheorem{lemma}[theorem]{Lemma}
\newtheorem{proposition}[theorem]{Proposition}
\newtheorem{corollary}[theorem]{Corollary}
\newtheorem{fact}[theorem]{Fact}
\newtheorem{definition}[theorem]{Definition}
\newtheorem{remark}[theorem]{Remark}
\newtheorem{conjecture}[theorem]{Conjecture}
\newtheorem{claim}[theorem]{Claim}
\newtheorem{example}[theorem]{Example}
\newenvironment{proof}{\noindent \textbf{Proof:}}{$\Box$}

\newcommand{\ignore}[1]{}

\newcommand{\enote}[1]{} \newcommand{\knote}[1]{}
\newcommand{\rnote}[1]{}



% \newcommand{\enote}[1]{{\bf [[Elchanan:} {\emph{#1}}{\bf ]]}}
% \newcommand{\knote}[1]{{\bf [[Krzysztof:} {\emph{#1}}{\bf ]]}}
% \newcommand{\rnote}[1]{{\bf [[Ryan:} {\emph{#1}}{\bf ]]}}



\DeclareMathOperator{\Support}{Supp} \DeclareMathOperator{\Opt}{Opt}
\DeclareMathOperator{\Ordo}{\mathcal{O}}
\newcommand{\MaxkCSP}{\textsc{Max $k$-CSP}}
\newcommand{\MaxkCSPq}{\textsc{Max $k$-CSP$_{q}$}}
\newcommand{\MaxCSP}[1]{\textsc{Max CSP}(#1)} \renewcommand{\Pr}{{\bf
    P}} \renewcommand{\P}{{\bf P}} \newcommand{\Px}{\mathop{\bf P\/}}
\newcommand{\E}{{\bf E}} \newcommand{\Cov}{{\bf Cov}}
\newcommand{\Var}{{\bf Var}} \newcommand{\Varx}{\mathop{\bf Var\/}}

\newcommand{\bits}{\{-1,1\}}

\newcommand{\nsmaja}{\textstyle{\frac{2}{\pi}} \arcsin \rho}

\newcommand{\Inf}{\mathrm{Inf}} \newcommand{\I}{\mathrm{I}}
\newcommand{\J}{\mathrm{J}}

\newcommand{\eps}{\epsilon} \newcommand{\lam}{\lambda}

% \newcommand{\trunc}{\ell_{2,[-1,1]}}
\newcommand{\trunc}{\zeta} \newcommand{\truncprod}{\chi}

\newcommand{\N}{\mathbb N} \newcommand{\R}{\mathbb R}\newcommand{\C}{\mathbb C}
\newcommand{\Z}{\mathbb Z} \newcommand{\CalE}{{\mathcal{E}}}
\newcommand{\CalC}{{\mathcal{C}}} \newcommand{\CalM}{{\mathcal{M}}}
\newcommand{\CalR}{{\mathcal{R}}} \newcommand{\CalS}{{\mathcal{S}}}
\newcommand{\CalV}{{\mathcal{V}}}
\newcommand{\cF}{\mathcal{F}}
\newcommand{\CalX}{{\boldsymbol{\mathcal{X}}}}
\newcommand{\CalG}{{\boldsymbol{\mathcal{G}}}}
\newcommand{\CalH}{{\boldsymbol{\mathcal{H}}}}
\newcommand{\CalY}{{\boldsymbol{\mathcal{Y}}}}
\newcommand{\CalZ}{{\boldsymbol{\mathcal{Z}}}}
\newcommand{\CalW}{{\boldsymbol{\mathcal{W}}}}
\newcommand{\CalF}{{\mathcal{Z}}}
% \newcommand{\boldG}{{\boldsymbol G}}
% \newcommand{\boldQ}{{\boldsymbol Q}}
% \newcommand{\boldP}{{\boldsymbol P}}
% \newcommand{\boldR}{{\boldsymbol R}}
% \newcommand{\boldS}{{\boldsymbol S}}
% \newcommand{\boldX}{{\boldsymbol X}}
% \newcommand{\boldB}{{\boldsymbol B}}
% \newcommand{\boldY}{{\boldsymbol Y}}
% \newcommand{\boldZ}{{\boldsymbol Z}}
% \newcommand{\boldV}{{\boldsymbol V}}
\newcommand{\boldi}{{\boldsymbol i}} \newcommand{\boldj}{{\boldsymbol
    j}} \newcommand{\boldk}{{\boldsymbol k}}
\newcommand{\boldr}{{\boldsymbol r}}
\newcommand{\boldsigma}{{\boldsymbol \sigma}}
\newcommand{\boldupsilon}{{\boldsymbol \upsilon}}
\newcommand{\hone}{{\boldsymbol{H1}}}
\newcommand{\htwo}{\boldsymbol{H2}}
\newcommand{\hthree}{\boldsymbol{H3}}
\newcommand{\hfour}{\boldsymbol{H4}}


\newcommand{\sgn}{\mathrm{sgn}} \newcommand{\Maj}{\mathrm{Maj}}
\newcommand{\Acyc}{\mathrm{Acyc}}
\newcommand{\UniqMax}{\mathrm{UniqMax}}
\newcommand{\Thr}{\mathrm{Thr}} \newcommand{\littlesum}{{\textstyle
    \sum}}

\newcommand{\half}{{\textstyle \frac12}}
\newcommand{\third}{{\textstyle \frac13}}
\newcommand{\fourth}{{\textstyle \frac14}}

\newcommand{\Stab}{\mathbb{S}}
\newcommand{\StabThr}[2]{\Gamma_{#1}(#2)}
\newcommand{\StabThrmin}[2]{{\underline{\Gamma}}_{#1}(#2)}
\newcommand{\StabThrmax}[2]{{\overline{\Gamma}}_{#1}(#2)}
\newcommand{\TestFcn}{\Psi}


\renewcommand{\phi}{\varphi}
\newcommand{\de}{\mathrm{det}}

\begin{document}
\title{Character Theory - Exercise 3}

 \author{Omer Tamuz, 035696574}
\maketitle

\begin{enumerate}
\item Let $\chi$ be a character of a finite group $G$ and let $X$ be a
  representation affording $\chi$.  Let $\de \chi:G \to \C$ be defined
  by $(\de\chi)(g) = \de X(g)$. We will first show that $\de\chi$ is a
  linear character of $G$.
  \begin{claim}
    $(\de\chi)(gh) = (\de\chi)(g)(\de\chi)(h)$.
  \end{claim}
  \begin{proof}
    By the definition of $\det\chi$ we have that
    \begin{align*}
      (\de\chi)(gh) &= \de X(gh).
    \end{align*}
    Since $X$ is a representation then $X(gh)=X(g)X(h)$ and hence 
    \begin{align*}
      (\de\chi)(gh) &= \de (X(g)X(h)).
    \end{align*}
    Finally, since $\de (AB) = \de A \cdot \de B$ then
    \begin{align*}
      (\de\chi)(gh) &= (\de\chi)(g)(\de\chi)(h).      
    \end{align*}
  \end{proof}
  
  Let $Y$ be another representation affording $\chi$. Then we've shown
  in class that for any $g \in G$ it holds that $Y(g)$ and $X(g)$ are
  similar, and so in particular have the same determinant. Hence
  $\de\chi$ is independent of the representation.
\item
  \begin{enumerate}
  \item 

    Since $G$ is of order 8 then $G'$ is of order either $1$, $2$, $4$
    or $8$. The smallest perfect group is $A_5$, and therefor $|G'|
    \neq 8$. Also, $G$ is non-abelian, and therefore $|G'| \neq 1$. We
    have shown in the past that $G/G'$ cannot be cyclic (unless
    trivial), and so we have that $|G'|=2$ and furthermore $G/G'$ is
    isomorphic to $\Z_2 \times \Z_2$.

    Since the number of linear characters is equal to $|G/G'|$ then
    these number four. To make the sum of squares of the degrees equal
    8 there must then be only one single degree 2 character.

    Since $G'$ is of order 2 we can safely name its non-unit member
    $-1$. Define $\chi :G \to \C$ by
    \begin{align*}
      \chi(g) =
      \begin{cases}
        2 &\;\;g=1\\
        -2 &\;\;g=-1\\
        0 &\;\;\mathrm{otherwise}
      \end{cases}
    \end{align*}
    Then $\chi$ satisfies the conditions of the answer. It remains to
    be shown that it is an irreducible character. Since $1$ and $-1$
    are in the center of $G$ then they are the only members of their
    conjugacy classes. Hence $\chi$ is a class function.

    Since, as we've shown in class, $G'=\cap_{\chi(1)=1}\ker\chi$ then
    the rest of the characters map $-1$ to 1. Hence $\chi$ is
    orthogonal to them. Finally, $(\chi,\chi)=1$, and so it must be
    the last degree two irreducible character.

  \item Let $X$ be a representation of $D_8$ 
    affording $\chi$. Since $-1$ is in the center of $D_8$
    then $X(-1)=-I_2$.
 \end{enumerate}
\end{enumerate}

\end{document}


