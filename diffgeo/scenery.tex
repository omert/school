\documentclass[11pt]{article} \usepackage{amssymb}
\usepackage{amsfonts} \usepackage{amsmath} \usepackage{bm}
\usepackage{latexsym} \usepackage{epsfig}

\setlength{\textwidth}{6.5 in} \setlength{\textheight}{8.25in}
\setlength{\oddsidemargin}{0in} \setlength{\topmargin}{0in}
\addtolength{\textheight}{.8in} \addtolength{\voffset}{-.5in}

\newtheorem{theorem}{Theorem}[section]
\newtheorem{lemma}[theorem]{Lemma}
\newtheorem{proposition}[theorem]{Proposition}
\newtheorem{corollary}[theorem]{Corollary}
\newtheorem{fact}[theorem]{Fact}
\newtheorem{definition}[theorem]{Definition}
\newtheorem{remark}[theorem]{Remark}
\newtheorem{conjecture}[theorem]{Conjecture}
\newtheorem{example}[theorem]{Example}
\newenvironment{proof}{\noindent \textbf{Proof:}}{$\Box$}

\newcommand{\ignore}[1]{}

\newcommand{\enote}[1]{} \newcommand{\knote}[1]{}
\newcommand{\rnote}[1]{}



% \newcommand{\enote}[1]{{\bf [[Elchanan:} {\emph{#1}}{\bf ]]}}
% \newcommand{\knote}[1]{{\bf [[Krzysztof:} {\emph{#1}}{\bf ]]}}
% \newcommand{\rnote}[1]{{\bf [[Ryan:} {\emph{#1}}{\bf ]]}}



\DeclareMathOperator{\Support}{Supp} \DeclareMathOperator{\Opt}{Opt}
\DeclareMathOperator{\Ordo}{\mathcal{O}}
\newcommand{\MaxkCSP}{\textsc{Max $k$-CSP}}
\newcommand{\MaxkCSPq}{\textsc{Max $k$-CSP$_{q}$}}
\newcommand{\MaxCSP}[1]{\textsc{Max CSP}(#1)} \renewcommand{\Pr}{{\bf
    P}} \renewcommand{\P}{{\bf P}} \newcommand{\Px}{\mathop{\bf P\/}}
\newcommand{\E}{{\bf E}} \newcommand{\Cov}{{\bf Cov}}
\newcommand{\Var}{{\bf Var}} \newcommand{\Varx}{\mathop{\bf Var\/}}

\newcommand{\bits}{\{-1,1\}}

\newcommand{\nsmaja}{\textstyle{\frac{2}{\pi}} \arcsin \rho}

\newcommand{\Inf}{\mathrm{Inf}} \newcommand{\I}{\mathrm{I}}
\newcommand{\J}{\mathrm{J}}

\newcommand{\eps}{\epsilon} \newcommand{\lam}{\lambda}

% \newcommand{\trunc}{\ell_{2,[-1,1]}}
\newcommand{\trunc}{\zeta} \newcommand{\truncprod}{\chi}

\newcommand{\N}{\mathbb N} \newcommand{\R}{\mathbb R}
\newcommand{\Z}{\mathbb Z} \newcommand{\CalE}{{\mathcal{E}}}
\newcommand{\CalC}{{\mathcal{C}}} \newcommand{\CalM}{{\mathcal{M}}}
\newcommand{\CalR}{{\mathcal{R}}} \newcommand{\CalS}{{\mathcal{S}}}
\newcommand{\CalV}{{\mathcal{V}}}
\newcommand{\CalX}{{\boldsymbol{\mathcal{X}}}}
\newcommand{\CalG}{{\boldsymbol{\mathcal{G}}}}
\newcommand{\CalH}{{\boldsymbol{\mathcal{H}}}}
\newcommand{\CalY}{{\boldsymbol{\mathcal{Y}}}}
\newcommand{\CalZ}{{\boldsymbol{\mathcal{Z}}}}
\newcommand{\CalW}{{\boldsymbol{\mathcal{W}}}}
\newcommand{\CalF}{{\mathcal{Z}}}
% \newcommand{\boldG}{{\boldsymbol G}}
% \newcommand{\boldQ}{{\boldsymbol Q}}
% \newcommand{\boldP}{{\boldsymbol P}}
% \newcommand{\boldR}{{\boldsymbol R}}
% \newcommand{\boldS}{{\boldsymbol S}}
% \newcommand{\boldX}{{\boldsymbol X}}
% \newcommand{\boldB}{{\boldsymbol B}}
% \newcommand{\boldY}{{\boldsymbol Y}}
% \newcommand{\boldZ}{{\boldsymbol Z}}
% \newcommand{\boldV}{{\boldsymbol V}}
\newcommand{\boldi}{{\boldsymbol i}} \newcommand{\boldj}{{\boldsymbol
    j}} \newcommand{\boldk}{{\boldsymbol k}}
\newcommand{\boldr}{{\boldsymbol r}}
\newcommand{\boldsigma}{{\boldsymbol \sigma}}
\newcommand{\boldupsilon}{{\boldsymbol \upsilon}}
\newcommand{\hone}{{\boldsymbol{H1}}}
\newcommand{\htwo}{\boldsymbol{H2}}
\newcommand{\hthree}{\boldsymbol{H3}}
\newcommand{\hfour}{\boldsymbol{H4}}


\newcommand{\sgn}{\mathrm{sgn}} \newcommand{\Maj}{\mathrm{Maj}}
\newcommand{\Acyc}{\mathrm{Acyc}}
\newcommand{\UniqMax}{\mathrm{UniqMax}}
\newcommand{\Thr}{\mathrm{Thr}} \newcommand{\littlesum}{{\textstyle
    \sum}}

\newcommand{\half}{{\textstyle \frac12}}
\newcommand{\third}{{\textstyle \frac13}}
\newcommand{\fourth}{{\textstyle \frac14}}

\newcommand{\Stab}{\mathbb{S}}
\newcommand{\StabThr}[2]{\Gamma_{#1}(#2)}
\newcommand{\StabThrmin}[2]{{\underline{\Gamma}}_{#1}(#2)}
\newcommand{\StabThrmax}[2]{{\overline{\Gamma}}_{#1}(#2)}
\newcommand{\TestFcn}{\Psi}

\renewcommand{\phi}{\varphi}

\begin{document}
\title{Scenery Reconstruction}

% \author{Elchanan Mossel\footnote{Supported by a Sloan fellowship in
%     Mathematics, by BSF grant 2004105, NSF Career Award (DMS 054829)
%     and by ONR award N00014-07-1-0506.
%     Part %of this work was carried out while the author was visiting IPAM, UCLA}
%     \\U.C.\ Berkeley and Weizmann Institute
%     \\mossel@stat.berkeley.edu} \date{\today}
\maketitle

\section{Notation and Basic Facts}
Some standard definitions are described below. 

Given functions $f:S^1\to\R$ and $g:S^1\to\R$:
\begin{itemize}
\item Fourier transform is denoted by $\mathcal{F}$ and $\hat{f}$ is the Fourier transform of $f$:
  \begin{equation*}
    \mathcal{F}[f]_l=\hat{f}_l = \int_{\theta=0}^{2\pi}{d\theta e^{-il\theta}f(\theta)}
  \end{equation*}
\item Inner product is denoted by $\left<f,g\right>$:
\begin{eqnarray*}
\left<f,g\right>:={1\over 2\pi}\int_{\theta=0}^{2\pi}{d\theta f(\theta)g(\theta)}\\
\left<\hat{f},\hat{g}\right>:=\sum_{l=-\infty}^\infty{\hat{f}_l^*\hat{g}_l}
\end{eqnarray*}
\item Convolution is denoted by $*$:
  \begin{equation*}
    (f*g)(\phi)={1 \over 2\pi}\int_{\theta=0}^{2\pi}{d\theta f(\theta)g(\theta-\phi)}
  \end{equation*}
\item $k$-fold convolution is denoted by $f^{(k)}$:
  \begin{equation}
    f^{(k)}=\underbrace{f*f*\cdots*f}_{k\; \mathrm{times}}
  \end{equation}
\item The auto-correlation function of $f$ is $f*f$.
\end{itemize}

Some basic facts:
\begin{itemize}
\item $\left<f,g\right>=\left<\hat{f},\hat{g}\right>$
\item $\mathcal{F}[f*f]=\widehat{f*f}=|\hat{f}|^2$
\item $\mathcal{F}[f^{(k)}]=|\hat{f}|^k$, for even $k$.
\item $(f*f)(0)=\left<f,f\right>$
\end{itemize}
\section{General Case}
We examine a general case of a random walk on the unit circle, 
where we want to reconstruct a function $f(\theta)$, 
and at each step the random walk makes
a jump $\phi$ according to a distribution $g(\phi)$. We denote $\theta_k$ as the
position of the random walk at time $k$, and $r_k=f(\theta_k)$ the value
of $f$ at time $k$.  

The expected correlation product of $r_k$'s of consecutive steps of the random 
walk is:
\begin{eqnarray*}
  a_1&=&\E[r_kr_{k+1}]\\
     &=&\E[f(\theta_k)f(\theta_{k+1})]\\
     &=&{1\over 2\pi}\int_{\phi=0}^{2\pi}{d\phi g(\phi){1\over 2\pi}\int_{\theta=0}^{2\pi}{d\theta f(\theta)f(\theta-\phi)}}\\
     &=&\left<g,f*f\right>\\
     &=&\left<\hat{g},|\hat{f}|^2\right>
\end{eqnarray*}

and in general the product of $r_k$'s separated by $m$ steps is
\begin{eqnarray*}
  a_m&=&\E[r_kr_{k+m}]\\
     &=&\left<g^{(m)},f*f\right>\\
     &=&\left<\hat{g}^m,|\hat{f}|^2\right>.
\end{eqnarray*}

This also holds for $m=0$, where $\hat{g}^0$ is constant.

Since the expected value $\E[r_kr_{k+m}]$ can be estimated from the $r_k$'s,
the autocorrelation of $f$, $f*f$, can be reconstructed in the linear
space spanned by the $\hat{g}^m$'s.

Note that $f$ cannot be, in general, reconstructed from $f*f$, since the
phases of the Fourier transform elements $\hat{f}_l$'s are lost. 
{\bf In particular, two functions 
that are identical up to a shift or a reflection have the same 
auto-correlation function}.

\section{$n$-cycle}
The case of $g={1\over 2}\left(\delta(\phi-2\pi/n)+\delta(\phi+2\pi/n)\right)$ 
corresponds to
a random walk on the $n$-cycle. Then the values of $f$ are only sampled at
integer multiples of $2\pi/n$. In this case, the dimension of the vector
space of such functions is $n$, so $n$ linearly independent functions
$g^{(m)}$ are sufficient to reconstruct $f*f$. Indeed, the set
$\{g^{(m)}\:|\:m=0,\cdots,n-1\}$ is linearly independent in this case.
However, it seems likely that a larger set, in particular
$\{g^{(m)}\:|\:m=0,\cdots,n^2\}$, would provide better estimation
for a given sample size. For $m>n^2$, the $g^{(m)}$'s are close to constant,
and perhaps no more information can be obtained.

\subsection{$f:\Z_n\to\{0,1\}$}
If $f:\Z_n\to\{0,1\}$, then:
\begin{conjecture}
$f$ is reconstructible from $f*f$, up to shifts and reflections. (Actually,
sometimes two functions seem to fit an auto-correlation funtion...)
\end{conjecture}
\begin{conjecture}
$f$ is reconstructible from $f*f$, up to shifts and reflections, 
in polynomial time in $n$.
\end{conjecture}
\begin{conjecture}
$f*f$ can be reconstructed to arbitrary precision from $\left\{s_m\;|\;m=0,\cdots,O(n^2)\right\}$, where
\begin{equation}
   s_m={1\over T}\sum_{k=0}^T[r_kr_{k+m}],
\end{equation}
and $T=O(n^6)$.

\end{conjecture}
We've implemented an algorithm which seems to consistently reconstruct
a binary scenery in $O(n^6)$, supporting the conjectures above.

\end{document}



