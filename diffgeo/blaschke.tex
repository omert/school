\documentclass[11pt]{book} \usepackage{amssymb}
\usepackage{amsfonts} \usepackage{amsmath} \usepackage{bm}
\usepackage{latexsym} \usepackage{epsfig}

\setlength{\textwidth}{6.5 in} \setlength{\textheight}{8.25in}
\setlength{\oddsidemargin}{0in} \setlength{\topmargin}{0in}
\addtolength{\textheight}{.8in} \addtolength{\voffset}{-.5in}

\newtheorem{theorem}{Theorem}[section]
\newtheorem{lemma}[theorem]{Lemma}
\newtheorem{proposition}[theorem]{Proposition}
\newtheorem{corollary}[theorem]{Corollary}
\newtheorem{fact}[theorem]{Fact}
\newtheorem{definition}[theorem]{Definition}
\newtheorem{remark}[theorem]{Remark}
\newtheorem{conjecture}[theorem]{Conjecture}
\newtheorem{claim}[theorem]{Claim}
\newtheorem{example}[theorem]{Example}
\newenvironment{proof}{\noindent \textbf{Proof:}}{$\Box$}

\newcommand{\ignore}[1]{}

\newcommand{\enote}[1]{} \newcommand{\knote}[1]{}
\newcommand{\rnote}[1]{}



% \newcommand{\enote}[1]{{\bf [[Elchanan:} {\emph{#1}}{\bf ]]}}
% \newcommand{\knote}[1]{{\bf [[Krzysztof:} {\emph{#1}}{\bf ]]}}
% \newcommand{\rnote}[1]{{\bf [[Ryan:} {\emph{#1}}{\bf ]]}}



\DeclareMathOperator{\Support}{Supp} \DeclareMathOperator{\Opt}{Opt}
\DeclareMathOperator{\Ordo}{\mathcal{O}}
\newcommand{\MaxkCSP}{\textsc{Max $k$-CSP}}
\newcommand{\MaxkCSPq}{\textsc{Max $k$-CSP$_{q}$}}
\newcommand{\MaxCSP}[1]{\textsc{Max CSP}(#1)} \renewcommand{\Pr}{{\bf
    P}} \renewcommand{\P}{{\bf P}} \newcommand{\Px}{\mathop{\bf P\/}}
\newcommand{\E}{{\bf E}} \newcommand{\Cov}{{\bf Cov}}
\newcommand{\Var}{{\bf Var}} \newcommand{\Varx}{\mathop{\bf Var\/}}

\newcommand{\bits}{\{-1,1\}}

\newcommand{\nsmaja}{\textstyle{\frac{2}{\pi}} \arcsin \rho}

\newcommand{\Inf}{\mathrm{Inf}} \newcommand{\I}{\mathrm{I}}
\newcommand{\J}{\mathrm{J}}

\newcommand{\eps}{\epsilon} \newcommand{\lam}{\lambda}

% \newcommand{\trunc}{\ell_{2,[-1,1]}}
\newcommand{\trunc}{\zeta} \newcommand{\truncprod}{\chi}

\newcommand{\N}{\mathbb N} \newcommand{\R}{\mathbb R}
\newcommand{\Z}{\mathbb Z} \newcommand{\CalE}{{\mathcal{E}}}
\newcommand{\CalC}{{\mathcal{C}}} \newcommand{\CalM}{{\mathcal{M}}}
\newcommand{\CalR}{{\mathcal{R}}} \newcommand{\CalS}{{\mathcal{S}}}
\newcommand{\CalV}{{\mathcal{V}}}
\newcommand{\CalX}{{\boldsymbol{\mathcal{X}}}}
\newcommand{\CalG}{{\boldsymbol{\mathcal{G}}}}
\newcommand{\CalH}{{\boldsymbol{\mathcal{H}}}}
\newcommand{\CalY}{{\boldsymbol{\mathcal{Y}}}}
\newcommand{\CalZ}{{\boldsymbol{\mathcal{Z}}}}
\newcommand{\CalW}{{\boldsymbol{\mathcal{W}}}}
\newcommand{\CalF}{{\mathcal{Z}}}
% \newcommand{\boldG}{{\boldsymbol G}}
% \newcommand{\boldQ}{{\boldsymbol Q}}
% \newcommand{\boldP}{{\boldsymbol P}}
% \newcommand{\boldR}{{\boldsymbol R}}
% \newcommand{\boldS}{{\boldsymbol S}}
% \newcommand{\boldX}{{\boldsymbol X}}
% \newcommand{\boldB}{{\boldsymbol B}}
% \newcommand{\boldY}{{\boldsymbol Y}}
% \newcommand{\boldZ}{{\boldsymbol Z}}
% \newcommand{\boldV}{{\boldsymbol V}}
\newcommand{\boldi}{{\boldsymbol i}} \newcommand{\boldj}{{\boldsymbol
    j}} \newcommand{\boldk}{{\boldsymbol k}}
\newcommand{\boldr}{{\boldsymbol r}}
\newcommand{\boldsigma}{{\boldsymbol \sigma}}
\newcommand{\boldupsilon}{{\boldsymbol \upsilon}}
\newcommand{\hone}{{\boldsymbol{H1}}}
\newcommand{\htwo}{\boldsymbol{H2}}
\newcommand{\hthree}{\boldsymbol{H3}}
\newcommand{\hfour}{\boldsymbol{H4}}


\newcommand{\sgn}{\mathrm{sgn}} \newcommand{\Maj}{\mathrm{Maj}}
\newcommand{\Acyc}{\mathrm{Acyc}}
\newcommand{\UniqMax}{\mathrm{UniqMax}}
\newcommand{\Thr}{\mathrm{Thr}} \newcommand{\littlesum}{{\textstyle
    \sum}}

\newcommand{\half}{{\textstyle \frac12}}
\newcommand{\third}{{\textstyle \frac13}}
\newcommand{\fourth}{{\textstyle \frac14}}

\newcommand{\Stab}{\mathbb{S}}
\newcommand{\StabThr}[2]{\Gamma_{#1}(#2)}
\newcommand{\StabThrmin}[2]{{\underline{\Gamma}}_{#1}(#2)}
\newcommand{\StabThrmax}[2]{{\overline{\Gamma}}_{#1}(#2)}
\newcommand{\TestFcn}{\Psi}

\renewcommand{\phi}{\varphi}

\begin{document}
\title{Affine Differential Geometry}

 \author{Wilhelm Blaschke, Translated by Omer Tamuz}
\maketitle

\chapter{Introduction to Curves in the Plane}
\section{Affine Mapping}
We would like to start by recalling some well known facts of analytical 
geometry.

For our purposes, it will be useful to represent the points of a plane with
a parallel coordinate system. We choose two intersecting lines 
in the plane and mark on each some direction as positive. We call these lines
the $x_1$ and $x_2$ axes. Through a remaining point of the plane we pass
parallels to the axes, and assign the corresponding segments of the axes 
the numbers $x_1$ and $x_2$. We have thus defined a bijection between 
pairs of real number $x_1$ and $x_2$, to the points of the plane.

We will thus deal with ``real'' points and lines, and in fact consider only
those points and lines that in the framework of projective geometry are 
considered ``proper'', as opposed to ``improper'' or infinitely remote elements.

In place of the pair $x_1$, $x_2$ we use the shorter vector notation and 
speak simply of the point ${\bf x}$.

There exists a mapping ${\bf x}\to {\bf x}^*$ between the points of plane,
which is given by a system of linear relations between their coordinates:
\begin{equation}
  \label{eq:affine_trans}
  \begin{array}{rcl}
    x_1^* &=& c_{10}+c_{11}x_1+c_{12}x_2\\
    x_2^* &=& c_{20}+c_{21}x_1+c_{22}x_2.
  \end{array}
\end{equation}
This system of equations is solvable for $x$ when the determinant 
\begin{equation}
  d=c_{11}c_{22}-c_{21}c_{12}=
  \begin{vmatrix}
    c_{11} & c_{12}\\
    c_{21} & c_{22}
  \end{vmatrix}
\end{equation}
is different than zero ($\neq 0$). Then, to each ${\bf x}^*$ corresponds 
one and only one ${\bf x}$. Following Euler and M\"obius, this kind of mapping
is called an ``affine mapping'', ``affine transformation'' or just ``affinity''.

The inverse transformation, which one can determine by 
solving~\ref{eq:affine_trans}, has again the same form: 
\begin{equation}
  \label{eq:affine_trans_inv}
  \begin{array}{rcl}
    x_1 &=& c_{10}^*+c_{11}^*x_1^*+c_{12}^*x_2^*\\
    x_2 &=& c_{20}^*+c_{21}^*x_1^*+c_{22}^*x_2^*,
  \end{array}
\end{equation}
where
\begin{equation}
  \label{eq:inv_c}
  \begin{array}{lll}
    c_{10}^*=-{c_{10}c_{22}-c_{20}c_{12} \over d},& c_{11}^*=+{c_{22}\over d},&c_{12}^*=-{c_{12}\over d},\\
    c_{20}^*=-{c_{10}c_{21}-c_{20}c_{11} \over d},& c_{21}^*=+{c_{21}\over d},&c_{22}^*=-{c_{11}\over d}.
  \end{array}
\end{equation}
The determinant
\begin{equation}
  \label{eq:inv_det}
  d^*=c_{11}^*c_{22}^*-c_{12}^*c_{21}^*={1\over d}
\end{equation}
is different than zero and hence the inverse transformation is also affine.

The points ${\bf x}$ of a line $g$ are mapped by an affine transformation to
the points ${\bf x}^*$ of a line $g^*$.
\end{document}


