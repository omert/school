\documentclass[11pt]{book} \usepackage{amssymb}
\usepackage{amsfonts} \usepackage{amsmath} \usepackage{bm}
\usepackage{latexsym} \usepackage{epsfig}


\newtheorem{theorem}{Theorem}[section]
\newtheorem{lemma}[theorem]{Lemma}
\newtheorem{proposition}[theorem]{Proposition}
\newtheorem{corollary}[theorem]{Corollary}
\newtheorem{fact}[theorem]{Fact}
\newtheorem{definition}[theorem]{Definition}
\newtheorem{remark}[theorem]{Remark}
\newtheorem{conjecture}[theorem]{Conjecture}
\newtheorem{claim}[theorem]{Claim}
\newtheorem{example}[theorem]{Example}
\newenvironment{proof}{\noindent \textbf{Proof:}}{$\Box$}

\newcommand{\myvec}[1]{\mathfrak{#1}}

\begin{document}
\title{Affine Differential Geometry}

 \author{Wilhelm Blaschke, Translated by Omer Tamuz}
\maketitle

\chapter{Local Theory of Planar Curves}
\section{Affine Mapping}
We would like to start by recalling some well known facts of analytical 
geometry.

For our purposes, it will be useful to represent the points of a plane with
a parallel coordinate system. We choose two intersecting lines\footnote{We
refer to straight lines simply as ``lines'' throughout the text}
in the plane and mark on each some direction as positive. We call these lines
the $x_1$ and $x_2$ axes. Through a remaining point of the plane we pass
parallels to the axes, and assign the corresponding segments of the axes 
the numbers $x_1$ and $x_2$. We have thus defined a bijection between 
pairs of real number $x_1$ and $x_2$, to the points of the plane.

We will thus deal with ``real'' points and lines, and in fact consider only
those points and lines that in the framework of projective geometry are 
considered ``proper'', as opposed to ``improper'' or infinitely remote elements.

In place of the pair $x_1$, $x_2$ we use the shorter vector notation and 
speak simply of the point $\myvec{x}$.

There exists a mapping $\myvec{x}\to \myvec{x}^*$ between the points of plane,
which is given by a system of linear relations between their coordinates:
\begin{equation}
  \label{eq:affine_trans}
  \begin{array}{rcl}
    x_1^* &=& c_{10}+c_{11}x_1+c_{12}x_2\\
    x_2^* &=& c_{20}+c_{21}x_1+c_{22}x_2.
  \end{array}
\end{equation}
This system of equations is solvable for $x$ when the determinant 
\begin{equation}
  d=c_{11}c_{22}-c_{21}c_{12}=
  \begin{vmatrix}
    c_{11} & c_{12}\\
    c_{21} & c_{22}
  \end{vmatrix}
\end{equation}
is different than zero ($\neq 0$). Then, to each $\myvec{x}^*$ corresponds 
one and only one $\myvec{x}$. Following Euler \cite{Euler:1749} and 
M\"obius \cite{Mobius:1827}, this kind of mapping
is called an ``affine mapping'', ``affine transformation'' or just ``affinity''.
 
The inverse transformation, which one can determine by 
solving~\ref{eq:affine_trans}, has again the same form: 
\begin{equation}
  \label{eq:affine_trans_inv}
  \begin{array}{rcl}
    x_1 &=& c_{10}^*+c_{11}^*x_1^*+c_{12}^*x_2^*\\
    x_2 &=& c_{20}^*+c_{21}^*x_1^*+c_{22}^*x_2^*,
  \end{array}
\end{equation}
where
\begin{equation}
  \label{eq:inv_c}
  \begin{array}{lll}
    c_{10}^*=-{c_{10}c_{22}-c_{20}c_{12} \over d},& c_{11}^*=+{c_{22}\over d},&c_{12}^*=-{c_{12}\over d},\\
    c_{20}^*=-{c_{10}c_{21}-c_{20}c_{11} \over d},& c_{21}^*=+{c_{21}\over d},&c_{22}^*=-{c_{11}\over d}.
  \end{array}
\end{equation}
The determinant
\begin{equation}
  \label{eq:inv_det}
  d^*=c_{11}^*c_{22}^*-c_{12}^*c_{21}^*={1\over d}
\end{equation}
is different than zero and hence the inverse transformation is also affine.

The points $\myvec{x}$ of a line $\mathfrak{g}$ are mapped by an affine 
transformation to
the points $\myvec{x}^*$ of a line $\mathfrak{g}^*$. In fact, a line 
$\mathfrak{g}$ can be
represented by a linear equation ($g_1$, $g_2$ not both zero):
\begin{equation}
  \label{eq:line}
  g_0+g_1x_1+g_2x_2=0.
\end{equation}
The substitution of~\ref{eq:inv_c} into this equation results, because of the
invertibilty of the transformation, in an equation which is not trivially
satisfied, and thus again the locus of $\myvec{x}^*$ is a line.

It further follows from~\ref{eq:affine_trans} that affine transformations are
continuous mappings. That is, as $\myvec{x}$ tends to $\myvec{x}_0$ the 
corresponding $\myvec{x}^*$ tends to $\myvec{x}^*_0$.

Using M\"obius nets ~\cite{Mobius:1827} one can show that the last
two properties, whose dependence or independence remain to be decided, are
the definitive properties of affine transformations: {\em Affine mappings are
the only coordianate transformations (in the plane) which are without
exception bijective, continuous and map lines to lines.}

Affine transformations preserve parallelism, since parallel lines on the plane
are characterized by not having common points. Furthermore, given two 
affinities $\myvec{x}\to\myvec{x}^*$ and $\myvec{x}^*\to\myvec{x}^{**}$ , their
``product'' $\myvec{x}\to\myvec{x}^{**}$ is again an affinity, since all the
characteristics of an affinity hold. Naturally, this can be derived 
from~\eqref{eq:affine_trans}. Now a set of transformations that is closed under
the transformation product is called a group. {\em Hence the 
affinities~\eqref{eq:affine_trans} form a group.}

An affinity~\eqref{eq:affine_trans} can be decomposed into a ``displacement''
\begin{equation}
  \label{eq:decompose_displacement}
  \begin{array}{ll}
    x_1^*=c_{10}+x_1,& x_2^*=c_{20}+x_2
  \end{array}
\end{equation}
and a preceding ``homogenous affinity''
\begin{equation}
  \label{eq:decompose_homogenous}
  \begin{array}{ll}
    \begin{array}{l}
      x_1^*=c_{11}x_1+c_{12}x_2,\\
      x_2^*=c_{21}x_1+c_{22}x_2,
    \end{array}
    & c_{11}c_{22}-c_{12}c_{21}\neq 0.
  \end{array}
\end{equation}

Evidently the displacements or ``translations'' alone, and likewise the 
homogenuous affinities by themselves, constitute a group of transformations.
We thus have two ``subgroups'' of the general affine group considered above.

For our purposes an additional subgroup is particularly important: namely the
so called ``equiareal affinities''.  


\bibliographystyle{plain} \bibliography{blaschke}

\end{document}


