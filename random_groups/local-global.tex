\documentclass[11pt]{article} \usepackage{amssymb}
\usepackage{amsfonts} \usepackage{amsmath} \usepackage{bm}
\usepackage{latexsym} \usepackage{epsfig}

\setlength{\textwidth}{6.5 in} \setlength{\textheight}{8.25in}
\setlength{\oddsidemargin}{0in} \setlength{\topmargin}{0in}
\addtolength{\textheight}{.8in} \addtolength{\voffset}{-.5in}

\newtheorem{theorem}{Theorem}[section]
\newtheorem{lemma}[theorem]{Lemma}
\newtheorem{proposition}[theorem]{Proposition}
\newtheorem{corollary}[theorem]{Corollary}
\newtheorem{fact}[theorem]{Fact}
\newtheorem{definition}[theorem]{Definition}
\newtheorem{remark}[theorem]{Remark}
\newtheorem{conjecture}[theorem]{Conjecture}
\newtheorem{example}[theorem]{Example}
\newenvironment{proof}{\noindent \textbf{Proof:}}{$\Box$}

\newcommand{\ignore}[1]{}

\newcommand{\enote}[1]{} \newcommand{\knote}[1]{}
\newcommand{\rnote}[1]{}



% \newcommand{\enote}[1]{{\bf [[Elchanan:} {\emph{#1}}{\bf ]]}}
% \newcommand{\knote}[1]{{\bf [[Krzysztof:} {\emph{#1}}{\bf ]]}}
% \newcommand{\rnote}[1]{{\bf [[Ryan:} {\emph{#1}}{\bf ]]}}



\DeclareMathOperator{\Support}{Supp} \DeclareMathOperator{\Opt}{Opt}
\DeclareMathOperator{\Ordo}{\mathcal{O}}
\newcommand{\MaxkCSP}{\textsc{Max $k$-CSP}}
\newcommand{\MaxkCSPq}{\textsc{Max $k$-CSP$_{q}$}}
\newcommand{\MaxCSP}[1]{\textsc{Max CSP}(#1)} \renewcommand{\Pr}{{\bf
    P}} \renewcommand{\P}{{\bf P}} \newcommand{\Px}{\mathop{\bf P\/}}
\newcommand{\E}{{\bf E}} \newcommand{\Cov}{{\bf Cov}}
\newcommand{\Var}{{\bf Var}} \newcommand{\Varx}{\mathop{\bf Var\/}}

\newcommand{\bits}{\{-1,1\}}

\newcommand{\nsmaja}{\textstyle{\frac{2}{\pi}} \arcsin \rho}

\newcommand{\Inf}{\mathrm{Inf}} \newcommand{\I}{\mathrm{I}}
\newcommand{\J}{\mathrm{J}}

\newcommand{\eps}{\epsilon} \newcommand{\lam}{\lambda}

% \newcommand{\trunc}{\ell_{2,[-1,1]}}
\newcommand{\trunc}{\zeta} \newcommand{\truncprod}{\chi}

\newcommand{\N}{\mathbb N} \newcommand{\R}{\mathbb R}
\newcommand{\Z}{\mathbb Z} \newcommand{\CalE}{{\mathcal{E}}}
\newcommand{\CalC}{{\mathcal{C}}} \newcommand{\CalM}{{\mathcal{M}}}
\newcommand{\CalR}{{\mathcal{R}}} \newcommand{\CalS}{{\mathcal{S}}}
\newcommand{\CalV}{{\mathcal{V}}}
\newcommand{\CalX}{{\boldsymbol{\mathcal{X}}}}
\newcommand{\CalG}{{\boldsymbol{\mathcal{G}}}}
\newcommand{\CalH}{{\boldsymbol{\mathcal{H}}}}
\newcommand{\CalY}{{\boldsymbol{\mathcal{Y}}}}
\newcommand{\CalZ}{{\boldsymbol{\mathcal{Z}}}}
\newcommand{\CalW}{{\boldsymbol{\mathcal{W}}}}
\newcommand{\CalF}{{\mathcal{Z}}}
% \newcommand{\boldG}{{\boldsymbol G}}
% \newcommand{\boldQ}{{\boldsymbol Q}}
% \newcommand{\boldP}{{\boldsymbol P}}
% \newcommand{\boldR}{{\boldsymbol R}}
% \newcommand{\boldS}{{\boldsymbol S}}
% \newcommand{\boldX}{{\boldsymbol X}}
% \newcommand{\boldB}{{\boldsymbol B}}
% \newcommand{\boldY}{{\boldsymbol Y}}
% \newcommand{\boldZ}{{\boldsymbol Z}}
% \newcommand{\boldV}{{\boldsymbol V}}
\newcommand{\boldi}{{\boldsymbol i}} \newcommand{\boldj}{{\boldsymbol
    j}} \newcommand{\boldk}{{\boldsymbol k}}
\newcommand{\boldr}{{\boldsymbol r}}
\newcommand{\boldsigma}{{\boldsymbol \sigma}}
\newcommand{\boldupsilon}{{\boldsymbol \upsilon}}
\newcommand{\hone}{{\boldsymbol{H1}}}
\newcommand{\htwo}{\boldsymbol{H2}}
\newcommand{\hthree}{\boldsymbol{H3}}
\newcommand{\hfour}{\boldsymbol{H4}}


\newcommand{\sgn}{\mathrm{sgn}} \newcommand{\Maj}{\mathrm{Maj}}
\newcommand{\Acyc}{\mathrm{Acyc}}
\newcommand{\UniqMax}{\mathrm{UniqMax}}
\newcommand{\Thr}{\mathrm{Thr}} \newcommand{\littlesum}{{\textstyle
    \sum}}

\newcommand{\half}{{\textstyle \frac12}}
\newcommand{\third}{{\textstyle \frac13}}
\newcommand{\fourth}{{\textstyle \frac14}}

\newcommand{\Stab}{\mathbb{S}}
\newcommand{\StabThr}[2]{\Gamma_{#1}(#2)}
\newcommand{\StabThrmin}[2]{{\underline{\Gamma}}_{#1}(#2)}
\newcommand{\StabThrmax}[2]{{\overline{\Gamma}}_{#1}(#2)}
\newcommand{\TestFcn}{\Psi}

\renewcommand{\phi}{\varphi}

\begin{document}
\title{Random Groups - The Gromov Papasoglu Local Global Theorem}

 \author{Omer Tamuz, 035696574}
\maketitle

\begin{enumerate}
\item Definition of group presentation
  \begin{itemize}
  \item $G=<S|R>$
  \item $R$ closed under cyclic permutation, inverse.
  \item Why {\bf normal} group generated by relators.
  \item Notation: $w\in F$. $l(w)$: length of reduced presentation.
    $N(R)$: normal subgroup generated by $R$.
    $$ l(w)=\min_n\left\{n\mbox{ s.t. } w=\sum_{i=1}^nu_i^{-1}r_iu_i\right\}$$
  \end{itemize}
\item Van Kampen diagrams
  \begin{itemize}
  \item Example: free group (only edges), $abc$ (face, face and edge),
    $aba^{-1}b^{-1}$ (concatenating faces)
  \item Concatenating faces
  \item Van Kampen theorem: boundaries are all words equivalent to $e$.
  \end{itemize} 
\item Word hyperbolicity
  \begin{enumerate}
  \item Previous hyperbolicity definition: thin triangles
    \begin{itemize}
    \item Thin triangles means little area for large perimeter.
    \item Local condition (negative curvature) on manifold means global 
      property.
    \end{itemize}
  \item Area of a van Kampen diagram is the same as minimum number of relators 
    in proof that word is equal to $e$.
  \item Generator invariant - assume relators are of length three. Given
    relator $abcd$ add generator $x$ and substitute relator with $xb^{-1}a^{-1}$ and
    $xcd$.
  \item definition: $\exists C:\: A(w)\leq Cl(w)$. In diagrams:
    $\exists C:\: \partial D\leq C|D|$.
  \item Word hyperbolicity equivalent to hyperbolicity - non-trivial.
  \end{enumerate}
  \item Local-global theorem statement
    \begin{itemize}
    \item In non-hyperbolic space $A\sim l^2$. 
    \item Wishful thinking: if for balls of size $< 1,000,000$
      we knew that $S<10^{-10} r^2$ then space is hyperbolic. 
      then space is hyperbolic. Can't work: doesn't work for small radii,
      1,000,000 can't be a global constant.
    \item Same for van Kampen diagrams: $A(w)<10^{-10}l(w)$.
    \item Correct formulation: Assume there exists an integer $K>0$ such that
      for all $w\in N(R)$ for which $\half K^2\leq A(w)\leq 240K^2$ it holds that
      $A(w)<{1\over 20,000}l(w)^2$. Then $A(w)\leq Kl(w)$ whenever $w\in N(R)$ and $A(w)>K^2$. 
      (It also implies
      $A(w)\leq K^2l(w)$ for all $w\in N(R)$).    
    \end{itemize}
  \item Local-global theorem proof
    \begin{itemize}
    \item The group $X$: all $w\in N(R)$ such that $A(w)>K^2$ and $A(w)>Kl(w)$.
    \item $w$: of all the words in $X$ with minimal area, the one with 
      minimal length. $D$ its diagram.
    \item $w$ can't be far from $A(w)=Kl(w)$: 
      $A(w)\leq Kl(w)+K+1$: remove one boundary face, $D'\not \in X$,
      $A(D') = A(D)-1\leq K(l(w)+1)$ since otherwise $A(D)=K^2$ (can't be - chart).
    \item $l(w)\geq 238K$. Try and prove theorem by finding sub-diagram of $D$ that
      is in the {\em forbidden zone}.
    \item If $D$ can be separated into two large sub diagrams then each
      must have $A(D')<Kl'$, but their sum must equal $A(D)$, so $A(D')$ has
      to be close to $Kl'$.
      \begin{lemma}
        Let $S$ separate $D_1$ from $D_2$, $A(D_2)>K^2$, $l=l(D_1\cap w)$. 
        Then:
        \begin{equation}
          A(D_1)\geq Kl-Kl(S).  
        \end{equation}
      \end{lemma}
      \begin{proof}
        $A(D_1)=A(D)-A(D_2)\geq Kl(w)-A(D_2)$ and
        by $w$'s minimality $A(D_2)\leq K(l(w)-l-l(S))$.
      \end{proof}
    \item $D$ might be very big. So we cut it into pieces of size of order 
      $K^2$:
      Pick $P_1$ to $P_n$ distanced $20K$ 
      (except the last and first, between $20K$
      and $40K$).
    \item $n\geq {l(w)\over 20K}-1$. $n$ is at least 10, since $l(w)$ is at least $238K$.
    \item $B_r(Pi)$: ball of all faces within radius $r$ around $P_i$. Consider
      balls of radius $6K$.
    \item 2 cases:
      \begin{enumerate}
      \item $i\neq j\to B_{6K}(Pi)\cap B_{6K}(P_j)=\emptyset$ and 
        $\forall i:\:diam_w(B_{6K}(P_i)\cap w)\leq 20K$.
        \begin{itemize}
        \item Second condition implies $\overline{B_{6K}(P_i)}$ has one 
          component.
        \item $S_r$ separates $B_r(P_i)$ from the other component.
        \item Three sub-cases:
          \begin{enumerate}
          \item $l(S_r(P_i))<K$ for some $i$ and some $K \leq r \leq 2K$. 
            \begin{itemize}
            \item $w'$ word of $B_r(P_i)$ and $D'$ its length.
            \item $2K\leq l(w') \leq 21K$, because $r$ is at least $K$ and 
              ball does not
              intersect next ball. 
            \item By lemma: $A(D')\geq K^2$.
            \item Then by hypothesis on $w$ we have 
              $A(D')\leq Kl(w')\leq 21K^2$.
            \item Since $l(w')\leq 21K$ then $l(w')^2/20,000<K^2$ and so
              $A(D')>l(w')^2/20,000$.
            \item $w'$ in {\em forbidden zone}.
            \end{itemize}
          \item  The previous assumption is false and 
            $l(S_r(P_i))<80K$ for some $i$ and some $2K \leq r \leq 3K$. 
            \begin{itemize}
            \item $l(w')\leq 80K+10K+10K=100K$. Hence  $l(w')^2/100^2 \leq K^2$.
            \item Previous assumption is false, so $l(S_t(P_i))>K$ for 
              $K \leq t \leq 2K -1$.
            \item Each cell in $B_t(P_i)$ intersects $S_t(P_i)$ in at most
              two places, so $A(D')>K^2/2$.
            \item Since $K^2\geq l(w')^2/100^2$ then $A(D') \geq l(w')/20,000$.
            \item By hypothesis on $w$ we have $A(D') \leq Kl(w') \leq 100K^2$.
            \item $w'$ is in the {\em forbidden zone}.
            \end{itemize}
          \item The previous two assumptions are false.
            \begin{itemize}
            \item $A(B_{3K}(P_i)) \geq 40K^2$ by summation on $l(S)/2$.
            \item 
              $$A(D)\geq \sum_i 40K^2\geq \left({l(w)\over 20K}-1\right)40K^2=2Kl(w)-40K^2>Kl(w)+K+1$$
              contradiction, (last inequality since $l(w)>238K$).
            \end{itemize}
          \end{enumerate}
        \end{itemize}
      \item $B_{6K}(Pi)\cap B_{6K}(P_j)\neq\emptyset$ for some 
        $i\neq j$  or $diam_w(B_{6K}(P_i)\cap w) > 20K$ for some $i$.
        \begin{itemize}
        \item There exists an arc $S$ with $l(S)<12K$ separating $D$ 
          into two regions $D_1,D_2$ such that $l(D_i\cap w) \geq 20K$.
        \item Choose $S$ such that $\min_il(D_i\cap w)$ is minimized. Let $D'$ 
          be the smaller of $D_1,D_2$. Let $w'$ be its boundary.
        \item $m\leq {l(D'\cap w)\over 20K}-3$. $m\geq 1$ since 
          $l(D'\cap w) \geq 20K$.
        \item Let $Q=\{P_i\ldots, P_{i+m}\}$ are the points in $D'$.
        \item By minimality of $S$ we have 
          $B_{3K}(P_{i+s})\cap B_{3K}(P_{i+t})=\emptyset$ for $1\leq s < t \leq m$.
        \item For same reason at most two of the $3K$ balls intersect
          $S$. Hence $diam_{w'}(B_{3K}(P_i)\cap w') \leq 20K$.
        \item Consider only the elements in $Q$ for which this is true.
          Now argue as in the first case, with three sub-cases:
          \begin{enumerate}
          \item In first sub-case argument is identical to the previous first 
            sub-case.
          \item Same for second.
          \item Third, we prove that $D'$ is in the {\em forbidden zone}:
            \begin{itemize}
            \item 
              \begin{equation}
                \label{eq:dprime_area}
                A(D')\geq \sum_i 40K^2\geq \left({l(w')\over 20K}-6\right)40K^2
              \end{equation}
            \item If $A(D')>K^2$ then by minimality of $w$ we have 
              $A(D')<Kl(w')$. 
            \item Else the lemma applies and 
              $A(D')\geq K \cdot 20K - K \cdot 8K=12K^2$ and again 
              $A(D')<Kl(w')$. 
            \item Combining $A(D')<Kl(w')$ and \eqref{eq:dprime_area} we have
              $l(w')\leq 240K$ and so $A(D')\leq 240K$.
            \item $A(D')>l(w')/20,000$ and so $w'$ is in the 
              {\em forbidden zone}.
            \end{itemize}
          \end{enumerate}
          
        \end{itemize}
      \end{enumerate}
    \end{itemize}
\end{enumerate}
\begin{figure}[htp]
  \label{fig:axes}
  \centering
  \includegraphics{al_chart.pdf}    
\end{figure}
\end{document}




