\documentclass[11pt]{article} \usepackage{amssymb}
\usepackage{amsfonts} \usepackage{amsmath} \usepackage{amsthm} \usepackage{bm}
\usepackage{latexsym} \usepackage{epsfig}

\setlength{\textwidth}{6.5 in} \setlength{\textheight}{8.25in}
\setlength{\oddsidemargin}{0in} \setlength{\topmargin}{0in}
\addtolength{\textheight}{.8in} \addtolength{\voffset}{-.5in}

\newtheorem*{theorem*}{Theorem}
\newtheorem{theorem}{Theorem}[section]
\newtheorem{lemma}[theorem]{Lemma}
\newtheorem*{proposition*}{Proposition}

\newtheorem{proposition}[theorem]{Proposition}
\newtheorem{corollary}[theorem]{Corollary}
\newtheorem{fact}[theorem]{Fact}
\newtheorem{definition}[theorem]{Definition}
\newtheorem{remark}[theorem]{Remark}
\newtheorem{conjecture}[theorem]{Conjecture}
\newtheorem{example}[theorem]{Example}
%\newenvironment{proof}{\noindent \textbf{Proof:}}{$\Box$}

\newcommand{\ignore}[1]{}

\newcommand{\enote}[1]{} \newcommand{\knote}[1]{}
\newcommand{\rnote}[1]{}


\DeclareMathOperator{\Support}{Supp} \DeclareMathOperator{\Opt}{Opt}
\DeclareMathOperator{\Ordo}{\mathcal{O}}
\newcommand{\MaxkCSP}{\textsc{Max $k$-CSP}}
\newcommand{\MaxkCSPq}{\textsc{Max $k$-CSP$_{q}$}}
\newcommand{\MaxCSP}[1]{\textsc{Max CSP}(#1)} \renewcommand{\Pr}{{\bf
    P}} \renewcommand{\P}{{\bf P}} \newcommand{\Px}{\mathop{\bf P\/}}
\newcommand{\E}{{\bf E}} \newcommand{\Cov}{{\bf Cov}}
\newcommand{\Var}{{\bf Var}} \newcommand{\Varx}{\mathop{\bf Var\/}}

\newcommand{\bits}{\{-1,1\}}

\newcommand{\nsmaja}{\textstyle{\frac{2}{\pi}} \arcsin \rho}

\newcommand{\Inf}{\mathrm{Inf}} \newcommand{\I}{\mathrm{I}}
\newcommand{\J}{\mathrm{J}}

\newcommand{\eps}{\epsilon} \newcommand{\lam}{\lambda}

% \newcommand{\trunc}{\ell_{2,[-1,1]}}
\newcommand{\trunc}{\zeta} \newcommand{\truncprod}{\chi}

\newcommand{\N}{\mathbb N} \newcommand{\R}{\mathbb R}
\newcommand{\Z}{\mathbb Z} \newcommand{\CalE}{{\mathcal{E}}}
\newcommand{\CalC}{{\mathcal{C}}} \newcommand{\CalM}{{\mathcal{M}}}
\newcommand{\CalR}{{\mathcal{R}}} \newcommand{\CalS}{{\mathcal{S}}}
\newcommand{\CalV}{{\mathcal{V}}}
\newcommand{\CalX}{{\boldsymbol{\mathcal{X}}}}
\newcommand{\CalG}{{\boldsymbol{\mathcal{G}}}}
\newcommand{\CalH}{{\boldsymbol{\mathcal{H}}}}
\newcommand{\CalY}{{\boldsymbol{\mathcal{Y}}}}
\newcommand{\CalZ}{{\boldsymbol{\mathcal{Z}}}}
\newcommand{\CalW}{{\boldsymbol{\mathcal{W}}}}
\newcommand{\CalF}{{\mathcal{Z}}}
% \newcommand{\boldG}{{\boldsymbol G}}
% \newcommand{\boldQ}{{\boldsymbol Q}}
% \newcommand{\boldP}{{\boldsymbol P}}
% \newcommand{\boldR}{{\boldsymbol R}}
% \newcommand{\boldS}{{\boldsymbol S}}
% \newcommand{\boldX}{{\boldsymbol X}}
% \newcommand{\boldB}{{\boldsymbol B}}
% \newcommand{\boldY}{{\boldsymbol Y}}
% \newcommand{\boldZ}{{\boldsymbol Z}}
% \newcommand{\boldV}{{\boldsymbol V}}
\newcommand{\boldi}{{\boldsymbol i}} \newcommand{\boldj}{{\boldsymbol
    j}} \newcommand{\boldk}{{\boldsymbol k}}
\newcommand{\boldr}{{\boldsymbol r}}
\newcommand{\boldsigma}{{\boldsymbol \sigma}}
\newcommand{\boldupsilon}{{\boldsymbol \upsilon}}
\newcommand{\hone}{{\boldsymbol{H1}}}
\newcommand{\htwo}{\boldsymbol{H2}}
\newcommand{\hthree}{\boldsymbol{H3}}
\newcommand{\hfour}{\boldsymbol{H4}}


\newcommand{\sgn}{\mathrm{sgn}} \newcommand{\Maj}{\mathrm{Maj}}
\newcommand{\Acyc}{\mathrm{Acyc}}
\newcommand{\UniqMax}{\mathrm{UniqMax}}
\newcommand{\Thr}{\mathrm{Thr}} \newcommand{\littlesum}{{\textstyle
    \sum}}

\newcommand{\half}{{\textstyle \frac12}}
\newcommand{\third}{{\textstyle \frac13}}
\newcommand{\fourth}{{\textstyle \frac14}}
\newcommand{\fifth}{{\textstyle \frac15}}

\newcommand{\Stab}{\mathbb{S}}
\newcommand{\StabThr}[2]{\Gamma_{#1}(#2)}
\newcommand{\StabThrmin}[2]{{\underline{\Gamma}}_{#1}(#2)}
\newcommand{\StabThrmax}[2]{{\overline{\Gamma}}_{#1}(#2)}
\newcommand{\TestFcn}{\Psi}

\renewcommand{\phi}{\varphi}

\begin{document}
\title{Combinatorial Statistics - Homework Set Solution}

\date{\today}
\maketitle
\subsection{Reconstruction}
\begin{itemize}
\item
{\em Find an estimation procedure which is guaranteed to estimate $s$ with probability at least $1-\delta$ using at most 
$C_1 \log n + C_2 \log (1/\delta)$ samples. What are the values of $C_1$ and $C_2$?  }

Given $k$ samples, the probability that the majority of  the $X_1(j)$'s are 
different than $s_1$
is lower than $2\exp(-2k/16)$, by the large deviation result above. 
The probability of an error in at least one of the $s_i$'s is then at most
$p=1-(1-2\exp(-2k/16))^n<2n\exp(-2k/16)$. Hence if 
$p<\delta$ then $2n\exp(-2k/16)<\delta$, or $k>8\log n+8\log(1/\delta)+8\log 2$.

\item
{\em Show that there exist constants $C_3$ and $C_4$ such that for all $n$ and $\delta < 1/4$ and for all estimation algorithms 
which estimate $s$ with probability at least $1-\delta$ it holds that $k \geq C_3 \log n + C_4 \log (1/\delta)$. 
What are the values of $C_3$ and $C_4$ you have obtained? }

Let $A$ be an estimation algorithm. Assume $A$'s estimate for $s_i$ depends
only on the $X_i(j)$'s, otherwise $A$ is certainly suboptimal. Now when
$s_1=1$, there must exist a set $X_1(j)$ such that $A$'s 
estimate of $s_1$ is wrong (otherwise $A$ would never estimate $s_1$ 
correctly when $s_1=0$). This probability for this outcome is at most $4^{-k}$,
and the same holds for the rest of the $s_i$'s. Hence the probability of an
error in at least one of the $s_i$'s is 
$1-\left(1-4^{-k}\right)^n<n4^{-k}$. For this to be smaller than $\delta$ we must have
$k>4\log n+4\log(1/\delta)$.

\end{itemize}

\subsection{Group Testing}
\begin{itemize}
\item
{\em Can you isolate at most $0.1\%$ of the population that contains all the carriers with probability of error at most $10^{-7}$ 
and cost at most $5 \times 10^{7}\$$?}

Whenever testing a group of people, apply the following procedure: test twice,
and if both tests have the same result, return that. If not, test a third time
and return the majority result.

Now divide the population into 200 groups of equal size, and test each group 
using the procedure above. Remove from the population any group that tested 
negative for infection. Apply repeatedly to the remaining population, until 
$10^5$ (=$0.1\%$) are left.

Assume first that no group is incorrectly identified. Then at least 100 groups 
are recognized as healthy at each iteration, and hence the size of the 
potentially infected population decreases by a factor of at least two at each
iteration. The number of iterations required is therefore at most
$\lceil\log_21000\rceil=10$. At each iteration $2\cdot200$ tests are performed,
and so the total cost is at most $10\cdot400\cdot\$10^4=\$4\cdot10^7$.

The probability that a group is falsely removed (i.e. its members are marked
as uninfected even though they are infected) is $10^{-2\cdot6}=10^{-12}$. 
Therefore, by union bound, the probability that no group (of the 2000) tested
is falsely removed is at most $2000\cdot 10^{-12}<10^{-7}$.

{\bf Note.} A more cost efficient procedure exists: repeatedly choose a group of
size one hundredth of the population and remove it if it twice tests to be
not infected. This procedure is, however, slightly more difficult to analyze.


\item
{\em Can you find all of the infected individuals with probability of error 
$10^{-7}$
and cost at most $2 \times 10^{8}\$$?}

Use the procedure above, until the remainning population is of size 100. The
number of iterations needed is $\lceil\log_21,000,000\rceil=20$, and so the cost will be
at most $20\cdot 400 \cdot \$10^4=\$8\cdot 10^7$.

The probability of error is at most $4000\cdot 10^{-12}<10^{-7}$.
\end{itemize}






\end{document}


















