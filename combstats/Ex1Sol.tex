\documentclass[11pt]{article} \usepackage{amssymb}
\usepackage{amsfonts} \usepackage{amsmath} \usepackage{amsthm} \usepackage{bm}
\usepackage{latexsym} \usepackage{epsfig}

\setlength{\textwidth}{6.5 in} \setlength{\textheight}{8.25in}
\setlength{\oddsidemargin}{0in} \setlength{\topmargin}{0in}
\addtolength{\textheight}{.8in} \addtolength{\voffset}{-.5in}

\newtheorem*{theorem*}{Theorem}
\newtheorem{theorem}{Theorem}[section]
\newtheorem{lemma}[theorem]{Lemma}
\newtheorem*{proposition*}{Proposition}

\newtheorem{proposition}[theorem]{Proposition}
\newtheorem{corollary}[theorem]{Corollary}
\newtheorem{fact}[theorem]{Fact}
\newtheorem{definition}[theorem]{Definition}
\newtheorem{remark}[theorem]{Remark}
\newtheorem{conjecture}[theorem]{Conjecture}
\newtheorem{example}[theorem]{Example}
%\newenvironment{proof}{\noindent \textbf{Proof:}}{$\Box$}

\newcommand{\ignore}[1]{}

\newcommand{\enote}[1]{} \newcommand{\knote}[1]{}
\newcommand{\rnote}[1]{}


\DeclareMathOperator{\Support}{Supp} \DeclareMathOperator{\Opt}{Opt}
\DeclareMathOperator{\Ordo}{\mathcal{O}}
\newcommand{\MaxkCSP}{\textsc{Max $k$-CSP}}
\newcommand{\MaxkCSPq}{\textsc{Max $k$-CSP$_{q}$}}
\newcommand{\MaxCSP}[1]{\textsc{Max CSP}(#1)} \renewcommand{\Pr}{{\bf
    P}} \renewcommand{\P}{{\bf P}} \newcommand{\Px}{\mathop{\bf P\/}}
\newcommand{\E}{{\bf E}} \newcommand{\Cov}{{\bf Cov}}
\newcommand{\Var}{{\bf Var}} \newcommand{\Varx}{\mathop{\bf Var\/}}

\newcommand{\bits}{\{-1,1\}}

\newcommand{\nsmaja}{\textstyle{\frac{2}{\pi}} \arcsin \rho}

\newcommand{\Inf}{\mathrm{Inf}} \newcommand{\I}{\mathrm{I}}
\newcommand{\J}{\mathrm{J}}

\newcommand{\eps}{\epsilon} \newcommand{\lam}{\lambda}

% \newcommand{\trunc}{\ell_{2,[-1,1]}}
\newcommand{\trunc}{\zeta} \newcommand{\truncprod}{\chi}

\newcommand{\N}{\mathbb N} \newcommand{\R}{\mathbb R}
\newcommand{\Z}{\mathbb Z} \newcommand{\CalE}{{\mathcal{E}}}
\newcommand{\CalC}{{\mathcal{C}}} \newcommand{\CalM}{{\mathcal{M}}}
\newcommand{\CalR}{{\mathcal{R}}} \newcommand{\CalS}{{\mathcal{S}}}
\newcommand{\CalV}{{\mathcal{V}}}
\newcommand{\CalX}{{\boldsymbol{\mathcal{X}}}}
\newcommand{\CalG}{{\boldsymbol{\mathcal{G}}}}
\newcommand{\CalH}{{\boldsymbol{\mathcal{H}}}}
\newcommand{\CalY}{{\boldsymbol{\mathcal{Y}}}}
\newcommand{\CalZ}{{\boldsymbol{\mathcal{Z}}}}
\newcommand{\CalW}{{\boldsymbol{\mathcal{W}}}}
\newcommand{\CalF}{{\mathcal{Z}}}
% \newcommand{\boldG}{{\boldsymbol G}}
% \newcommand{\boldQ}{{\boldsymbol Q}}
% \newcommand{\boldP}{{\boldsymbol P}}
% \newcommand{\boldR}{{\boldsymbol R}}
% \newcommand{\boldS}{{\boldsymbol S}}
% \newcommand{\boldX}{{\boldsymbol X}}
% \newcommand{\boldB}{{\boldsymbol B}}
% \newcommand{\boldY}{{\boldsymbol Y}}
% \newcommand{\boldZ}{{\boldsymbol Z}}
% \newcommand{\boldV}{{\boldsymbol V}}
\newcommand{\boldi}{{\boldsymbol i}} \newcommand{\boldj}{{\boldsymbol
    j}} \newcommand{\boldk}{{\boldsymbol k}}
\newcommand{\boldr}{{\boldsymbol r}}
\newcommand{\boldsigma}{{\boldsymbol \sigma}}
\newcommand{\boldupsilon}{{\boldsymbol \upsilon}}
\newcommand{\hone}{{\boldsymbol{H1}}}
\newcommand{\htwo}{\boldsymbol{H2}}
\newcommand{\hthree}{\boldsymbol{H3}}
\newcommand{\hfour}{\boldsymbol{H4}}


\newcommand{\sgn}{\mathrm{sgn}} \newcommand{\Maj}{\mathrm{Maj}}
\newcommand{\Acyc}{\mathrm{Acyc}}
\newcommand{\UniqMax}{\mathrm{UniqMax}}
\newcommand{\Thr}{\mathrm{Thr}} \newcommand{\littlesum}{{\textstyle
    \sum}}

\newcommand{\half}{{\textstyle \frac12}}
\newcommand{\third}{{\textstyle \frac13}}
\newcommand{\fourth}{{\textstyle \frac14}}
\newcommand{\fifth}{{\textstyle \frac15}}

\newcommand{\Stab}{\mathbb{S}}
\newcommand{\StabThr}[2]{\Gamma_{#1}(#2)}
\newcommand{\StabThrmin}[2]{{\underline{\Gamma}}_{#1}(#2)}
\newcommand{\StabThrmax}[2]{{\overline{\Gamma}}_{#1}(#2)}
\newcommand{\TestFcn}{\Psi}

\renewcommand{\phi}{\varphi}

\begin{document}
\title{Combinatorial Statistics - Homework Set Solution}

\date{\today}
\maketitle
\subsection{Reconstruction}
\begin{itemize}
\item
{\em Find an estimation procedure which is guaranteed to estimate $s$ with probability at least $1-\delta$ using at most
$C (\log n + \log (1/\delta))$ samples. What value of $C$ have you have obtained?
}

Given $k$ samples, the probability that the majority of  the $X_i(1)$'s are 
different than $s_1$ is equal to the probability that their average is more than
$1/4$ away from their expectation, and so
is lower than $2\exp(-2k/4^2)$, by the given large deviation result. 
The probability of an error in at least one of the $s_j$'s is then at most
$p=1-(1-2\exp(-k/8))^n<2n\exp(-k/8)$. Hence if 
$p<\delta$ then $2n\exp(-k/8)<\delta$, or $k>8\left(\log n+\log(1/\delta)\right)+8\log 2$.

\item
{\em Show that there exist a constant $C'$ such that for all $n$ and $\delta < 1/4$ and for all estimation algorithms of $s$ with
probability of success at least $1-\delta$ it holds that $k \geq C'( \log n + \log (1/\delta))$.
What value of $C'$ have you obtained?
 }

Let $k(n,\delta)$ be such that for any algorithm $A$ that estimates $s$, there
exists a distribution over the possible values $s$ such that $A$ requires
$k(n,\delta)$ samples to estimate $s$ with probability $1-\delta$. Then in particular any 
algorithm requires $k(n,\delta)$ samples when $S$ is distributed by the uniform 
distribution. We will show that the majority algoritm $M$, with a given number
of samples, succeeds with higher probability then any other algorithm,
when $S$ is distributed
uniformly. Then, since $M$ {\bf has equal probability of success for any 
distribution over $S$}, we'll conclude that the number of samples $M$ requires
is a lower bound of the number of samples required by any algorithm. {\em
Note: This point was not made in any of the solutions.}

Let $A$ be an algorithm that differs from $M$ for at least one set of values of
the $X_i(j)$'s. Call this set $z=\{z_i(j)\}$, and assume WLOG that $A(z)_1\neq M(z)_1$. 
Define an algorithm $B$ by 
$B(x)=A(x)$ for $x\neq z$,  $B(z)_1=M(z)=1- A(z)_1$ and for $j>1$ again 
$B(z)_j=A(z)_j$. Then, assuming $S$ is distributed uniformly:
\begin{eqnarray*}
\P[\mbox{B succeeds}]-\P[\mbox{A succeeds}] &=& 
\P[s=B]-\P[s=A] 
\\ &=& \sum_x\P[x]\P[s=B(x)|x]-\sum_x\P[x]\P[s=A(x)|x]
\\ &=& P[z]\P[s=B(z)|z]-\P[z]\P[s=A(z)|z]
\end{eqnarray*}
Since the different $s_j$'s are independent we can write:
\begin{eqnarray*}
 &=& \P[z]\left(\prod_j\P[s_j=B(z)_j|z(j)]-\prod_j\P[s_j=A(z)_j|z(j)]\right)
\\ &=& \P[z]\prod_{j>1}\P[s_j=B(z)_j|z(j)]\cdot\left(\P[s_1=M(z)_1|z(1)]-\P[s_1=1-M(z)_1|z(1)]\right)  
\end{eqnarray*}
Now, because both $S$ and $X$ are distributed uniformly then
 $\P[s_1=M(z)_1|z(1)]=\P[z(1)|s_1=M(z)_1]$. Since by the defintion of $\P[z|s]$ it
is easy to see that $\P[z(1)|s_1=M(z)_1]>\P[z(1)|s_1=1-M(z)_1]$, then
the probability that $B$ succeeds is greater than that of $A$, and $M$ is
the optimal algorithm.

To lower bound the sample size $M$ needs, we note that when $s_j=1$, $M$ 
is certainly wrong when $x_i(j)=0$ for all $i$.
This probability for this outcome is $4^{-k}$,
and so the probability of an
error in at least one of the $s_i$'s is 
$1-\left(1-4^{-k}\right)^n<n4^{-k}$. For this to be smaller than $\delta$ we must have
$k>4\left(\log n+\log(1/\delta)\right)$.
\end{itemize}

\subsection{Group Testing}
Whenever testing a group of people, apply the following procedure: test five
times,
and return the majority result. Call this the ``quintuple test''. The
probability that the quintuple test fails is less than $10^{-16}$.

\begin{itemize}
\item
{\em Devise a procedure which would result in a division of the population to two sets of size $5 \times 10^7$ where all the infected individuals belong to the second set and further: 1) the probability of error of the
procedure is at most $10^{-6}$? 2) the cost of the test is at most $\$  10^7 $ ?
}

Now divide the population into 200 groups of equal size, and test each group 
using the procedure above. Assuming each quintuple test succeeds, we have at
most 100 groups with infected individuals and the population is divided 
appropriately. The probability that all tests succeed is more than 
$1-10^{13}$, and the cost was $10^7$.

\item
{\em Can you isolate at most $0.1\%$ of the population that contains all the carriers with probability of error at most $10^{-6}$
and cost at most $\$ 10^{8}$?
}

Apply the above procedure repeatedly to the part of the population that includes
infected individuals. After ten iterations the desired division will be
achieved, at cost $10^8$ and probability of success higher than $1-10^{-11}$.

\item
{\em Can you find all of the infected individuals with probability of error
$10^{-7}$
and cost at most $\$2 \times 10^{8}$?
}

As above, apply the procedure 20 times. The cost is $2\cdot 10^8$ and 
the probability of success is at least $1-10^9$.

\end{itemize}






\end{document}


















