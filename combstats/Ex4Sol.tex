\documentclass[11pt]{article} \usepackage{amssymb}
\usepackage{amsfonts} \usepackage{amsmath} \usepackage{amsthm} \usepackage{bm}
\usepackage{latexsym} \usepackage{epsfig}

\setlength{\textwidth}{6.5 in} \setlength{\textheight}{8.25in}
\setlength{\oddsidemargin}{0in} \setlength{\topmargin}{0in}
\addtolength{\textheight}{.8in} \addtolength{\voffset}{-.5in}

\newtheorem*{theorem*}{Theorem}
\newtheorem{theorem}{Theorem}[section]
\newtheorem{lemma}[theorem]{Lemma}
\newtheorem*{proposition*}{Proposition}

\newtheorem{proposition}[theorem]{Proposition}
\newtheorem{corollary}[theorem]{Corollary}
\newtheorem{fact}[theorem]{Fact}
\newtheorem{definition}[theorem]{Definition}
\newtheorem{remark}[theorem]{Remark}
\newtheorem{conjecture}[theorem]{Conjecture}
\newtheorem{example}[theorem]{Example}
%\newenvironment{proof}{\noindent \textbf{Proof:}}{$\Box$}

\newcommand{\ignore}[1]{}

\newcommand{\enote}[1]{} \newcommand{\knote}[1]{}
\newcommand{\rnote}[1]{}


\DeclareMathOperator{\Support}{Supp} \DeclareMathOperator{\Opt}{Opt}
\DeclareMathOperator{\Ordo}{\mathcal{O}}
\newcommand{\MaxkCSP}{\textsc{Max $k$-CSP}}
\newcommand{\MaxkCSPq}{\textsc{Max $k$-CSP$_{q}$}}
\newcommand{\MaxCSP}[1]{\textsc{Max CSP}(#1)} \renewcommand{\Pr}{{\bf
    P}} \renewcommand{\P}{{\bf P}} \newcommand{\Px}{\mathop{\bf P\/}}
\newcommand{\E}{{\bf E}} \newcommand{\Cov}{{\bf Cov}}
\newcommand{\Var}{{\bf Var}} \newcommand{\Varx}{\mathop{\bf Var\/}}

\newcommand{\bits}{\{-1,1\}}

\newcommand{\nsmaja}{\textstyle{\frac{2}{\pi}} \arcsin \rho}

\newcommand{\Inf}{\mathrm{Inf}} \newcommand{\I}{\mathrm{I}}
\newcommand{\J}{\mathrm{J}}

\newcommand{\eps}{\epsilon} \newcommand{\lam}{\lambda}

% \newcommand{\trunc}{\ell_{2,[-1,1]}}
\newcommand{\trunc}{\zeta} \newcommand{\truncprod}{\chi}

\newcommand{\N}{\mathbb N} \newcommand{\R}{\mathbb R}
\newcommand{\Z}{\mathbb Z} \newcommand{\CalE}{{\mathcal{E}}}
\newcommand{\CalC}{{\mathcal{C}}} \newcommand{\CalM}{{\mathcal{M}}}
\newcommand{\CalR}{{\mathcal{R}}} \newcommand{\CalS}{{\mathcal{S}}}
\newcommand{\CalV}{{\mathcal{V}}}
\newcommand{\CalX}{{\boldsymbol{\mathcal{X}}}}
\newcommand{\CalG}{{\boldsymbol{\mathcal{G}}}}
\newcommand{\CalH}{{\boldsymbol{\mathcal{H}}}}
\newcommand{\CalY}{{\boldsymbol{\mathcal{Y}}}}
\newcommand{\CalZ}{{\boldsymbol{\mathcal{Z}}}}
\newcommand{\CalW}{{\boldsymbol{\mathcal{W}}}}
\newcommand{\CalF}{{\mathcal{Z}}}
% \newcommand{\boldG}{{\boldsymbol G}}
% \newcommand{\boldQ}{{\boldsymbol Q}}
% \newcommand{\boldP}{{\boldsymbol P}}
% \newcommand{\boldR}{{\boldsymbol R}}
% \newcommand{\boldS}{{\boldsymbol S}}
% \newcommand{\boldX}{{\boldsymbol X}}
% \newcommand{\boldB}{{\boldsymbol B}}
% \newcommand{\boldY}{{\boldsymbol Y}}
% \newcommand{\boldZ}{{\boldsymbol Z}}
% \newcommand{\boldV}{{\boldsymbol V}}
\newcommand{\boldi}{{\boldsymbol i}} \newcommand{\boldj}{{\boldsymbol
    j}} \newcommand{\boldk}{{\boldsymbol k}}
\newcommand{\boldr}{{\boldsymbol r}}
\newcommand{\boldsigma}{{\boldsymbol \sigma}}
\newcommand{\boldupsilon}{{\boldsymbol \upsilon}}
\newcommand{\hone}{{\boldsymbol{H1}}}
\newcommand{\htwo}{\boldsymbol{H2}}
\newcommand{\hthree}{\boldsymbol{H3}}
\newcommand{\hfour}{\boldsymbol{H4}}


\newcommand{\sgn}{\mathrm{sgn}} \newcommand{\Maj}{\mathrm{Maj}}
\newcommand{\Acyc}{\mathrm{Acyc}}
\newcommand{\UniqMax}{\mathrm{UniqMax}}
\newcommand{\Thr}{\mathrm{Thr}} \newcommand{\littlesum}{{\textstyle
    \sum}}

\newcommand{\half}{{\textstyle \frac12}}
\newcommand{\third}{{\textstyle \frac13}}
\newcommand{\fourth}{{\textstyle \frac14}}
\newcommand{\fifth}{{\textstyle \frac15}}

\newcommand{\Stab}{\mathbb{S}}
\newcommand{\StabThr}[2]{\Gamma_{#1}(#2)}
\newcommand{\StabThrmin}[2]{{\underline{\Gamma}}_{#1}(#2)}
\newcommand{\StabThrmax}[2]{{\overline{\Gamma}}_{#1}(#2)}
\newcommand{\TestFcn}{\Psi}

\renewcommand{\phi}{\varphi}

\begin{document}
\title{Combinatorial Statistics - Homework Set 4 Solution}

\date{\today}
\maketitle
\subsection{Maximum Likelihood on Tree Markov Random Fields}
\begin{itemize}
\item This problem can be solved with a variant of the Belief
  Propagation algorithm shown in class. We will describe the solution
  but not prove its correctness. Note that this solution is
  essentially equivalent to (but perhaps not as efficient as) a
  Dynamic Programming solution which finds the maximum likelihood
  assignment from the leaves up.
   
  In analogy with the definitions of BP that were shown in class, we
  make the following definitions:
  \begin{enumerate}
  \item The message from $v$ to $C$, $\mu_{v\to C}(x)$, is defined to
    be the maximum likelihood of any assignment such that
    $\sigma_v=x$, in the MRF where $C$ is removed.
  \item The message from $C$ to $v$, $\mu_{C\to v}$, is defined to be
    the maximum likelihood of any assignment in which $\sigma_v=x$, in
    the MRF where all cliques containing $v$ but $C$ are removed.
  \end{enumerate}
  Following a line of reasoning analogous to the one demonstrated in
  class for BP, one can prove that
  \begin{equation}
    \label{eq:bp1}
    \mu_{v\to C}(x) \equiv
    \prod_{\substack{C'\neq C\\ v\in C'}}\mu_{C'\to v}(x)
  \end{equation}
  and
  \begin{equation}
    \label{eq:bp2}
    \mu_{C\to v}(x) \equiv
    \max_{\sigma_C:\sigma_v=x}
    \left\{
      \psi_C(\sigma_C)
      \prod_{\substack{w\in C\\ q\neq v}}\mu_{w\to C}(\sigma_w)
    \right\}.
  \end{equation}
  One can now calculate $\mu_{v\to C}$ and $\mu_{C\to v}$ in time
  equal to that of the BP calculation.  Then any assignment $x$ for
  $v$ that maximizes $\prod_{C:v\in C}\mu_{C\to v}(x)$ is part of a
  maximum likelihood assignment. We can then pick some vertex $v$,
  assign it such an assignment $x$, and repeat the calculation above
  for an MRF for which $v$ is fixed to $x$.
  \end{equation}
  
\item By the symmetry of the model, we may assign 1 to some vertex
  $r$. Iterating until all vertices are assigned, we choose the sign
  of a yet-to-be-assigned vertex $v$, neighboring an assigned vertex
  $u$, to be such that $\beta_{uv}\sigma_u\sigma_v$ is positive, thus
  maximizing that edge's $(u,v)$ contribution to the
  likelihood. Finally, since each edge's contribution is maximal, then
  the entire assignment is maximal.

\item We choose some spanning tree $T$ on the graph, and run the same
  algorithm as above. Let $e=(u,v)$ be some edge of the graph. If $e$
  is in $T$ then $\beta_{uv}\sigma_u\sigma_v$ is positive and its
  contribution is maximal. Otherwise consider the cycle containing $e=e(1)$
  and the path from $u$ to $v$ in $T$, $e(2),\ldots,e(k)$. Then 
  \begin{align*}
    \beta_{uv}\sigma_u\sigma_v &=
    {\prod_{i=1}^k\beta_{e(i)}\sigma_{u_i}\sigma_{v_i}
      \over
      \prod_{i=2}^k\beta_{e(i)}\sigma_{u_i}\sigma_{v_i}
    }
    \\ &= 
    {\prod_{i=1}^k\beta_{e(i)}\prod_{i=1}^k\sigma_{u_i}^2
      \over
      \prod_{i=2}^k\beta_{e(i)}\sigma_{u_i}\sigma_{v_i}
    }
    \\ &= 
    {\prod_{i=1}^k\beta_{e(i)}
      \over
      \prod_{i=2}^k\beta_{e(i)}\sigma_{u_i}\sigma_{v_i}
    }.
  \end{align*}
  Now, the numerator is positive by the stipulation on the graph's
  cycles, and the denominator is positive by our assignments, and so
  it follows that  $\beta_{uv}\sigma_u\sigma_v$ is positive in this
  case, too.
\end{itemize}


\subsection{The Complexity of BP and Small Tree Width Graphs}
\begin{itemize}
\item We omit the answer to this question, as it is straightforward
  and everyone seems to have gotten it right.
\item Briefly: We consider each set $D_i$ as a clique and run BP on
  this representation of the MRF. Running time will be exponential in
  the maximum size of a set $D_i$. 
\item We run the algorithm above, where each column is a set
  $D_i$. Since the size of the sets is $n$, the running time is $2^{O(n)}$.
\end{itemize}

\end{document}


















