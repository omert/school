\documentclass[11pt]{article} \usepackage{amssymb}
\usepackage{amsfonts} \usepackage{amsmath} \usepackage{bm}
\usepackage{latexsym} \usepackage{epsfig}

\setlength{\textwidth}{6.5 in} \setlength{\textheight}{8.25in}
\setlength{\oddsidemargin}{0in} \setlength{\topmargin}{0in}
\addtolength{\textheight}{.8in} \addtolength{\voffset}{-.5in}

\newtheorem{theorem}{Theorem}[section]
\newtheorem{lemma}[theorem]{Lemma}
\newtheorem{proposition}[theorem]{Proposition}
\newtheorem{corollary}[theorem]{Corollary}
\newtheorem{fact}[theorem]{Fact}
\newtheorem{definition}[theorem]{Definition}
\newtheorem{remark}[theorem]{Remark}
\newtheorem{conjecture}[theorem]{Conjecture}
\newtheorem{example}[theorem]{Example}
\newenvironment{proof}{\noindent \textbf{Proof:}}{$\Box$}

\newcommand{\ignore}[1]{}

\newcommand{\enote}[1]{} \newcommand{\knote}[1]{}
\newcommand{\rnote}[1]{}



% \newcommand{\enote}[1]{{\bf [[Elchanan:} {\emph{#1}}{\bf ]]}}
% \newcommand{\knote}[1]{{\bf [[Krzysztof:} {\emph{#1}}{\bf ]]}}
% \newcommand{\rnote}[1]{{\bf [[Ryan:} {\emph{#1}}{\bf ]]}}



\DeclareMathOperator{\Support}{Supp} \DeclareMathOperator{\Opt}{Opt}
\DeclareMathOperator{\Ordo}{\mathcal{O}}
\newcommand{\MaxkCSP}{\textsc{Max $k$-CSP}}
\newcommand{\MaxkCSPq}{\textsc{Max $k$-CSP$_{q}$}}
\newcommand{\MaxCSP}[1]{\textsc{Max CSP}(#1)} \renewcommand{\Pr}{{\bf
    P}} \renewcommand{\P}{{\bf P}} \newcommand{\Px}{\mathop{\bf P\/}}
\newcommand{\E}{{\bf E}} \newcommand{\Cov}{{\bf Cov}}
\newcommand{\Var}{{\bf Var}} \newcommand{\Varx}{\mathop{\bf Var\/}}

\newcommand{\bits}{\{-1,1\}}

\newcommand{\nsmaja}{\textstyle{\frac{2}{\pi}} \arcsin \rho}

\newcommand{\Inf}{\mathrm{Inf}} \newcommand{\I}{\mathrm{I}}
\newcommand{\J}{\mathrm{J}}

\newcommand{\eps}{\epsilon} \newcommand{\lam}{\lambda}

% \newcommand{\trunc}{\ell_{2,[-1,1]}}
\newcommand{\trunc}{\zeta} \newcommand{\truncprod}{\chi}

\newcommand{\N}{\mathbb N} \newcommand{\R}{\mathbb R}
\newcommand{\Z}{\mathbb Z} \newcommand{\CalE}{{\mathcal{E}}}
\newcommand{\CalC}{{\mathcal{C}}} \newcommand{\CalM}{{\mathcal{M}}}
\newcommand{\CalR}{{\mathcal{R}}} \newcommand{\CalS}{{\mathcal{S}}}
\newcommand{\CalV}{{\mathcal{V}}}
\newcommand{\CalX}{{\boldsymbol{\mathcal{X}}}}
\newcommand{\CalG}{{\boldsymbol{\mathcal{G}}}}
\newcommand{\CalH}{{\boldsymbol{\mathcal{H}}}}
\newcommand{\CalY}{{\boldsymbol{\mathcal{Y}}}}
\newcommand{\CalZ}{{\boldsymbol{\mathcal{Z}}}}
\newcommand{\CalW}{{\boldsymbol{\mathcal{W}}}}
\newcommand{\CalF}{{\mathcal{Z}}}
% \newcommand{\boldG}{{\boldsymbol G}}
% \newcommand{\boldQ}{{\boldsymbol Q}}
% \newcommand{\boldP}{{\boldsymbol P}}
% \newcommand{\boldR}{{\boldsymbol R}}
% \newcommand{\boldS}{{\boldsymbol S}}
% \newcommand{\boldX}{{\boldsymbol X}}
% \newcommand{\boldB}{{\boldsymbol B}}
% \newcommand{\boldY}{{\boldsymbol Y}}
% \newcommand{\boldZ}{{\boldsymbol Z}}
% \newcommand{\boldV}{{\boldsymbol V}}
\newcommand{\boldi}{{\boldsymbol i}} \newcommand{\boldj}{{\boldsymbol
    j}} \newcommand{\boldk}{{\boldsymbol k}}
\newcommand{\boldr}{{\boldsymbol r}}
\newcommand{\boldsigma}{{\boldsymbol \sigma}}
\newcommand{\boldupsilon}{{\boldsymbol \upsilon}}
\newcommand{\hone}{{\boldsymbol{H1}}}
\newcommand{\htwo}{\boldsymbol{H2}}
\newcommand{\hthree}{\boldsymbol{H3}}
\newcommand{\hfour}{\boldsymbol{H4}}


\newcommand{\sgn}{\mathrm{sgn}} \newcommand{\Maj}{\mathrm{Maj}}
\newcommand{\Acyc}{\mathrm{Acyc}}
\newcommand{\UniqMax}{\mathrm{UniqMax}}
\newcommand{\Thr}{\mathrm{Thr}} \newcommand{\littlesum}{{\textstyle
    \sum}}

\newcommand{\half}{{\textstyle \frac12}}
\newcommand{\third}{{\textstyle \frac13}}
\newcommand{\fourth}{{\textstyle \frac14}}

\newcommand{\Stab}{\mathbb{S}}
\newcommand{\StabThr}[2]{\Gamma_{#1}(#2)}
\newcommand{\StabThrmin}[2]{{\underline{\Gamma}}_{#1}(#2)}
\newcommand{\StabThrmax}[2]{{\overline{\Gamma}}_{#1}(#2)}
\newcommand{\TestFcn}{\Psi}

\renewcommand{\phi}{\varphi}

\begin{document}
\title{RWRE - Exercise 1}

 \author{Omer Tamuz, 035696574}
\maketitle

Suppose $\{\omega_x\}$ is a stationary, finite state, irreducible MC.
\section{}
Prove  $\{\omega_x\}$ is ergodic.
\section{}
Let the MC have two states: $\omega_-$ and $\omega_+$. Let the transition
probability from $\omega_-$ to $\omega_+$ be $p$, and the transition
probability in the other direction be $q$. Then the stationary distribution
is $P(\omega_-)={q\over q+p}$, $P(\omega_+)={p\over q+p}$. 
The expected value of $\rho$ is:
\begin{equation*}
  \E\rho
    = {1\over q+p}\left(q{1-\omega_- \over \omega_-}+
                        p{1-\omega_+ \over \omega_+}\right).
\end{equation*}

The speed is:
\begin{eqnarray*}
  v &=& {1-\E\rho \over 1+\E\rho}\\
  v &=& {1-{1\over q+p}\left(q{1-\omega_- \over \omega_-}+
                        p{1-\omega_+ \over \omega_+}\right) 
          \over 1+{1\over q+p}\left(q{1-\omega_- \over \omega_-}+
                        p{1-\omega_+ \over \omega_+}\right)}\\
  v &=& {q+p-q{1-\omega_- \over \omega_-}- p{1-\omega_+ \over \omega_+}
          \over q+p+q{1-\omega_- \over \omega_-}+p{1-\omega_+ \over \omega_+}}\\
  v &=& {q{2\omega_--1 \over \omega_-}+ p{2\omega_+-1 \over \omega_+}
          \over q{1 \over \omega_-}+p{1 \over \omega_+}}\\
  v &=& {2(q+p)-q/\omega_--p/\omega_+
          \over q{1 \over \omega_-}+p{1 \over \omega_+}}\\
  v &=& 2{q+p \over q/\omega_-+p/\omega_+}-1\\
  v &=& {2\over\E[1/\omega]}-1\\
\end{eqnarray*}
\section{}
The i.i.d. case is obtained when $p=1-q$. 

Let $A=10000$. 
Let $\gamma$ be the unique solution in $[0,1]$ of the equation 
$\log\gamma=A\log(1-\gamma)$. 
Then $\gamma<0.001$, and $\log(\gamma/(1-\gamma))=(A-1)\log(1-\gamma)$.
If $\omega_-=\gamma$, $\omega_+=1-\gamma$, with $P(\omega_+)=1-p=2/3$, then:
\begin{equation*}
  \E\omega={2\over 3}(1-\gamma)+{1\over 3}\gamma
          ={2\over 3}-{1\over 3}\gamma>{1\over 2},
\end{equation*}
whereas 
\begin{eqnarray*}
  \E[\log\rho]&=&{2\over 3}\log({\gamma\over 1-\gamma})+
                 {1\over 3}\log({1-\gamma\over \gamma})\\
  \E[\log\rho]&=&{1\over 3}\log(\gamma)-
                 {1\over 3}\log(1-\gamma)\\
  \E[\log\rho]&=&{1\over 3}(A-1)\log(1-\gamma)\\
              &<&0
\end{eqnarray*} 
and
\begin{eqnarray*}
  \E\rho
  &=& {2\over 3}{\gamma \over 1-\gamma}+{1\over 3}{1-\gamma \over \gamma}\\
  &=& {2\over 3}{\gamma^2 \over \gamma(1-\gamma)}
      +{1\over 3}{(1-\gamma)^2 \over \gamma(1-\gamma)}\\
\end{eqnarray*}

Then 
$\E\omega=(1-p)\omega_-+p\omega_+$. 
\end{document}




